\chapter*{Chapter 111: \\
	Alcoholic Beverages}
    \addstarredchapter{Chapter 111: Alcoholic Beverages}
    \vspace{-1cm}
    \minitoc
    \pagebreak

\subchapter{GENERAL PROVISIONS}

\section{Definitions\footnote{(‘83 Code, SEC. 5.01)}}
\index{ALCOHOLIC BEVERAGES!GENERAL PROVISIONS!Definitions}
For the purpose of this chapter the following definitions shall apply, unless the context clearly indicates or requires a different meaning.

\begin{description}
    \item[ALCOHOLIC BEVERAGE] Any beverage containing more than one-half of one percent alcohol by volume, including, but not limited to, “beer,” “wine,” and “liquor” as defined in this section.
    \item[APPLICANT] Any person making an application for a license under this chapter.
    \item[APPLICATION] A form with blanks or spaces thereon, to be filled in and completed by the applicant as his or her request for a license, furnished by the city and uniformly required as a prerequisite to the consideration of the issuance of a license for a business. \footnote{(Ord. 31, 2nd Series, effective 4-16-86)}
    \item[BEER] Malt liquor containing not less than one-half of one percent alcohol by volume nor more than 3.2\% alcohol by weight. (This definition includes so-called “malt coolers” with the alcoholic content limits stated herein.)
    \item[BREWER] A person who manufactures beer for sale.
    \item[CHURCH] A building which is primarily used as a place where persons of the same faith regularly assemble for the public worship of God. \footnote{(Ord. 31, 2nd Series, effective 4-16-86)}
    \item[CLUB] An incorporated organization organized under the laws of the state for civic, fraternal, social, or business purposes, for intellectual improvement, or for the promotion of sports, or a congressionally chartered veterans’ organization, which has more than 50 members; has owned or rented a building or space in a building for more than one year that is suitable and adequate for the accommodation of its members; and is directed by a board of directors, executive committee, or other similar body chosen by the members at a meeting held for that purpose.  No member, officer, agent, or employee shall receive any profit from the distribution or sale of beverages to the members of the club, or their guests, beyond a reasonable salary or wages fixed and voted each year by the governing body.  The club or congressionally chartered veterans’ organization must have been in existence for at least three years.
    \item[COMMISSIONER] The Minnesota Commissioner of Public Safety. \footnote{(Ord. 31, 2nd Series, effective 4-16-86)}
    \item[EXCLUSIVE LIQUOR STORE] An establishment used exclusively for the sale of liquor except for the incidental sale of ice, tobacco, beer, beverages for mixing with liquor, soft drinks, cork extraction devices, and books and videos on the use of alcoholic beverages in the preparation of food, and the establishment may offer recorded or live entertainment.  \textbf{EXCLUSIVE LIQUOR STORE} also includes an on-sale or combination on-sale and off-sale liquor establishment which sells food for on-premise consumption when authorized by the city. \footnote{(Ord. 54, 2nd Series, effective 11-26-88)}
    \item[FULL MENU] At least two meals, consisting of a luncheon and a dinner, during the hours of wine and liquor sales on at last five days each calendar week, exclusive of Sundays; the luncheon must consist of at least four items served for at least two hours between 11:00 a.m. and 2:00 p.m. and the dinner must be served of at least four items served for at least four hours between 5:00 p.m. and midnight.
    \item[HOTEL/MOTEL] An establishment where food and lodging are regularly furnished to transients and which has a dining room serving the general public at tables and having facilities for seating at least 30 guests at one time, and at least ten guest rooms. \footnote{(Ord. 102, 2nd Series, effective 5-13-95)}
    \item[LICENSE] A document, issued by the city, to an applicant permitting him or her to carry on and transact the business stated therein.
    \item[LICENSE FEE] The money paid to the city pursuant to an application and prior to issuance of a license to transact and carry on the business stated therein. \footnote{(Ord. 31, 2nd Series, effective 4-16-86)}
    \item[LICENSED BED AND BREAKFAST] An establishment licensed under city code, SEC.SEC. 114.01 through 114.03. \footnote{(Ord. 79, 2nd Series, effective 8-18-92)}
    \item[LICENSED PREMISES] The space or structure described in the issued license.  Unless limited in the issued license, in the case of a restaurant or a club licensed for on-sales of alcoholic beverages and located on a golf course, LICENSED PREMISES means the entire golf course except for areas where motor vehicles are regularly parked or operated. \footnote{(Ord. 102, 2nd Series, effective 5-13-95)}
    \item[LICENSEE] An applicant who, pursuant to his or her approved application, holds a valid, current, unexpired license, which has neither been revoked nor is then under suspension, from the city for carrying on the business stated therein.
    \item[LIQUOR] Ethyl alcohol and distilled, fermented, spirituous, vinous and malt beverages containing in excess of 3.2\% of alcohol by weight. (This definition includes so-called “wine coolers” and “malt coolers” with the alcoholic content limits stated herein.)
    \item[MALT LIQUOR] Any beer, ale, or other beverage made from malt by fermentation and containing not less than one-half of one percent alcohol by volume.
    \item[MANUFACTURER] Every person who, by any process of manufacture, fermenting, brewing, distilling, refining, rectifying, blending, or by the combination of different materials, prepares or produces alcoholic beverages for sale. \footnote{(Ord. 31, 2nd Series, effective 4-16-86)}
    \item[MINOR] Any natural person who has not attained the age of 21 years. \footnote{(Ord. 63, 2nd Series, effective 7-21-90)}
    \item[OFF-SALE] The sale of alcoholic beverages in original packages for consumption off the licensed premises only.
    \item[ON-SALE] The sale of alcoholic beverages for consumption on the licensed premises only.\
    \item[PACKAGE and ORIGINAL PACKAGE] Any container or receptacle holding alcoholic beverages, which container or receptacle is corked, capped or sealed by a manufacturer or wholesaler.
    \item[RESTAURANT] An establishment, other than a hotel, under the control of a single proprietor or manager, where meals are regularly prepared on the premises and served at tables to the general public, and having seating capacity for at least 30 guests. \footnote{(Ord. 102, 2nd Series, effective 5-13-95)}
    \item[SALE, SELL and SOLD] All barters and all manners or means of furnishing beer, wine or liquor to persons, including the furnishing in violation or evasion of law.
    \item[WHOLESALER] Any person engaged in the business of selling alcoholic beverages to a licensee from a stock maintained in a warehouse. \footnote{(Ord. 31, 2nd Series, effective 4-16-86)}
    \item[WINE] The product made from the normal alcoholic fermentation of grapes, including still wine, sparkling and carbonated wine, wine made from condensed grape must, wine made from other agricultural products than sound, ripe grapes, imitation wine, compounds sold as wine, vermouth, cider, perry and sake. (This definition includes “wine coolers” with the alcoholic content limits stated herein.) For purposes of on-sale wine licenses, WINE may contain up to 14\% alcohol by volume for consumption with the sale of food.  For all other purposes, WINE is a product containing not less than one-half of one percent nor more than 24\% alcohol by volume for nonindustrial use. \footnote{(Ord. 102, 2nd Series, effective 5-13-95)}
\end{description}

\subchapter{GENERAL LICENSING PROVISIONS}

\setcounter{section}{14}
\section{Application and Fees}
\index{ALCOHOLIC BEVERAGES!GENERAL LICENSING PROVISIONS!Application and Fees}
\subsection{}
All applications shall be made at the office of the Clerk-Treasurer upon forms prescribed by the city, or if by the Commissioner, then together with the additional information as the Council may desire. Information required may vary with the type of business organization making application. All questions asked or information required by the application forms shall be answered fully and completely by the applicant. Every application for the issuance or renewal of an alcoholic beverage license must include a copy of each summons received by the applicant during the preceding year under M.S. \textsection 340A.802, as it may be amended from time to time.\footnote{(Ord. 54, 2nd Series, effective 11-26-88)}
\subsection{}
It is unlawful for any applicant to intentionally make a false statement or omission upon any application form. Any false statement in the application, or any willful omission to state any information called for on the application form shall, upon discovery of the falsehood, work an automatic refusal of license, or if already issued, shall render any license issued pursuant thereto void and of no effect to protect the applicant from prosecution for violation of this chapter, or any part thereof.
\subsection{}
At the time of the initial application, an applicant for an on-sale liquor license, an applicant for an beer license, or an applicant for an on-sale wine license shall pay to the city an application fee as established by Council, which fee shall be considered an application and investigation fee, not refundable to applicant, to cover the costs of the city in processing the application and the investigation thereof. No fee shall be required of an applicant for a temporary beer license. Should the Council or the Bureau of Criminal Apprehension determine that a comprehensive background investigation of an applicant for an on-sale liquor license is necessary, if the investigation is conducted within the state or the actual cost of the investigation does not exceed current limits then the applicant shall pay to the city an investigation fee as established by the City Council Fee Schedule. If the investigation is required outside the state an increased investigation fee takes effect, as established by the City Council Fee Schedule.\footnote{(‘83 Code, SEC. 5.02, Subds. 1, 2, 3) (See annual Fee Schedule.)}

\section{Action on Application}
\index{ALCOHOLIC BEVERAGES!GENERAL LICENSING PROVISIONS!Granting}
\subsection{Granting}
The Council may approve any application for the period of the remainder of the then current license year or for the entire ensuing license year.  All applications including proposed license periods must be consistent with this chapter.  Prior to consideration of any application for a license, the applicant shall pay the license fee, and if applicable, pay the investigation fee.  Upon rejection of any application for a license, or upon withdrawal of an application before approval of the issuance by the Council, the license fee shall be refunded to the applicant.  Failure to pay any portion of a fee when due shall be cause for revocation.
\subsection{Issuing}
If an application is approved, the Clerk-Treasurer shall forthwith issue a license pursuant thereto in the form prescribed by the city or the proper Department of the State of Minnesota, as the case may be, and upon payment of the license fee.  All licenses shall be on a calendar year basis unless otherwise specified herein.  For licenses issued and which are to become effective other than on the first day of the licensed year, the fee to be paid with the application shall be a pro rata share of the annual license fee.  Licenses shall be valid only at one location and on the premises therein described.
\subsection{License Refund in Certain Cases}
In the event that, during the license year, the licensed premises shall be destroyed or so damaged by fire, or otherwise, that the licensee shall cease to carry on the licensed business, or in case the business of the licensee shall cease by reason of his or her illness or death, or if it shall become unlawful for the licensee to carry on the licensed business under his or her license, except when the license is revoked, the city shall, upon the happening of any event, refund to the licensee, or to his or her estate, the part of the license fee paid by him as corresponds to the time the license had yet to run.  In the event of death of the licensee, his or her personal representative is hereby authorized to continue operation of the business for not more than 90 days after the death of the licensee.
\subsection{Transfer}
A license shall be transferable between persons upon consent of the Council and payment of the investigation fee.  No license shall be transferable to a different location without prior consent of the Council and payment of the fee for a duplicate license.  It is unlawful to make any transfer in violation of this division (D).\footnote{(Ord. 31, 2nd Series, effective 4-16-86)}
\subsection{Refusal and Termination}
The Council may, in its sole discretion and for any reasonable cause, refuse to grant any application.  No license shall be granted to a person of questionable moral character or business reputation.  Licenses shall terminate only by expiration or revocation.
\subsection{Revocation or Suspension\footnote{(Ord. 102, 2nd Series, effective 5-13-95)}}
For any license granted under the provisions of this chapter, the Council may revoke, suspend for a period not to exceed 60 days, impose a civil fine not to exceed \$2,000, or any combination of these sanctions, for each violation on a finding that the licensee has failed to comply with a statute, regulation or provision of the city code relating to alcoholic beverages. The Council shall revoke the license upon conviction of any licensee or agent or employee of a licensee for violating any law relating to the sale or possession of beer, wine or liquor upon premises of the licensee, or if the revocation is mandatory by statute. If it shall be made to appear at the hearing thereon that the violation was not willful, the Council may order suspension; provided that revocation shall be ordered upon the third violation or offense. No suspension or revocation shall take effect until the licensee has been afforded an opportunity for a hearing before the Council, a committee of the Council, or a hearing under the Administrative Procedures Act, as may be determined by the Council in action calling the hearing. The hearing shall be called by the Council upon written notice to the licensee served in person or by certified mail not less than 15 nor more than 30 days prior to the hearing date, stating the time, place and purpose thereof. As additional restrictions or regulations on licensees under this chapter, and in addition to grounds for revocation or suspension stated in the city code or statute, the following shall also be grounds for the action:
\begin{enumerate}[{\indent}1)]
    \item That the licensee suffered or permitted illegal acts upon licensed premises unrelated to the sale of beer, wine or liquor; 
    \item That the licensee had knowledge of the illegal acts upon licensed premises, but failed to report the same to police; 
    \item That the licensee failed or refused to cooperate fully with police in investigating the alleged illegal acts upon licensed premises; or
    \item That the activities of the licensee created a serious danger to public health, safety, or welfare.
\end{enumerate}
\subsection{Corporate Applicants and Licensees}
A corporate applicant, at the time of application, shall furnish the city with a list of all persons that have an interest in the corporation and the extent of the interest.  The list shall name all shareholders and show the number of shares held by each, either individually or beneficially for others.  It is the duty of each corporate licensee to notify the city Clerk-Treasurer in writing of any change in legal ownership, or beneficial interest in the corporation or in the shares.  Any change in the ownership or beneficial interest in the shares entitled to be voted at a meeting of the shareholders of a corporate licensee, which results in the change of voting control of the corporation by the persons owning the shares therein, shall be deemed equivalent to a transfer of the license issued to the corporation, and any license shall be revoked 30 days after any change in ownership or beneficial interest of shares unless the Council has been notified of the change in writing and has approved it by appropriate action.  The Council, or any officer of the city designated by it, may at any reasonable time examine the stock transfer records and minute books of any corporate licensee in order to verify and identify the shareholders, and the Council or its designated officer may examine the business records of any other licensee to the extent necessary to disclose the interest which persons other than the licensee have in the licensed business.  The Council may revoke any license issued upon its determination that a change of ownership of shares in a corporate licensee or any change of ownership of any interest in the business of any other licensee has actually resulted in the change of control of the licensed business so as materially to affect the integrity and character of its management and its operation, but no action shall be taken until after a hearing by the Council on notice to the licensee.\footnote{(‘83 Code, SEC. 5.02, Subd. 4)}

\section{Disqualifications}
\index{ALCOHOLIC BEVERAGES!GENERAL LICENSING PROVISIONS!Disqualifications}
\subsection{}
No license under this chapter may be issued, or renewed, to:
\begin{enumerate}[{\indent}1)]
    \item A person who within five years of the license application has been convicted of any felony or a willful violation of a federal or state law, or local ordinance governing the manufacture, sale, distribution, or possession for sale or distribution, of alcoholic beverages; 
    \item A person who has had an alcoholic beverage license revoked within five years of the license application, or to any person who at the time of the violation owns any interest, whether as a holder of more than 5\% of the capital stock of a corporate licensee, as a partner or otherwise, in the premises or in the business conducted thereon, or to a corporation, partnership, association, enterprise, business, or firm in which any person is in any manner interested; 
    \item A person under the age of 21 years; or
    \item A person not of good moral character and repute.
\end{enumerate}
\subsection{}
No person holding a license from the Commissioner as a manufacturer, brewer (except as provided by statute), wholesaler or importer, may have a direct or indirect interest, in whole or in part, in a business holding an alcoholic beverage license from the city.\footnote{(‘83 Code, SEC. 5.02, Subd. 8)  (Ord. 127, 2nd Series, effective 5-16-98)}
\subsection{}
No license under this chapter shall be granted for operation on any premises upon which taxes, assessments, or installments thereof, or other financial claims of the city are owed.\footnote{(‘83 Code, SEC. 5.04)}

\section{Issuance of License to One Person or Premises}
\index{ALCOHOLIC BEVERAGES!GENERAL LICENSING PROVISIONS!Issuance of License to One Person or Premises}
\subsection{Limitation on Issuance of Licenses to One Person or Place}
No more than one off-sale liquor license shall be issued to any one person or for any one place.\footnote{(‘83 Code, SEC. 5.05)  (Ord. 76, 2nd Series, effective 1-18-92)}
\subsection{Premises Licensed}
A license issued under the provisions of this chapter shall be valid only for the premises described in the license, and all transactions relating to a sale under the license must take place within the space or structure.\footnote{(‘83 Code, SEC. 5.07)  (Ord. 102, 2nd Series, effective 5-13-95)}

\section{Financial Responsibility of Licensees}
\index{ALCOHOLIC BEVERAGES!GENERAL LICENSING PROVISIONS!Financial Responsibility of Licensees}
\subsection{Proof\footnote{(Ord. 31, 2nd Series, effective 4-16-86)}}
No beer, wine or liquor license shall be issued or renewed unless and until the applicant has provided proof of financial responsibility imposed by Minnesota Statutes, by filing with the city a certificate that there is in effect an insurance policy or pool providing minimum coverages of:
\begin{enumerate}[{\indent}1)]
    \item \$50,000 because of bodily injury to any one person in any one occurrence, and, subject to the limit for one person, in the amount of \$100,000 because of bodily injury to two or more persons in any one occurrence, and in the amount of \$10,000 because of injury to or destruction of property of others in any one occurrence; and 
    \item \$50,000 for loss of means of support of any one person in any one occurrence; and, subject to the limit for one person, \$100,000 for loss of means of support of two or more persons in any one occurrence.
\end{enumerate}
\subsection{Exception}
This section does not apply to on-sale beer licensees with sales of beer of less than \$10,000 for the preceding year, nor to off-sale beer licensees with sales of beer of less than \$20,000 for the preceding year, nor does it apply to holders of on-sale wine licenses with sales of wine of less than \$10,000 for the preceding year.  An affidavit of the licensee shall be required to establish the exemption under this division (B).
\subsection{Documents Submitted to Commissioner}
All proofs of financial responsibility and exemption affidavits filed with the city under this section shall be submitted by the city to the Minnesota Commissioner of Public Safety.\footnote{(‘83 Code, SEC. 5.13)}

\section{Insurance Certificate Requirements}
\index{ALCOHOLIC BEVERAGES!GENERAL LICENSING PROVISIONS!Insurance Certificate Requirements}
Whenever an insurance certificate is required by this chapter the applicant shall file with the Clerk-Treasurer a certificate of insurance showing that the limits are at least as high as required, that coverage is effective for at least the license term approved, and that the insurance will not be cancelled or terminated without 30 days’ written notice served upon the Clerk-Treasurer.  Cancellation or termination of the coverage shall be grounds for license revocation.\footnote{(‘83 Code, SEC. 5.14)  (Ord. 7, 2nd Series, effective 5-15-84)}

\section{Posting of License Required}
\index{ALCOHOLIC BEVERAGES!GENERAL LICENSING PROVISIONS!Posting of License Required}
All licensees shall conspicuously post their licenses in their places of business.\footnote{(‘83 Code, SEC. 5.02, Subd. 6)  Penalty, see SEC. 110.99}

\section{Resident Manager or Agent}
\index{ALCOHOLIC BEVERAGES!GENERAL LICENSING PROVISIONS!Resident Manager or Agent}
Before a license is issued under this chapter to an individual who is a non-resident of the city, to more than one individual whether or not they are residents of the city, or to a corporation, partnership, or association, the applicant or applicants shall appoint in writing a natural person who is a resident of the city as its manager or agent.  The resident manager or agent shall, by the terms of his or her written consent, take full responsibility for the conduct of the licensed premises, and, serve as agent for service of notices and other process relating to the license.  The manager or agent must be a person who, by reason of age, character, reputation, and other attributes, could qualify individually as a licensee.  If the manager or agent ceases to be a resident of the city or ceases to act in the capacity for the licensee without appointment of a successor, the license issued pursuant to the appointment shall be subject to revocation or suspension.\footnote{(‘83 Code, SEC. 5.02, Subd. 7)}

\section{Duplicate License; Renewal of License}
\index{ALCOHOLIC BEVERAGES!GENERAL LICENSING PROVISIONS!Duplicate License; Renewal of License}
\subsection{}
Duplicates of all original licenses under this chapter may be issued by the Clerk-Treasurer without action by the Council, upon licensee’s affidavit that the original has been lost, and upon payment of a fee of \$2 for issuance of the duplicate.  All duplicate licenses shall be clearly marked \textbf{DUPLICATE}.\footnote{(‘83 Code, SEC. 5.02, Subd. 5)}
\subsection{}
Applications for renewal of all licenses under this chapter shall be made at least 60 days prior to the date of expiration of the license, and shall contain the information as is required by the city.  This time requirement may be waived by the Council for good and sufficient cause.  All license fees for the ensuing license year shall be paid in the form of cash, certified check, cashier’s check or money order by the fifteenth day of December prior thereto.\footnote{(‘83 Code, SEC. 5.03)}

\section{Conditional Licenses}
\index{ALCOHOLIC BEVERAGES!GENERAL LICENSING PROVISIONS!Conditional Licenses}
It shall be a condition upon every license issued under this chapter that the circumstances under which license is issued shall remain substantially the same during the entire period of the license.  Any change in circumstance adversely affecting the eligibility of the holder of the license to be then issued the license shall be grounds for revocation, suspension, or civil fine.  Notwithstanding any provision of law to the contrary, the Council may, upon a finding of the necessity therefor, place the special conditions and restrictions, in addition to those stated in this chapter, upon any license as it, in its discretion, may deem reasonable and justified.\footnote{(‘83 Code, SEC. 5.06)  (Ord. 39, 2nd Series, effective 4-23-87)}

\section{License Fee Increases}
\index{ALCOHOLIC BEVERAGES!GENERAL LICENSING PROVISIONS!License Fee Increases}
No license fee for on-sale or off-sale beer, on-sale or off-sale liquor (including clubs), or on-sale wine, shall be increased except after notice and hearing thereon.  Notice of the proposed increase shall be mailed at least 30 days before the hearing date to all then-current licensees and persons, if any, whose applications for the licenses are then pending before the Council.\footnote{(Ord. 63, 2nd Series, effective 7-21-90)}

\subchapter{\mbox{OPERATION OF LICENSED PREMISES;} \mbox{PROHIBITED CONDUCT}}

\setcounter{section}{39}
\section{Unlawful Acts}
\index{ALCOHOLIC BEVERAGES!OPERATION OF LICENSED PREMISES; PROHIBITED CONDUCT!Unlawful Acts}
\subsection{Consumption}
It is unlawful for any person to consume, or any licensee to permit consumption of, beer, wine or liquor on licensed premises more than 20 minutes after the hour when a sale thereof can legally be made.
\subsection{Removal of Containers}
It is unlawful for any on-sale licensee to permit any glass, bottle or other container, containing beer, wine or liquor in any quantity, to remain upon any table, bar, stool or other place where customers are served, more than 20 minutes after the hour when a sale thereof can legally be made.
\subsection{Closing}
It is unlawful for any person, other than an on-sale licensee or his or her bona fide employee actually engaged in the performance of his or her duties, to be on premises licensed under this chapter more than 30 minutes after the legal time for making licensed sales.  Provided, however, that this division shall not apply to licensees, employees of licensees and patrons on licensed premises for the sole purpose of preparing, serving or consuming food or beverages other than beer, wine or liquor.\footnote{(‘83 Code, SEC. 5.08)  Penalty, see SEC. 110.99}

\section{Conduct of Place of Business; Sale by Employee}
\index{ALCOHOLIC BEVERAGES!OPERATION OF LICENSED PREMISES; PROHIBITED CONDUCT!Conduct of Place of Business; Sale by Employee}
\subsection{}
Except as herein provided, every licensee under this chapter shall be responsible for the conduct of his or her place of business and shall maintain conditions of sobriety and order therein.\footnote{(‘83 Code, SEC. 5.09)}
\subsection{}
Any sale of an alcoholic beverage in or from any premises licensed under this chapter by any employee authorized to make the sale in or from the place is the act of the employer as well as of the person actually making the sale; and every employer is liable to all of the penalties, except criminal penalties, provided by law for the sale, equally with the person actually making the sale.\footnote{(‘83 Code, SEC. 5.10)  (Ord. 54, 2nd Series, effective 11-26-88)}

\section{Premises Open to Inspection}
\index{ALCOHOLIC BEVERAGES!OPERATION OF LICENSED PREMISES; PROHIBITED CONDUCT!Premises Open to Inspection}
\subsection{}
All premises licensed under this chapter shall at all times be open to inspection by any police officer to determine whether or not this chapter and all other laws are being observed.  All persons, as a condition to being issued the license, consent to the inspection by the officers and without a warrant for searches or seizures.
\subsection{}
It is unlawful for any licensee, or agent or employee of a licensee, to hinder or prevent a police officer from making the inspection.\footnote{(‘83 Code, SEC. 5.11)}

\section{Gambling Prohibited}
\index{ALCOHOLIC BEVERAGES!OPERATION OF LICENSED PREMISES; PROHIBITED CONDUCT!Gambling Prohibited}
It is unlawful for any licensee to keep, possess, or operate, or permit the keeping, possession, or operation on licensed premises of dice or any other gambling device, or permit raffles to be conducted, except as are licensed by the Charitable Gambling Control Board and then only except as it complies with the established policy of the city.\footnote{(‘83 Code, SEC. 5.15)  Penalty, see SEC. 110.99}

\section{Unlawful Acts by Minors}
\index{ALCOHOLIC BEVERAGES!OPERATION OF LICENSED PREMISES; PROHIBITED CONDUCT!Unlawful Acts by Minors}
\subsection{Consumption}
It is unlawful for any:
\begin{enumerate}[{\indent}1)]
    \item Licensee to permit any minor to consume alcoholic beverages on licensed premises.
    \item Minor to consume alcoholic beverages except in the household of the minor’s parent or guardian, and then only with the consent of the parent or guardian.
\end{enumerate}
\subsection{Purchasing}
It is unlawful for any person:
\begin{enumerate}[{\indent}1)]
    \item To sell, barter, furnish, or give alcoholic beverages to a minor unless the person is the parent or guardian of the minor, and then only for consumption in the household of the parent or guardian.
    \item Minor to purchase or attempt to purchase any alcoholic beverage.
    \item To induce a minor to purchase or procure any alcoholic beverage.
\end{enumerate}
\subsection{Possession}
It is unlawful for a minor to possess any alcoholic beverage with the intent to consume it at a place other than the household of the minor’s parent or guardian.  Possession of an alcoholic beverage by a minor at a place other than the household of the parent or guardian is prima facie evidence of intent to consume it at a place other than the household of his or her parent or guardian.\footnote{(Ord. 31, 2nd Series, effective 4-16-86)}
\subsection{Entering Licensed Premises}
It is unlawful for any minor, as defined in this chapter, to enter licensed premises for the purpose of purchasing or consuming any alcoholic beverage.  It is unlawful for a licensee to permit a person under the age of 18 years to enter licensed premises unless the person is attending a social event, consuming a meal, performing work for the establishment not including the serving or selling of alcoholic beverages, or in the company of a parent or guardian.\footnote{(Ord. 95, 2nd Series, effective 10-15-94)}
\subsection{Misrepresentation of Age}
It is unlawful for a minor to misrepresent his or her age for the purpose of purchasing an alcoholic beverage.\footnote{(Ord. 31, 2nd Series, effective 4-16-86)}
\subsection{Proof of Age}
Proof of age for purchasing or consuming alcoholic beverages may be established only by a valid driver’s license or identification card issued by Minnesota, another state, or a province of Canada, and including the photograph and date of birth of the licensed person; or by a valid military identification card issued by the United States Department of Defense; or, in the case of a foreign national, from a nation other than Canada, by a valid passport.\footnote{(Ord. 102, 2nd Series, effective 5-13-95)}
\subsection{Qualification for License}
No minor shall qualify for a license.\footnote{(Ord. 31, 2nd Series, effective 4-16-86) (‘83 Code, SEC. 5.16)}

\section{Nudity or Obscenity Prohibited}
\index{ALCOHOLIC BEVERAGES!OPERATION OF LICENSED PREMISES; PROHIBITED CONDUCT!Nudity or Obscenity Prohibited}
\subsection{Definitions}
For the purpose of this section the following definitions shall apply, unless the context clearly indicates or requires a different meaning.
\begin{description}
    \item[NUDITY] Uncovered, or less than opaquely covered, post-pubertal human genitals, pubic areas, the post pubertal human female breast below a point immediately above the top of the areola, or the covered human male genitals in a discernibly turgid state.  For purposes of this definition, a female breast is considered uncovered if the nipple only or the nipple and the areola only are covered.
    \item[OBSCENE PERFORMANCE] A play, motion picture, dance, show or other presentation, whether pictured, animated or live, performed before an audience and which in whole or in part depicts or reveals nudity, sexual conduct, sexual excitement or sado-masochistic abuse, or which includes obscenities or explicit verbal descriptions or narrative accounts of sexual conduct.
    \item[OBSCENITIES] Those slang words currently generally rejected for regular use in mixed society, that are used to refer to genitals, female breasts, sexual conduct or excretory functions or products, either that have no other meaning or that in context are clearly used for their bodily, sexual or excretory meaning.
    \item[SADO-MASOCHISTIC ABUSE] Flagellation or torture by or upon a person who is nude or clad in undergarments or in revealing or bizarre costume, or the condition of being fettered, bound or otherwise physically restrained on the part of one so clothed.
    \item[SEXUAL CONDUCT] Human masturbation, sexual intercourse, or any touching of the genitals, pubic areas or buttocks of the human male or female, or the breasts of the female, whether alone or between members of the same or opposite sex or between humans and animals in an act of apparent sexual stimulation or gratification.
    \item[SEXUAL EXCITEMENT] The condition of human male or female genitals or the breasts of the female when in a state of sexual stimulation, or the sensual experiences of humans engaging in or witnessing sexual conduct or nudity.
\end{description}
\subsection{Unlawful Act}
It is unlawful for any person issued a license provided for in this chapter to permit upon licensed premises any nudity, obscene performance, or continued use of obscenities by any agent, employee, patron or other person.\footnote{(‘83 Code, SEC. 5.90)  Penalty, see SEC. 110.99}

\section{Sales of Confections Containing Alcohol or Liqueur-filled Candy}
\index{ALCOHOLIC BEVERAGES!OPERATION OF LICENSED PREMISES; PROHIBITED CONDUCT!Sales of Confections Containing Alcohol or Liqueur-filled Candy}
\subsection{Confections Containing Alcohol}
It is unlawful for any person to sell a confection containing alcohol to any person under the age of 21 years.  For purposes of this section, \textbf{CONFECTION CONTAINING ALCOHOL} means a confection containing or bearing not more than 5\% alcohol by volume where the alcohol is in a nonliquid form by reason of being mixed with other substances in the manufacture of the confection, does not include “liqueur-filled candy” as herein defined, and may be sold only by an exclusive liquor store licensed under this chapter or a business establishment that derives more than 50\% of its gross sales from the sale of confections.\footnote{(‘83 Code, SEC. 5.20)}
\subsection{Liqueur-filled Candy}
It is unlawful for any person to sell liqueur-filled candy to any person under the age of 21 years.  For purposes of this section, \textbf{LIQUEUR-FILLED CANDY} means any confectionery containing more than one-half of one percent alcohol by volume in liquid form that is intended for or capable of beverage use, and may be sold only by an eligible licensee under this chapter.\footnote{(‘83 Code, SEC. 5.21)  (Ord. 102, 2nd Series, effective 5-13-95)}\\

\subchapter{BEER LICENSES}

\setcounter{section}{59}
\section{Beer License Required}
\index{ALCOHOLIC BEVERAGES!BEER LICENSES!Beer License Required}
It is unlawful for any person, directly or indirectly, on any pretense or by any device, to sell, barter, keep for sale, or otherwise dispose of beer, as part of a commercial transaction, without a license therefor from the city. This section shall not apply to sales by manufacturers to wholesalers or to sales by wholesalers to persons holding beer licenses from the city.  Annual on-sale beer licenses may be issued only to drug stores, restaurants, hotels, bowling centers, clubs, and establishments used exclusively for the sale of beer with the incidental sale of tobacco and soft drinks.  Any person licensed to sell liquor on-sale shall not be required to obtain an on-sale beer license, and may sell beer on-sale without an additional license.  Any person licensed to sell liquor off-sale shall not be required to obtain an off-sale beer license, and may sell beer off-sale without an additional license.\footnote{(‘83 Code, SEC. 5.30)  (Am. Ord. 54, 2nd Series, effective 11-26-88)  Penalty, see SEC. 110.99}

\section{Beer License Fees}
\index{ALCOHOLIC BEVERAGES!BEER LICENSES!Beer License Fees}
\subsection{}
The annual on-sale beer license fee shall be set by resolution by the City Council. See Fee Schedule.
\subsection{}
The annual off-sale beer license fee shall be set by resolution by the City Council. See Fee Schedule.
\subsection{}
The temporary, two day, on-sale beer license fee shall be set by resolution by the City Council. See Fee Schedule.\footnote{(‘83 Code, SEC. 5.31)  (Ord. 154, 2nd Series, passed 12-10-02) \textbf{Cross-reference:} License fee increases, see SEC. 111.025}

\section{Temporary Beer License}
\index{ALCOHOLIC BEVERAGES!BEER LICENSES!Temporary Beer License}
\subsection{Applicant}
A club or charitable, religious, or non-profit organization, duly incorporated as a non-profit or religious corporation under the laws of the State of Minnesota, and having its registered office and principal place of activity within the city, shall qualify for a temporary on-sale beer license, for serving beer on and off school grounds, and in and out of school buildings.  The license may provide that the licensee may contract with the holder of a full-year on-sale licensee, issued by the city, for beer catering services.\footnote{(Ord. 54, 2nd Series, effective 11-26-88)}
\subsection{Conditions}
\subsubsection{}
An application for a temporary license shall state the exact dates and place of proposed temporary sale.
\subsubsection{}
The Council may, but at no time shall it be under any obligation whatsoever to, grant a temporary beer license on premises owned or controlled by the city.  Any license may be conditioned, qualified or restricted as the Council sees fit.  If the premises to be licensed are owned or under the control of the city, the applicant shall file with the city, prior to issuance of the license, a certificate of liability insurance coverage in at least the sum of \$50,000 for injury to any one person and \$100,000 for injury to more than one person, and \$10,000 for property damage, naming the city as an insured during the license period.  The license shall be issued only on the condition that the applicant will not sell in excess of \$10,000 (retail value) worth of beer in any calendar year, and thereupon shall be exempt from proof of financial responsibility as provided for herein.
\subsubsection{}
The applicant shall comply with all other restrictions, limitations and regulations for the sale of beer under the city code and statutes.\footnote{(Ord. 127, 2nd Series, effective 5-16-98) (‘83 Code, SEC. 5.32)  Penalty, see SEC. 110.99}

\section{Beer License Restrictions and Regulations}
\index{ALCOHOLIC BEVERAGES!BEER LICENSES!Beer License Restrictions and Regulations}
\subsection{}
No licensee shall, during the effective period of the license, be the owner or holder of a federal retail liquor dealer’s tax stamp for the sale of intoxicating liquor, unless the owner or holder also holds a liquor license from the city, and ownership or holding thereof shall be grounds for immediate revocation, without a hearing.\footnote{(Ord. 31, 2nd Series, effective 4-16-86)}
\subsection{}
No person who has not attained the age of 18 years shall be employed to sell or serve beer in any on-sale establishment.\footnote{(Ord. 54, 2nd Series, effective 11-26-88)}
\subsection{}
Except as otherwise provided in this chapter, no license shall be granted for any building within 100 feet of any public elementary or secondary school structure, or within 100 feet of any church structure.
\subsection{}
Every license shall be granted subject to the provisions of this chapter and all other applicable provisions of the city code and other laws relating to the operation of licensee’s business.\footnote{(Ord. 31, 2nd Series, effective 4-16-86) (‘83 Code, SEC. 5.33)}

\section{Hours and Days of Beer Sales}
\index{ALCOHOLIC BEVERAGES!BEER LICENSES!Hours and Days of Beer Sales}
No sale of beer shall be made between the hours of 1:00 a.m. and 8:00 a.m. on the days of Monday through Saturday, nor between the hours of 1:00 a.m. and 12:00 noon on Sunday.\footnote{(‘83 Code, SEC. 5.34)  (Ord. 54, 2nd Series, effective 11-26-88)  Penalty, see SEC. 110.99}

\section{Unlawful Acts Involving Beer\footnote{(‘83 Code, SEC. 5.35)  (Ord. 31, 2nd Series, effective 4-16-86)  Penalty, see SEC. 110.99}}
\index{ALCOHOLIC BEVERAGES!BEER LICENSES!Unlawful Acts Involving Beer}
It is unlawful for any:
\begin{enumerate}[{\indent}A)]
    \item Person to knowingly induce another to make an illegal sale or purchase of beer.
    \item Licensee to sell or serve beer to any person who is obviously intoxicated.
    \item Licensee to fail, where doubt could exist, to require adequate proof of age of a person upon licensed premises.
    \item Licensee to sell beer on any day, or during any hour, when sales are not permitted by law.
    \item Licensee to allow consumption of beer on licensed premises on any day when sales of beer are not permitted by law.
    \item Person to purchase beer on any day, or during any hour, when sales of beer are not permitted by law.
\end{enumerate}

\subchapter{LIQUOR LICENSES}

\setcounter{section}{79}
\section{Liquor License Required}
\index{ALCOHOLIC BEVERAGES!LIQUOR LICENSES!Liquor License Required}
\subsection{}
It is unlawful for any person, directly or indirectly, on any pretense or by any device, to sell, barter, keep for sale, or otherwise dispose of liquor, as part of a commercial transaction, without a license therefor from the city. This section shall not apply to:
\begin{enumerate}[{\indent}1)]
    \item The potable liquors as are intended for therapeutic purposes and not as a beverage; 
    \item Industrial alcohol and its compounds not prepared or used for beverage purposes; 
    \item Wine in the possession of a person duly licensed under this chapter as an on-sale wine licensee;
    \item Sales by manufacturers to wholesalers duly licensed as such by the Commissioner; 
    \item Sales by wholesalers to persons holding liquor licenses from the city; or 
    \item The providing by a person holding an off-sale license from the city of samples of malt liquor, wine, liqueurs, and cordials which the licensee currently has in stock and is offering for sale to the general public, provided the malt liquor, wine, liqueur, and cordial samples are dispensed at no charge and consumed on the licensed premises during permitted hours of off-sale in a quantity less than 100 milliliters of malt liquor per variety per customer, 50 milliliters of wine per variety per customer and 25 milliliters of liqueur or cordial per variety per customer.
\end{enumerate}
\subsection{}
Any person licensed to sell liquor on-sale shall not be required to obtain an on-sale beer license, and may sell beer on-sale without an additional license.  The city may issue annual on-sale liquor licenses only to the following\footnote{(‘83 Code, SEC. 5.50) (Ord. 59, 2nd Series, effective 8-12-89)}
:
\begin{enumerate}[{\indent}1)]
    \item Hotels; 
    \item Restaurants; 
    \item Bowling centers; 
    \item Clubs or congressionally chartered veterans’ organizations, provided that liquor sales will be made only to members and bona fide guests; and 
    \item Exclusive liquor stores.
\end{enumerate}

\section{Liquor License Fees}
\index{ALCOHOLIC BEVERAGES!LIQUOR LICENSES!Liquor License Fees}
\subsection{}
The annual on-sale liquor license fees shall be, as follows:
\begin{enumerate}[{\indent}1)]
    \item For restaurants, hotels or motels providing a full menu shall be set by resolution by the City Council. See Fee Schedule.
    \item For exclusive liquor stores, restaurants, hotels or motels not providing a full menu shall be set by resolution by the City Council. See Fee Schedule.
\end{enumerate}
\subsection{}
The annual off-sale liquor license fee shall be set by resolution by the City Council. See Fee Schedule.
\subsection{}
The annual Sunday liquor license fee shall be set by resolution by the City Council. See Fee Schedule.
\subsection{}
The annual club on-sale liquor license fee shall be set by resolution by the City Council. In the case of a club with a membership of 200 or less the annual license fee shall be set at a lesser amount. See Fee Schedule.
\subsection{}
The sports or convention facilities liquor license fee shall be set by resolution by the City Council. See Fee Schedule.\footnote{(Ord. 31, 2nd Series, effective 4-16-86)}
\subsection{}
The temporary liquor license fee shall be set by resolution by the City Council. See Fee Schedule.\footnote{(Ord. 63, 2nd Series, effective 7-21-90) (‘83 Code, SEC. 5.51)}

\section{Liquor License Restrictions and Regulations}
\index{ALCOHOLIC BEVERAGES!LIQUOR LICENSES!Liquor License Restrictions and Regulations}
\subsection{}
No license shall be effective until a permit shall be issued to a licensee under the laws of the United States, if the permit be required under the laws or the State of Minnesota.\footnote{(Ord. 31, 2nd Series, effective 4-16-86)}
\subsection{}
No person under 18 years of age may sell or serve liquor on licensed premises.\footnote{(Ord. 54, 2nd Series, effective 11-26-88)}
\subsection{}
No licensee shall sell, offer for sale, or keep for sale, liquor in any original package which has been refilled or partly refilled.
\subsection{}
No licensee shall display liquor to the public during hours when the sale of liquor is prohibited.
\subsection{}
No license shall be granted for any building within 100 feet of any public elementary or secondary school structure or within 100 feet of any church structure.\footnote{(Ord. 31, 2nd Series, effective 4-16-86)}
\subsection{}
No more than a total of six off-sale liquor licenses shall be outstanding at any one time.  In order to qualify for an off-sale license, the applicant shall, among other things, be the holder of an on-sale license and the operator of an on-sale establishment.  For purposes of this division, the person who owns and controls on a day-to-day basis at least 51\% of an applicant which is a corporation, partnership, or other business organization, may be deemed to be the holder of the license and operator of the establishment.\footnote{(Ord. 39, 2nd Series, effective 4-23-87)}
\subsection{}
No on-sale liquor license shall be granted to any person which does not have invested or does not propose to invest in the fixtures and structure of the licensed establishment, exclusive of land, at least \$150,000 on a replacement cost, less depreciation basis.  The Council may provide for an independent appraisal, at the expense of the applicant, as an aid- in determining the value.  If this provision is not complied with within one year from the date of initial issuance of the license, the same shall be grounds for refusal or revocation of the license.
\subsection{}
Every license shall be granted subject to the provisions of this chapter and all other applicable provisions of the city code and other laws relating to the operation of the licensed business.\footnote{(Ord. 31, 2nd Series, effective 4-16-86) (‘83 Code, SEC. 5.52)}

\section{Hours and Days of Liquor Sales}
\index{ALCOHOLIC BEVERAGES!LIQUOR LICENSES!Hours and Days of Liquor Sales}
No on-sale shall be made after 1:00 a.m. on Sunday, nor between the hours of 1:00 a.m. and 8:00 a.m. on Monday through Saturday.  No off-sale shall be made on Sunday nor before 8:00 a.m. or after 10:00 p.m. on Monday through Saturday, nor on Thanksgiving Day or Christmas Day, December 25.  No sale of liquor shall be made after 8:00 p.m. on December 24.  This section does not prohibit sales during hours when on-sale is permitted on Sunday as stated in SEC. 111.084 of this chapter.\footnote{(‘83 Code, SEC. 5.53)}

\section{Sunday Sales}
\index{ALCOHOLIC BEVERAGES!LIQUOR LICENSES!Sunday Sales}
The electorate of the city having heretofore authorized the same, a Sunday on-sale liquor license may be issued to hotels, motels, restaurants, bowling centers, and clubs, which have facilities for serving not less than 30 guests at one time.  The hours of the sales shall be from 12:00 noon, or 10:00 a.m. if the licensee is in compliance with the Minnesota Clean Air Act, on Sunday to 1:00 a.m. on Monday in conjunction with the serving of food.\footnote{(‘83 Code, SEC. 5.54)  (Ord. 59, 2nd Series, effective 8-12-89)}

\section{Unlawful Acts Involving Liquor\footnote{(‘83 Code, SEC. 5.55)  (Ord. 31, 2nd Series, effective 4-16-86)}}
\index{ALCOHOLIC BEVERAGES!LIQUOR LICENSES!Unlawful Acts Involving Liquor}
It is unlawful for any:
\begin{enumerate}[{\indent}A)]
    \item Person to knowingly induce another to make an illegal sale or purchase of liquor.
    \item Licensee to sell liquor on any day, or during any hour, when sales of liquor are not permitted by law.
    \item Person to purchase liquor on any day, or during any hour, when sales of liquor are not permitted by law.
    \item Licensee to sell or serve liquor to any person who is obviously intoxicated.
    \item Licensee to fail, where doubt could exist, to require adequate proof of age of a person upon licensed premises.
\end{enumerate}

\section{Sports or Convention Facilities License}
\index{ALCOHOLIC BEVERAGES!LIQUOR LICENSES!Sports or Convention Facilities License}
The Council may authorize any holder of an on-sale liquor license issued by the city or by an adjacent municipality to sell liquor at any convention, banquet, conference, meeting or social affair conducted on the premises of a sports or convention facility owned by the city, or instrumentality thereof having independent policy-making and appropriating authority and located within the city.  The licensee must be engaged to sell liquor at such an event by the person or organization permitted to use the premises, and may sell liquor only to persons attending the event.  The licensee shall not sell liquor to any person attending or participating in any amateur athletic event.  The sales may be limited to designated areas of the facility.  All the sales shall be subject to all laws relating thereto.  No club licensee shall qualify to serve liquor in the facilities.\footnote{(‘83 Code, SEC. 5.56)}

\section{Temporary Liquor License}
\index{ALCOHOLIC BEVERAGES!LIQUOR LICENSES!Temporary Liquor License}
\subsection{License Authorized}
Notwithstanding any provision of the city code to the contrary, the Council may issue a license for the temporary on-sale of liquor in connection with a social event sponsored by the licensee.  The license will provide that the licensee shall contract with the holder of a full-year on-sale license, issued by the city, for liquor catering services.\footnote{(Ord. 63, 2nd Series, effective 7-21-90)}
\subsection{Applicant}
The applicant for a license under this section must be a club or charitable, religious, or other non-profit organization in existence for at least three years, or a political committee registered under M.S. § 10A.14, as it may be amended from time to time.\footnote{(Ord. 102, 2nd Series, effective 5-13-95)}
\subsection{Terms and Conditions of License}
\subsubsection{}
No license is valid until approved by the Commissioner.\footnote{(Ord. 63, 2nd Series, effective 7-21-90)}
\subsubsection{}
No license shall be issued for more than four consecutive days.\footnote{(Ord. 127, 2nd Series, effective 5-16-98)}
\subsubsection{}
No (temporary) license shall issue until the city is furnished with written proof that the licensee has dram shop coverage in the amount provided for in this chapter, and that the coverage is in force on the premises where liquor is to be served.
\subsubsection{}
All licenses and licensees are subject to all provisions of statutes and the city code relating to liquor sale and licensing.  The licensee shall provide proof of financial responsibility coverage and, in the case of catering by a full-year on-sale licensee, the caterer shall provide proof of the extension of the coverage to the licensed premises.
\subsubsection{}
Licenses may authorize sales on premises other than those owned or permanently occupied by the licensee.\footnote{(Ord. 63, 2nd Series, effective 7-21-90)}
\subsubsection{}
No more than three four-day, four three-day or six two-day licenses in any combination not to exceed 12 days per year may be issued to any one organization or registered political committee, or for any one location within a 12-month period.
\subsection{Insurance Required}
The Council may, but at no time shall it be under any obligation whatsoever to, grant a temporary liquor license on premises owned or controlled by the city.  Any license may be conditioned, qualified or restricted as the Council sees fit.  If the premises to be licensed are owned or under the control of the city, the applicant shall file with the city, prior to issuance of the license, a certificate of liability insurance coverage in at least the sum of \$50,000 for injury to any one person, \$100,000 for injury to more than one person, and \$10,000 for property damage, naming the city as an insured during the license period.\footnote{(Ord. 127, 2nd Series, effective 5-16-98) (‘83 Code, SEC. 5.57)}\\

\subchapter{ON-SALE WINE}

\setcounter{section}{99}
\section{On-Sale Wine License Provisions}
\index{ALCOHOLIC BEVERAGES!ON-SALE WINE!On-Sale Wine License Provisions}
\subsection{On-Sale Wine License Required}
It is unlawful for any person to sell, or keep or offer for sale, any wine without a license therefor from the city. This section shall not apply to possession or handling for sale or otherwise of sacramental wine or any representative of any religious order or for use in connection with a legitimate religious ceremony, to sales by manufacturers to wholesalers duly licensed as such by the State of Minnesota, to sales by wholesalers to persons holding on-sale or off-sale liquor licenses from the city, or to sales by wholesalers to persons holding on-sale wine licenses from the city.
\subsection{On-Sale Wine License Fee}
The annual on-sale wine license fee shall be set by resolution by the City Council. See Fee Schedule.\footnote{(Ord. 3, 2nd Series, effective 2-25-84)}
\subsection{On-Sale Wine License Restrictions and Regulations}
\subsubsection{}
No license shall be granted to a wholesaler or manufacturer of wine, or to anyone holding a financial interest in the manufacture or wholesaling.
\subsubsection{}
No license shall be effective until a permit shall be issued to a licensee under the laws of the United States, if the permit be required under the laws or the State of Minnesota.
\subsubsection{}
No person under 18 years of age may sell or serve wine on licensed premises.\footnote{(Ord. 54, 2nd Series, effective 11-26-88)}
\subsubsection{}
No licensee shall display wine to the public on days or during hours when the sale of wine is prohibited.
\subsubsection{}
No license shall be granted for any building within 100 feet of any public elementary or secondary school structure or within 100 feet of any church structure.\footnote{(Ord. 7, 2nd Series, effective 5-15-84)}
\subsubsection{}
No more than one license shall be held by any person.  For the purpose of this division, any person owning a beneficial interest of 5\%, or more, of any licensed establishment shall be considered a licensee.
\subsubsection{}
No on-sale wine license shall be granted to any person which does not have invested or does not propose to invest in the fixtures and structure of the licensed establishment, exclusive of land, at least \$150,000 for a restaurant or \$50,000 for a licensed bed and breakfast facility, on a replacement cost, less depreciation basis.  The Council may provide for an independent appraisal, at the expense of the applicant, as an aid in determining the value.  If this provision is not complied with within one year from the date of initial issuance of the license, the same shall be grounds for refusal or revocation of the license.
\subsubsection{}
On-sale wine licenses shall be granted only to restaurants and licensed bed and breakfast facilities as defined in this chapter.  Provided, however, for purposes of this section, the restaurant shall have appropriate facilities for seating not less than 30 guests at one time.\footnote{(Ord. 79, 2nd Series, effective 8-18-92)}
\subsubsection{}
Every license shall be granted subject to the provisions of this chapter and all other applicable provisions of the city code and other laws relating to the operation of the licensed business.\footnote{(‘83 Code, SEC. 5.70)  Penalty, see SEC. 110.99}

\section{Hours and Days of Sales by On-Sale Wine Licensees}
\index{ALCOHOLIC BEVERAGES!ON-SALE WINE!Hours and Days of Sales by On-Sale Wine Licensees}
No on-sale sale of wine shall be made between 1:00 a.m. and 12:00 noon on Sunday, nor between 12:00 midnight and 8:00 a.m. on Monday, nor between the hours of 1:00 a.m. and 8:00 a.m. on Tuesday through Saturday, nor between the hours of 8:00 p.m. on December 24 and 8:00 a.m. on December 25.\footnote{(‘83 Code, SEC. 5.71)  (Ord. 54, 2nd Series, effective 11-26-88)  Penalty, see SEC. 110.99}

\section{Unlawful Acts Involving Wine\footnote{(‘83 Code, SEC. 5.72)  Penalty, see SEC. 110.99}}
\index{ALCOHOLIC BEVERAGES!ON-SALE WINE!Unlawful Acts Involving Wine}
It is unlawful for any:
\begin{enumerate}[{\indent}A)]
    \item Person to knowingly induce another to make an illegal sale or purchase of wine.
    \item Licensee to sell wine on any day, or during any hour, when sales of wine are not permitted by law.
    \item Person to purchase wine on any day, or during any hour, when sales of wine are not permitted by law.
    \item Licensee to sell or serve wine to any person who is obviously intoxicated.
    \item Licensee to fail, where doubt could exist, to require adequate proof of age of a person upon licensed premises.
    \item Licensee to sell wine except in conjunction with the sale of food.\footnote{(Ord. 31, 2nd Series, effective 4-16-86)}
    \item Licensed bed and breakfast facility licensee to furnish wine to any person who is not a registered guest of the facility.\footnote{(Ord. 79, 2nd Series, effective 8-18-92)}
\end{enumerate}

\section{On-Sale Wine License not Required for Bed and Breakfast Facility}
\index{ALCOHOLIC BEVERAGES!ON-SALE WINE!On-Sale Wine License not Required for Bed and Breakfast Facility}
No on-sale wine license is required for a bed and breakfast facility as defined in Chapter 152 and registered with the Commissioner, provided the facility provides no more than two glasses per day each containing not more than four fluid ounces of wine at no additional charge to a person renting a room at the facility.  Wine so furnished may be consumed on the premises of the bed and breakfast facility.\footnote{(‘83 Code, SEC. 5.73)  (Ord. 127, 2nd Series, effective 5-16-98)}\\

\subchapter{CONSUMPTION AND DISPLAY}

\setcounter{section}{114}
\section{One Day License}
\index{ALCOHOLIC BEVERAGES!CONSUMPTION AND DISPLAY!One Day License}
\subsection{License Required}
Any non-profit organization desiring to serve liquids for the purpose of mixing with liquor and permitting the consumption and display of liquor in conjunction with a social activity sponsored by it, shall first obtain a license therefor from the city. It is unlawful for any organization to fail to obtain the license.
\subsection{Term}
The term of the license shall be one day only.
\subsection{Limitation on Number}
No more than ten licenses shall be issued in any calendar year.
\subsection{License Fee}
The fee for the one-day license shall be set by resolution by the City Council. See Fee Schedule.
\subsection{Approval}
In addition to Council approval, the license must be approved by the Commissioner of Public Safety.\footnote{(‘83 Code, SEC. 5.80)  Penalty, see SEC. 110.99}

\section{Bottle Clubs}
\index{ALCOHOLIC BEVERAGES!CONSUMPTION AND DISPLAY!Bottle Clubs}
\subsection{Definition}
For purposes of this section, the term \textbf{BOTTLE CLUB} is a “club” as defined in this chapter, or an unincorporated society which, except for its lack of incorporation, otherwise meets the requirements of a club, and which is not otherwise licensed for the sale of liquor, either on-sale or off-sale or both.
\subsection{Consumption and Display License Required}
It is unlawful for any bottle club or for any business establishment to allow the consumption or display of liquor or the serving of any liquid for the purpose of mixing liquor without a license therefor from the city, but a bottle club as herein defined and licensed may permit its members to bring and keep a personal supply of liquor in lockers assigned to the members.
\subsection{Consumption and Display License Fee}
The annual consumption and display license fee shall be set by resolution by the City Council. See Fee Schedule.
\subsection{Consumption and Display Restrictions and Regulations}
\subsubsection{}
Every bottle, container or other receptacle containing liquor stored by a member of a bottle club shall have attached to it a label signed by the member of the club, shall be kept in a locker designated to the use of the member, and no other liquor shall be on bottle club premises.
\subsubsection{}
It is unlawful for any club member who is a minor to be assigned a locker for the storage of liquor or to consume or display liquor on any premises under control by the club.\footnote{(Ord. 54, 2nd Series, effective 11-26-88)}
\subsubsection{}
It is unlawful to consume or allow consumption or display of liquor in any bottle club or business establishment between the hours of 1:00 a.m. and 12:00 noon on Sundays, and between 1:00 a.m. and 8:00 a.m. on Monday through Saturday.\footnote{(Ord. 31, 2nd Series, effective 4-16-86)}
\subsubsection{}
No license shall be issued to any bottle club when a member of the board, management, executive committee, or other similar body chosen by its members, or when a business establishment or the owner thereof holds a federal retail liquor dealer’s special tax stamp for the sale of liquor.
\subsubsection{}
Liquor sold, served or displayed in violation of this section shall be subject to seizure for purposes of evidence.
\subsubsection{}
No license shall be issued unless the licensee can seat and serve food to not less than 500 persons at one time and in one room.
\subsection{Other Licenses}
An on-sale liquor or wine licensee may also be licensed for consumption and display.
\subsection{State Permit Required}
It is unlawful for any person or business establishment, directly or indirectly, or upon any pretense or by any device, to allow the consumption or display of liquor, or the serving of any liquid for the purpose of mixing of liquor, without first having obtained a permit therefor from the State of Minnesota.  The state permit shall expire on March 31 of each year.\footnote{(Ord. 80, 2nd Series, effective 11-18-92) (‘83 Code, SEC. 5.81)}
