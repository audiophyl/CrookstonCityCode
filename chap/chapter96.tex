\chapter*{Chapter 96: \\
	Streets and Sidewalks}
    \addstarredchapter{Chapter 96: Streets and Sidewalks}
    \vfill
    \minitoc
    \pagebreak

\section{General Provisions Applicable to this Chapter}
\index{STREETS AND SIDEWALKS!General Provisions Applicable to this Chapter}
\subsection{Definitions}
Except as otherwise defined in the city code, or where the context clearly indicates a contrary intent, the words and terms defined in M.S. Chapter 169, as it may be amended from time to time, shall be applicable to this chapter.\footnote{(‘83 Code, SEC. 7.01)}
\subsection{Application}
Except as otherwise provided in this code, the provisions of this chapter are applicable to the drivers of all vehicles and animals upon streets, including, but not limited to, those owned or operated by the United States, the State of Minnesota, or any county, town, city, district, or other political subdivision.\footnote{(‘83 Code, SEC. 7.02)}
\subsection{Scope}
Except as otherwise provided in this code, the provisions of this chapter relate exclusively to the streets and alleys in the city, and the operation and parking of vehicles refer exclusively to the operation and parking of vehicles upon the streets and alleys.\footnote{(‘83 Code, SEC. 7.03, Subd. 1)  (Ord. 15, 2nd Series, effective 5-18-85)}

\section{Obstructions in Streets}
\index{STREETS AND SIDEWALKS!Obstructions}
\subsection{Obstructions}
It is a misdemeanor for any person to place, deposit, display or offer for sale, any fence, goods or other obstructions upon, over, across or under any street without first having obtained a written permit from the Council, and then only in compliance in all respects with the terms and conditions of the permit, and taking precautionary measures for the protection of the public.  An electrical cord or device of any kind is hereby included, but not by way of limitation, within the definition of an obstruction.
\subsection{Fires}
It is a misdemeanor for any person to build or maintain a fire upon a street.
\subsection{Dumping in Streets}
It is a misdemeanor for any person to throw or deposit in any street any nails, dirt, glass or glassware, cans, discarded cloth or clothing, metal scraps, garbage, leaves, grass or tree limbs, paper or paper products, shreds or rubbish, oil, grease or other petroleum products, or to empty any water containing salt or other injurious chemical thereon.  It is a violation of this section to haul any material, inadequately enclosed or covered, thereby permitting the same to fall upon streets.  It is also a violation of this section to place or store any building materials or waste resulting from building construction or demolition on any street without first having obtained a written permit from the Council.
\subsection{Signs and Other Structures}
It is a misdemeanor for any person to place or maintain a sign, advertisement, or other structure in any street without first having obtained a written permit from the city.  In a district zoned for commercial or industrial enterprises special permission allowing an applicant to erect and maintain signs overhanging the street may be granted upon the terms and conditions as may be set forth in the zoning or construction provisions of the city code.
\subsection{Placing Snow or Ice in a Roadway or on a Sidewalk}
\subsubsection{}
It is a misdemeanor for any person, not acting under a specific contract with the city or without special permission from the city, to remove snow or ice from private property and place the same in any roadway or on a sidewalk.
\subsubsection{}
Where permission is granted by the city the person to whom the permission is granted shall be initially responsible for payment of all direct or indirect costs of removing the snow or ice from the street or sidewalk.  If not paid, collection shall be by civil action or assessment against the benefitted property as any other special assessment.
\subsection{Continuing Violation}
Each day that any person continues in violation of this section shall be a separate offense and punishable as such.
\subsection{Condition}
Before granting any permit under any of the provisions of this section, the Council may impose insurance or bonding conditions thereon as it, considering the projected danger to public or private property or to persons, deems proper for safeguarding persons and property.  The insurance or bond shall also protect the city from any suit, action or cause of action arising by reason of the obstruction.\footnote{(‘83 Code, SEC. 7.06)  Penalty, see SEC. 96.99}

\section{Construction and Reconstruction of Roadway Surfacing, Sidewalk, Curb and Gutter}
\index{STREETS AND SIDEWALKS!Construction and Reconstruction of Roadway Surfacing, Sidewalk, Curb and Gutter}
\subsection{Methods of Procedure}
\subsubsection{}
Abutting or affected property owners may contract for, construct or reconstruct roadway surfacing, sidewalk or curb and gutter in accordance with this section if advance payment is made therefor or arrangements for payment considered adequate by the city are completed in advance.
\subsubsection{}
The city may, with or without petition by the methods set forth in the local improvement code of Minnesota Statutes, presently beginning with M.S. \textsection 429.011, as the same may from time to time be amended.
\subsection{Permit Required}
It is a misdemeanor to construct or reconstruct a sidewalk, curb and gutter, driveway, or roadway surfacing in any street or other public property in the city without a permit from the city. Application for the permit shall be made in the form approved by the city and shall sufficiently describe the contemplated improvements, the contemplated date of beginning of work, and the length of time required to complete the same, provided, that no permit shall be required for any improvement ordered installed by the Council. All applications shall be referred by the Building Official to the Public Works Director and no permit shall be issued until approval has been received from the Public Works Director. All applications shall contain an agreement by the applicant to be bound by this chapter and plans and specifications consistent with the provisions of this chapter and good engineering practices shall also accompany the application. A permit from the city shall not relieve the holder from damages to the person or property of another caused by the work.
\subsection{Specifications and Standards}
All construction and reconstruction of roadway surfacing, sidewalk and curb and gutter improvements, including curb cuts, shall be strictly in accordance with specifications and standards on file in the office of the City Engineer and open to inspection and copying there.  The specifications and standards may be amended from time to time by the city, but shall be uniformly enforced.
\subsection{Inspection}
The Building Official shall inspect the improvements as deemed necessary or advisable.  Any work not done according to the applicable specifications and standards shall be removed and corrected at the expense of the permit holder.  Any work done hereunder may be stopped by the Building Official if found to be unsatisfactory or not in accordance with the specifications and standards, but this shall not place a continuing burden upon the city to inspect or supervise the work.
\subsection{Insurance Required}
No permit shall be issued until the applicant has furnished the city with evidence of public liability insurance in the amount of \$100,000 for the injury of one person, \$300,000 for any occurrence, and \$25,000 for property damage.\footnote{(Ord. 9, 2nd Series, effective 5-15-84)  (‘83 Code, SEC. 7.05)  Penalty, see SEC. 96.99}

\section{Street Openings or Excavations}
\index{STREETS AND SIDEWALKS!Street Openings or Excavations}
It is a misdemeanor for any person, except a city employee acting within the course and scope of his or her employment or a contractor acting within the course and scope of a contract with the city, to make any excavation, opening or tunnel in, over, across or upon a street or other public property without first having obtained a written permit from the Building Official as herein provided.
\subsection{Application}
Application for a permit to make a street excavation shall describe with reasonable particularity the name and address of the applicant, the place, purpose and size of the excavation, and the other information as may be necessary or desirable to facilitate the investigation hereinafter provided for, and shall be filed with the Building Official.
\subsection{Investigation and Payment of Estimated Costs}
Upon receipt of the application, the Building Official shall cause the investigation to be made as he or she may deem necessary to determine estimated cost of repair, such as back-filling, compacting, resurfacing and replacement, and the conditions as to the time of commencement of work, manner of procedure and time limitation upon the excavation.  The foregoing estimated costs shall include permanent and temporary repairs due to weather or other conditions, and the cost of the investigation shall be included in the estimate.
\subsection{Protection of the City and the Public}
\subsubsection{Non-completion or Abandonment}
Work shall progress expeditiously to completion in accordance with any time limitation placed thereon so as to avoid unnecessary inconvenience to the public.  In the event that work is not performed in accordance therewith, or shall cease or be abandoned without due cause, the city may, after six hours notice in writing to the holder of the permit of its intention to do so, correct the work, fill the excavation and repair the public property, and the cost thereof shall be paid by the person holding the permit.
\subsubsection{Insurance}
Prior to commencement of the work described in the application, the applicant shall furnish the city satisfactory evidence in writing that the applicant will keep in effect public liability insurance of not less than \$100,000 for any person, \$300,000 for any occurrence and property damage insurance of not less than \$25,000, issued by an insurance company authorized to do business in the State of Minnesota on which the city is named as a co-insured.
\subsubsection{Indemnification}
Before issuance of a permit, the applicant shall, in writing, agree to indemnify and hold the city harmless from any liability for injury or damage arising out of the action of the applicant in performance of the work, or any expense whatsoever incurred by the city incident to a claim or action brought or commenced by any person arising therefrom.
\subsection{Issuance of Permit}
The Building Official shall issue the permit after completion of the investigation, determination of all estimated costs as aforesaid; agreement by the applicant to the conditions of time and manner as aforesaid; agreement in writing by the applicant to pay all actual cost of repairs over and above the estimate, and, agreement in writing by the applicant to be bound by all of the provisions of this section.  No permit shall be issued until the applicant has paid in full (or, in the alternative, furnished either an irrevocable and unconditional letter of credit drawn on a national bank and approved by the City Attorney or a performance bond approved by the City Attorney) the amount of the Building Official’s estimated cost of the work to be performed, together with the investigation, inspection and permit fees as are fixed and determined by resolution of the Council.\footnote{(Ord. 9, 2nd Series, effective 5-15-84)}
\subsection{Repairs}
All temporary and permanent repairs, including back-filling, compacting and resurfacing shall be made, or contracted for, by the city in a manner prescribed by the City Engineer and an accurate account of costs thereof shall be kept.
\subsection{Cost Adjustment}
Within 60 days following completion of the permanent repairs the Clerk-Treasurer shall determine actual costs of repairs, including cost of investigation, and prepare and furnish to the permit holder an itemized statement thereof and claim additional payment from, or make refund (without interest) to, the permit holder, as the case may be.
\subsection{Alternate Method of Charging}
In lieu of the above provisions relating to cost and cost adjustment for street openings, the city may charge on the basis of surface square feet removed, excavated cubic feet, or a combination of surface square feet and excavated cubic feet, on an established unit price uniformly charged.
\subsection{Two-year Period}
The applicant for the permit to make a street opening or excavation shall be responsible for the proper performance of the work on the right-of-way for a two-year period.  The city shall inspect the work performed and determine if the final inspection after the two-year period is acceptable.\footnote{(‘83 Code, SEC. 7.07)  Penalty, see SEC. 96.99}

\section{Curb Set-back Regulation}
\index{STREETS AND SIDEWALKS!Curb Set-back Regulation}
\subsection{Permit Required}
It is a misdemeanor for any person to hereafter remove, or cause to be removed, any curb from its position abutting upon the roadway to another position without first making application to the Building Official.  In determining whether or not to issue the permit, the Building Official shall consider all relevant factors including, but not necessarily limited to, the effect on public and franchised utilities, the inconvenience, if any, upon abutting property owners, vehicular and pedestrian safety, the number of curb cuts presently on the same parcel or property, and the effect on public parking.\footnote{(Ord. 40, 2nd Series, effective 4-23-87)}
\subsection{Agreement Required}
No permit shall be issued until the applicant, and abutting landowner if other than applicant, shall enter into a written agreement with the city agreeing to pay all costs of constructing and maintaining the setback area in at least as good condition as the abutting roadway, and further agreeing to demolish and remove the set-back and reconstruct the area as was at the expense of the landowner, his or her heirs or assigns if the area ever, in the Council’s opinion becomes a public hazard.  The agreement shall be recorded in the office of the County Recorder, and shall run with the adjoining land.
\subsection{Sign-posting}
“ANGLE PARKING ONLY” signs shall be erected and maintained at the expense of the adjoining landowner in all set-back areas now in use or hereafter constructed.  It is unlawful for any person to park other than at an angle in the set-back areas, as angle parking is herein described and allowed.
\subsection{Public Rights Preserved}
The set-back parking areas shall be kept open for public parking and the abutting landowner shall at no time acquire any special interest or control of or in such areas.\footnote{(‘83 Code, SEC. 7.09)  Penalty, see SEC. 96.99}

\section{Improvement of Street; Sewer and Water Lateral Installation Required}
\index{STREETS AND SIDEWALKS!Improvement of Street; Sewer and Water Lateral Installation Required}
\subsection{Requirement of Sewer and Water Laterals}
No petition for the improvement of a street shall be considered by the Council if the petition contemplates constructing therein any part of a pavement or stabilized base, or curb and gutter, unless all sewer and water main installations shall have been made therein, including the installation of service laterals to the curb, if the area along the street will be served by the utilities installed in the street.
\subsection{Sewer System Service and Water Main Service Laterals}
No sewer system shall be hereafter constructed or extended unless service laterals to platted lots and frontage facing thereon shall be extended simultaneously with construction of mains.
\subsection{Waiver}
The Council may waive the requirements of this section only if it finds the effects thereof are burdensome and upon the notice and hearing as the Council may deem necessary or proper.\footnote{(‘83 Code, SEC. 7.11)  Penalty, see SEC. 96.99}

\section{Painting or Coloring Street, Sidewalk or Curb and Gutter}
\index{STREETS AND SIDEWALKS!Painting or Coloring Street, Sidewalk or Curb and Gutter}
It is unlawful for any person to paint, letter or color any street, sidewalk or curb and gutter for advertising purposes, or to paint or color any street, sidewalk or curb and gutter for any purpose, except as the same may be done by city employees acting within the course or scope of their employment. Provided, however, that this provision shall not apply to uniformly coloring concrete or other surfacing, or uniformly painted house numbers, as the coloring may be approved by the city.\footnote{(‘83 Code, SEC. 7.12)  Penalty, see SEC. 96.99}

\section{Sidewalk Maintenance and Repair}
\index{STREETS AND SIDEWALKS!Sidewalk Maintenance and Repair}
\subsection{Primary Responsibility}
It is the primary responsibility of the owner of property upon which there is abutting any sidewalk to keep and maintain the sidewalk in safe and serviceable condition.
\subsection{Construction, Reconstruction and Repair Specifications}
All construction, reconstruction or repair of sidewalks shall be done in strict accordance with specifications on file in the office of the Public Works Director.
\subsection{Notice - No Emergency}
Where, in the opinion of the Building Official, no emergency exists, notice of the required repair or reconstruction shall be given to the owner of the abutting property.  The notice shall require completion of the work within 90 days, and shall be mailed to the owner or owners shown to be on the records of the County Officer who mails tax statements.
\subsection{Notice - Emergency}
Where, in the opinion of the Building Official, an emergency exists, notice of the required repair or reconstruction shall be given to the owner of the abutting property.  The notice shall require completion of the work within ten days, and shall be mailed to the owner or owners shown to be on the records of the County Officer who mails tax statements.
\subsection{Failure of Owner to Reconstruct or Make Repairs}
If the owner of the abutting property fails to make repairs or accomplish reconstruction as herein required, the Clerk-Treasurer shall report the failure to the Council and the Council may order the work to be done under its direction and the cost thereof assessed to the abutting property owner as any other special assessment.
\subsection{Duty to Inspect}
In order to accomplish the purpose of this section, it shall be the duty of the Building Official to inspect sidewalks within the city, or cause the same to be inspected under his or her direction.\footnote{(Ord. 9, 2nd Series, effective 5-15-84) (‘83 Code, SEC. 7.14)  Penalty, see SEC. 96.99}

\section{Ice, Snow, Dirt and Refuse on Sidewalks}
\index{STREETS AND SIDEWALKS!Ice, Snow, Dirt and Refuse on Sidewalks}
Sidewalks shall be kept free of ice, snow, dirt and refuse in accordance with the provisions of city code.\footnote{(‘83 Code, SEC. 7.15)  (Ord. 40, 2nd Series, effective 4-23-87)  Penalty, see SEC. 96.99}

\section{Haul Fee}
\index{STREETS AND SIDEWALKS!Haul Free}
\subsection{Policy and Purpose}
The city has determined that public works and public improvements which require the transport of a significant amount of material over city streets cause damage to and excessive wear and tear of the streets.  It is consistent with the public interest that the costs associated with the use of city streets are paid as a part of the costs of the public works or improvements.
\subsection{Haul Fee}
The City Administrator may require contractors, as a part of any contract for a public work or public improvement let by the city either alone or in cooperation with any other governmental entity, a fee, payable to the city, for the use of city streets.  The amount of the fee shall be \$1 per cubic yard or \$0.67 per ton of material (as estimated by the Public Works Director) to be transported upon city streets in connection with the contract.\footnote{(Ord. 149, 2nd Series, passed 4-23-02)}

\setcounter{section}{98}
\section{Penalty\footnote{(‘83 Code, SEC. 7.99)}}
\index{STREETS AND SIDEWALKS!Penalty}
Every person violates a section, division, or provision of this title when he or she performs an act thereby prohibited or declared unlawful, or fails to act when the failure is thereby prohibited or declared unlawful, and upon conviction thereof, shall be punished as follows:
\begin{enumerate}[{\indent}A)]
    \item Where the specific section, division, or provision specifically makes violation a misdemeanor, the person shall be punished as for a misdemeanor; where a violation is committed in a manner or under circumstances so as to endanger or be likely to endanger any person or property, the person shall be punished as for a misdemeanor; where the person stands convicted of violation of any provision of this chapter, exclusive of violations relating to the standing or parking of an unattended vehicle, within the immediate preceding 12-month period for the third or subsequent time, he or she shall be punished as for a misdemeanor.  The penalty which may be imposed for any crime which is a misdemeanor under this code, including Minnesota Statutes specifically adopted by reference, shall be a sentence of not more than 90 days or a fine of not more than \$1,000, or both.
    \item As to any violation not constituting a misdemeanor under the provisions of division (A) above,  the person shall be punished as for a petty misdemeanor.  The penalty which may be imposed for any petty offense which is a petty misdemeanor shall be a sentence of a fine of not more than \$300.
    \item As to any violation of a provision adopted by reference, he or she shall be punished as specified in the provision, so adopted.
    \item In the case of a misdemeanor or a petty misdemeanor, the costs of prosecution may be added.  A separate offense shall be deemed committed upon each day during which a violation occurs or continues.
\end{enumerate}
