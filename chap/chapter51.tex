\chapter*{Chapter 51: \\
	Solid Waste}
    \addstarredchapter{Chapter 51: Solid Waste}
    \vfill
    \minitoc
    \textbf{\emph{{Cross-reference:}}}\\
    \indent\emph{Penalty surcharge, see SEC. 50.98}
    \pagebreak

\section{Definition}
\index{SOLID WASTE!Definition}
For the purpose of this chapter the following definition shall apply, unless the context clearly indicates or requires a different meaning.
\begin{description}
    \item[REFUSE] Includes all drained organic material resulting from the preparation of food and spoiled or decayed food from any source, bottles, cans, glassware, paper or paper products, crockery, ashes, rags and discarded clothing, and tree or lawn clippings.
\end{description}

\section{Unlawful Acts}
\index{SOLID WASTE!Unlawful Acts}
\subsection{}
It is unlawful for any person to store or deposit refuse except as provided in this chapter and the regulations adopted by the Council under this chapter.
\subsection{}
It is unlawful for any person to transport refuse over any street, for hire, except by special permit from the Council, or acting within the course and scope of a written contract with the city, or his or her employment with the city.
\subsection{}
It is unlawful for any person to transport refuse on any street unless it is carried in a vehicle equipped with a leak-proof body or container and completely covered with a heavy canvas or top to prevent loss of contents.
\subsection{}
It is unlawful for any person to use the refuse container or dumpster of another without express written permission.  Permission to use a refuse container or dumpster cannot be implied because the refuse container or dumpster is empty or is accessible to the public.
\subsection{}
It is unlawful for any person to deposit refuse from any source, rubbish, offal or the body of a dead animal in any place other than a sanitary landfill.
\subsection{}
It is unlawful for any person occupying property on which collectible refuse is produced or accumulated to refuse or fail to obtain the collection service provided under this chapter.  In unusual cases where the collection or disposal of refuse by collection service is impractical, an alternative type of disposal may be permitted by the Public Works Director upon the occupant’s or owner’s request therefor.\footnote{Penalty, see SEC. 50.99}

\section{Storage Containers}
\index{SOLID WASTE!Storage Containers}
Except as otherwise stated in this section, all refuse shall be stored in plastic bags approved by the city and purchased from the city or its designated distributors.  Tree clippings may be stored in tied bundles no longer than four feet.  Lawn clippings, leaves and other garden and lawn waste may be stored in containers protected from wind and other elements.  Recyclable materials designated by the city for collection may be stored in recycling containers approved by the city or, if the city has not approved containers for recycling, in bags or boxes.\footnote{Penalty, see SEC. 50.99}

\section{Collection and Disposal of Refuse}
\index{SOLID WASTE!Collection and Disposal of Refuse}
The city shall provide for collection and disposal of refuse from residential properties in a sanitary manner to insure the health, safety and general welfare of its residents.  Containers shall be placed at the designated collection points on days specified by the city.

\section{Additional Rules and Regulations}
\index{SOLID WASTE!Additional Rules and Regulations}
The Council may by resolution adopt additional rules and regulations not inconsistent with this chapter as may be necessary or helpful to the effective collection and disposal of refuse in the city.

\section{Property of the City}
\index{SOLID WASTE!Property of the City}
All materials at public disposal sites are the property of the city.  It is unlawful for any person to separate, collect, carry off or dispose of the materials except by direction of the city.\footnote{Penalty, see SEC. 50.99}

\section{Deposit of Materials Other Than Refuse}
\index{SOLID WASTE!Deposit of Materials Other Than Refuse}
Materials other than refuse may be deposited at the city designated disposal site upon payment of charges therefor.
