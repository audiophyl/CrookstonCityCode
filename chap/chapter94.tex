\chapter*{Chapter 94: \\
	Health and Safety; Nuisances}
    \addstarredchapter{Chapter 94: Health and Safety; Nuisances}
    \minitoc
    \pagebreak

\begin{center}
\emph{\textbf{\LARGE{GENERAL PROVISIONS}}}
\end{center}

\section{Assessable Current Services}
\subsection{Definition}
\textbf{CURRENT SERVICE} as used in this section, means one or more of the following: 
\begin{enumerate}[{\indent}1)]
    \item Remove snow, ice, dirt, and refuse from sidewalks; 
    \item Eliminate weeds and cut grass on private property and on non-traveled portions of abutting streets; 
    \item Remove or eliminate public health or safety hazards from private property; 
    \item Repair abutting sidewalks; 
    \item Street sprinkling, street flushing, light street oiling, or other dust treatment of streets; 
    \item Trimming and care of trees and removal of unsound trees from public streets or private property; 
    \item The operation of a street lighting system; 
    \item Remove and dispose of or eliminate junk from private property; and
    \item Fees of Rental Certification Program; including late fees, re-inspection fees and re-scheduling fees.
\end{enumerate}
\subsection{Responsibility of Owner or Occupant}
It is unlawful for the owner or occupant of private property to fail to and it is the primary responsibility of all owners and occupants of private property to: 
\begin{enumerate}[{\indent}1)]
    \item Remove snow, ice, dirt, and refuse from adjacent sidewalks; 
    \item Eliminate weeds and cut grass thereon and non-traveled portions of abutting streets; 
    \item Remove or eliminate junk or public health and safety hazards therefrom; 
    \item Repair abutting sidewalks; and
    \item Removal and storage of junk vehicles.
\end{enumerate}
\subsection{Ice, Snow, Dirt and Refuse on Sidewalks}
All ice and snow within 48 hours after it ceases to be deposited thereon, and all dirt and refuse deposited thereon, shall be removed by the owner or occupant of abutting private property.  If ice, snow, dirt or refuse is not so removed the city may do so and keep a record of the cost attributable to each property.
\subsection{Weeds and Grass}
The city shall mail notice to affected property owners to cut and remove all weeds whether noxious or not, and cut all grass, on private property having attained a height of six inches, and on non-traveled portions of abutting streets, within three (3) days after the notice. If weeds are not so cut and removed, or if grass is not so cut, the city may do so and keep a record of the cost attributable to each property. See Section 94.35.
\subsection{Junk and Public Health and Safety Hazards}
\subsubsection{Junk}  As used in this section, \emph{JUNK} means and includes:
\begin{enumerate}[{\indent}a)]
    \item Any unlicensed, unregistered, or inoperable (including, but not limited to the lack of component parts) motor vehicle, motorized vehicle, bicycle, boat, outboard motor, or trailer and all furniture, household furnishings, or appliances, or parts or components thereof, metal, paper, or rags, unless housed within an enclosed garage or storage building; 
    \item Inoperable (including, but not limited to, the lack of component parts) agricultural implements or parts or components thereof, machines and mechanical equipment of all kinds or parts or components thereof, unless housed within a building or on the premises of a licensed junk dealer and bi-products or waste from manufacturing operations of all kinds; 
    \item Used lumber or waste resulting from building construction, renovation, remodeling, or demolition; or 
    \item Felled trees and tree branches that are not immediately processed into lumber, wood for fuel, fence components, or other such ultimate use.
\end{enumerate}
\subsubsection{Public Health and Safety Hazards}
As used in this section, \emph{PUBLIC HEALTH AND SAFETY HAZARDS} means:
\begin{enumerate}[{\indent}a)]
    \item Organic or inorganic material resulting from the manufacture, preparation, or serving of food or food products, spoiled, decayed, or waste food from any source, bottles, cans, glassware, paper, or paper products, crockery, ashes, rags, and discarded clothing, or human or household waste of all kinds not stored in the manner provided in city code, Chapter 51 for the storage of refuse; 
    \item Carcasses of animals not buried or destroyed within 24 hours after death; or 
    \item Any condition which unreasonably annoys, injures, or endangers the safety, health, morals, comfort, or repose of any considerable number of members of the public 
\end{enumerate}
\subsubsection{Procedure for Removal or Elimination}
\paragraph{Observation, Report, and Investigation}
Any condition whether or not lawful relating to junk or public health and safety hazard permitted or maintained on private property and reported to the city will be referred to the City Administrator, who will assign the report to the city employee or department he or she deems appropriate for investigation.
\paragraph{Notice-No Emergency}
Where, in the opinion of the City Administrator, junk or a public health and safety hazard should be removed or eliminated, and no emergency exists, notice of the required removal action will be given to the owner and occupant of the property.  The notice will require completion of the removal within ten days.
\paragraph{Notice-Emergency}
Where, in the opinion of the City Administrator, junk or a public health and safety hazard should be removed or eliminated, and an emergency exists, no notice or such notice of the required removal as is reasonable under the circumstances will be given to the owner and occupant of the property.
\paragraph{Failure of Owner and Occupant to Remove or Eliminate}
If the owner and occupant do not so remove or eliminate the junk or public health and safety hazard, the city may do so or cause the same to be done and keep a record of the cost thereof.
\subsection{Repair of Sidewalks}
\subsubsection{Notice-No Emergency}
Where, in the opinion of the city, no emergency exists, notice of required repair or reconstruction shall be given to the owner and occupant of the abutting property.  The notice shall require completion of the work within 90 days.
\subsubsection{Notice-Emergency}
Where, in the opinion of the city, an emergency exists, notice of required repair or reconstruction shall be given to the owner and occupant of the abutting property.  The notice shall require completion of the work within ten days.
\subsubsection{Failure of Owner and Occupant to Repair or Reconstruct}
If the owner and occupant do not so repair or reconstruct the sidewalk, the city may do so and keep a record of the cost thereof.
\subsection{Street Sprinkling, Street Flushing, and Tree Care}
\subsubsection{}
The Council may each year determine what streets shall be sprinkled or flushed, oiled, or given other dust treatment during the year and the kind of work to be done on each.  The Council may also determine from time to time the streets on which trees (other than diseased trees) shall be trimmed and cared for, the kind of work to be done, and what unsound trees shall be removed.  The City Administrator shall, under the Council’s direction, publish notice that the Council will meet to consider the projects.  The notice shall be published in the official newspaper at least once no less than two weeks prior to the meeting of the Council and shall state the date, time, and place of the meeting, the streets affected and the particular projects proposed and the estimated cost of each project, either in total or on the basis of the proposed assessment per front foot or otherwise.
\subsubsection{}
At the hearing the Council shall hear property owners with reference to the scope and desirability of the proposed projects.  The Council shall thereupon adopt a resolution confirming the original projects with the modifications as it considers desirable and shall provide for doing the work by day labor or by contract.  The city shall keep a record of the cost and the portion of the cost properly attributable to each lot and parcel of property abutting on the street.
\subsection{Street Lighting System}
The City Administrator may keep a record of the cost of operation of the street lighting system for the twelve months preceding September 1 of each year and the proportion of the cost properly attributable during that period to each lot and parcel of property abutting on the street in which the system is located.
\subsection{Personal Liability}
The owner and occupant of the property on or adjacent to which a current service has been performed are individually and jointly personally liable for the entire cost of the service, including any administrative fees, fines or penalties and any other obligation owed to the city, which relates to the current service. When the service has been completed and the cost, along with any administrative fees, fines or penalties and any other obligation owed to the city which relate to the service, determined, the city may prepare a bill and mail it to the owner and occupant and the amount will then be due and payable.
\subsection{Assessment}
Charges for any current services and for administrative fees, fines or penalties and for any other obligation owed to the city which relate to the current services unpaid after billing, and after notice and hearing, may be certified to the County Auditor and collected as any other special assessments.\\
\emph{(‘83 Code, SEC. 2.72)  (Ord. 71, 2nd Series, effective 7-13-91; Ord. 113, 2nd Series, effective 10-1-96; Am. Ord. 137, 2nd Series, passed 12-8-98)  Penalty, see SEC. 10.99}

\section{Tree Diseases}
\subsection{Trees Constituting Nuisance Declared}
The following are public nuisances whenever they may be found within the city:
\begin{enumerate}[{\indent}1)]
    \item Any living or standing elm tree or part thereof infected to any degree with the Dutch Elm disease fungus Ceratocystis Ulmi (Buisman) Moreau or which harbors any of the elm bark beetles Scolytus Multistriatus (Eichh.) or Hylungopinus Rufipes (Marsh);
    \item Any dead elm tree or part thereof, including branches, stumps, firewood or other elm material from which the bark has not been removed and burned or sprayed with an effective elm bark beetle insecticide;
    \item Any living or standing oak tree or part thereof infected to any degree with the Oak Wilt fungus Ceratocystis fagacearum;
    \item Any dead oak tree or part thereof which in the opinion of the designated officer constitutes a hazard, including but not limited to logs, branches, stumps, roots, firewood or other oak material which has not been stripped of its bark and burned or sprayed with an effective fungicide;
    \item Any other shade tree with an epidemic disease.
\end{enumerate}
\subsection{Abatement of Nuisance}
It is unlawful for any person to permit any public nuisance as defined in division (A) of this section to remain on any premises the person owns or controls within the city. The City Council may by resolution order the nuisance abated. Before action is taken on that resolution, the City Council shall publish notice of its intention to meet to consider taking action to abate the nuisance. This notice shall be mailed to the affected property owner and published once no less than one week prior to the meeting. The notice shall state the time and place of the meeting, the street affected, action proposed, the estimated cost of the abatement, and the proposed basis of assessment, if any, of costs. At the hearing or adjournment thereof, the City Council shall hear any property owner with reference to the scope and desirability of the proposed project. The City Council shall thereafter adopt a resolution confirming the original resolution with modifications as it considers desirable and provide for the doing of the work by day labor or by contract.
\subsection{Record of Costs}
The Clerk-Treasurer shall keep a record of the costs of abatement done under this section for all work done for which assessments are to be made, stating and certifying the description of the land, lots, parcels involved, and the amount chargeable to each.
\subsection{Unpaid Charges}
The Clerk-Treasurer shall list the total unpaid charges for each abatement against each separate lot or parcel to which they are attributable under this section. The City Council may then spread the charges or any portion thereof against the property involved as a special assessment as authorized by M.S. § 429.101, as it may be amended from time, to time and other pertinent statutes for certification to the County Auditor and collection the following year along with the current taxes.\\
\emph{Penalty, see SEC. 10.99}


\begin{center}
\emph{\textbf{\LARGE{NUISANCES}}}
\end{center}
\setcounter{section}{14}
\section{Public Nuisance}
Whoever by his or her act or failure to perform a legal duty intentionally does any of the following is guilty of maintaining a public nuisance, which is a misdemeanor:
\begin{enumerate}[{\indent}A)]
    \item Maintains or permits a condition which unreasonably annoys, injures or endangers the safety, health, morals, comfort or repose of any considerable number of members of the public;
    \item Interferes with, obstructs or renders dangerous for passage any public highway or right-of-way, or waters used by the public; or
    \item Is guilty of any other act or omission declared by law or SEC. 94.16, 94.17 or 94.18, or any other part of this code to be a public nuisance and for which no sentence is specifically provided.
\end{enumerate}
\emph{Penalty, see SEC. 10.99}
\section{Public Nuisances Affecting Health}
The following are hereby declared to be nuisances affecting health:
\begin{enumerate}[{\indent}A)]
    \item Exposed accumulation of decayed or unwholesome food or vegetable matter;
    \item All diseased animals running at large;
    \item All ponds or pools of stagnant water;
    \item Carcasses of animals not buried or destroyed within 24 hours after death;
    \item Accumulations of manure, refuse or other debris;
    \item Privy vaults and garbage cans which are not rodent-free or fly-tight or which are so maintained as to constitute a health hazard or to emit foul and disagreeable odors;
    \item The pollution of any public well or cistern, stream or lake, canal or body of water by sewage, industrial waste or other substances;
    \item All noxious weeds and other rank growths of vegetation upon public or private property;
    \item Dense smoke, noxious fumes, gas and soot, or cinders, in unreasonable quantities;
    \item All public exposure of people having a contagious disease; and
    \item Any offensive trade or business as defined by statute not operating under local license.
\end{enumerate}
\emph{Penalty, see SEC. 10.99}
\section{Public Nuisances Affecting Morals and Decency}
The following are hereby declared to be nuisances affecting public morals and decency:
\begin{enumerate}[{\indent}A)]
    \item All gambling devices, slot machines and punch boards, except as otherwise authorized by federal, state or local law;
    \item Betting, bookmaking and all apparatus used in those occupations;
    \item All houses kept for the purpose of prostitution or promiscuous sexual intercourse, gambling houses, houses of ill fame and bawdy houses;
    \item All places where intoxicating liquor is manufactured or disposed of in violation of law or where, in violation of law, people are permitted to resort for the purpose of drinking intoxicating liquor, or where intoxicating liquor is kept for sale or other disposition in violation of law, and all liquor and other property used for maintaining that place;
    \item Any vehicle used for the unlawful transportation of intoxicating liquor, or for promiscuous sexual intercourse, or any other immoral or illegal purpose.
\end{enumerate}
\emph{Penalty, see SEC. 10.99}
\section{Public Nuisances Affecting Peace and Safety}
The following are declared to be nuisances affecting public peace and safety:
\subsection{}
All snow and ice not removed from public sidewalks 48 hours after the snow or other precipitation causing the condition has ceased to fall;
\subsection{}
All trees, hedges, billboards or other obstructions which prevent people from having a clear view of all traffic approaching an intersection;
\subsection{}
All wires and limbs of trees which are so close to the surface of a sidewalk or street as to constitute a danger to pedestrians or vehicles;
\subsection{}
\subsubsection{Prohibited Generally}
It shall be unlawful for any person to make or cause to be made any loud, unnecessary or unusual noise which either annoys, disturbs, or affects the comfort, repose, health or peace of others.
\subsubsection{Prohibited Specifically}
The following acts are declared to be loud, disturbing and unnecessary noises in violation of Section 94.18(D)(1), but such enumeration shall not be deemed to be exclusive.
\paragraph{Horns and Signaling Devices}
The sounding of any horn or signaling device on any automobile, motorcycle or other vehicles, except as a danger Warning.
\paragraph{Radios; Tape and Disk Players} The using, operating or permitting to be played of a radio receiving set, tape or disk player, or other machine or device for the producing or reproducing of sound in such manner, considering the time, place and purpose for which the sound is produced, as to disturb the peace, quiet or repose of a person or persons of ordinary sensibilities.
\subparagraph{}
The play, use or operation of any radio, tape or disk player, or other machine or device for the production or reproduction of sound in such a manner which is plainly audible at a distance of 50 feet from such machine or device shall be prima facie evidence of a violation of this section.
\subparagraph{}
When an unlawful sound violating this section is produced or reproduced by a machine or device that is located in or on a vehicle, the vehicle’s owner is guilty of the violation; provided, however, that if the vehicle’s owner is not present at the time of violation, the person in charge or control of the vehicle at the time of the violation is guilty of the violation.
\subparagraph{}
Section 94.18(D)(1) shall not apply to sound produced by the following:
\begin{enumerate}[{\indent}A.]
    \item Amplifying equipment used in connection with activities which are authorized, sponsored or permitted by the city, including sporting or other public events as long as this activity is conducted pursuant to the conditions of the license, permit or contract authorizing such activity.
    \item Church bells or chimes.
    \item School bells.
    \item Machines or devices for the production of sound on or in authorized emergency vehicles.
    \item Governmental warning systems.
    \item Sounds emanating from the operation of motor vehicles on a public highway; aircraft; outdoor implements such as power lawn mowers, snow blowers, power hedge clippers, and power saws; pile drivers or jackhammers and other construction equipment, except during the hours of 10:00 pm. to 7:00 a.m.; and, sounds emanating from lawful and proper activities at school grounds, playgrounds, parks or places wherein athletic contests take place.
    \item Noise created exclusively in the performance of emergency work to preserve the public health, safety or welfare, or in the performance of emergency work when all reasonable actions are taken to minimize the amount of noise.
\end{enumerate}
\paragraph{Exhaust and Muffler}
No person shall operate or permit the operation of a motor vehicle upon a street, highway or alley in the City with an engine the exhaust system of which has been altered, modified or repaired, including the use of an engine retarding (Jake Brake) or a muffler or resonant kit, such that the noise emitted by the engine’s exhaust system is increased to make loud explosive noises or unusual noise which either annoys, disturbs, or affects the comfort, repose, health or peace of others.
\subsection{}
The discharging of the exhaust or permitting the discharging of the exhaust of any stationary internal combustion engine, motor boat, motor vehicle, motorcycle, all terrain vehicle, snowmobile or any recreational device except through a muffler or other device that effectively prevents loud or explosive noises therefrom and complies with all applicable state laws and regulations.
\subsection{}
The using or operation or permitting the using or operation of any radio receiving set, musical instrument, phonograph, paging system, machine or other device for producing or reproduction of sound in a distinctly and loudly audible manner so as to disturb the peace, quiet and comfort of any person nearby. Operation of any device referred to above between the hours of 10:00 p.m. and 7:00 a.m. in a manner so as to be plainly audible at the property line of the structure or building in which it is located, or at a distance of 50 feet if the source is located outside a structure or building shall be prima facie evidence of violation of this section.
\subsection{}
The participation in a party or gathering of people giving rise to noise which disturbs the peace, quiet or repose of the occupants of adjoining or other property.
\subsection{}
Obstructions and excavations affecting the ordinary public use of streets, alleys, sidewalks or public grounds except under conditions as are permitted by this code or other applicable law;
\subsection{}
 Radio aerials or television antennae erected or maintained in a dangerous manner;
\subsection{}
 Any use of property abutting on a public street or sidewalk or any use of a public street or sidewalk which causes large crowds of people to gather, obstructing traffic and the free use of the street or sidewalk;
\subsection{}
 All hanging signs, awnings and other similar structures over streets and sidewalks, so situated so as to endanger public safety, or not constructed and maintained as provided by ordinance;
\subsection{}
 The allowing of rain water, ice or snow to fall from any building or structure upon any street or sidewalk or to flow across any sidewalk;
\subsection{}
 Any barbed wire fence less than six feet above the ground and within three feet of a public sidewalk or way;
\subsection{}
 All dangerous, unguarded machinery in any public place, or so situated or operated on private property as to attract the public;
\subsection{}
 Waste water cast upon or permitted to flow upon streets or other public properties;
\subsection{}
 Accumulations in the open of discarded or disused machinery, household appliances, automobile bodies or other material in a manner conducive to the harboring of rats, mice, snakes or vermin, or the rank growth of vegetation among the items so accumulated, or in a manner creating fire, health or safety hazards from accumulation;
\subsection{}
 Any well, hole or similar excavation which is left uncovered or in another condition as to constitute a hazard to any child or other person coming on the premises where it is located;
\subsection{}
 Obstruction to the free flow of water in a natural waterway or a public street drain, gutter or ditch with trash of other materials;
\subsection{}
 The placing or throwing on any street, sidewalk or other public property of any glass, tacks, nails, bottles or other substance which may injure any person or animal or damage any pneumatic tire when passing over the substance;
\subsection{}
 The depositing of garbage or refuse on a public right-of-way or on adjacent private property; and
\subsection{}
 All other conditions or things which are likely to cause injury to the person or property of anyone.
\emph{Penalty, see SEC. 10.99}
\section{Duties of City Officers}
Any appropriate city officer or department shall enforce the provisions relating to nuisances.  Any city officer shall have the power to inspect private premises and take all reasonable precautions to prevent the commission and maintenance of public nuisances.
\section{Abatement}
\subsection{Enforcement}
Sections of this chapter enforced by the Police Department shall be enforced through citation, summons, complaint, or arrest, to the appropriate county or district court.
\subsection{Notice}
For all violations of this chapter enforced by city officers other than the Police Department, written notice of violation; notice of the time, date, place and subject of any hearing before the City Council; notice of City Council order; and notice of motion for summary enforcement hearing shall be given as set forth in this section.
\subsubsection{Notice of Violation}
Written notice of violation shall be served by a city officer on the owner of record or occupant of the premises either in person or by certified or registered mail.  If the premises is not occupied, the owner of record is unknown, or the owner of record or occupant refuses to accept notice of violation, notice of violation shall be served by posting it on the premises.
\subsubsection{Notice of City Council Hearing}
Written notice of any City Council hearing to determine or abate a nuisance shall be served on the owner of record and occupant of the premises either in person or by certified or registered mail.  If the premises is not occupied, the owner of record is unknown, or the owner of record or occupant refuses to accept notice of the City Council hearing, notice of City Council hearing shall be served by posting it on the premises.
\subsubsection{Notice of City Council Order}
Except for those cases determined by the city to require summary enforcement, written notice of any City Council order shall be made as provided in M.S. \textsection 463.17 (Hazardous and Substandard Building Act), as it may be amended from time to time.
\subsubsection{Notice of Motion for Summary Enforcement}
Written notice of any motion for summary enforcement shall be made as provided for in M.S. \textsection 463.17 (Hazardous and Substandard Building Act), as it may be amended from time to time.
\subsection{Procedure}
Whenever a proper city official determines that a public nuisance is being maintained or exists on the premises in the city, the officer shall notify in writing the owner of record or occupant of the premises of the fact and order that the nuisance be terminated or abated.  The notice of violation shall specify the steps to be taken to abate the nuisance and the time within which the nuisance is to be abated.  If the notice of violation is not complied with within the time specified, the officer shall report that fact forthwith to the City Council.  Thereafter, the City Council may, after notice to the owner or occupant and an opportunity to be heard, determine that the condition identified in the notice of violation is a nuisance and further order that if the nuisance is not abated within the time prescribed by the City Council, the city may seek injunctive relief by serving a copy of the City Council order and notice of motion for summary enforcement.
\subsection{Emergency Procedure; Summary Enforcement}
In cases of emergency, where delay in abatement required to complete the notice and procedure requirements set forth in divisions (B) and (C) of this section will permit a continuing nuisance to unreasonably endanger public health safety or welfare, the City Council may order summary enforcement and abate the nuisance.  To proceed with summary enforcement, the officer shall determine that a public nuisance exists or is being maintained on premises in the city and that delay in abatement of the nuisance will unreasonably endanger public health, safety or welfare.  The officer shall notify in writing the occupant or owner of the premises of the nature of the nuisance and of the city’s intention to seek summary enforcement and the time and place of the City Council meeting to consider the question of summary enforcement.  The City Council shall determine whether or not the condition identified in the notice to the owner or occupant is a nuisance, whether public health, safety or welfare will be unreasonably endangered by delay in abatement required to complete the procedure set forth in division (B) of this section, and may order that the nuisance be immediately terminated or abated.  If the nuisance is not immediately terminated or abated, the City Council may order summary enforcement and abate the nuisance.
\subsection{Immediate Abatement}
Nothing in this section shall prevent the city, without notice or other process, from immediately abating any condition which poses an imminent and serious hazard to human life or safety.\\
\emph{Penalty, see SEC. 10.99}
\section{Recovery of Cost}
\subsection{Personal Liability}
The owner of premises on which a nuisance has been abated by the city shall be personally liable for the cost to the city of the abatement, including administrative costs.  As soon as the work has been completed and the cost determined, the Clerk-Treasurer or other official shall prepare a bill for the cost and mail it to the owner.  Thereupon the amount shall be immediately due and payable at the office of the Clerk-Treasurer.
\subsection{Assessment}
If the nuisance is a public health or safety hazard on private property, the accumulation of snow and ice on public sidewalks, the growth of weeds on private property or outside the traveled portion of streets, or unsound or insect-infected trees, the Clerk-Treasurer shall, on or before September 1 next following abatement of the nuisance, list the total unpaid charges along with all other the charges as well as other charges for current services to be assessed under M.S. \textsection 429.101, as it may be amended from time to time, against each separate lot or parcel to which the charges are attributable.  The City Council may then spread the charges against the property under that statute and other pertinent statutes for certification to the County Auditor and collection along with current taxes the following year or in annual installments, not exceeding ten, as the City Council may determine in each case.\\
\emph{Penalty, see SEC. 10.99}

\begin{center}
\emph{\textbf{\LARGE{WEEDS}}}
\end{center}
\setcounter{section}{34}
\section{Weeds and Grass}
\subsection{}
On or before June 1 of each year, after the effective date of this subdivision and at such other times as ordered by resolution of the City Council, the City Clerk Treasurer shall publish once in the official newspaper a notice directing owners and occupants of property within the City to cut and remove all weeds, whether noxious or not, and all grass outside the traveled portion of any street or alley adjacent to the property having attained a height of six (6) inches and stating that if not so cut and removed, the city may do so at the expense of the owner and occupant and, if not paid, the charge for such work may be made a special assessment against the property concerned. Failure of the City Clerk-Treasurer to properly publish the general notice does not relieve a person from the necessity of full compliance with Chapter 94 Section 94.01 and related provisions. The city shall also mail notice to affected property owners to cut and remove all weeds, whether noxious or not, and cut all grass, on private property and on non-traveled portions of any abutting street or alley, having attained a height of six (6) inches, within three (3) days after such notice. If weeds are not so cut and removed, or if grass is not so cut, the city may do so and keep a record oftbe cost attributable to each property. See Section 94.01(D).
\subsection{}
This Subdivision shall not apply to: “Natural areas” which shall be defined as densely wooded areas, bogs and marshes, as well as undisturbed lands, an area where the land and vegetation appear not to have been graded, landscaped, mowed or otherwise disturbed by human or mechanical means at any time. The weed inspector or assignee shall use reasonable judgment in determining what constitutes this type of area based on the present appearance of the area and research as to the history of the area, if such information is available. “Natural areas” may also be created as follows:
\begin{enumerate}[{\indent}1)]
    \item Those individuals who wish to naturalize portions of their property must submit a Landscaping Management Plan for approval.  Such Plan will require a review before the Parks and Recreation Department, which will make recommendations to approve or disapprove the Plan to the City Council.  The Landscaping Management Plan will require a public hearing.
    \item In the event that a notice of a possible violation of this Subdivision has been sent, and the individual wishes to submit a request to naturalize the area that is designated in the notice, the individual may, within the required time frame for correction of the violation, submit a Landscaping Management Plan.  The City will take no enforcement action until a final determination has been made on the Plan.
    \item Landscaping Management Plan means a written plan relating to the management of the natural area, which contains a plot plan of the area upon which grass and other growth will exceed six (6) inches in height or length, a statement of intent and purpose for the area, a general description of the vegetation types, plant succession involved and specific management techniques to be employed.
    \item Notwithstanding the fact that approval has been given for a Landscaping Management Plan, the Fire Chief may order the cutting of such natural area at any time when in the exercise of his/her official duties, he/she determines that the growth has become so hazardous as to cause a danger to the safety of the inhabitants of any residential structure on said property or to the citizens and residents of the neighborhood in which the Landscaping Management Plan has been approved.
\end{enumerate}
\subsection{}
Transmission line or utility easements.
\subsection{}
Park land.
\subsection{}
Storm water/sanitary ponds.
\subsection{}
Land used for agricultural purposes.
\subsection{}
Compost areas when the compost is in a compost enclosure which is of adequate construction to allow for the decomposition of the material and the compost is screened from view of adjacent property owners.
\subsection{}
An area that is steeply sloped as to make mowing or cutting of the vegetation not reasonable or practical for equipment operation or safety.
\subsection{}
Non-noxious weeds and grass and herbaceous vegetation within fifty (50) feet of natural or altered creeks, rivers and stream corridors, including riparian buffer strips that convey water.\\
Notwithstanding any provision of this Subdivision to the contrary, noxious weeds and other vegetation must be eliminated or cut as otherwise required by law or regulation.\\
\emph{(‘83 Code, SEC. 2.72, Subd. 4)  (Ord. 156, 2nd Series, effective 5-16-2003)  Penalty, see SEC. 10.99}

\begin{center}
\emph{\textbf{\LARGE{OPEN BURNING}}}
\end{center}
\setcounter{section}{59}
\section{Definitions}
For the purpose of this chapter, the following definitions shall apply unless the context clearly indicates or requires a different meaning.
\begin{description}
    \item[FIRE CHIEF] The Fire Chief of the Fire Department which provides fire protection services to the city.
    \item[OPEN BURNING] The burning of any matter if the resultant combustion products are emitted directly to the atmosphere without passing through a stack, duct or chimney, except a “recreational fire” as defined herein. Mobile cooking devices such as manufactured hibachis, charcoal grills, wood smokers, and propane or natural gas devices are not defined as “open burning.”  
    \item[STARTER FUELS] Dry, untreated, unpainted, kindling, branches, cardboard or charcoal fire starter. Paraffin candles and alcohols are permitted as starter fuels and as aids to ignition only.  Propane gas torches or other clean gas burning devices causing minimal pollution must be used to start an open burn.
    \item[WOOD] Dry, clean fuel only such as twigs, branches, limbs, “presto logs,” charcoal, cord wood or untreated dimensional lumber. The term does not include wood that is green with leaves or needles, rotten, wet, oil soaked, or treated with paint, glue or preservatives. Clean pallets may be used for recreational fires when cut into three foot lengths.
    \item[RECREATIONAL FIRES] A fire set with approved starter fuel no more than 3 feet or less in diameter and a flame height of 2 feet or less contained within the border of a “recreational fire site” using dry, clean wood; producing little detectable smoke, odor or soot beyond the property line; conducted with an adult tending the fire at all times; for recreational, ceremonial, food preparation for social purposes; extinguished completely before quitting the occasion; and respecting weather conditions, neighbors, burning bans, and air quality so that nuisance, health or safety hazards will not be created. No more than one recreational fire is allowed on any property at one time.
    \item[RECREATIONAL FIRE SITES] An area no more than 3 feet or less in diameter and contained completely surrounded by non-combustible and non-smoke or odor producing material, either of natural rock, cement, brick, tile or blocks or ferrous metal only in an area which is depressed below ground, on the ground, or on a raised bed. Included are outdoor wood burning fireplaces. Burning barrels are not a “recreation fire site” as defined herein.
\end{description}
\section{Prohibited Activities}
\subsection{}
No person shall conduct, cause or permit open burning oils, petro fuels, rubber, plastics, chemically treated materials, or other materials which produce excessive or noxious smoke such as tires, railroad ties, treated, painted or glued wood composite shingles, tar paper, insulation, composition board, sheetrock, wiring, paint or paint fillers.
\subsection{}
No person shall conduct, cause or permit open burning of hazardous waste or salvage operations, open burning of solid waste generated from an industrial or manufacturing process or from a service or commercial establishment or building material generated from demolition of commercial or institutional structures.
\subsection{}
No person shall conduct, cause or permit open burning of discarded material resulting from the handling, processing, storage, preparation, serving or consumption of food.
\subsection{}
No person shall conduct, cause or permit open burning of any leaves or grass clippings.
\subsection{}
No person shall conduct, cause, permit or maintain a recreational fire site:
\begin{enumerate}[{\indent}1)]
    \item Three feet or more in diameter and a flame height of two feet or more contained within the border of a “recreational fire site”;
    \item That is not completely surrounded by natural rock, cement, brick, tile, blocks or ferrous metal, which is depressed below ground, on the ground or on a raised bed;
    \item In a burn barrel;
    \item Closer than 15 feet to any structure; or
    \item In excess of one recreational fire site per residence.
\end{enumerate}
\subsection{}
No person shall conduct, cause, permit or maintain a recreational fire:
\begin{enumerate}[{\indent}1)]
    \item Using materials other than dry and clean wood;
    \item That is deemed by the Chief of the Fire Department or his/her designee to be hazardous to public safety or property;
    \item That is unattended at any time;
    \item That produces excessive smoke, odor or soot; or
    \item That creates a nuisance.
\end{enumerate}
\subsection{}
No person shall conduct, cause, permit or maintain a recreational fire and/or fire site at any time in which a burning ban has been imposed that includes the City of Crookston.\\
\emph{Penalty, see SEC. 10.99}
\section{Permit Required for Open Burning}
No person shall start or allow any open burning on any property in the city without first having obtained an open burn permit.
Penalty, see SEC. 10.99
\section{Purposes Allowed for Open Burning}
\subsection{}
Open burn permits may be issued only for the following purposes:
\begin{enumerate}[{\indent}1)]
    \item Elimination of fire or health hazard that cannot be abated by other practical means.
    \item Ground thawing for utility repair and construction.
    \item Disposal of vegetative matter for managing forest, prairie or wildlife habitat, and in the development and maintenance of land and rights-of-way where chipping, composting, landspreading or other alternative methods are not practical.
    \item Disposal of diseased trees generated on site, diseased or infected nursery stock, diseased bee hives.
    \item Disposal of unpainted, untreated, non-glued lumber and wood shakes generated from construction, where recycling, reuse, removal or other alternative disposal methods are not practical.
    \item Recreation fire for public purpose.
\end{enumerate}
\section{Permit Application for Open Burning; Permit Fees}
\subsection{}
Open burning permits shall be obtained by making application on a form prescribed by the Department of Natural Resources (DNR) and adopted by the Fire Department.  The permit application shall be presented to the Fire Chief for reviewing and processing those applications.
\subsection{}
An open burning permit may require the payment of a fee.  Permit fees shall be in the amount established by ordinance, and as it may be amended from time to time.\\
\emph{Penalty, see SEC. 10.99}
\section{Permit Process for Open Burning}
Upon receipt of the completed open burning permit application and permit fee, the Fire Chief shall schedule a preliminary site inspection to locate the proposed burn site, note special conditions, and set dates and time of permitted burn and review fire safety considerations.
\section{Permit Holder Responsibility}
\subsection{}
Prior to starting an open burn, the permit holder shall be responsible for confirming that no burning ban or air quality alert is in effect.  Every open burn event shall be constantly attended by the permit holder or his or her competent representative.  The open burning site shall have available, appropriate communication and fire suppression equipment as set out in the fire safety plan.
\subsection{}
The open burn fire shall be completely extinguished before the permit holder or his or her representative leaves the site.  No fire may be allowed to smolder with no person present.  It is the responsibility of the permit holder to have a valid permit, as required by this subchapter, available for inspection on the site by the Police Department, Fire Department, MPCA representative or DNR forest officer.
\subsection{}
The permit holder is responsible for compliance and implementation of all general conditions, special conditions, and the burn event safety plan as established in the permit issued.  The permit holder shall be responsible for all costs incurred as a result of the burn, including but not limited to fire suppression and administrative fees.\\
\emph{Penalty, see SEC. 10.99}
\section{Revocation of Open Burning Permit}
The open burning permit is subject to revocation at the discretion of DNR forest officer or the Fire Chief.  Reasons for revocation include but are not limited to a fire hazard existing or developing during the course of the burn, any of the conditions of the permit being violated during the course of the burn, pollution or nuisance conditions developing during the course of the burn, or a fire smoldering with no flame present.\\
\emph{Penalty, see SEC. 10.99}
\section{Denial of Open Burning Permit}
If established criteria for the issuance of an open burning permit are not met during review of the application, it is determined that a practical alternative method for disposal of the material exists, or a pollution or nuisance condition would result, or if a burn event safety plan cannot be drafted to the satisfaction of the Fire Chief, these officers may deny the application for the open burn permit.
\section{Burning Ban or Air Quality Alert}
No open burning  will be permitted when the city or DNR has officially declared a burning ban due to potential hazardous fire conditions or when the MPCA has declared an Air Quality Alert.\\
\emph{Penalty, see SEC. 10.99}
\section{Rules and Laws Adopted by Reference}
The provisions of M.S. \textsection 88.16 to \textsection 88.22, as amended from time to time, and the Minnesota State Fire Code, are hereby adopted by reference and made a part of this subchapter as if fully set forth at this point.
