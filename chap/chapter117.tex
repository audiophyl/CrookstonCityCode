\chapter*{Chapter 117: \\
	Peddlers and Solicitors}
    \addstarredchapter{Chapter 117: Peddlers and Solicitors}
    \minitoc
    \pagebreak

\section{Definitions}
For the purpose of this chapter, the following definitions shall apply unless the context clearly indicates or requires a different meaning.
\begin{description}
    \item[PEDDLER] A person who goes from house-to-house, door-to-door, business-to-business, street-to-street, or any other type of place-to-place, for the purpose of offering for sale, displaying or exposing for sale, selling or attempting to sell, and delivering immediately upon sale, the goods, wares, products, merchandise or other personnel property that the person is carrying or otherwise transporting. The term peddler shall mean the same as the term hawker.
    \item[PERSON] Any natural individual, group, organization, corporation, partnership or association. As applied to groups, organizations, corporations, partnerships and associations, the term shall include each member, officer, partner, associate, agent or employee.
    \item[REGULAR BUSINESS DAY] Any day during which the city hall is normally open for the purpose of conducting public business. Holidays defined by state law shall not be counted as regular business days.
    \item[SOLICITOR] A person who goes from house-to-house, door-to-door, business-to-business, street-to-street, or any other type of place-to-place, for the purpose of obtaining or attempting to obtain orders for goods, wares, products, merchandise, other personal property or services of which he or she may be carrying or transporting samples, or that may be described in a catalog or by other means, and for which delivery or performance shall occur at a later time. The absence of samples or catalogs shall not remove a person from the scope of this provision if the actual purpose of the person's activity is to obtain or attempt to obtain orders as discussed above. The term shall mean the same as the term “canvasser."
    \item[TRANSIENT MERCHANT] A person who temporarily sets up business out of a vehicle, trailer, boxcar, tent, other portable shelter, or empty store front for the purpose of exposing or displaying for sale, selling or attempting to sell, and delivering, goods, wares, products, merchandise or other personal property and who does not remain or intend to remain in any one location for more than 14 consecutive days.
\end{description}

\section{Exceptions to Definitions}
\subsection{}
For the purpose of the requirements of this chapter, the terms \textbf{PEDDLER}, \textbf{SOLICITOR}, and \textbf{TRANSIENT MERCHANT} shall not apply to any person selling or attempting to sell at wholesale any goods, wares, products, merchandise or other personal property to a retailer of the items being sold by the wholesaler. The terms also shall not apply to any person who makes initial contacts with other people for the purpose of establishing or trying to establish a regular customer delivery route for the delivery of perishable food and dairy products such as baked goods and milk, nor shall they apply to any person making deliveries of perishable food and dairy products to the customers on his or her established regular delivery route. No license or registration will be required of any person desiring to engage in a regulated activity for a charitable, religious, patriotic, or philanthropic purpose not involving the use of a professional fundraiser.
\subsection{}
In addition, persons conducting the type of sales commonly known as garage sales, rummage sales, or estate sales, as well as those persons participating in an organized multi-person bazaar or flea market, shall be exempt from the definitions of \textbf{PEDDLERS}, \textbf{SOLICITORS}, and \textbf{TRANSIENT MERCHANTS}, as shall be anyone conducting an auction as a properly licensed auctioneer, or any officer of the court conducting a court-ordered sale. Exemption from the definitions for the scope of this chapter shall not excuse any person from complying with any other applicable statutory provision or local ordinance.

\section{Licensing; Exemptions}
\subsection{County License Required}
No person shall conduct business as a peddler, solicitor or transient merchant within the city limits without first having obtained the appropriate license from the county as required by MS. Chapter 329 as it may be amended from time to time.
\subsection{City License Required}
Except as otherwise provided for by this chapter, no person shall conduct business as either a peddler or a transient merchant without first having obtained a license from the city. Solicitors need not be licensed, but are still required to register pursuant to SEC. 117.07.
\subsection{Application}
Application for a city license to conduct business as a peddler or transient merchant shalt be made at least 14 regular business days before the applicant desires to begin conducting business. Application for a license shall be made on a form approved by the City Council and available from the office of the Clerk-Treasurer. All applications shall be signed by the applicant. All applications shall include the following information:
\begin{enumerate}[{\indent}1)]
    \item Applicant’s full legal name and date of birth.
    \item All other names under which the applicant conducts business or to which applicant officially answers.
    \item A physical description of the applicant (hair color, eye color, height, weight, distinguishing marks and features, and the like).
    \item Full address of applicant's permanent residence.
    \item Telephone number of applicant’s permanent residence.
    \item Full legal name of any and all business operations owned, managed or operated by applicant, or for which the applicant is an employee or agent.
    \item Full address of applicant's regular place of business (if any).
    \item Any and all business related telephone numbers of the applicant.
    \item The type of business for which the applicant is applying for a license.
    \item Whether the applicant is applying for an annual or daily license.
    \item The dates during which the applicant intends to conduct business, and if the applicant is applying for a daily license, the number of days he or she will be conducting business in the city (maximum 14 consecutive days).
    \item Any and all addresses and telephone numbers where the applicant can be reached while conducting business within the city, including the location where a transient merchant intends to set up business.
    \item A statement as to whether or not the applicant has been convicted within the last five years of any felony, gross misdemeanor, or misdemeanor for violation of any state or federal statute or any local ordinance, other than traffic offenses.
    \item A list of the three most recent locations where the applicant has conducted business as a peddler or transient merchant.
    \item Proof of any required licenses.
    \item Written permission of the property owner or the property owner’s agent for any property to be used by a transient merchant.
    \item A general description of the items to be sold or services to be provided.
    \item All additional information deemed necessary by the City Council.
    \item The applicant's driver's license number or other acceptable form of identification.
    \item The license plate number, registration information and vehicle identification number for any vehicle to be used in conjunction with the licensed business and a description of the vehicle.
\end{enumerate}
\subsection{Fee}
The fee established by resolution, as amended from time to time shall accompany all applications for a license under this chapter.
\subsection{Procedure}
Upon receipt of the completed application and payment of the license fee, the Clerk-Treasurer, within two regular business days, must determine if the application is complete. An application is determined to be complete only if all required information is provided. If the Clerk-Treasurer determines that the application is incomplete, the Clerk-Treasurer must inform the applicant of the required necessary information that is missing. If the application is complete, the Clerk-Treasurer must order any investigation, including background checks, necessary to verify the information provided with the application. Within ten regular business days of receiving a complete application the Clerk-Treasurer must issue the license unless there exist grounds for denying the license under SEC. 117.04, in which case the Clerk-Treasurer must deny the license. If the Clerk-Treasurer denies the license, the applicant must be notified in writing of the decision, the reason for denial, and of the applicant’s right to appeal the denial by requesting, within 20 days of receiving notice of rejection, a public hearing before the City Council. The City Council shall hear the appeal within 20 days of the date of the request. The decision of the City Council following the public hearing can be appealed by petitioning the Minnesota Court of Appeals for a Writ of Certiorari.
\subsection{Duration}
An annual license granted under this chapter shall be valid for one calendar year from the date of issue. All other licenses granted under this chapter shall be valid only during the time period indicated on the license.
\subsection{License Exemptions}
\subsubsection{}
No license shall be required for any person to sell or attempt to sell, or to take or attempt to take orders for, any product grown, produced, cultivated, or raised on any farm.
\subsubsection{}
No license shall be required of any person going from house-to-house, door-todoor, business-to-business, street-to-street, or other type of place-to-place when the activity is for the purpose of exercising that person's State or Federal Constitutional rights such as the freedom of speech. press, religion and the like, except that this exemption may be lost if the person's exercise of Constitutional rights is merely incidental to a commercial activity.
\subsubsection{}
Professional fundraisers working on behalf of an otherwise exempt person or group shall not be exempt from the licensing requirements of this chapter.\\
\emph{Penalty, see SEC. 110.99}

\section{License Ineligibility}
The following shall be grounds for denying a license under this chapter:
\begin{enumerate}[{\indent}A)]
    \item The failure of the applicant to obtain and show proof of having obtained any required county license.
    \item The failure of the applicant to truthfully provide any of the information requested by the city as a part of the application, or the failure to sign the application, or the failure to pay the required fee at the time of application.
    \item The conviction of the applicant within the past five years from the date of application for any violation of any federal or state statute or regulation, or of any local ordinance, which adversely reflects on the person‘s ability to conduct the business for which the license is being sought in an honest and legal manner. Those violations shall include but not be limited to burglary, theft, larceny, swindling, fraud, unlawful business practices, and any form of actual or threatened physical harm against another person.
    \item The revocation within the past five years of any license issued to the applicant for the purpose of conducting business as a peddler, solicitor or transient merchant.
    \item The applicant is found to have a bad business reputation. Evidence of a bad business reputation shall include, but not be limited to, the existence of more than three complaints against the applicant with the Better Business Bureau, the Attorney General’s Office, or other similar business or consumer rights office or agency, within the preceding 12 months, or three complaints filed against the applicant within the preceding five years.
    \item The applicant is included on any state or national registry for sexual or predatory offenders.
\end{enumerate}

\section{License Suspension}
\subsection{Generally}
Any license issued under this section may be suspended at the discretion of the Clerk Treasurer or Crookston Police Chief for violation of any of the following:
\begin{enumerate}[{\indent}1)]
    \item Fraud, misrepresentation or incorrect statements on the appiication form.
    \item Fraud, misrepresentation or false statements made during the course of the licensed activity.
    \item Convicti0n of any offense for which granting of a license could have been denied under SEC.117.04.
    \item Violation of any provision of this chapter.
\end{enumerate}
\subsection{Multiple Persons under One License}
The suspension of any license issued for the purpose of authorizing multiple persons to conduct business as peddlers or transient merchants on behalf of the licensee shall serve as a suspension of each authorized person's authority to conduct business as a peddler or transient merchant on behalf of the licensee whose license is suspended or revoked.
\subsection{Notice}
Prior to suspending any license issued under this chapter, the city shall provide the license holder with notice of the alleged violations and inform the licensee of his or her right to a hearing on the alleged violation.
\subsection{Appeals}
Any person whose license is suspended under this section shall have the right to an administrative appeal under SEC. 30.03 and, thereafter, the right to appeal the decision in court.

\section{License Revocation}
\subsection{Generally}
Any license suspended under SEC. 117.05 may be revoked at the discretion of the City Council for violation of any of the following;
\begin{enumerate}[{\indent}1)]
    \item Fraud, misrepresentation or incorrect statements on the appiication form.
    \item Fraud, misrepresentation or false statements made during the course of the licensed activity.
    \item Conviction of any offense for which granting of a license could have been denied under SEC.117.04.
    \item Violation of any provision of this chapter.
\end{enumerate}
\subsection{Multiple Persons under One License}
The revocation of any license issued for the purpose of authorizing multiple persons to conduct business as peddlers or transient merchants on behalf of the licensee shall serve as a revocation of each authorized person’s authority to conduct business as a peddler or transient merchant on behalf of the licensee whose license is revoked.
\subsection{Notice}
Prior to revoking any license issued under this chapter, the city shall provide the license holder with written notice of the alleged violations and inform the licensee of his or her right to a hearing on the alleged violation. Notice shall be delivered in person or by mail to the permanent residential address listed on the license application, or if no residential address is listed, to the business address provided on the license application.
\subsection{Public Hearing}
Upon receiving the notice provided in division (C) of this section, the licensee shall have the right to request a public hearing. If no request for a hearing is received by the Clerk-Treasurer within ten regular business days following the service of the notice, the city may proceed with the revocation. For the purpose of mailed notices, service shall be considered complete as of the date the notice is placed in the mail. If a public hearing is requested within the stated timeframe, a hearing shall be scheduled within 20 days from the date of the request. Within three regular business days of the hearing, the City Council shall notify the licensee of its decision.
\subsection{Appeals}
Any person whose license is revoked under this section shall have the right to appeal that decision in court.\\
\emph{Penalty, see SEC. 110.99}

\section{License Transferability}
No license issued under this chapter shall be transferred to any person other than the person to whom the license was issued.\\
\emph{Penalty, see SEC. 110.99}

\section{Registration}
All solicitors, and any person exempt from the licensing requirements of this chapter under SEC. 117.03, shall be required to register with the city. Registration shall be made on the same form required for a license application, but no fee shall be required. Immediately upon completion of the registration form, the Clerk-Treasurer shall issue to the registrant a Certificate of registration as proof of the registration. Certificates of registration shall be nontransferable.\\
\emph{Penalty, see SEC. 110.99}

\section{Prohibited Activities}
No peddler, solicitor or transient merchant shall conduct business in any of the following manners:
\begin{enumerate}[{\indent}A)]
    \item Calling attention to his or her business or items to be sold by means of blowing any horn or whistle, ringing any bell, crying out, or by any other noise, so as to be unreasonably audible within an enclosed structure.
    \item Obstructing the free flow of either vehicular or pedestrian traffic on any street, alley, sidewalk or other public right-of-way.
    \item Conducting business in a way as to create a threat to the health, safety and welfare of any individual or the general public.
    \item Conducting business before 8:00 am. or after 8:00 pm.
    \item Failing to provide proof of license or registration, and identification, when requested; or using the license or registration of another person.
    \item Making any false or misleading statements about the product or service being sold, including untrue statements of endorsement. No peddler, solicitor or transient merchant shall claim to have the endorsement of the city solely based on the city having issued a license or certificate of registration to that person.
    \item Remaining on the property of another when requested to leave, or to otherwise conduct business in a manner a reasonable person would find obscene, threatening, intimidating or abusive.\\
        \emph{Penalty, see SEC. 110.99}
    \item Violations of other non-traffic related city ordinances, state statutes, or federal law.
    \item Selling products not contained in the manufacturer's original packaging.
\end{enumerate}

\section{Exclusion by Placard}
No peddler, solicitor or transient merchant, unless invited to do so by the property owner or tenant, shall enter the property of another for the purpose of conducting business as a peddler, solicitor or transient merchant when the property is marked with a sign or placard at least four inches long and four inches wide with print of at least 48 point in size stating “No Peddlers, Solicitors or Transient Merchants," or “Peddlers, Solicitors, and Transient Merchants Prohibited," or other comparable statement. No person other than the property owner or tenant shall remove, deface or otherwise tamper with any sign or placard under this section.\\
\emph{Penalty, see SEC. 110.99}

\section{Aggressive Solicitation}
\subsection{Policy}
As an aid in the interpretation and enforcement of this section, the City Council finds that:
\begin{enumerate}[{\indent}1)]
    \item Aggressive solicitation is disturbing and disruptive to residents and businesses and contributes to the loss of access to and enjoyment of public places and to a sense of fear, intimidation and disorder.
    \item Aggressive solicitation may include approaching or following pedestrians, repetitive soliciting despite refusals, the use of abusive or profane language to cause fear and intimidation, unwanted physical contact, or the intentional blocking of pedestrian and vehicular traffic.
    \item The presence of individuals who solicit money from persons in places that are confined, difficult to avoid, or where a person might find it necessary to wait, is especially troublesome because of the enhanced fear of crime.
    \item This section is intended to protect citizens from the disruption, fear and intimidation accompanying certain kinds of solicitation, and not to limit constitutionally protected activity.\end{enumerate}
\subsection{Definitions}
For the purpose of this section, the following definitions shall apply, unless the context clearly indicates or requires a different meaning.
\begin{description}
    \item[SOLICITATION], as used in this section, means any plea made in person where:
        \begin{enumerate}[{\indent}a)]
            \item A person by vocal appeal requests an immediate donation of money or other item from another person; or
            \item A person verbally offers or actively provides an item or service of little or no value to another in exchange for a donation, under circumstances where a reasonable person would understand that the transaction is in substance a donation.
        \end{enumerate}
    However, solicitation shall not include the act of passively standing, sitting, or engaging in a performance of art with a sign or other indication that a donation is being sought, without any vocal request other than in response to an inquiry by another person.
    \item[CONVENIENCE STORE], as used in this section, means a retail establishment offering for sale prepackaged food products, household items, and other goods commonly associated with them, with a gross floor area of less than seven thousand five hundred (7,500) square feet.
    \item[PUBLIC ENTERTAINMENT VENUE] means a place that is open to the public (whether or not upon payment of a fee for admission and whether or not the management reserves the right to exclude individual members of the public) for entertainment but does not include a shop.  The term includes, but is not limited to, cinemas, theatres, concert halls, electronic games centers, indoor sports centers (including a bowling alley), art galleries, museums, and premises upon which any display or exhibition promoted as such is conducted.
    \item[SHOP] means premises used for the sale or displaying or offering for sale of goods or food to a member of the public, whether on a wholesale or retail basis and includes a warehouse (other than where the premises is temporarily used as a public entertainment venue).
\end{description}
\subsection{Prohibitions}
\subsubsection{}
It shall be unlawful in a public place to engage in an act of solicitation when the person being solicited is present at any of the following locations:
\begin{enumerate}[{\indent}a)]
    \item In a restroom.
    \item At a bus stop or shelter.
    \item At or within ten (10) feet in any direction from a crosswalk.
    \item In any public transportation vehicle or public transportation facility.
    \item In a vehicle which is parked or stopped on a public street or alley.
    \item In a sidewalk café or an outdoor seating area of a restaurant.
    \item In a line waiting to be admitted to a commercial or government establishment.
    \item Within eighty (80) feet in any direction from an automatic teller machine or entrance to a bank, other financial institution, or check cashing business.
    \item On any park land, or in any park, playground, or public entertainment venue, including within fifty (50) feet of entry ways or exits thereto.
    \item At or within ten (10) feet in any direction of the property on which is located a gasoline filling station.
    \item At or within ten (10) feet in any direction of the property on which is located a liquor store, including any establishment with an off-sale license under Chapter 111 of this code, and not including any establishment with an on-sale license.
    \item At or within ten (10) feet in any direction of the property on which a convenience store is located.
    \item In the Crookston Sports Center, including within fifty (50) feet of entry ways or exits thereto.
\end{enumerate}
\subsubsection{}
It shall be unlawful in a public place to engage in an act of solicitation in a manner that incorporates any of the following methods:
\begin{enumerate}[{\indent}a)]
    \item Intentionally touching or causing physical contact with the solicited person without that person’s consent.
    \item Intentionally blocking the path of the solicited person, or the entrance to any building or vehicle.
    \item Following behind, ahead or alongside a person who walks away from the solicitor after being solicited, with the intent to intimidate or continue solicitation.
    \item Using obscene, profane, or abusive language or gestures toward the solicited person.
    \item Approaching the solicited person in a manner that:
        \begin{enumerate}[{\indent}i)]
            \item Is intended to or is likely to cause a reasonable person to fear imminent bodily harm or the commission of a criminal act upon property in the person’s possession; or
            \item Is intended to or is likely to intimidate a reasonable person into responding affirmatively to the solicitation.
        \end{enumerate}
    \item Solicitation while under the influence of alcohol or drugs.
    \item Soliciting in a group of two (2) or more persons.
\end{enumerate}
\subsubsection{}
It shall be unlawful in a public place to engage in an act of solicitation on any day after sunset, or before sunrise.
\subsection{Penalties}
Each act of solicitation prohibited by this section shall constitute a separate violation of this section.  Each violation shall be punishable as a misdemeanor.
\subsection{Severance}
If any section, sentence, clause, or phrase of this law is held invalid or unconstitutional by any court of competent jurisdiction, it shall in no way effect the validity of any remaining portions of this law.
