\chapter*{Chapter 114: \\
	Bed and Breakfast Inns and Other Rentals}
    \addstarredchapter{Chapter 114: Bed and Breakfast Inns and Other Rentals}
    \minitoc
    \pagebreak

\begin{center}
    \emph{\textbf{\LARGE{BED AND BREAKFAST INNS}}}
\end{center}

\section{Definition}
The term \emph{BED AND BREAKFAST INN} has the meaning set forth in city code SEC. 152.003.\\
\emph{(‘83 Code, SEC. 6.73, Subd. 3)  (Ord. 38, 2nd Series, effective 9-16-86)}
\section{License Required}
It is unlawful for any person to operate or maintain, or to allow the operation or maintaining upon property which he or she owns or controls, a bed and breakfast inn without having a license therefor from the city.\\
\emph{(‘83 Code, SEC. 6.73, Subd. 1)  (Ord. 38, 2nd Series, effective 9-16-86)  Penalty, see SEC. 110.99}\\
\emph{\textbf{Cross-reference:} On-sale wine license not required for bed and breakfast facility, see SEC. 111.103}
\section{Restriction}
No license shall be granted for any person to operate or maintain a bed and breakfast inn contrary to any zoning provision of the city code, or other law.\\
\emph{(‘83 Code, SEC. 6.73, Subd. 2)  (Ord. 38, 2nd Series, effective 9-16-86)}

\begin{center}
    \emph{\textbf{\LARGE{RENTAL HOUSING}}}
\end{center}

\setcounter{section}{14}
\section{License Required}
It is unlawful for any person as owner, landlord, agent or manager within the city to rent or cause to be rented any dwelling unit (as defined in city code Chapter 152) without first having obtained for the dwelling unit a license or temporary certificate from the city. It is also unlawful for any person to occupy any dwelling unit unless the unit has a license or temporary certificate from the city.\\
\emph{(‘83 Code, SEC. 6.34, Subd. 1)  Penalty, see SEC. 110.99}
\section{Temporary Certificate}
\subsection{}
Upon receipt of a completed application for a license, with tender of any appropriate license and inspection fee, the Rental Licensing Official may issue a temporary certificate indicating that a license has been applied for, and that the license will be issued or denied after the dwelling unit has been inspected for compliance with the applicable laws and regulations.  A temporary certificate authorizes continued occupancy of the dwelling unit pending the issuance or denial of the applied for license.  Dwelling units that are converted to rental usage after the effective date of the license requirement provided for in this subchapter are required to make application for a temporary certificate prior to occupancy.\\
\emph{(‘83 Code, SEC. 6.34, Subd. 2)}
\subsection{}
Temporary certificates or licenses issued under this subchapter will expire on December 31 of each year.\\
\emph{(‘83 Code, SEC. 6.34, Subd. 4)}
\section{Code Compliance}
No license may be issued unless each dwelling unit for which it is issued meets all minimum applicable housing, building, fire and safety regulations.\\
\emph{(‘83 Code, SEC. 6.34, Subd. 3)}
\section{Display of License and Certificate}
Licenses under this subchapter must be prominently and publicly displayed on or in the dwelling unit or public area of the structure containing the dwelling.\\
\emph{(‘83 Code, SEC. 6.34, Subd. 5)  Penalty, see SEC. 110.99}
