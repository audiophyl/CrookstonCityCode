\chapter*{Chapter 71: \\
	Traffic Rules}
    \addstarredchapter{Chapter 71: Traffic Rules}
    \minitoc
    \pagebreak

\subchapter{GENERAL TRAFFIC RULES}

\section{Driving Contrary to Lane Restrictions or Posted Prohibitions}
It is a misdemeanor for any person to drive a vehicle contrary to lane restrictions or prohibitions painted on any street, or contrary to sign-posted, fenced, or barricaded restrictions or prohibitions.\footnote{(‘83 Code, SEC. 7.04, Subd. 3) Penalty, see SEC. 70.99}
\section{Load Limits; Operation in Violation of}
The City Engineer may from time to time impose upon vehicular traffic on any part or all of the streets the load limits as may be necessary or desirable. The limits, and the specific extent or weight to which loads are limited, shall be clearly and legibly sign-posted thereon. It is a misdemeanor for any person to operate a vehicle on any street in violation of the limitation so posted.\footnote{(‘83 Code, SEC. 7.10) Penalty, see SEC. 70.99}
\section{Truck Routes}
It is unlawful for any person to drive a tractor, agricultural implement, truck over 9,000 pounds gross vehicle weight, truck-trailer, tractor-trailer or truck tractor in through traffic, upon any street except those which have been designated and sign-posted as truck routes.  For the purpose of this title, \textbf{THROUGH TRAFFIC} means originating without the city and with a destination without the city, as distinguished from \textbf{LOCAL TRAFFIC} which means traffic either originating or having a destination within the city.\footnote{(‘83 Code, SEC. 8.02) Penalty, see SEC. 70.99}

\section{Driving on Sidewalk, Walkway or Bicycle Trail Prohibited}
It is unlawful for any person to drive or operate a motorized vehicle on any public sidewalk or public property designated for use as a pedestrian walkway or bicycle trail, except when crossing the same for ingress and egress to private property lying on the other side thereof.\footnote{(‘83 Code, SEC. 7.13) Penalty, see SEC. 70.99}
\section{Exhibition Driving}
\subsection{}
It is prima facie evidence of exhibition driving when a motor vehicle stops, starts, accelerates, decelerates, or turns at an unnecessary rate of speed so as to cause tires to squeal, gears to grind, soil to be thrown, engine backfire, fishtailing or skidding, or, as to two-wheeled or three-wheeled motor vehicles, the front wheel to lose contact with the ground or roadway surface.
\subsection{}
It is a misdemeanor for any person to do any exhibition driving on any street, parking lot, or other public or private property, except when an emergency creates necessity for the operation to prevent injury to persons or damage to property; provided, that this section shall not apply to driving on a racetrack.  For purposes of this section, a \textbf{RACETRACK} means any track or premises whereon motorized vehicles, horses, dogs, or other animals or fowl legally compete in a race or timed contest for an audience, the members of which have directly or indirectly paid a consideration for admission.\footnote{(‘83 Code, SEC. 8.03) Penalty, see SEC. 70.99}\\

\subchapter{RECREATIONAL MOTOR VEHICLES}

\setcounter{section}{14}
\section{Definitions}
For the purpose of this subchapter, the following definitions shall apply unless the context clearly indicates or requires a different meaning.
\begin{description}
    \item[MOTORIZED BICYCLE] means a bicycle that is propelled by a motor of a piston displacement capacity of 50 cubic centimeters or less, and a maximum of two brake horsepower, which is capable of a maximum speed of not more than 30 mph on a flat surface with not more than one percent grade in any direction when the motor is engaged. "Motorized bicycle" includes an electric-assisted bicycle as defined in Minnesota Statutes, Section 169.01, Subd. 4b.
    \item[ALL-TERRAIN VEHICLE or ATV] means a motorized flotation-tired vehicle of not less than three low pressure tires, but not more than six tires, that is limited in engine displacement of less than 960 cubic centimeters and includes a class 1 all-terrain vehicle and class 2 all-terrain vehicle.
    \item[CLASS 1 ALL-TERRAIN VEHICLE] means an all-terrain vehicle that has a total dry weight of less than 1,000 pounds.
    \item[CLASS 2 ALL-TERRAIN VEHICLE] means an all-terrain vehicie that has a total dry weight of 1,000 to 1,800 pounds.
    \item[SNOWMOBILE] means a self‑propelled vehicle designed for travel on snow or ice steered by skis or runners.
    \item[OFF-ROAD VEHICLE] means a motor‑driven recreational vehicle capable of cross‑country travel on natural terrain without benefit of a road or trail.
    \item[RECREATIONAL MOTOR VEHICLE] means any self‑propelled vehicle and any vehicle propelled or drawn by a self‑propelled vehicle used for recreational purposes, including, but not limited to, a snowmobile, all‑terrain vehicle or off‑road vehicle.
    \item[OWNER] means a person, other than a lien holder, having a property interest in, or title to, a recreational motor vehicle, who is entitled to the use or possession thereof.
    \item[OPERATE] means to ride in or on and have control of a recreational motor vehicle.
    \item[OPERATOR] means the person who operates or is in actual physical control of a recreational motor vehicle.
\end{description}

\section{Recreational Motor Vehicle Operating Restrictions\footnote{(‘83 Code, SEC. 8.04, Subd. 2) (Am. Ord. 132, passed 6-9-98) Penalty, see SEC. 70.99}}
It is unlawful for any person to operate a recreational motor vehicle as follows:
\begin{enumerate}[{\indent}A)]
    \item On a public sidewalk or walkway provided or used for pedestrian travel.
    \item On private property of another without lawful authority or written permission of the owner or occupant.
    \item Except for snowmobiles on sign‑posted snowmobile trails, on any lands owned or occupied by a public body or on frozen waters, including, but not limited to, dikes, school grounds, park property, playgrounds, recreational areas, private roads, platted but unimproved roads, utility easements, public trails and golf courses.
    \item While the operator is under the influence of liquor or narcotics, or habit-forming drugs.
    \item At a rate of speed greater than reasonable or proper under all of the surrounding circumstances.
    \item In a careless, reckless or negligent manner so as to endanger the person or property of another or cause injury or damage thereto.
    \item Towing any person or thing on a public street or highway except through the use of a rigid tow bard attached to the rear of an automobile.
    \item At a speed greater than 10 mph when within 100 feet of any lakeshore, except in channels, or of a fisherman, ice house, skating rink, or sliding area, nor where the operation would conflict with the lawful use of property or would endanger other persons or property.
    \item In a manner so as to create a loud, unnecessary or unusual noise, which disturbs, annoys or interferes with the peace and quiet of other persons.
    \item Chasing, running over, or killing any animal, wild or domestic.
    \item During the hours between 11:00 o'clock P.M. of one day and 7:00 o'clock A.M. of the day next following.
\end{enumerate}

\section{Additional Operating Regulations for Snowmobiles}
\subsection{}
It is unlawful for any person to operate a snowmobile upon the roadway, shoulder or inside bank or slope of any State of County Highway within the City limits of Crookston. Operation in the ditch or on the outside bank within the right-of-way of any State or County Highway within the City of Crookston except interstate highways or freeways is permitted in conformance with State law, unless the roadway directly abuts a public sidewalk or walkway or property used for private purposes. It is unlawful for any person to operate a snowmobile on the roadway of any local street except for travel directly to and from the City limits, a gas station and the place where it is principally garaged/stored and then only on the right-hand side of such street or highway and in the same direction as the local traffic on the nearest lane of the roadway adjacent thereto. Operation must be in conformance to all applicable state and local traffic laws.
\subsection{}
A snowmobile may make a direct crossing of a street or highway except an interstate highway or freeway, provided:
\begin{enumerate}
    \item The crossing is made at an angle of approximately 90 degrees to the direction of the street or highway and at a place where no obstruction prevents a quick and safe crossing.
    \item The snowmobile is brought to a complete stop before crossing the shoulder or main traveled way of the highway.
    \item The driver yields the right of way to all oncoming traffic which constitutes an immediate hazard.
    \item In crossing a divided street or highway, the crossing is made only at an intersection of such street or highway with another public street or highway.
    \item If the crossing is made between the hours of one‑half hour after sunset to one‑half hour before sunrise or in conditions of reduced visibility, only if both front and rear lights are on.
\end{enumerate}
\subsection{}
No snowmobile shall enter any intersection without making a complete stop. The operator shall then yield the right of way to any vehicles or pedestrians which constitute an immediate hazard.
\subsection{}
Notwithstanding any prohibition in this Section, a snowmobile may be operated on a public thoroughfare in an emergency during the period of time when, and at locations where, snow upon the roadway renders travel by automobile impractical.
\subsection{}
No person under fourteen (14) years of age shall operate on streets or highways or make a direct crossing of a street or highway as the operator of a snowmobile. A person fourteen (14) years of age or older, but less than eighteen (18) years of age, may operate a snowmobile on streets or highways as permitted under this Section and make a direct crossing thereof only if he has in his immediate possession a valid snowmobile safety certificate issued by the Commissioner of Conservation as provided by Minnesota Statutes 1969, Section 84.86. It is unlawful for the owner of a snowmobile to permit the snowmobile to be operated contrary to the provisions of this Subparagraph.\footnote{(‘83 Code, SEC. 8.04, Subd. 4) (Am. Ord. 132, passed 6-9-98) Penalty, see SEC. 70.99}

\section{Owner Responsibility}
\subsection{}
It is unlawful for the owner of any recreational motor vehicle to permit its operation on private property without written permission of the owner or occupant, on City property without the written permission of the Council, or on other public property without written permission of the body in charge thereof. For purposes of this Section, the owner shall be conclusively presumed to have given such permission unless the recreational motor vehicle so operated shall have been reported stolen to a law enforcement agency.
\subsection{}
Every person leaving a recreational motor vehicle in a public place shall lock the ignition, remove the key and take the same with him.\footnote{(‘83 Code, SEC. 8.04, Subd. 3) (Am. Ord. 132, passed 6-9-98) Penalty, see SEC. 70.99}
\section{Snowmobile Equipment}
It is unlawful for any person to operate a snowmobile unless it is equipped with the following:
\subsection{}
Standard mufflers which are properly attached and in constant operation, and which reduce the noise of operation of the motor to the minimum necessary for operation. Mufflers shall comply with Regulation CONS. 55 which is hereby adopted by reference as it existed on September 1, 1970. No person shall use a muffler cut­out, by‑pass, straight pipe or similar device on a snowmobile motor, and the exhaust system shall not emit or produce a sharp popping or crackling sound.
\subsection{}
Brakes adequate to control the movement of and to stop and hold the snowmobile under any conditions of the operation.
\subsection{}
A safety or so‑called "deadman" throttle in operating condition, so that when pressure is removed from the accelerator or throttle, the motor is disengaged from the driving track.
\subsection{}
At least one clear lamp attached to the front, with sufficient intensity to reveal persons and vehicles at a distance of at least 100 feet ahead during the hours of darkness under normal atmospheric conditions. Such head lamp shall be so aimed that glaring rays are not projected in to the eyes of an oncoming vehicle operator. It shall also be equipped with at least one read tail lamp having a minimum candle power of sufficient intensity to exhibit a red light plainly visible from a distance of 500 feet to the rear during the hours of darkness under normal atmospheric conditions. The equipment to be in operating condition when the vehicle is operated between the hours of one-half hour after sunset to one half hour before sunrise or at times of reduced visibility.
\subsection{}
Reflective material at least sixteen inches on each side, forward of the handlebars, so as to reflect or beam light at a 90 degree angle.\footnote{(‘83 Code, SEC. 8.04, Subd. 5) (Am. Ord. 132, passed 6-12-98) Penalty, see SEC. 70.99}

\section{Additional ATV Operating Regulations}
It is unlawful for any person to operate an all-terrain vehicle on the roadway of any local street except for the purpose of snow removal if the ATV is equipped with a plow and except for travel directly to and from the City limits, a gas station and the place where it is principally garaged/stored and then only on the right-hand side of such street or highway and in the same direction as the local traffic on the nearest lane of the roadway adjacent thereto. Operation must be in conformance with all applicable state and local traffic laws.\\

\subchapter{SPECIAL VEHICLES}

\setcounter{section}{29}
\section{Special Vehicle Use on Roadway}
\subsection{Definitions}
For the purpose of this Section, the following definitions shall apply unless the context clearly indicates or requires a different meaning:
\begin{description}
    \item[Driver] The person driving and having physical control over the motorized golf cart or mini-truck and being the holder of a permit under this Section.
    \item[Motorized Golf Cart] A three or four wheel vehicle designed commercially for the primary purpose of golfing, powered by gas or electricity which does not exceed 20 mph.
    \item[Mini-Truck] As defined in Minnesota Statutes, Section 169.01, Subd. 40(a), a motor vehicle that has four wheels; is propelled by an electric motor with a rated power of 7,500 watts or less or an internal combustion engine with a piston displacement capacity of 660 cubic centimeters or less; has a total dry weight of 900 to 2,200 pounds; contains an enclosed cabin and a seat for the vehicle operator; commonly resembles a pickup truck or van, including a cargo area or bed located at the rear of the vehicle; and was not originally manufactured to meet federal motor vehicle safety standards required of motor vehicles in the Code of Federal Regulations, title 49, sections 571.101 to 571.404, and successor requirements.  A mini-truck does not include:  a neighborhood electric vehicle or a medium-speed electric vehicle as defined by Minnesota Statutes, Section 169.011, Subds. 47 and 39; or a motor vehicle that meets or exceeds the regulations in the Code of Federal Regulations, title 49, section 571.500, and successor requirements.
    \item[Designated Roadways] All City roadways.
\end{description}
\subsection{}
No person shall operate a motorized golf cart or mini-truck on designated roadways without obtaining a permit as provided in this Section; and, no person shall operate a motorized golf cart or mini-truck on roadways which are not designated roadways, alleys, and other public property.
\subsection{}
Every application for a permit shall be made on a form supplied by the City and shall contain all of the following information:
\begin{enumerate}[{\indent}1)]
    \item The name and address of the applicant.
    \item The nature of the applicant’s physical handicap, if any.
    \item Model name, make, and year and number of the motorized golf cart or mini-truck.
    \item Current driver’s license or reason for not having a current license.
    \item If applicant does not have a Minnesota driver’s license, applicant’s date of birth.
    \item Other information as the City may require.
\end{enumerate}
\subsection{}
All fees for applications, if any, and for permits under this Section shall be fixed and determined by the Council, adopted by resolution, and uniformly enforced.  The fees may, from time-to-time, be amended by the Council by resolution.  A copy of the resolution setting forth currently effective fees shall be kept on file in the office of the Clerk-Treasurer, and open to inspection during regular business hours.  For the purpose of fixing the fees, the Council may subdivide and categorize permits under a specific permit requirement, provided, that any subdivision or categorization shall be included in the resolution authorized by this subdivision.
\subsection{}
Permits shall be granted for a period of one (1) year and may be renewed annually January 1 to December 31.
\subsection{}
No permit shall be granted or renewed unless the following conditions are met:
\begin{enumerate}[{\indent}1)]
    \item The applicant must demonstrate that he or she currently holds or has held a valid Minnesota driver’s license to operate a mini-truck.
    \item The applicant must demonstrate that he or she is at least 16 years of age to operate a motorized golf car.
    \item The applicant may be required to submit a certificate signed by a physician that the applicant is able to safely operate a motorized golf cart on designated roadways.
    \item The applicant must provide evidence of insurance in compliance with the provisions of Minnesota Statutes concerning insurance coverage for the motorized golf cart or mini-truck.
    \item The applicant has not had his or her driver’s license revoked as a result of criminal proceedings.
\end{enumerate}
\subsection{}
Motorized golf carts and mini-trucks may only be operated on designated roadways, not state or federal highways, except to cross at designated intersections.
\subsection{}
Motorized golf carts may only be operated on designated roadways from sunrise to sunset.  They shall not be operated in inclement weather conditions or at any time when there is insufficient light to clearly see persons and vehicles on the roadway at a distance of 500 feet.
\subsection{}
Motorized golf carts shall display the slow-moving vehicle emblem provided for in Minnesota Statutes, Section 169.045, as it may be amended from time to time, when operated on designated roadways.
\subsection{}
Motorized golf carts and mini-trucks shall be equipped with a rearview mirror to provide the driver with adequate vision from behind as required by Minnesota Statutes, Section 169.70, as it may be amended from time to time.
\subsection{}
The operator of a motorized golf cart or mini-truck may cross any street or highway intersecting a designated roadway.
\subsection{}
Every person operating a motorized golf cart or a mini-truck under permit on designated roadways has all of the rights and duties applicable to the driver of any other vehicle under the provisions of Minnesota Statutes, Chapter 169, as it may be amended from time to time, except when these provisions cannot reasonably be applied to motorized golf carts or mini-trucks and except as otherwise specifically provided in Minnesota Statutes, Section 169.045(7), as it may be amended from time to time.
\subsection{}
The Council may suspend or revoke a permit granted under this Section upon a finding that the holder thereof has violated any of the provisions of this Section or Minnesota Statutes, Chapter 169, as it may be amended from time to time, or if there is evidence that the permit holder cannot safely operate the motorized golf cart or mini-truck on designated roadways.
\subsection{}
The number of occupants on the motorized golf cart or mini-truck may not exceed the design occupant load.
\subsection{}
Authorized City staff may operate City-owned motorized golf carts and mini-trucks without obtaining a permit within the City on City streets, sidewalks, trails, rights-of-way, and public property when conducting City business.
\subsection{}
Mini-truck equipment requirements:
\subsubsection{}
A mini-truck may be operated under permit on designated roadways if it is equipped with all of the following:
\begin{enumerate}[{\indent}a)]
    \item At least two headlamps.
    \item At least two tail lamps.
    \item Front and rear turn-signal lamps.
    \item An exterior mirror mounted on the driver’s side of the vehicle and either an exterior mirror mounted on the passenger’s side of the vehicle or an interior mirror.
    \item A windshield.
    \item A seat belt for the driver and front passenger.
    \item A parking brake.
\end{enumerate}
