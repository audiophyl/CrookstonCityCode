\chapter*{Chapter 118: \\
	Taxicabs}
    \addstarredchapter{Chapter 118: Taxicabs}
    \minitoc
    \pagebreak

\section{Definitions}
For the purpose of this chapter the following definitions shall apply, unless the context clearly indicates or requires a different meaning.
\begin{description}
    \item[DRIVER] The person driving and having physical control over a taxicab whether he or she is the licensee or in the employ of the licensed operator.
    \item[OPERATOR] A licensee owning or otherwise having control of one or more taxicabs.
    \item[TAXICAB] Any passenger conveyance being driven, on call or traversing a scheduled or unscheduled route for public use or hire upon payment of a fare or at regular fare rates, but not including such as are designed for mass transportation as buses, trains or streetcars.
\end{description}
\emph{(‘83 Code, SEC. 6.50, Subd. 1)}
\section{License Required}
It is unlawful for any person to drive or operate a taxicab without a license therefor from the city.\\
\emph{(‘83 Code, SEC. 6.50, Subd. 2)  Penalty, see SEC. 110.99}
\section{License Issuance and Display}
All licenses shall be issued for specific conveyances, except as otherwise herein provided.  License tags, including number and year for which issued, shall be plainly visible from the front of the conveyance.  Both sides of every licensed taxicab, when in use, shall be plainly and permanently marked as such with a painted sign or appurtenances showing the full or abbreviated name of the licensed operator.\\
\emph{(‘83 Code, SEC. 6.50, Subd. 3)  Penalty, see SEC. 110.99}
\section{Insurance Required}
Before a taxicab license is issued by the Council, and at all times effective during the licensed period, the licensee shall have and maintain public liability and bodily injury insurance in the amount of \$50,000 for any one person and \$100,000 for two or more persons injured in any one accident, as well as \$10,000 property damage insurance.\\
\emph{(‘83 Code, SEC. 6.50, Subd. 4)  Penalty, see SEC. 110.99}
\section{Schedule of Rates}
Each applicant shall file with the Clerk-Treasurer, before a taxicab license is issued or renewed, a schedule of proposed maximum rates to be charged by him or her during the licensed period for which the application is made. The schedule of proposed maximum rates, or a compromise schedule thereof, shall be approved by the Council before granting the license. The schedule shall be posted in a conspicuous place in the taxicab in full view of passengers riding therein. Nothing herein shall prevent a taxicab licensee from petitioning the Council for review of the rates during the licensed period, and the Council may likewise consider the petition and make new rates effective at any time.  No taxicab licensee shall charge rates in excess of maximum rates approved by the Council.\\
\emph{(‘83 Code, SEC. 6.50, Subd. 5)  Penalty, see SEC. 110.99}
\section{Mechanical Condition}
Before issuing a taxicab license, the applicant shall present to the Council a certificate signed by a competent and experienced mechanic showing that the taxicab conveyance is in good mechanical condition, that it is thoroughly safe for transportation of passengers and that it is in neat and clean condition.  The similar certificate may be required from time to time during the licensed period.  In lieu of the certificate the Council may accept the report of the Chief of Police relative thereto.\\
\emph{(‘83 Code, SEC. 6.50, Subd. 6)}
