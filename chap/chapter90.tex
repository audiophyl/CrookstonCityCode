\chapter*{Chapter 90: \\
	Abandoned, Unlicensed, Unregistered, or Inoperable Vehicles}
    \addstarredchapter{Chapter 90: Abandoned, Unlicensed, Unregistered, or Inoperale Vehicles}
    \minitoc
    \pagebreak

\section{Definitions}
For the purpose of this chapter the following definitions shall apply, unless the context clearly indicates or requires a different meaning.\footnote{(‘83 Code, SEC. 2.70, Subd. 1, A.)}
\begin{description}
\item[ABANDONED MOTOR VEHICLE] A motor vehicle as defined in M.S. Chapter 169.01, subdivision 3, as it may be amended from time to time, that has remained for a period of more than 48 hours on public property illegally, or more than four hours on public property if that property is properly posted; or has remained on private property for a period of time, as determined under M.S. Chapter 168B.04, Subd. 2, as it may be amended from time to time, without the consent of the person in control of the property; and is lacking vital component parts, or is in an inoperable condition such that it has no substantial potential further use consistent with its usual function unless it is kept in an enclosed garage or storage building.  It shall also mean any other motor vehicle defined as “unauthorized” in M.S. Chapter 168B.011, Subd. 4, as it may be amended from time to time, or a motor vehicle defined as “junk” in M.S. Chapter 168B.011, Subd. 3, as it may be defined from time to time, or a motor vehicle voluntarily surrendered by its owner to and accepted by the city.  A classic car or pioneer car, as defined in M.S. Chapter 168, as it may be amended from time to time, shall not be considered an \textbf{ABANDONED MOTOR VEHICLE} within the meaning of this section.  Vehicles on the premises of junk yards or automobile graveyards, which are licensed and maintained in accordance with the city code, shall not be considered \textbf{ABANDONED MOTOR VEHICLES} within the meaning of this section.
\item[ABANDONED, UNLICENSED, UNREGISTERED OR INOPERABLE VEHICLES] Defined in Chapter 94, Section 94.01 (E) (a and b).
\item[VITAL COMPONENT PARTS] Those parts of a motor vehicle that are essential to the mechanical functioning of the vehicle, including, but not limited to, the motor, drive train and wheels as defined in M.S. Chapter 168B.011, Subd. 14, as it may be amended from time to time.
\end{description}

\section{Abandoning Vehicle Unlawful}
It is unlawful for any person to abandon a motor vehicle on any public or private property without the consent of the person in control of the property.  For the purpose of this section, a \textbf{MOTOR VEHICLE} is as defined in M.S. Chapter 169.01, Subd. 3, as it may be amended from time to time.\footnote{(‘83 Code, SEC. 10.36)  Penalty, see SEC. 10.99}

\section{Impoundment; Disposal}
\subsection{Custody}
The city may take into custody and impound any abandoned motor vehicle.
\subsection{Custody}
Custody.  The city may take into custody and impound any unlicensed, unregistered, or inoperable motor vehicle after proper notice according to Chapter 94, Sec. 94.01 (E)(3)(b).
\subsection{Disposal}
The city will dispose of impounded vehicles in the manner indicated in SEC. 90.04 through SEC. 90.07 of this chapter.\footnote{(‘83 Code, SEC. 2.70, Subd. 1, B., C.)}

\section{Notice of Impoundment}
\subsection{}
When the city has impounded a motor vehicle under SEC. 90.03 of this chapter, the city shall give notice of the taking within five days.  The notice shall set forth the date and place of the taking, the year, make, model and serial number of the abandoned, unlicensed, unregistered, or inoperable motor vehicle, if the information can be reasonably obtained, and the place where the vehicle is being held, shall inform the owner and any lien holders of their right to reclaim the vehicle under SEC. 90.05 of this chapter, and shall state that failure of the owner or lien holder to exercise their right to reclaim the vehicle and contents be deemed a waiver by them of all rights, title and interest in the vehicle and a consent to the sale of the vehicle and contents at a public auction pursuant to SEC. 90.06 of this chapter.
\subsection{}
The notice shall be sent by mail to the registered owner and property owner, if any, of the abandoned motor vehicle and to all readily identifiable lien holders of record.\footnote{(‘83 Code, SEC. 2.70, Subd. 1, D.)}

\section{Right to Reclaim}
\subsection{}
The owner or any lien holder after providing proof of ownership, title, insurance and current license tabs of an abandoned motor vehicle, other than an unauthorized vehicle as defined in M.S. Chapter 168B.011, Subd. 4, as it may be amended from time to time, shall have a right to reclaim the vehicle from the city upon payment of all towing and storage charges resulting from taking the vehicle into custody within 15 days after the date of the notice required by this chapter.
\subsection{}
The owner or lien holder of any unauthorized motor vehicle, as defined in M.S. Chapter 168B.011, Subd. 4, as it may be amended from time to time, shall have the right to reclaim the vehicle upon payment of all towing and storage charges within 15 days after the date of the initial notice required by this chapter.
\subsection{}
Nothing in this chapter shall be construed to impair any lien of a garage keeper under the laws of this state, or the right of the lien holder to foreclose.  For the purposes of this section \textbf{GARAGE KEEPER} is an operator of a parking place or establishment, an operator of a motor vehicle storage facility, or an operator of an establishment for the servicing, repair or maintenance of motor vehicles.\footnote{(‘83 Code, SEC. 2.70, Subd. 1, E.)}

\section{Public Sale}
\subsection{}
An abandoned, unlicensed, unregistered, or inoperable motor vehicle and contents taken into custody and not reclaimed under SEC. 90.05 of this chapter shall be sold to the highest bidder at public auction or sale, following notification by mail to owner and any lien holder and one published notice published at least seven days prior to the auction or sale. The purchaser shall be given a receipt in a form prescribed by the Registrar of Motor Vehicles which shall be sufficient title to dispose of the vehicle. The receipt shall also entitle the purchaser to register the vehicle and receive a certificate of title, free and clear of all liens and claims of ownership.  Before such a vehicle is issued a new certificate of title, it must receive a motor vehicle safety check.
\subsection{}
From the proceeds of the sale of an abandoned motor vehicle, the city shall reimburse itself for the cost of towing, preserving and storing the vehicle, and all administrative, notice and publication costs incurred pursuant to this chapter.\footnote{(‘83 Code, SEC. 2.70, Subd. 1, F.)}

\section{Disposal of Vehicles Not Sold}
\subsection{}
Where no bid has been received for an abandoned motor vehicle, the city may dispose of it in accordance with this chapter. \footnote{(‘83 Code, SEC. 2.70, Subd. 1, G.)}
\subsection{Contracts and Disposal}
\subsubsection{}
The city may contract with any qualified person for collection, storage, incineration, volume reduction, transportation or other services necessary to prepare abandoned motor vehicles and other scrap metal for recycling or other methods of disposal.
\subsubsection{}
Where the city enters into a contract with a person duly licensed by the Minnesota Pollution Control Agency, the Agency shall review the contract to determine whether it conforms to the Agency’s plan for solid waste disposal.  A contract that does so conform may be approved by the Agency.  Where a contract has been approved, the Agency may reimburse the city for the costs incurred under the contract which have not been reimbursed.
\subsubsection{}
If the city utilizes its own equipment and personnel for disposal of the abandoned motor vehicle, it shall be entitled to reimbursement for the cost thereof along with its other costs as herein provided.\footnote{(‘83 Code, SEC. 2.70, Subd. 1, H.)}\footnote{\textbf{Cross-reference:} City employees who may not purchase excess or unclaimed property, see SEC. 36.04}
