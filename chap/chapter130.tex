\chapter*{Chapter 130: \\
	General Offenses}
    \addstarredchapter{Chapter 130: General Offenses}
    \minitoc
    \emph{\textbf{Cross-Reference:}\\
        {\indent}Chapter 120: Display, Sale, Storage, Possession, and Use of Fireworks}\\
    \pagebreak

\section{Use of Weapons and Fireworks}
\subsection{Acts Prohibited}
It is unlawful for any person to:
\begin{enumerate}[{\indent}1)]
    \item Recklessly handle or use a gun or other dangerous weapon or explosive so as to endanger the safety of another;
    \item Intentionally point a gun of any kind, capable of injuring or killing a human being and whether loaded or unloaded, at or toward another;
    \item Manufacture or sell for any unlawful purpose any weapon known as a slung-shot or sand club;
    \item Manufacture, transfer or possess metal knuckles or a switch blade knife opening automatically;
    \item Possess any other dangerous article or substance for the purpose of being used unlawfully as a weapon against another;
    \item Sell or have in his or her possession any device designed to silence or muffle the discharge of a firearm;
    \item Permit, as a parent or guardian, any child under 14 years of age to handle or use, outside of the parent’s or guardian’s presence, a firearm or air gun of any kind, or any ammunition or explosive;
    \item Furnish a minor under 18 years of age with a firearm, air gun, ammunition, or explosive without the written consent of his or her parent or guardian or the Police Department; or
    \item Possess, sell, transfer, or have in possession for sale or transfer, any weapon commonly known as a throwing star, nun chuck or sharp stud.  For the purposes of this division,  a “throwing star” means a circular metallic device with any number of points projecting from the edge; a “nun chuck” means a pair of wood sticks or metallic rods separated by chain links attached to one end of each such stick or rod; and a “sharp stud” means a circular piece of metal attached to a wrist band, glove, belt or other material which protrudes one-fourth inch, or more, from the material to which it is attached, and with the protruding portion pyramidal in shape, sharp or pointed.
\end{enumerate}
\emph{(Ord. 33, 2nd Series, effective 4-16-86)}
\subsection{Exception}
Nothing in division (A) of this section shall prohibit the possession of the articles therein mentioned if the purpose of the possession is for public exhibition by museums or collectors of art.
\subsection{Discharge of Firearms and Explosives}
It is unlawful for any person to fire or discharge any cannon, gun, pistol or other firearm, firecracker, sky rocket or other fireworks, air gun, air rifle, or other similar device commonly referred to as a BB gun.
\subsection{Exception}
Nothing in division (C) of this section shall apply to a display of fireworks by an organization or group of organizations authorized in writing by the Council, or to a peace officer in the discharge of his or her duty, or to a person in the lawful defense of his or her person or family.  This section shall not apply to the discharge of firearms in a range authorized in writing by the Council.
\subsection{Possession and Sale of Fireworks}
It is unlawful for any person to sell, have in possession for the purpose of sale, except as allowed in division (D) of this section, any firecrackers, sky rockets or other fireworks.
\subsection{Use of Bow and Arrow}
It is unlawful for any person to shoot a bow and arrow except in the physical education program in a school supervised by a member of its faculty, a community-wide supervised class or event specifically authorized by the Chief of Police, or a bow and arrow range or other places as authorized by the Council.\\
\emph{(‘83 Code, SEC. 10.10)  Penalty, see SEC. 130.99}
\section{Consumption and Possession of Alcoholic Beverages on Public Property and Private Parking Lots to which the Public has Access}
It is unlawful for any person to consume, or possess in an unsealed container, any alcoholic beverage as that term is defined in Chapter 111 on any city park, street, public property, or private parking lot to which the public has access, except on the premises when and where permission has been specifically granted or licensed by the Council.  Provided, that this section shall not apply to the possession of an unsealed container in a motor vehicle when the container is kept in the trunk of the vehicle if it is equipped with a trunk, or kept in some other area of the vehicle not normally occupied by the driver or passengers, if the motor vehicle is not equipped with a trunk.  For the purpose of this section, a utility or glove compartment shall be deemed to be within the area occupied by the driver or passengers.\\
\emph{(‘83 Code, SEC. 5.17)  (Ord. 54, 2nd Series, effective 11-26-88)  Penalty, see SEC. 130.99}
\section{Alcoholic Beverages in Certain Buildings and Grounds}
It is unlawful for any person to introduce upon, or have in his or her possession upon, or in, any public elementary or secondary school ground, or any public elementary or secondary school building, any alcoholic beverage, except for experiments in laboratories and except for those organizations who have been issued temporary licenses to sell alcoholic beverages, and for any person to possess alcoholic beverages as a result of a purchase from those organizations holding temporary licenses.\\
\emph{(‘83 Code, SEC. 5.18)  (Ord. 80, 2nd Series, effective 11-18-92)  Penalty, see SEC. 130.99}
\section{Curfew}
\subsection{Purpose and Findings}
The Council has determined that there has been an increase in juvenile violence, vandalism, gang activity and crime by persons under the age of 18 in the city.  Persons under the age of 18 are particularly susceptible by their lack of maturity and experience to participate in unlawful activities and to be victims of older perpetrators of crime.  The city has an obligation to provide for the protection of minors from each other and from other persons, for the enforcement of parental control over and responsibility for children, for the protection of the general public, and for the reduction of the incidents of juvenile criminal activities.  A curfew for those under the age of 18 is in the interest of public health, safety and general welfare and will help to attain the foregoing objectives.
\subsection{Definitions}
For the purpose of this section the following definitions shall apply, unless the context clearly indicates or requires a different meaning.
\begin{description}
    \item[CURFEW HOURS] 11:00 p.m. - 6:00 a.m. Sunday through Thursday and 12:01 a.m. - 6:00 a.m. Friday and Saturday.
    \item[EMERGENCY] An unforeseen combination of circumstances or the resulting state that calls for immediate action; the term includes, but is not limited to, a fire, a natural disaster, an automobile accident, or any situation requiring immediate action to prevent serious bodily injury or loss of life.
    \item[ESTABLISHMENT] Any privately owned place of business operated for a profit to which the public is invited, including, but not limited to, any place of amusement or entertainment.
    \item[GUARDIAN] A person who, under court order, is the guardian of the person of a minor, or a public or private agency with whom a minor has been placed by a court.
    \item[MINOR] Any person under 18 years of age.
    \item[OPERATOR] Any individual, firm, association, partnership, or corporation operating, managing or conducting any establishment.  The term includes the members or partners of an association or partnership and the officers of a corporation.
    \item[PARENT] A person who is a natural parent, adoptive parent or step-parent of another person, or at least 18 years of age and authorized by a parent or guardian to have the care and custody of a minor.
    \item[PUBLIC PLACE] Any place to which the public or a substantial group of the public has access including, but not limited to, streets, highways and the common areas of schools, hospitals, apartment houses, office buildings, transport facilities and shops.
    \item[REMAIN] To linger or stay, or fail to leave premises when requested to do so by a police officer or the owner, operator or other person in control of the premises.
    \item[SERIOUS BODILY INJURY] Bodily injury that creates a substantial risk of death or that causes death, serious permanent disfigurement or protracted loss or impairment of the function of any bodily member or organ.
\end{description}
\subsection{Restrictions}
\subsubsection{}
It is unlawful for any minor to remain in any public place or on the premises of any establishment within the city during curfew hours.
\subsubsection{}
It is unlawful for any parent or guardian of a minor to knowingly permit, or by insufficient control or supervision allow, the minor to remain in any public place or on the premises of any establishment within the city during curfew hours.  The term “knowingly” includes knowledge which a parent or guardian should reasonably be expected to have concerning the whereabouts of a minor in the legal custody of that parent or guardian.
\subsubsection{}
It is unlawful for any owner, operator or any employee of an establishment to knowingly allow a minor to remain upon the premises of the establishment during curfew hours.
\subsection{Exceptions}
\subsubsection{}
The following constitute valid exceptions to the operation of the curfew.
\begin{enumerate}[{\indent}a)]
    \item The minor was accompanied by the minor’s parent or guardian;
    \item The minor was on an errand at the direction of the minor’s parent or guardian, without any detour or stops;
    \item The minor was in a motor vehicle involved in interstate travel;
    \item The minor was engaged in an employment activity, or going to or returning home from an employment activity, without any detour or stop;
    \item The minor was involved in an emergency;
    \item The minor was on the sidewalk abutting the minor’s residence or abutting the residence of a neighbor located immediately adjacent to the minor’s residence if the neighbor did not complain to the Police Department about the minor’s presence;
    \item The minor was attending an official school, religious or other recreational activity supervised by adults and sponsored by the city, a civic organization or another similar entity that takes responsibility for the minor, or going to or returning home from, without any detour or stop, an official school, religious or other recreational activity supervised by adults and sponsored by the city, a civic organization or another similar entity that takes responsibility for the minor;
    \item The minor was exercising First Amendment rights protected by the United States Constitution, such as the free exercise of religion, freedom of speech, and the right of assembly; or
    \item The minor was married or had been married.
\end{enumerate}
\subsubsection{}
It is a defense to prosecution under this section that the owner, operator or employee of an establishment promptly notified the Police Department that a minor was present on the premises of the establishment during curfew hours and refused to leave.
\subsection{Enforcement}  Before taking any enforcement action under this section, a police officer must ask the apparent offender’s age and reason for being in the public place.  The officer shall not issue a citation or make an arrest under this section unless the officer reasonably believes that an offense has occurred and, based on any response and other circumstances, no defense in division (D) is present.\\
\emph{(‘83 Code, SEC. 10.12)  (Ord. 96, 2nd Series, effective 12-28-94)}
\section{Disorderly Conduct}
It is unlawful for any person, in a public or private place, knowing, or having reasonable grounds to know, that it will, or will tend to, alarm, anger or disturb others or provoke any assault or breach of the peace, to do the following:
\begin{enumerate}[{\indent}A)]
    \item Engage in brawling or fighting.
    \item Disturb an assembly or meeting, not unlawful in its character.
    \item Engage in offensive, obscene or abusive language or in boisterous and noisy conduct tending reasonably to arouse alarm, anger or resentment in others.
    \item Willfully and lewdly expose his or her person or the private parts thereof, or procure another to so expose himself; and any open or gross lewdness or lascivious behavior, or any act of public indecency.
    \item Whether or not posted with signs so prohibiting, voluntarily enter the waters of any river or public swimming pool at any time when said waters are not properly supervised by trained life-saving personnel in attendance for that purpose, or enter the waters without being garbed in a bathing suit sufficient to cover his or her person and equal to the standards generally adopted and accepted by the public.
    \item Urinate or defecate in a place other than:
        \begin{enumerate}[{\indent\indent}1)]
            \item If on public property then in a plumbing fixture provided for that purpose;
            \item If on the private property of another then in a plumbing fixture provided for that purpose; or
            \item If on private property not owned or controlled by another, then within a building.
        \end{enumerate}
    \item Cause the making or production of an unnecessary noise by shouting or by any other means or mechanism including the blowing of any automobile or other vehicle horn.
    \item Use a sound amplifier upon streets and public property without prior written permission from the city.
    \item Use a flash or spotlight in a manner so as to annoy or endanger others.
    \item Cause defacement, destruction, or otherwise damage to any premises or any property located thereon.
    \item Strew, scatter, litter, throw, dispose of or deposit any refuse, garbage, or rubbish unto any premises except into receptacles provided for the purpose.
    \item Enter any motor vehicle of another without the consent of the owner or operator.
    \item Fail or refuse to vacate or leave any premises after being requested or ordered, whether orally or in writing, to do so, by the owner, or person in charge thereof, or by any law enforcement agent or official; provided, however, that this provision shall not apply to any person who is owner or tenant of the premises involved nor to any law enforcement or other government official who may be present thereon at that time as part of his or her official duty, nor shall it include the spouse, children, employee or tenant of the owner or occupier.
\end{enumerate}
\emph{(‘83 Code, SEC. 10.11)  Penalty, see SEC. 130.99}
\section{Noisy Parties}
\subsection{}
It is unlawful for any person or persons to congregate on any private lands because of, or participate in, any party or gathering of people from which noise emanates of a sufficient volume or of the nature as to disturb the peace, quiet or repose of other persons. Any owner or person in lawful possession or control of the private lands who has knowledge of the disturbance and fails to immediately abate said disturbance shall be guilty of a violation of this section.
\subsection{}
It is unlawful for any person or persons to congregate on any private lands of another because of, or participate in, any party or gathering of people in the absence of the owner of the private lands being present, without first having obtained written permission from the landowner or other person in lawful possession of private lands. The written permission shall at all times be in the possession of one or more persons at the site of the congregation. The document containing the written permission must bear the signature of the landowner and date of the permitted use. Failure to display written permission upon request shall be considered prima facie evidence of an absence of permission from the owner.
\subsection{}
A violation of division (A) or (B) of this section shall give a peace officer the authority to order all persons present, other than persons identifying themselves as the owner or person in lawful possession or control of the land, to immediately disperse. Any person who shall refuse to leave after being ordered to do so by a peace officer shall be guilty of a violation of this section.\\
\emph{(‘83 Code, SEC. 10.17)  (Ord. 20, 2nd Series, effective 5-18-85)  Penalty, see SEC. 130.99}
\section{Trespass}
\subsection{Definitions}
For the purpose of this section the following definitions shall apply, unless the context clearly indicates or requires a different meaning.
\begin{description}
    \item[ALCOHOL] Beer, wine or liquor as defined in Chapter 111 of this code.
    \item[CONTROLLED SUBSTANCE] Has the same meaning given in Minnesota Statutes.
    \item[INVITATION] The landowner or other person in lawful possession of the land is then and there present and states that he or she has given his or her consent, endorsement, ratification or permission; or, that the landowner or other person in lawful possession of the land has executed a written document giving his or her consent, endorsement, ratification or permission for those present to be upon the land, which written document contains the signature of the owner or possessor of the land and the date of permitted entry, is at all times in the possession of a person present, and displayed to a peace officer immediately upon request.
    \item[MOTOR VEHICLE] Has the same meaning given in Minnesota Statutes.
\end{description}
\subsection{Unlawful Acts}
\subsubsection{}
It is unlawful for any person or persons to enter without invitation onto the private land of another for the purpose of consuming alcohol or a controlled substance.
\subsubsection{}
It is unlawful for any person or persons to bring a motor vehicle onto the private land of another without invitation for the purpose of facilitating the consumption of alcohol or a controlled substance.
\subsubsection{}
It is unlawful for any person to represent, orally or in writing, that he or she is the owner or in lawful possession of the land upon which entry is made, unless he or she is actually the owner or possessor.
\subsection{Purpose}
In determining the purpose of an entry without invitation by a person or motor vehicle, factors to be considered include but are not limited to the following:
\begin{enumerate}[{\indent}1)]
    \item Time of day; 
    \item Presence of containers intended to contain or containing alcohol; 
    \item Presence of equipment used to dispense alcoholic beverages; 
    \item Presence of paraphernalia containing identifiable residues of a controlled substance; 
    \item Noise level; 
    \item Lighting; 
    \item Identified physiological responses; and 
    \item Conduct of persons in the presence of a peace officer.
\end{enumerate}
\emph{(‘83 Code, SEC. 10.16)  Penalty, see SEC. 130.99}
\section{Exposure of Unused Container}
It is unlawful for any person, being the owner or in possession or control thereof, to permit an unused refrigerator, ice box, or other container, sufficiently large to retain any child and with doors which fasten automatically when closed, to expose the same accessible to children, without removing the doors, lids, hinges or latches.\\
\emph{(‘83 Code, SEC. 10.10)  Penalty, see SEC. 130.99}
\section{Radio and Television Interference}
It is unlawful for any person to maintain, use or operate any apparatus or device whether electrical, mechanical or of any other type, so as to cause interference with radio or television reception. This section shall not apply to electro-medical devices provided that they are equipped so far as reasonably possible with apparatus tending to reduce the interference.\\
\emph{(‘83 Code, SEC. 10.37)  Penalty, see SEC. 130.99}

\setcounter{section}{98}
\section{Penalty}
Every person violates a section, division or provision of this chapter when he or she performs an act thereby prohibited or declared unlawful, or fails to act when the failure is thereby prohibited or declared unlawful, or performs an act prohibited or declared unlawful or fails to act when the failure is prohibited or declared unlawful by a code adopted by reference by this chapter, and upon conviction thereof, shall be punished as for a misdemeanor except as otherwise stated in specific provisions hereof. The penalty which may be imposed for any crime which is a misdemeanor under this code, including Minnesota Statutes specifically adopted by reference, shall be a sentence of not more than 90 days or a fine of not more than \$1,000, or both. The costs of prosecution may be added. A separate offense shall be deemed committed upon each day during which a violation occurs or continues.\\
\emph{(‘83 Code, SEC. 10.99)}
