%\documentclass[code.tex]{subfiles}
%\begin{document}
\chapter*{Chapter 34: \newline
	Finance and Taxation}
\addcontentsline{toc}{chapter}{Chapter 34: Finance and Taxation}

\centerline{\textbf{\emph{\LARGE{ASSESSMENTS; FUNDS}}}}
\section{Prepayment of Assessments}
\subsection{}
After the adoption of an assessment roll pursuant to M.S. Chapter 429, as it may be amended from time to time, and before certification of the assessment roll to the County Auditor, the City Clerk-Treasurer, or other authorized official, is authorized and directed to accept total prepayment of the assessment.  As provided by law, such prepayment may be accepted only during the 30-day period following approval of the assessment roll.
\subsection{}
This section shall apply to all assessment rolls which, on the effective date hereof, have been adopted by the Council but not yet certified to the County Auditor, and to all assessment rolls subsequently adopted by the Council.\newline
\emph{(‘83 Code, SEC. 2.73)}
\section{Special Assessment Policy}
The Council may, by resolution, adopt, from time to time amend, or repeal a special assessment policy.\newline
\emph{(‘83 Code, SEC. 2.75)}
\section{Airport Fund}
\subsection{Establishment}
There shall be established in the treasury of the city an Airport Fund.
\subsection{Revenues and Expenditures}
Any and all receipts derived by the city from property acquired, established or maintained as an airport, or available for the landing and take-off of aircraft, shall be paid into the fund, and all disbursements and expenditures from the fund shall be made only upon resolution of the Council.  The fund shall be maintained by taxation, if necessary.\newline
\emph{(‘83 Code, SEC. 2.77)}
\section{Trust or Escrow Funds for Fire or Explosion Losses}
The city may use the proceeds from losses arising from fire or explosion of insured real property located in the city that are held in a trust or escrow account to secure, repair or demolish damaged or destroyed structures and clear the property in question, so that the structure and property are in compliance with local code requirements and applicable city code provisions.  Any unused portion of the retained proceeds shall be returned to the insured.  In addition, the regulatory and procedural provisions of M.S. §  65A.50, Trust or Escrow Accounts; Insured Real Property Fire or Explosion Loss Proceeds, as it may be amended from time to time, are hereby incorporated herein and adopted by reference, including the penalty provision thereof.\newline
\emph{(‘83 Code, SEC. 2.79)  (Ord. 124, 2nd Series, effective 5-16-98)}\newline


\centerline{\textbf{\emph{\LARGE{LODGING TAX}}}}
\setcounter{section}{14}
\section{Purpose}
The promotion of the city as a tourist or convention center would benefit the economy and recognition of the city.  The creation of a tax on lodging at hotels, motels, rooming houses, tourist courts and resorts for the purpose of providing funding for a convention and visitor’s bureau to promote the city’s tourism and convention industry is in the best interest of the city and its citizens.\newline
\emph{(Ord. 139, 2nd Series, passed 5-9-00)}
\section{Definitions}
For the purpose of this subchapter the following definitions shall apply, unless the context clearly indicates or requires a different meaning.
\begin{description}
\item[LODGING FACILITY] A hotel, motel, rooming house, tourist court, or resort, as those terms are commonly understood, where accommodations are furnished for consideration, other than the renting or leasing of such accommodations for a continuous period of 30 days or more.  The furnishing of rooms by any legally constituted religious, educational or nonprofit organization shall not constitute lodging for the purposes of this subchapter.
\item[OPERATOR] The person who is the proprietor of the “lodging facility,” whether in the capacity of owner, lessee, sublessee, licensee or any officer, agent or employee of the person.
\item[PERSON] Any individual, corporation, partnership, association, limited liability company or partnership, estate, receiver, trustee, executor, administrator, assignee, syndicate or any other combination of individuals.  Whenever the term \textbf{PERSON} is used in any provision of this section prescribing and imposing a penalty, the term as applied to a corporation, partnership, association, limited liability company or partnership, or any other combination of individuals, shall mean the officers or partners thereof as the case may be.\newline
\emph{(Ord. 139, 2nd Series, passed 5-9-00)}
\end{description}
\section{Imposition of Tax}
\subsection{}
Pursuant to the authority granted under M.S. § 469.190, as it may be amended from time to time, there is imposed a tax in the amount of 3\% on the gross receipts from the furnishing for consideration of accommodations at any lodging facility in the city.
\subsection{}
The tax imposed under this subchapter shall be paid by the individual occupying the accommodations through the operator of the lodging facility at the time the charge for the accommodations is paid.  The tax constitutes a debt owed to the city by the operator which is extinguished only upon payment of the tax to the city.\newline
\emph{(Ord. 139, 2nd Series, passed 5-9-00)}
\section{Collection}
\subsection{Operator's Duties}
Each operator shall collect the tax imposed under this subchapter at the time the rent charge for the accommodation is paid. The amount of the tax shall be separately stated from the rent so charged. The person paying the tax shall receive a receipt of payment from the operator.
\subsection{Reports}
On or before the twenty-fifth day of each month, each operator shall submit to the Clerk-Treasurer a report of the rental activities in the preceding calendar month. In the event the twenty-fifth day of the month falls on a Saturday, Sunday or legal holiday, the report shall be due on the next succeeding city business day. The report shall be on forms provided by the city and shall contain, at a minimum, the following:
\begin{enumerate}
\item Name, address, telephone number of the lodging facility.
\item Reporting period covered by the report.
\item Total gross amount of accommodation charges collected during reporting period.
\item The amount of lodging tax required to be collected and due during the reporting period.
\item Name, date of birth, and signature of the operator or the operator’s agent submitting the report and the address and telephone number of the operator or agent, if different from the lodging facility.
\item The total amount of any uncollectable accommodation rental charges.
\item Other information as the city from time to time may, in its discretion, require.
\end{enumerate}
\subsection{Payment of Tax}
Payment of the tax for the preceding calendar month shall be submitted by each operator to the city with the report provided for that reporting period.
\subsection{Alternative Accounting}
Upon written notice to the Clerk-Treasurer, an operator may, at the operator’s option, adopt any four-week accounting period other than a calendar month.  Submission of the required reports and tax payments shall not be later than 25 days after the end of any reporting period unless the twenty-fifth day falls on a Saturday, Sunday or legal holiday, in which case the report and payment shall be submitted no later than the next succeeding city business day.
\subsection{Uncollectable Charges}
The operator may off-set against the tax due in any reporting period the amount of any taxes imposed under this section previously paid as a result of any transaction which becomes uncollectable during such reporting period, but only in proportion to the portion of the amount which becomes uncollectable.
\subsection{Examination of Reports}
After a report is filed, the Clerk-Treasurer may make any reasonable examination of the records and accounts of the lodging facility for which the report is made which the Clerk-Treasurer deems necessary for determining the correctness of that report. The tax imposed on the basis of an examination shall be the tax paid. If the tax due is found to be greater than the tax paid, the operator submitting the report shall remit the difference to the city within ten days after receipt of written notice. The notice shall be given personally or sent by certified mail to the address shown on the report. If the tax paid is greater than the tax found to be due, the excess paid shall be refunded to the operator at the address shown on the report.\newline
\emph{(Ord. 139, 2nd Series, passed 5-9-00)}

\section{Refunds}
\subsection{}
Any operator may file a claim for a refund of taxes paid in any reporting period that exceed the amount actually due for that period. The claim shall be in writing directed to the Clerk-Treasurer and must be received by the Clerk-Treasurer no later than one year following the payment of the contested taxes.
\subsection{}
Upon receipt of the claim for refund, the Clerk-Treasurer shall determine the validity of the claim. The Clerk-Treasurer may approve the claim, deny the claim or make any other reasonable determination regarding the claim. The city shall refund any excess payments as determined by the Clerk-Treasurer. If no excess payments are found, the Clerk-Treasurer shall so notify the operator in writing.\newline
\emph{(Ord. 139, 2nd Series, passed 5-9-00)}

\section{Penalties}
\subsection{}
Any taxes not paid within 25 days following the close of a reporting period shall be subject to a penalty of 10\%.
\subsection{}
If an operator does not include the penalty with a late payment of the tax, the Clerk-Treasurer shall notify the operator in writing, served either personally or by certified mail, of the amount of penalty due. If the operator fails to pay the penalty within ten days of that notice, the Clerk-Treasurer may proceed to collect the penalty in the same manner as provided under this subchapter for the collection of unpaid taxes.
\subsection{}
If an operator refuses or fails to pay the tax imposed by this subchapter, including any penalties, within 30 days following the close of the reporting period for which the tax is due, the Clerk-Treasurer shall determine an estimate of the tax due. For the purposes of making the estimates, the operator shall grant the Clerk-Treasurer access to all relevant books and records relating to the lodging facility. The Clerk-Treasurer shall notify the operator personally or by certified mail of the amount due. Full payment of the amount determined by the Clerk-Treasurer shall be made within ten days of the notice.
\subsection{}
If an operator fails to pay a tax and/or penalty imposed under this subchapter within ten days of receipt of notice from the Clerk-Treasurer, the tax and/or penalty may be specially assessed against the property in the same manner as a special assessment for unpaid utility charges.
\subsection{}
As an alternative to a special assessment, the City Attorney shall have the express authority to commence any legal action or actions to collect the tax and/or penalty due and, in addition, all costs of the collection including, but not limited to, reasonable attorney’s fees.\newline
\emph{(Ord. 139, 2nd Series, passed 5-9-00)}

\section{Council Hearing}
\subsection{}
An operator aggrieved by any determination by or action(s) of the Clerk-Treasurer may request a hearing before the City Council.
\subsection{}
A request for a hearing shall be in writing and shall set forth the basis for the request.  The request shall be received by the Clerk-Treasurer not more than ten days following receipt by the operator of the notice or action giving rise to the request.  If no request for a hearing is received within the ten days, the determination of the Clerk-Treasurer shall be final.
\subsection{}
Upon receipt of a proper and timely request, a hearing shall be conducted at the next regular City Council meeting that is held at least 15 days after receipt of the request.  The operator shall receive written notice of the day, time and place of the hearing at least ten days in advance of the hearing date.
\subsection{}
The hearing shall be limited to the issues contained in the operator’s request.
\subsection{}
The City Council may affirm, deny or modify the determination of the Clerk-Treasurer.\newline
\emph{(Ord. 139, 2nd Series, passed 5-9-00)}

\section{Deposit of Revenue; Distribution}
\subsection{Separate Fund}
All revenue collected by the city pursuant to this subchapter shall be deposited in a separate fund created for the purpose until appropriately distributed in accordance with division (B) below of this section.
\subsection{Distribution}
\begin{enumerate}
\item 95\% of the gross proceeds from the tax collected pursuant to this section shall be used toward funding a local convention and tourism bureau for the purpose of marketing and promoting the city as a tourist and convention center.
\item The city shall retain up to 5\% of the gross proceeds, which may be transferred to the city’s general fund to defray the costs of administering and enforcing this subchapter.\newline
\emph{(Ord. 139, 2nd Series, passed 5-9-00)}
\end{enumerate}

\section{Violations}
In addition to any other penalties that may be provided by this chapter, any willful violation of any provision of this subchapter, or the failure to tender a required report or the tendering of a false report constitutes a misdemeanor.\newline
\emph{(Ord. 139, 2nd Series, passed 5-9-00)}


%\end{document}
