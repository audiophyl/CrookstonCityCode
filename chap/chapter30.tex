\chapter*{Chapter 30: \\
	City Council}
    \addstarredchapter{Chapter 30: City Council}
    \vfill
    \minitoc
    \pagebreak

\section{Authority and Purpose}
\index{CITY COUNCIL!Authority and Purpose}
Pursuant to authority granted by Charter, this chapter of the city code is enacted to set down for enforcement the government and good order of the city by and through the City Council.\footnote{(‘83 Code, SEC. 2.01)}

\section{Council Procedure}
\index{CITY COUNCIL!Council Procedure}
\subsection{}
The City Clerk-Treasurer shall prepare the following items:
\begin{enumerate}[{\indent}1)]
    \item An agenda for the forthcoming meeting;
    \item A compiled list of all claimants who have filed verified accounts claiming payment for goods or services rendered the city during the preceding month, the list to be called the “Claim Report” and bearing headings “Claimant,” “Purpose,” and “Amount;”
    \item A copy of all minutes to be considered; and
    \item Copies of the other proposals, communications, or other documents as the City Clerk-Treasurer deems necessary or proper for advance consideration by the Council.
\end{enumerate}
\subsection{}
The City Clerk-Treasurer shall cause to be mailed or delivered to each member of the Council copies of all documents.  Those items that are to appear on the Council agenda which are considered routine or about which no controversy or need for discussion can be foreseen by the Mayor and City Clerk-Treasurer may be placed on the consent agenda.  The Council may approve all actions on the consent agenda with one vote.  The motion to approve the consent agenda shall not be debated or discussed.  At the request of any individual member of the Council prior to a vote upon the consent agenda, an item shall be removed from the consent agenda and placed upon the regular agenda for debate or discussion.  Roberts’ Rules of Order (Newly Revised) shall govern all Council meetings as to procedural matters not set forth in the Charter or this code.
\subsection{}
The order of business at regular meetings shall be as follows:
\begin{enumerate}[{\indent}1)]
    \item CALL TO ORDER
    \item ROLL CALL
    \item CROOKSTON FORUM -- Individuals may address the Council about any item not contained on the regular agenda.   Maximum of 15 minutes is allotted for the Forum.  If the full 15 minutes are not needed for the Forum, the City Council will continue with the agenda.  The City Council will take no official action on items discussed at the Forum, with the exception of referral to staff or Commission for future report.
    \item PRESENTATIONS AND PUBLIC INFORMATION ANNOUNCEMENTS
    \item APPROVE AGENDA
    \item CONSENT AGENDA -- These items are considered to be routine and will be enacted by one motion.  There will be no separate discussion of these items unless a Council member or citizen so requests, in which event the item will be removed from the Consent Agenda and placed elsewhere on the agenda.
    \item PUBLIC HEARINGS
    \item REGULAR AGENDA
    \item REPORTS AND STAFF RECOMMENDATIONS
    \item ADJOURNMENT
\end{enumerate}
\subsection{}
Matters inappropriate for consideration at a meeting, or not in the order specified, shall not be considered except with the unanimous consent of the members of the Council, or scheduled public hearings or bid lettings at the time stated in the notice.  All claims for payment must be filed at or before 12:00 noon on the Wednesday preceding the regular Council meeting at which they are to be considered.\footnote{(‘83 Code, SEC. 2.02)  (Ord. 98, 2nd Series, effective 2-25-95)}

\section{Right to Administrative Appeal}
\index{CITY COUNCIL!Right to Administrative Appeal}
If any person shall be aggrieved by any administrative decision of the City Administrator or any other city official, or any board or commission not having within its structure an appellate procedure, the aggrieved person is entitled to a full hearing before the Council upon serving a written request therefore upon the Mayor and City Administrator at least ten days prior to any regular Council meeting. The request shall contain a general statement setting forth the administrative decision to be challenged by the appellant. At the hearing the appellant may present any evidence he or she deems pertinent to the appeal, but the city shall not be required to keep a verbatim record of the proceedings. The Mayor, or other officer presiding at the hearing, may, in the interest of justice or to comply with time requirements and on the Mayor’s own motion or the motion of the appellant, the City Administrator, or a member of the Council, adjourn the hearing to a more convenient time or place, but the time or place shall be fixed and determined before adjournment to avoid the necessity for formal notice of reconvening.\footnote{(‘83 Code, SEC. 2.04)}

\section{Rules of Procedure for Appeals and Other Hearings}
\index{CITY COUNCIL!Rules of Procedure for Appeals and Other Hearings}
The Council may adopt by resolution certain written rules of procedure to be followed in all administrative appeals and other hearings to be held before the Council or other bodies authorized to hold hearings and determine questions therein presented.  The rules of procedure shall be effective 30 days after adoption and shall be for the purpose of establishing and maintaining order and decorum in the proceedings.\footnote{(‘83 Code, SEC. 2.05)}

\section{Accounts, Claims or Demands}
\index{CITY COUNCIL!Accounts, Claims or Demands}
\subsection{Generally} Except as to an annual salary, fees of jurors or witnesses fixed by law, or wages or salaries of employees which have been fixed on an hourly, daily, weekly or monthly basis by the Council and which by law are authorized to be paid on a payroll basis, any account, claim or demand against the city which can be itemized in the ordinary course of business, the Council shall not audit or allow the claim until the person claiming payment, or his agent, reduces it to writing, in items, and signs a declaration to the effect that the account, claim or demand is just and correct and that no part of it has been paid.
\subsection{Discretionary Exception} The Council may, in its discretion, allow a claim prepared by the City Clerk-Treasurer prior to the declaration by the claimant if the declaration is made on the check by which the claim is paid.
\subsection{Form of Declaration} The declaration provided for in division (A) is sufficient in the following form: “I declare under the penalties of law that this account, claim or demand is just and correct and that no part of it has been paid. Signature of Claimant.”
\subsection{Form and Effect of Declaration on Check} The declaration provided for in division (B) shall be printed on the reverse side of the check, above the space for endorsement by the payee, as follows: “The undersigned payee, in endorsing this check declares that the same is received in payment of a just and correct claim against the city, and that no part of it has heretofore been paid.” When endorsed by the payee named in the check, the statement shall operate and shall be deemed sufficient as the required declaration of claim.\footnote{(‘83 Code, SEC. 2.07)}

\section{Salaries}
\index{CITY COUNCIL!Salaries}
\subsection{}
The Mayor shall receive an annual salary established by ordinance from time to time, which salary shall be paid to him or her in equal monthly installments.  In the absence or disability of the Mayor, the Vice-Chairperson of the Council, after having served in place of the Mayor a continuous period of 30 days, shall be entitled to receive the same amount of salary as the Mayor, for all service beyond the initial 30-day period.
\subsection{}
The Council members shall receive an annual salary established by ordinance from time to time, which salary shall be paid in equal monthly installments.  The Assistant Mayor shall not receive the Council member salary during any period for which he or she receives the Mayor salary as hereinabove provided.\footnote{(Ord. 58, 2nd Series, effective 1-1-90)}
\subsection{}
Nothing contained herein shall alter the rules set out in Section 2.07 of the City Charter regarding expenses.\footnote{(‘83 Code, SEC. 2.10)}

\section{City Seal}
\index{CITY COUNCIL!City Seal}
All contracts to which the city is a party may be sealed with the City Seal.  The seal shall be kept in the custody of the Clerk-Treasurer and affixed by the Clerk-Treasurer.  The official City Seal shall be a circular disc having engraved thereupon “CITY OF CROOKSTON” and other words, figures or emblems as the Council may, by resolution, designate.\footnote{(‘83 Code, SEC. 2.03)}

\section{Facsimile Signatures}
\index{CITY COUNCIL!Facsimile Signatures}
The Mayor and City Clerk-Treasurer are hereby authorized to request a depository of city funds to honor an order for payment when the instrument bears a facsimile of his or her signature, and to charge the same to the account designated thereon or upon which it is drawn, as effectively as though it were his or her manually written signature. The authority is granted only for the purpose of permitting the officers an economy of time and effort.\footnote{(‘83 Code, SEC. 2.06)}

\section{Interim Emergency Succession}
\index{CITY COUNCIL!Interim Emergency Succession}
\subsection{Purpose}
Due to the existing possibility of a nuclear attack or a natural disaster requiring a declaration of a state of emergency, it is found urgent and necessary to insure the continuity of duly elected and lawful leadership of the city to provide for the continuity of the government and the emergency interim succession of key governmental officials by providing a method for temporary emergency appointments to their offices.
\subsection{Succession to Local Offices}
In the event of a nuclear attack upon the United States, a natural disaster affecting the vicinity of the city, or a disaster in the nature of an accident or occurrence involving one or more members of the Council, the Mayor, Council and City Administrator shall be forthwith notified by any one of the persons and by any means available to gather at the City Hall.  In the event that safety or convenience dictate, an alternative place of meeting may be designated.  Those gathered shall proceed as follows:
\begin{enumerate}[{\indent}1)]
    \item By majority vote of those persons present, regardless of number, they shall elect a Chairperson and Secretary to preside and keep minutes, respectively.
    \item They shall review and record the specific facts relating to the nuclear attack or disaster and injuries to persons or damage to property already done, or the imminence thereof.
    \item They may, based on the facts, declare a state of emergency.
    \item By majority vote of those persons present, regardless of number, they shall fill all positions on the Council (including the office of Mayor) of those persons upon whom notice could not be served or who are unable to be present.
    \item The interim successors shall serve until the time as the duly elected official is again available and returns to his or her position, or the state of emergency has passed and a successor is designated and qualifies as required by law, whichever shall occur first.
\end{enumerate}
\subsection{Duties of the Interim Emergency Council}
The Interim Emergency Council shall exercise the powers and duties of their offices, and appoint other key government officials to serve during the emergency.\footnote{(‘83 Code, SEC. 2.09)}
