\chapter*{Chapter 113: \\
	Auctioneers, Junk Dealers and Pawnbrokers}
    \addstarredchapter{Chapter 113: Auctioneers, Junk Dealers and Pawnbrokers}
    \vfill
    \minitoc
    \pagebreak

\subchapter{AUCTIONEERS}

\section{License Required}
It is unlawful for any person to sell property at an auction without a license therefor from the city.\footnote{(‘83 Code, SEC. 6.56, Subd. 1)  Penalty, see SEC. 110.99}

\section{Exceptions}
This subchapter shall not apply to a person selling property owned by him or her for at least six months, nor to judicial sales or sales made under court order.\footnote{(‘83 Code, SEC. 6.56, Subd. 2)}

\subchapter{JUNK DEALERS}

\setcounter{section}{14}
\section{Definition}
The term \textbf{JUNK} as used in this subchapter means and includes, but is not limited to, scrap of all kinds such as metal, paper, rags and wood.\footnote{(‘83 Code, SEC. 6.54, Subd. 1)}

\section{License Required}
It is unlawful for any person to deal in junk without having a license therefor from the city.\footnote{(‘83 Code, SEC. 6.54, Subd. 2)  Penalty, see SEC. 110.99}

\section{Restriction}
No license shall be granted to any person for operation upon premises contrary to any zoning provision of the city code, or other law.\footnote{(‘83 Code, SEC. 6.54, Subd. 3)}

\subchapter{PAWNBROKERS}

\setcounter{section}{29}
\section{Definitions\footnote{(‘83 Code, SEC. 6.59, Subd. 1)}}
For the purpose of this subchapter the following definitions shall apply, unless the context clearly indicates or requires a different meaning.
\begin{description}
    \item[PAWN TRANSACTION] Any loan on the security of pledged goods or any purchase of pledged goods on the condition that the pledged goods are left with the pawnbroker and may be redeemed or repurchased by the seller for a fixed price within a fixed period of time.
    \item[PAWNBROKER] A person engaged in whole or in part in the business of lending money on the security of pledged goods left in pawn, or in the business of purchasing tangible personal property to be left in pawn on the condition that it may be redeemed or repurchased by the seller for a fixed price within a fixed period of time.
    \item[PAWNSHOP] The location at which or premises in which a pawnbroker regularly conducts business.
    \item[PLEDGED GOODS] Tangible personal property other than a chose in action, securities, bank drafts, or printed evidence of indebtedness, that are purchased by, deposited with, or otherwise actually delivered into the possession of a pawnbroker in connection with a pawn transaction.
\end{description}

\section{License required}
\subsection{}
It is unlawful for any person to engage in the business as a pawnbroker unless the person has a valid license authorizing engagement in the business. Any pawn transaction made without benefit of a license is void.
\subsection{}
A separate license is required for each place of business. The city may issue more than one license to a person if that person complies with this subchapter for each license.
\subsection{}
No expiration, revocation, suspension, or surrender of any license shall impair or affect the obligation of any pre-existing lawful contract between the licensee and any pledgor.
\subsection{}
The Chief of Police shall be notified by the city of any licensee whose license has expired or been surrendered, suspended, or revoked as provided by this subchapter.\footnote{(‘83 Code, SEC. 6.59, Subd. 2)  Penalty, see SEC. 110.99}

\section{Persons Disqualified; Unemployment Clearance Required\footnote{(‘83 Code, SEC. 6.59, Subd. 3)}}
\subsection{}
No license under this subchapter may be issued or renewed to a person who:
\begin{enumerate}[{\indent}1)]
    \item Is a minor; 
    \item Has been convicted of any crime related to the occupation of pawnbroker; or 
    \item Is not of good moral character or repute.
\end{enumerate}
\subsection{}
No license shall be granted, transferred, or renewed and shall be revoked if the Commissioner notifies the city that the licensee owes the state delinquent unemployment insurance contributions, reimbursements, or benefit overpayments.\footnote{(‘83 Code, SEC. 6.59, Subd. 4)}

\section{Change in Ownership}
Any change, directly or beneficially, in the ownership of any licensed pawnshop shall require the application for a new license and the new owner must satisfy all current eligibility requirements.\footnote{(‘83 Code, SEC. 6.59, Subd. 5)}

\section{Pawn Tickets}
\subsection{Entries of Pawn Tickets}
At the time of making the pawn or purchase transaction, the pawnbroker shall immediately and legibly record in English the following information by using ink or other indelible medium on forms or in a computerized record approved by the city:
\begin{enumerate}[{\indent}1)]
    \item A complete and accurate description of the property, including model and serial number if indicated on the property;
    \item The full name, residence address, residence telephone number, and date of birth of the pledgor or seller;
    \item The date and time of pawn or purchase transaction;
    \item The identification number and state of issue from one of the following forms of identification of the seller or pledgor: current valid Minnesota driver’s license; current valid Minnesota identification card; or current valid photo identification card issued by another state or Province of Canada;
    \item Description of the pledgor including approximate height, sex, and race;
    \item Amount advanced or paid;
    \item The maturity date of the pawn transaction and the amount due; and
    \item The monthly and annual interest rates, including all pawn fees and charges.
\end{enumerate}
\subsection{Printed Pawn Ticket\footnote{(‘83 Code, SEC. 6.59, Subd. 6)}}
The following shall be printed on all pawn tickets:
\begin{enumerate}[{\indent}1)]
    \item The statement that “Any personal property pledged to a pawnbroker within this state is subject to sale or disposal when there has been no payment made on the account for a period of not less than 60 days past the date of the pawn transaction, renewal, or extension; no further notice is necessary.  There is no obligation for the pledgor to redeem pledged goods”;
    \item The statement that “The pledgor of this item attests that it is not stolen, it has no liens or encumbrances against it, and the pledgor has the right to sell or pawn the item”;
    \item The statement that “This item is redeemable only by the pledgor to whom the receipt was issued, or any person identified in a written and notarized authorization to redeem the property identified in the receipt, or a person identified in writing by the pledgor at the time of the initial transaction and signed by the pledgor.” Written authorization for release of property to persons other than the original pledgor must be maintained along with the original transaction record; and
    \item A blank line for the pledgor’s signature.
\end{enumerate}

\section{Records; Prohibitions Concerning}
\subsection{}
The pledgor or seller shall sign a pawn ticket and receive an exact copy of the pawn ticket.
\subsection{}
The pawnbroker shall maintain on the premises a record of all transactions of pledged or purchased goods for a period of three years.  These records shall be a correct copy of the entries made of the pawn transactions.  A pawnbroker shall upon request provide to the appropriate law enforcement agency a complete record of pawn items.\footnote{(‘83 Code, SEC. 6.59, Subd. 7)}
\subsection{}
A pawnbroker and any clerk, agent, or employee of a pawnbroker shall not\footnote{(‘83 Code, SEC. 6.59, Subd. 10)  Penalty, see SEC. 110.99}:
\begin{enumerate}[{\indent}1)]
    \item Make any false entry in the records of pawn transactions;
    \item Falsify, obliterate, destroy, or remove from the place of business the records, books, or accounts relating to the licensee’s pawn transactions;
    \item Refuse to allow the appropriate law enforcement agency, the Attorney General, or any other duly authorized state or federal law enforcement officer to inspect the pawn records or any pawn goods in the person’s possession during the ordinary hours of business or other times acceptable to both parties;
    \item Fail to maintain a record of each pawn transaction for three years;
    \item Accept a pledge or purchase property from a person under the age of 18 years;
    \item Make any agreement requiring the personal liability of a pledgor or seller, or waiving any provision of this subchapter or providing for a maturity date less than one month after the date of the pawn transaction;
    \item Fail to return pledged goods to a pledgor or seller, or provide compensation as set forth in state statutes, upon payment of the full amount due the pawnbroker unless either the date of redemption is more than 60 days past the date of the pawn transaction, renewal, or extension and the pawnbroker has sold the pledged goods pursuant to state statutes, or the pledged goods have been taken into custody by a court or a law enforcement officer or agency;
    \item Sell or lease, or agree to sell or lease, pledged or purchased goods back to the pledgor or seller in the same, or a related, transaction;
    \item Sell or otherwise charge for insurance in connection with a pawn transaction; or
    \item Remove pledged goods from the pawnshop premises or other storage place approved by the city at any time before unredeemed, pledged goods are sold pursuant to statute.
\end{enumerate}

\section{Effect of Non-Redemption}
\subsection{}
A pledgor shall have no obligation to redeem pledged goods or make any payment on a pawn transaction.  Pledged goods not redeemed within at least 60 days of the date of the pawn transaction, renewal, or extension shall automatically be forfeited to the pawnbroker, and qualified right, title, and interest in and to the goods shall automatically vest in the pawnbroker.
\subsection{}
The pawnbroker’s right, title, and interest in the pledged goods under this division is qualified only by the pledgor’s right, while the pledged goods remain in possession of the pawnbroker and not sold to a third party, to redeem the goods by paying the loan plus fees and/or interest accrued up the date of redemption.
\subsection{}
A pawn transaction that involves holding only the title to property is subject to M.S. Chapter 168A or 336, as it may be amended from time to time.\footnote{(‘83 Code, SEC. 6.59, Subd. 8)}

\section{Permitted Charges}
\subsection{}
A pawnbroker may contract for and receive a pawnshop charge not to exceed 3\% per month of the principal amount advanced in the pawn transaction plus a reasonable fee for storage and services.  A fee for storage and services may not exceed the amount set by Council as provided for in the Fee Schedule if the property is not in the possession of the pawnbroker.
\subsection{}
The pawnshop charge allowed under this division shall be deemed earned, due, and owing as of the date of the pawn transaction and a like sum shall be deemed earned, due, and owing on the same day of the succeeding month.  However, if full payment is made more than two weeks before the next succeeding date the pawnbroker shall remit one-half of the pawnshop charge for that month to the pledgor.
\subsection{}
Interest shall not be deducted in advance, nor shall any loan be divided or split so as to yield greater interest or fees than would be permitted upon a single, consolidated loan or for otherwise evading any provisions of this subchapter.
\subsection{}
Any interest, charge, or fees contracted for or received, directly or indirectly, in excess of the amount permitted under this subchapter, shall be uncollectible and the pawn transaction shall be void.
\subsection{}
A schedule of charges permitted by this subchapter shall be posted on the pawnshop premises in a place clearly visible to the general public.\footnote{(‘83 Code, SEC. 6.59, Subd. 9)}

\section{Risk of Loss}
Any person to whom the receipt for pledged goods was issued, or any person identified in a written and notarized authorization to redeem the pledged goods identified in the receipt, or any person identified in writing by the pledgor at the time of the initial transaction and signed by the pledgor shall be entitled to redeem or repurchase the pledged goods described on the ticket.  In the event the goods are lost or damaged while in possession of the pawnbroker, the pawnbroker shall compensate the pledgor, in cash or replacement goods acceptable to the pledgor, for the fair market value of the lost of damaged goods.  Proof of compensation shall be a defense to any prosecution or civil action.\footnote{(‘83 Code, SEC. 6.59, Subd. 11)}

\section{Motor Vehicle Title Pawn Transactions}
\subsection{}
In addition to the other requirements of this subchapter, a pawnbroker who holds a title to a motor vehicle as part of a pawn transaction shall:
\begin{enumerate}[{\indent}1)]
    \item Be licensed as a used motor vehicle dealer under M.S. Chapter 168, as it may be amended from time to time, and post the license on the pawnshop premises;
    \item Verify that there are no liens or encumbrances against the motor vehicle with the Department of Public Safety; and
    \item Verify that the pledgor has automobile insurance on the motor vehicle as required by law.
\end{enumerate}
\subsection{}
A pawnbroker may not sell a motor vehicle covered by a pawn transaction until 90 days after recovery of the motor vehicle.\footnote{(‘83 Code, SEC. 6.59, Subd. 12)  Penalty, see SEC. 110.99}

\section{Violations}
A violation of this subchapter is a misdemeanor.\footnote{(‘83 Code, SEC. 6.59, Subd. 14)  (Ord. 128, 2nd Series, effective 5-16-98)  Penalty, see SEC. 110.99}
