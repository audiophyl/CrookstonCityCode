\chapter*{Chapter 50: \\
	Municipal and Public Utilities}
    \addstarredchapter{Chapter 50: Municipal and Public Utilities}
    \minitoc
    \pagebreak
    \textbf{\emph{{Cross-reference:}}}\\
    \emph{Public Works Department, see SEC. 32.080}
    \pagebreak

\subchapter{GENERAL PROVISIONS}

\section{Definitions}
For the purpose of this title the following definitions shall apply, unless the context clearly indicates or requires a different meaning.\footnote{(‘83 Code, SEC. 3.01)}
\begin{description}
\item[COMPANY, GRANTEE and FRANCHISEE] Any public utility system to which a franchise has been granted by the city.
\item[CONSUMER and CUSTOMER] Any user of a utility.
\item[MUNICIPAL UTILITY] Any city-owned utility system, including, but not by way of limitation, water, sewerage, and refuse service.
\item[SERVICE] Providing a particular utility to a customer or consumer.
\item[UTILITY] All utility services, whether the same be public city-owned facilities or furnished by public utility companies.
\end{description}

\section{Fixing Rates and Charges for Municipal Utilities}
All rates and charges for municipal utilities, including, but not by way of limitation, rates for service, permit fees, deposit, connection and meter testing fees, disconnection fees, reconnection fees including penalties for non-payment if any, and penalty surcharges for violation of utilities regulations, shall be fixed, determined and amended by the Council and adopted by resolution.  No resolution shall be adopted before a public hearing has been held thereon.  Notice of the public hearing shall be published once at least ten days prior thereto.  Notice of adoption of the resolution shall be published at least 30 days prior to the effective date thereof.  The resolution, containing the effective date thereof, shall be kept on file and open to inspection in the office of the City Clerk-Treasurer and shall be uniformly enforced.\footnote{(‘83 Code, SEC. 3.02) (Ord. 82, 2nd Series, effective 1-21-93)}
\section{Contractual Contents}
Provisions of this title relating to municipal utilities shall constitute portions of the contract between the city and all consumers of municipal utility services, and every consumer shall be deemed to assent to the same.  All contracts between franchisees and consumers of utility services other than municipal shall be in strict accord with the provisions of this title.\footnote{(‘83 Code, SEC. 3.04)}\\

\subchapter{\mbox{RULES AND REGULATIONS RELATING} \mbox{TO MUNICIPAL UTILITIES}}
\setcounter{section}{14}
\section{Billing, Payment and Delinquency}
All municipal utilities shall be billed monthly or quarterly and a utilities statement or statements shall be mailed to each consumer each month.  All utilities charges shall be delinquent if they are unpaid at the close of business on the fifth day of the month following the billing, provided, that if the fifth day of the month shall fall on a Saturday, Sunday or legal holiday, the time shall be extended to the close of business on the next succeeding day on which business is normally transacted.  If service is suspended due to delinquency it shall not be restored at that location until a reconnection charge has been paid for each utility reconnected in addition to amounts owed for service and penalties.\footnote{(‘83 Code, SEC. 3.05, Subd. 1)}
\section{Application, Connection and Sale of Service}
Application for municipal utility services shall be made upon forms supplied by the city, and strictly in accordance therewith.  No connection shall be made until consent has been received from the city to make the same.  All municipal utilities shall be sold and delivered to consumers under the then applicable rate applied to the amount of the utilities taken as metered or ascertained in connection with the rates.\footnote{(‘83 Code, SEC. 3.05, Subd. 2)}

\section{Discontinuance of Service}
All municipal utilities may be shut off or discontinued whenever it is found that:
\subsection{}
The owner or occupant of the premises served, or any person working on any connection with the municipal utility systems, has violated any requirement of the city code relative thereto, or any connection therewith; or,
\subsection{}
Any charge for a municipal utility service, or any other financial obligation imposed on the present owner or occupant of the premises served, is unpaid after due notice thereof; or,
\subsection{}
There is fraud or misrepresentation by the owner or occupant in connection with any application for service or delivery or charges therefor.
\subsection{}
No service of a residential customer shall be disconnected if the disconnection affects the primary heat source for the residential unit when the disconnection would occur during the period between October 15 and April 15, the customer has declared inability to pay on forms provided by the city, the household income of the customer is less than 185\% of the federal poverty level as documented by the customer to the city, and the customer’s account is current for the billing period immediately prior to October 15 or the customer has entered into a payment under the schedule.  The city shall, between August 15 and October 15 of each year, notify all residential customers of these provisions.\footnote{(‘83 Code, SEC. 3.05, Subd. 3)}

\section{Ownership of Municipal Utilities; Right of Entry}
\subsection{}
Ownership of all municipal utilities, plants, lines, mains, extensions and appurtenances thereto, shall be and remain in the city and no person shall own any part or portion thereof.  Provided, however, that private facilities and appurtenances constructed on private property are not intended to be included in municipal ownership.\footnote{(‘83 Code, SEC. 3.05, Subd. 4)}
\subsection{}
The city has the right to enter in and upon private property, including buildings and dwelling houses, in or upon which is installed a municipal utility, or connection therewith, at all times reasonable under the circumstances, for the purpose of reading utility meters, for the purpose of inspection and repair of meters or a utility system, or any part thereof, and for the purpose of connecting and disconnecting service.  If permission to enter the property is not granted, the Public Works Director is authorized to obtain an administrative search warrant from the district court.  However, the proper city employee shall have the right to unrestricted entry when it is determined that there is an emergency or threat of danger.\footnote{(‘83 Code, SEC. 3.05, Subd. 5)}

\section{Meter Accuracy}
All water service shall be supplied through a meter which shall accurately measure the amount thereof supplied to any consumer. The consumer shall supply a safe and proper place for the installation of the meters. Meters shall be tested for accuracy by the city upon the request of any consumer who believes his meter to be inaccurate, or upon its own initiative.  If, upon test, it appears that the meter overruns to the extent of 3\% or more, the city shall pay the cost of the tests and shall make a refund for overcharges collected since the last known date of accuracy but for not longer than six months, on the basis of the extent of the inaccuracy found to exist at the time of the tests.  If, upon test, it appears that the meter is slow to the extent of 3\% or more, the consumer shall pay for undercharges since the last known date of accuracy but for not longer than six months on the basis of the extent of the inaccuracy found to exist at the time of the test.  If, when any meter is tested upon the demand of a consumer, it is found to be accurate or slow or less than 3\% fast, the consumer shall pay the reasonable cost of the testing.\footnote{(‘83 Code, SEC. 3.05, Subd. 6)}

\section{Unlawful Acts}
\subsection{}
It is unlawful for any person to willfully or carelessly break, injure, mar, deface, disturb, or in any way interfere with any building, attachments, machinery, apparatus, equipment, fixture, or appurtenance of any municipal utility or municipal utility system, or commit any act tending to obstruct or impair the use of any municipal utility.
\subsection{}
It is unlawful for any person to make any connection with, opening into, use, or alter in any way any municipal utility system without first having applied for and received written permission to do so from the city.
\subsection{}
It is unlawful for any person to turn on or connect a utility when the same has been turned off or disconnected by the city for non-payment of a bill, or for any other reason, without first having obtained a permit to do so from the city.
\subsection{}
It is unlawful for any person to “jumper” or by any means or device fully or partially circumvent a municipal utility meter, or to knowingly use or consume unmetered utilities or use the services of any utility system, the use of which the proper billing authorities have no knowledge.\footnote{(‘83 Code, SEC. 3.05, Subd. 7)  Penalty, see SEC. 50.99}

\section{Municipal Utility Services and Charges a Lien}
\subsection{}
Payment for all municipal utility (as that term is defined in SEC. 50.01 of this code) service and charges shall be the primary responsibility of the fee owner of the premises served and shall be billed to the owner unless otherwise contracted for and authorized in writing by the fee owner and any other person (such as a tenant, contract purchaser, manager, etc.), as agent for the fee owner, and consented to by the city.  If the utility service and charges are for a single metered multi-unit rental residential building, the owner of the building shall be the customer of record and this responsibility shall not be waived by contract or otherwise.  The city may collect the same in a civil action or, in the alternative and at the option of the city, as otherwise provided in this section.
\subsection{}
Each account is hereby made a lien upon the premises served. All accounts which are more than 45 days delinquent may, when authorized by resolution of the Council, be certified by the Clerk-Treasurer to the County Auditor, and the Clerk-Treasurer in so certifying shall specify the amount thereof, the description of the premises served, and the name of the owner thereof. The amount so certified shall be extended by the Auditor on the tax rolls against the premises in the same manner as other taxes, and collected by the County Treasurer, and paid to the city along with other taxes.\footnote{(‘83 Code, SEC. 3.05, Subd. 8) (Ord. 125, 2nd Series, effective 5-16-98)}

\section{Municipal Utility Service Outside the City}
The city may furnish municipal utility service to consumers outside the city, provided, that the consumers specifically agree to all of the terms of the city code, including, but not limited to, rules, regulations and rates adopted thereunder and the right to specially assess delinquent services, charges and penalties.\footnote{(‘83 Code, SEC. 3.05, Subd. 9)}

\section{Delinquent Charges or Assessments}
No permit shall be granted to tap or connect with sewer or water mains when any assessment or connection charge for the sewer or water main against the property to be connected is in default or delinquent, unless a reasonable guarantee of payment to the city is made.  If the assessment or connection charges are payable in installments, no permit shall be granted unless all installments then due and payable have been paid.\footnote{(‘83 Code, SEC. 3.05, Subd. 10)}

\setcounter{section}{97}
\section{Violations; Penalty Surcharge}
\subsection{Purpose}
The city determines that violations of its municipal and public utilities regulations cause unnecessary expense and threaten the safety, health and general welfare of its residents.  Criminal penalties alone are not always an efficient, and therefore, effective means to assure compliance.  It is in the public interest to establish additional methods to insure and encourage compliance with and recover the increased costs resulting from violations of utilities regulations.
\subsection{Civil Penalty for Violation}
Any person violating any rule or regulation contained in or adopted by the Council under the provisions of this chapter shall pay to the city a penalty surcharge in an amount determined by the Council as provided in this chapter.  A separate violation is deemed committed upon each day during or on which a violation occurs or continues.  This penalty surcharge is in addition to the criminal fine or penalty provided for in SEC. 50.99.
\subsection{Determination of Violation}
The Public Works Director, or other proper city official, shall determine whether a violation has occurred and the person or persons responsible for the violation.  The Public Works Director may adopt rules containing procedures for the investigation and determination of alleged violations of utility regulations.
\subsection{Responsibility of Owner and Occupant}
It is the primary responsibility of all owners and occupants of private property to comply with the utility rules and regulations contained or adopted under the authority of this title.
\subsection{Collection}
The city may prepare a bill for the penalty surcharge amount and mail it to the person responsible for the violation and the amount shall then be due and payable.  The city may collect the penalty surcharge due by appropriate court action.  If the person violating the rule or regulation is a municipal utilities customer or an occupant of property owned by a municipal utilities customer, the amount of the penalty surcharge may be added to the customer’s municipal utility bill.\footnote{(‘83 Code, SEC. 3.99, Subd. 2) (Ord. 82, 2nd Series, effective 1-21-93)}
\section{Penalty}
Every person violates a section, division, or provision of this title when he or she performs an act thereby prohibited or declared unlawful, or fails to act when the failure is thereby prohibited or declared unlawful, and upon conviction thereof, shall be punished as for a misdemeanor except as otherwise stated in specific provisions hereof.\footnote{(‘83 Code, SEC. 3.99, Subd. 1)}
\subsection{}
The penalty which may be imposed for any crime which is a misdemeanor under this code, including Minnesota Statutes specifically adopted by reference, shall be a sentence of not more than 90 days or a fine of not more than \$1,000, or both.
\subsection{}
The costs of prosecution may be added.  A separate offense shall be deemed committed upon each day during which a violation occurs or continues.
