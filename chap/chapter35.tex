\chapter*{Chapter 35: \\
	Emergency Management}
    \addstarredchapter{Chapter 35: Emergency Management}
    \minitoc
    \pagebreak

\section{Policy and Purpose}
Because of the existing possibility of the occurrence of disasters of unprecedented size and destruction resulting from fire, flood, tornado, blizzard, destructive winds or other natural causes, or from sabotage, hostile action, or from hazardous material mishaps of catastrophic measure; and in order to insure that preparations of this city will be adequate to deal with those disasters, and generally, to provide for the common defense and to protect the public peace, health, and safety, and to preserve the lives and property of the people of this city, it is hereby found and declared to be necessary:
\subsection{}
To establish a city emergency management organization responsible for city planning and preparation for emergency government operations in time of disasters.
\subsection{}
To provide for the exercise of necessary powers during emergencies and disasters.
\subsection{}
To provide for the rendering of mutual aid between this city and other political subdivisions of this state and of other states with respect to the carrying out of emergency-preparedness functions.
\subsection{}
To comply with the provisions of M.S. \textsection 12.25, as it may be amended from time to time, which requires that each political subdivision of the state shall establish a local organization for emergency management.

\section{Definitions}
For the purpose of this chapter, the following definitions shall apply unless the context clearly indicates or requires a different meaning.
\begin{description}
    \item[DISASTER] A situation which creates an immediate and serious impairment to the health and safety of any person, or a situation which has resulted in or is likely to result in catastrophic loss to property, and for which traditional sources of relief and assistance within the affected area are unable to repair or prevent the injury or loss.
    \item[EMERGENCY] An unforeseen combination of circumstances which calls for immediate action to prevent from developing or occurring.
    \item[EMERGENCY MANAGEMENT] The preparation for and the carrying out of all emergency functions, other than functions for which military forces are primarily responsible, to prevent, minimize, and repair injury and damage resulting from disasters caused by fire, flood, tornado, and other acts of nature, or from sabotage, hostile action, or from industrial hazardous material mishaps. These functions include, without limitation, fire-fighting services, police services, emergency medical services, engineering, warning services, communications, radiological, and chemical, evacuation, congregate care, emergency transportation, existing or properly assigned functions of plant protection, temporary restoration of public utility services and other functions related to civil protection, together with all other activities necessary or incidental for carrying out the foregoing functions. Emergency management includes those activities sometimes referred to as “civil defense” functions.
    \item[EMERGENCY MANAGEMENT FORCES] The total personnel resources engaged in city-level emergency management functions in accordance with the provisions of this chapter or any rule or order thereunder. This includes personnel from city departments, authorized volunteers, and private organizations and agencies.
    \item[EMERGENCY MANAGEMENT ORGANIZATION] The staff responsible for coordinating city-level planning and preparation for disaster response. This organization provides city liaison and coordination with federal, state, and local jurisdictions relative to disaster preparedness activities and assures implementation of federal and state program requirements.
\end{description}

\section{Establishment of Emergency Management Organization}
There is hereby created within the city government an emergency management organization which shall be under the supervision and control of the City Emergency Management Director, called the Director. The Fire Chief shall serve as the Director unless otherwise approved by the Mayor and Council. The Director shall have direct responsibility for the organization, administration and operation of the emergency preparedness organization, subject to the direction and control of the City Administrator.

\section{Powers and Duties of Director}
\subsection{}
The Director, with the consent of the City Administrator, shall represent the city on any regional or state conference for emergency management. The Director, with the City Administrator, shall develop proposed mutual aid agreements with other political subdivisions of the state for reciprocal emergency management aid and assistance in an emergency too great to be dealt with unassisted, and shall present these agreements to the Council for its action.  These arrangements shall be consistent with the State Emergency Plan.
\subsection{}
The Director shall make studies and surveys of the human resources, industries, resources, and facilities of the city as deemed necessary to determine their adequacy for emergency management and to plan for their most efficient use in time of an emergency or disaster. The Director shall establish the economic stabilization systems and measures, service staffs, boards, and sub-boards required, in accordance with state and federal plans and directions subject to the approval of the City Administrator.
\subsection{}
The Director shall prepare a comprehensive emergency plan for the emergency preparedness of the city and shall present the plan to the City Administrator for approval.   When the Council has approved the plan, it shall be the duty of all city agencies and all emergency preparedness forces of the city to perform the duties and functions assigned by the plan as approved. The plan may be modified in like manner from time to time. The Director shall coordinate the emergency management activities of the city to the end that they shall be consistent and fully integrated with the emergency plans of the federal government and the state and correlated with emergency plans of the county and other political subdivisions within the state.
\subsection{}
In accordance with the State and City Emergency Plan, the Director shall institute training programs, public information programs and conduct practice warning alerts and emergency exercises as may be necessary to assure prompt and effective operation of the City Emergency Plan when a disaster occurs.
\subsection{}
The Director shall utilize the personnel, services, equipment, supplies, and facilities of existing departments and agencies of the city to the maximum extent practicable. The officers and personnel of all city departments and agencies shall, to the maximum extent practicable, cooperate with and extend services and facilities to the city’s emergency management organization and to the Governor upon request. The head of each department or agency in cooperation with the Director shall be responsible for the planning and programming of those emergency activities as will involve the utilization of the facilities of the department or agency.
\subsection{}
The Director shall, in cooperation with those city departments and agencies affected, assist in the organizing, recruiting, and training of emergency management personnel, which may be required on a volunteer basis to carry out the emergency plans of the city and state.   To the extent that emergency personnel are recruited to augment a regular city department or agency for emergencies, they shall be assigned to the departments or agencies and shall be under the administration and control of the department or agency.
\subsection{}
Consistent with the state emergency services law, the Director shall coordinate the activity of municipal emergency management organizations within the city and assist in establishing and conducting training programs as required to assure emergency operational capability in the several services as provided by M.S. \textsection 12.25, as it may be amended from time to time.
\subsection{}
The Director shall carry out all orders, rules, and regulations issued by the Governor with reference to emergency management.
\subsection{}
The Director shall prepare and submit reports on emergency preparedness activities when requested by the City Administrator.

\section{Local Emergencies}
\subsection{}
A local emergency may be declared only by the Mayor or his or her legal successor.   It shall not be continued for a period in excess of three days except by or with the consent of the Council. Any order, or proclamation declaring, continuing, or terminating a local emergency shall be given prompt and general publicity and shall be filed in the office of the City Clerk-Treasurer.
\subsection{}
A declaration of a local emergency shall invoke necessary portions of the response and recovery aspects of applicable local or inter-jurisdictional disaster plans, and may authorize aid and assistance thereunder.
\subsection{}
No jurisdictional agency or official may declare a local emergency unless expressly authorized by the agreement under which the agency functions. However, an inter-jurisdictional disaster agency shall provide aid and services in accordance with the agreement under which it functions.\\
\emph{Penalty, see SEC. 35.99}

\section{Emergency Regulations}
\subsection{}
Whenever necessary, to meet a declared emergency or to prepare for an emergency for which adequate regulations have not been adopted by the Governor or the Council, the Council may by resolution promulgate regulations, consistent with applicable federal or state law or regulation, respecting the conduct of persons and the use of property during emergencies, the repair, maintenance, and safeguarding of essential public services, emergency health, fire, and safety regulations, drills or practice periods required for preliminary training, and all other matters which are required to protect public safety, health, and welfare in declared emergencies.
\subsection{}
Every resolution of emergency regulations shall be in writing, shall be dated, shall refer to the particular emergency to which it pertains, if so limited, and shall be filed in the office of the City Clerk-Treasurer. A copy shall be kept posted and available for public inspection during business hours. Notice of the existence of these regulations and their availability for inspection at the City Clerk-Treasurer’s Office shall be conspicuously posted at the front of City Hall or other headquarters of the city or at other places in the affected area as the Council shall designate in the resolution. By resolution, the Council may modify or rescind a regulation.
\subsection{}
The Council may rescind any regulation by resolution at any time. If not sooner rescinded, every regulation shall expire at the end of 30 days after its effective date or at the end of the emergency to which it relates, whichever comes first. Any resolution, rule, or regulation inconsistent with an emergency regulation promulgated by the Council shall be suspended during the period of time and to the extent conflict exists.
\subsection{}
During a declared emergency, the city is, under the provisions of M.S. \textsection 12.31, as it may be amended from time to time and notwithstanding any statutory or Charter provision to the contrary, empowered, through its Council, acting within or without the corporate limits of the city, to enter into contracts and incur obligations necessary to combat the disaster by protecting the health and safety of persons and property and providing emergency assistance to the victims of a disaster. The city may exercise these powers in the light of the exigencies of the disaster without compliance with the time-consuming procedures and formalities prescribed by law pertaining to the performance of public work, entering rental equipment agreements, purchase of supplies and materials, limitations upon tax levies, and the appropriation and expenditure of public funds, including, but not limited to, publication of resolutions, publication of calls for bids, provisions of personnel laws and rules, provisions relating to low bids, and requirement for bids.\\
\emph{Penalty, see SEC. 35.99}

\section{Emergency Management a Government Function}
All functions and activities relating to emergency management are hereby declared to be governmental functions. The provisions of this section shall not affect the right of any person to receive benefits to which he would otherwise be entitled under this resolution or under the worker’s compensation law, or under any pension law, nor the right of any person to receive any benefits or compensation under any act of Congress.

\section{Participation in Labor Disputes or Politics}
The emergency management organization shall not participate in any form of political activity, nor shall it be employed directly or indirectly for political purposes, nor shall it be employed in a labor dispute.

\setcounter{section}{98}
\section{Penalty}
Any person who violates any provision of this chapter or any regulation adopted thereunder relating to acts, omissions, or conduct other than official acts of city employees or officers is guilty of a misdemeanor.
