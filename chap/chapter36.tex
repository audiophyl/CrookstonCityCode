\chapter*{Chapter 36: \\
	City Policy}
    \addstarredchapter{Chapter 36: City Policy}
    \minitoc
    \pagebreak
    
\section{Worker's Compensation Requirement}
\subsection{Contractors}
The city shall not enter into any contract for doing public work before receiving from all other contracting parties acceptable evidence of compliance with the worker’s compensation insurance coverage requirement of Minnesota Statutes.
\subsection{City Officers}
City officers.  All officers of the city elected or appointed for a regular term of office or to complete the unexpired portion of any regular term shall be included in the definition of “employee” as defined in Minnesota Statutes relating to coverage for purposes of worker’s compensation entitlement.\\
\emph{(‘83 Code, SEC. 2.11)  (Ord. 62, 2nd Series, effective 7-21-90)}

\section{Absentee Ballot Precinct}
Pursuant to the provisions of M.S. \textsection 203B.13, as it may be amended from time to time, the Council hereby authorizes the Polk County Commissioners to establish an Absentee Ballot Counting Precinct and ratifies the ordinance of the Polk County Board of Commissioners adopted on the 20th day of August, 1986.  The Absentee Ballot Precinct shall be located in the Polk County Courthouse for the purpose of receiving all absentee ballots for any state-wide primary or general election (or special referendum) held within the city.  The Absentee Ballot Precinct shall be under the direct charge and supervision of the Polk County Auditor and shall be administered pursuant to the Minnesota Election Laws without cost to the city.\\
\emph{(‘83 Code, SEC. 2.56)  (Ord. 51, 2nd Series, effective 9-27-88)}

\section{Franchise Ordinances}
\subsection{Definition}
The term \textbf{FRANCHISE} as used in this section shall be construed to mean any special privileges granted to any person in, over, upon, or under any of the streets or public places of the city, whether such privilege has heretofore been granted by it or by the State of Minnesota, or shall hereafter be granted by the city or by the State of Minnesota.
\subsection{Franchise Ordinances}
The Council may grant franchises by ordinance. Franchise rights shall always be subject to the superior right of the public to the use of streets and public places. All persons desiring to make any burdensome use of the streets or public places, inconsistent with the public’s right in such places, or desiring the privilege of placing in, over, upon, or under any street or public place any permanent or semipermanent fixtures for the purpose of constructing or operating railways, telegraphing, or transmitting electricity, or transporting by pneumatic tubes, or for furnishing to the city or its inhabitants or any portion thereof, transportation facilities, water, light, heat, power, gas, or any other the utility, or for any other purpose, shall be required to obtain a franchise before proceeding to make the use of the streets or public places or before proceeding to place the fixtures in such places.
\subsection{Power of Regulation Reserved}
The city shall have the right and power to regulate and control the exercise by any person, of any franchise however acquired, and whether such franchise has been heretofore granted by it or by the State of Minnesota.
\subsection{Conditions in Every Franchise}
All conditions specified in this section shall be a part of every franchise even though they may not be expressly contained in the franchise:
\begin{enumerate}
    \item That the grantee shall be subject to and will perform on its part all the terms of this section and will comply with all pertinent provisions of any City Charter and City Code, as the same may from time to time be amended.
    \item That the grantee shall in no case claim or pretend to exercise any power to fix fares, rates, and charges, but that the fares, rates, and charges shall at all times be just, fair and reasonable for the services rendered and shall in all cases be fixed and from time to time changed, unless regulated by an agency of the State of Minnesota, in the manner following:
    \begin{enumerate}[{\indent\indent}a)]
        \item A reasonable rate shall be construed to be one which will, with efficient management, normally yield above all operating expenses and depreciation, a fair return upon all money invested.
        \item If possible, maximum rates and charges shall be arrived at by direct negotiation with the Council.
        \item If direct negotiations fail to produce agreement, the Council shall, not less than 30 days before the expiration of any existing rate schedule or agreement, appoint an expert as its representative, the franchisee shall likewise appoint an expert as its representative and the two of them shall appoint a third person, preferably an expert, and the three of them shall constitute a board of arbitration. The board shall report its findings as soon as possible and the rates and charges it shall agree upon by majority vote shall be legal and binding, subject only to review by a court of competent jurisdiction upon application of one of the parties.
    \end{enumerate}
    \item That the Council shall have the right to require reasonable extensions of any public service system from time to time, and to make such rules and regulations as may be required to secure adequate and proper service and to provide sufficient accommodations for the public.
    \item That the grantee shall not issue any capital stock on account of the franchise or the value thereof, and that the grantee shall have no right to receive upon condemnation proceedings brought by the city to acquire the public utility exercising such franchise, any return on account of the franchise or its value.
    \item That no sale or lease of the franchise shall be effective until the assignee or lessee shall have filed with the city an instrument, duly executed, reciting the facts of such sale or lease, accepting the terms of the franchise, and agreeing to perform all the conditions required of the grantee thereunder.
    \item That every grant in the franchise contained of permission for the erection of poles, masts, or other fixtures in the streets and for the attachment of wires thereto, or for the laying of tracks in, or of pipes or conduits under the streets or public places, or for the placing in the streets or other public places of any permanent or semi-permanent fixtures whatsoever, shall be subject to the conditions that the Council shall have the power to require the alterations therein, or relocation or rerouting thereof, as the Council may at any time deem necessary for the safety, health, or convenience of the public, and particularly that it shall have the power to require the removal of poles, masts, and other fixtures bearing wires and the placing underground of all facilities for whatsoever purpose used.
    \item Every franchise shall contain a provision granting the city the right to acquire the same in accordance with statute.
    \item That the franchisee may be obligated by the city to pay the city fees to raise revenue or defray increased costs accruing as a result of utility operations, or both, including, but not limited to, a sum of money based upon gross operating revenues or gross earnings from its operations in the city.
\end{enumerate}
\subsection{Further Provisions of Franchises}
The enumeration and specification of particular matters which must be included in every franchise or renewal or extension thereof, shall not be construed as impairing the right of the city to insert in any franchise or renewal or extension thereof any further conditions and restrictions as the Council may deem proper to protect the city’s interests, nor shall anything contained in this section limit any right or power possessed by the city over existing franchises.\\
\emph{(‘83 Code, SEC. 2.74)}

\section{Disposal of Unclaimed and Excess Property}
\subsection{Disposal of Unclaimed Property}
\subsubsection{Definition}
The term \textbf{ABANDONED PROPERTY} means tangible or intangible property that has lawfully come into the possession of the city in the course of municipal operations, remains unclaimed by the owner, and has been in the possession of the city for at least 60 days and has been declared such by a resolution of the Council.
\subsubsection{Preliminary Notice}
If the City Administrator knows the identity and whereabouts of the owner, he shall serve written notice upon him at least 30 days prior to a declaration of abandonment by the Council.  If the city acquired possession from a prior holder, the identity and whereabouts of whom are known by the City Administrator notice shall also be served upon him. The notice shall describe the property and state that unless it is claimed and proof of ownership, or entitlement to possession established, the matter of declaring it abandoned property will be brought to the attention of the Council after the expiration of 30 days from the date of the notice.
\subsubsection{Notice and Sale}
Upon adoption of a resolution declaring certain property to be abandoned property, the City Administrator shall publish a notice thereof describing the same, together with the names (if known) and addresses (if known) of prior owners and holders thereof, and including a brief description of the property. The text of the notice shall also state the time, place and manner of sale of all the property, except cash and negotiables. The notice shall be published once at least three weeks prior to sale. Sale shall be made to the highest bidder at public auction or sale conducted in the manner directed by the Council in its resolution declaring property abandoned and stated in the notice.
\subsubsection{Fund and Claims Thereon}
All proceeds from the sale shall be paid into the general fund of the city and expenses thereof paid therefrom. The former owner, if he makes claim within eight months from the date of publication of the notice herein provided, and upon application and satisfactory proof of ownership, may be paid the amount of cash or negotiables or, in the case of property sold, the amount received therefor, less a pro rata share of the expenses of storage, publication of notice, and sale expenses, but without interest.  The payment shall be also made from the general fund.
\subsection{Disposal of Excess Property}
\subsubsection{Declaration of Surplus and Authorizing Sale of Property}
The City Administrator may, from time to time, recommend to the Council that certain personal property (chattels) owned by the city is no longer needed for a municipal purpose and should be sold. By action of the Council, the property shall be declared surplus, the value estimated and the City Administrator authorized to dispose of the property in the manner stated herein.
\subsubsection{Surplus Property with a Total Estimated Value of Less Than \$100}
The City Administrator may sell surplus property with a total value of less than \$100 through negotiated sale.
\subsubsection{Surplus Property with a Total Estimated Value Between \$100 and \$500}
The City Administrator shall offer for public sale, to the highest bidder, surplus property with a total estimated value of from \$100 to \$500. Notice of the public sale shall be given stating time and place of sale and generally describing the property to be sold at least ten days prior to the date of sale either by publication once in the official newspaper, or by posting in a conspicuous place in the City Hall at the City Administrator’s option. The sale may be by auction or sealed bids.
\subsubsection{Surplus Property with a Total Estimated Value Over \$500}
The City Administrator shall offer for public sale, to the highest bidder, surplus property with a total estimated value over \$500.  Notice of the public sale shall be given stating time and place of sale and generally describing property to be sold at least ten days prior to the date of sale by publication once in the official newspaper.  The sale shall be to the person submitting the highest bid.
\subsubsection{Receipts from Sales of Surplus Property}
All receipts from sales of surplus property under this section shall be placed in the general fund.
\subsection{Persons Who May Not Purchase -- Exception}
\subsubsection{}
No employee of the city who is a member of the administrative staff, department head, a member of the Council, or an advisor serving the city in a professional capacity, may be a purchaser of property under this section.  Other city employees may be purchasers if they are not directly involved in the sale, if they are the highest responsible bidder, and if at least one week’s published or posted notice of sale is given.
\subsubsection{}
It is unlawful for any person to be a purchaser of property under this section if the purchase is prohibited by the terms of this section.\\
\emph{(‘83 Code, SEC. 2.70, Subds. 2, 3, 4) Penalty, see SEC. 10.99}

\section{Administrative Offenses and Penalties}
\subsection{Purpose}
Administrative offense procedures established pursuant to this Section are intended to provide the public and the City with an informal, cost effective, and expeditious alternative to traditional criminal charges for violations of certain City Code provisions. The procedures are intended to be voluntary on the part of those who have been charged with an administrative offense. At any time prior to the payment of the administrative penalty as provided for in this Section, the individual may withdraw from participation in the procedures, in which event the City, in its discretion, may choose not to initiate an administrative offense and may bring criminal charges in the first instance. In the event a party participates in the administrative offense procedures but does not pay the monetary penalty which may be imposed, the City will seek to collect the costs of the administrative offense procedures as part of a subsequent criminal sentence in the event the party is charged and subsequently adjudicated guilty of the related criminal violation.
\subsection{Administrative Offense Defined}
An administrative offense is a violation of a provision of the City Code and is subject to the administrative penalties set forth in the schedule of offenses and penalties referred to in Subsection (H).
\subsection{Notice}
Any officer of the Crookston Police Department or any other person employed by the City, authorized in writing by the City Administrator, and having authority to enforce the City Code, shall, upon determining that there has been a violation, notify the violator, or, in the case of a vehicular violation, attach to the vehicle a notice of the violation. Said notice shall set forth the nature, date and time of the violation, the name of the official issuing the notice, and the amount of the scheduled penalty.
\subsection{Payment}
Once such notice of violation is given, the alleged violator may, within 14 days of the time of issuance of the notice, pay the amount of the scheduled penalty for the violation, or request in writing that the matter be referred to the Polk County District Court as is provided in Subsection (E). The penalty may be paid in person or by mail, and payment shall be deemed an admission of the violation.
\subsection{Referral to Polk County District Court}
Any person contesting an administrative offense pursuant to this Section may, within14 days of the time of issuance of the notice of violation, request in writing that the matter be referred to the Polk County District Court to be processed through the usual rules of Criminal Court. The request may be delivered in person or by mail to the Chief of Police and will be deemed effective when received.
\subsection{Failure to Pay}
In the event a person charged with an administrative offense fails to pay the penalty, a misdemeanor or petty misdemeanor charge may be brought against the alleged violator in accordance with applicable law.
\subsection{Disposition of Penalties}
All penalties collected pursuant to this Section shall be paid to the City Clerk-Treasurer and may be deposited in the City’s general fund.
\subsection{Offenses and Penalties}
Offenses which may be charged as an administrative offense and the penalties for such offenses may be established from time to time by resolution of the City Council. Copies of such resolutions shall be maintained in the office of the City Clerk-Treasurer.
\subsection{Subsequent Offenses}
In the event a person is charged with a subsequent administrative offense within a 12 month period after paying an administrative penalty for the same or substantially similar offense, the subsequent administrative penalty shall be increased by twenty five percent (25\%) above the previous administrative penalty, except as otherwise stated in the penalty schedule or by resolution.

\section{Criminal History Background Checks}
\subsection{Purpose}
The purpose and intent of this Section is to establish regulations that will allow law enforcement access to Minnesota’s Computerized Criminal History information for specified non-criminal purposes of background checks for applicants described in Subsection (B) of this Section.
\subsection{Criminal History Background Investigations}
The Crookston Police Department is hereby required, as the exclusive entity within the City, to do a criminal history background investigation on the following applicants:
\begin{enumerate}[{\indent}1)]
    \item All applicants for regular, full-time employment with the City or seasonal employment positions with the City that have interaction with children, finances, or positions that have access to security-sensitive areas, unless the City’s hiring authority concludes that a background investigation is not needed;
    \item All applicants for the position of taxicab and personal transportation vehicle driver as defined in Section 118.12; and
    \item All applicants for City licenses requiring a background check under the provisions of the City Code not set forth in subdivision (2) above.
\end{enumerate}
\subsection{}
In conducting the criminal history background investigation in order to screen applicants, the Police Department is authorized to access data maintained in the Minnesota Bureau of Criminal Apprehensions Computerized Criminal History information system in accordance with BCA policy. Any data that is accessed and acquired shall be maintained at the Police Department under the care and custody of the chief law enforcement official or his or her designee. A summary of the results of the Computerized Criminal History data may be released by the Police Department: for employment background investigations, to the hiring authority, including the City Council, the City Administrator or other City staff involved in the hiring process; or, for license background investigations, to the licensing authority, including the City Council, the City Administrator or other City staff involved in the license approval process.
\subsection{}
Before the investigation is undertaken, the applicant must authorize the Police Department by written consent to undertake the investigation. The written consent must fully comply with the provisions of Minn. Stat. Chap. 13 regarding the collection, maintenance and use of the information. Except for the positions set forth in Minnesota Statutes Section 364.09, the City will not reject an applicant for employment on the basis of the applicant’s prior conviction unless the crime is directly related to the position of employment sought and the conviction is for a felony, gross misdemeanor, or misdemeanor with a jail sentence. The City will not reject an applicant for a license on the basis of the applicant’s prior conviction unless the crime is directly related to the license sought and the conviction is for a felony, gross, misdemeanor, or misdemeanor with a jail sentence. If the City rejects the applicant’s request on this basis, the City shall notify the applicant in writing of the following:
\begin{enumerate}[{\indent}1)]
    \item The grounds and reason for the denial.
    \item The applicant complaint and grievance procedure set forth in Minnesota Statutes Section 364.06.
    \item The earliest date the applicant may reapply for employment or for the license.
    \item That all competent evidence of rehabilitation will be considered upon reapplication.
\end{enumerate}
