\chapter*{Chapter 152: \\
	Zoning}
    \addstarredchapter{Chapter 152: Zoning}
    \minitoc
    \pagebreak

\begin{center}
    \emph{\textbf{\LARGE{GENERAL PROVISIONS}}}
\end{center}

\section{Intent and Purpose}
This chapter is adopted for the purpose of:
\begin{enumerate}[{\indent}A)]
    \item Protecting the public health, safety, comfort, convenience and general welfare; 
    \item Promoting orderly development of the residential, commercial, industrial, recreational and public areas; 
    \item Conserving the natural and scenic beauty and attractiveness of the city; 
    \item Conserving and developing natural resources in the city; 
    \item Providing for the compatibility of different land uses and the most appropriate use of land throughout the city; 
    \item Minimizing environmental pollution; and
    \item Conserving energy through the use of solar systems and the encouragement of solar and earth-sheltered structures for commercial, industrial, and residential uses.
\end{enumerate}
\emph{(‘83 Code, SEC. 11.01)}
\section{Rules of Language Construction}
Whenever a word or term defined hereinafter appears in the text of this chapter, its meaning shall be construed as set forth in the definition. All measured distances expressed in feet shall be to the nearest tenth of a foot.\\
\emph{(‘83 Code, SEC. 11.02)}
\section{Definitions}
For the purpose of this chapter the following definitions shall apply, unless the context clearly indicates or requires a different meaning.
\begin{description}
    \item[ACCESSORY USE OR STRUCTURE] A use or structure on the same lot with, and of a nature customarily incidental and subordinate to, the principal use or structure. \emph{(Ord. 86, 2nd Series, effective 12-23-93)}
    \item[AGRICULTURAL BUILDING OR STRUCTURE] Any building or structure existing or erected which is used principally for agricultural purposes, with the exception of dwelling units.
    \item[AGRICULTURAL USE] The use of land for the growing and/or production of field crops, livestock, and livestock products for the production of income, including but not limited to the following:
        \begin{enumerate}[{\indent\indent}1)]
            \item Field crops, including barley, soy beans, corn, hay, oats, potatoes, rye, sorghum, and sunflowers.
            \item Livestock, including dairy and beef cattle, goats, horses, sheep, hogs, poultry, game birds and other animals including dogs, ponies, deer, rabbits and mink.
            \item Livestock products, including milk, butter, cheese, eggs, meat, fur and honey.
        \end{enumerate}
    \item[APARTMENT] A room or suite of rooms with cooking facilities available which is occupied as a residence by a single family, or a group of individuals living together as a single family unit.  This includes any unit in buildings with more than two dwelling units.
    \item[APARTMENT BUILDING, HIGH RISE] An apartment building three or more stories in height, whose upper floors are accessible by elevator, and whose dwelling units are accessible through common corridors.
    \item[APARTMENT BUILDING, WALKUP] An apartment building not more than three stories above grade whose upper floors are accessible by stairs, and whose dwelling units are usually accessible through common corridors.
    \item[AUTO OR MOTOR VEHICLE REDUCTION YARD] A lot or yard where one or more unlicensed motor vehicle(s), or the remains thereof, are kept for the purpose of dismantling, wrecking, crushing, repairing, rebuilding, sale of parts, sale as scrap, storage, or abandonment. (See also JUNK YARD).
    \item[AUTOMOBILE SERVICE STATION] A building designed primarily for the supplying of motor fuel, oil, lubrication and accessories to motor vehicles or any portion thereof. \emph{(Ord. 537, effective 7-1-83)}
    \item[BASEMENT] Any area of a structure, including crawl space, having its floor or base subgrade (below ground level) on all four sides, regardless of the depth of excavation below ground level. \emph{(Ord. 86, 2nd Series, effective 12-23-93)}
    \item[BED AND BREAKFAST INN] An owner occupied building at least 75 years old designed for and used as a single-family or two-family dwelling that provides four or fewer lodging rooms accommodating no more than eight adults, in which meals are provided to overnight guests, and that is open to the traveling public for a stay not to exceed 20 days. \emph{(Ord. 37, 2nd Series, effective 9-16-86)}
    \item[BOARDING HOUSE (ROOMING OR LODGING HOUSE)] A building other than a motel or hotel where, for compensation and by prearrangement for definite periods, meals or lodging are provided for three or more persons, but not to exceed 20 persons.
    \item[BUILDING] Any structure having a roof which may provide shelter or enclosure of persons, animals, chattel, or property of any kind and when the structures are divided by party walls without openings, each portion of the building so separated shall be deemed a separate building.
    \item[BUILDING LINE] A line parallel to the street right-of-way line at any story level of a building and representing the minimum distance which all or any part of the building is set back from the right-of-way.
    \item[BUILDING HEIGHT] The vertical distance to be measured from the average grade of a building line to the top, to the cornice of a flat roof, to the deck line of a mansard roof, to a point on the roof directly above the highest wall of a shed roof, to the uppermost point on a round or other arch type roof, to the mean distance of the highest gable on a pitched or hip roof.
    \item[BUILDING SETBACK] The minimum horizontal distance between the building and a lot line, or the normal high water mark of a stream or river.
    \item[BUSINESS] Any occupation, employment or enterprise wherein merchandise is exhibited or sold, or where services are offered for compensation.
    \item[CARPORT] An automobile shelter having two or more sides open. Floor of carport shall be of approved non-combustible material.
    \item[CELLAR] That portion of a building having more than one-half of the floor-to-ceiling height below the average grade of the adjoining ground.
    \item[CHURCH]  A building, together with its accessory buildings and uses, where persons regularly assemble for religious worship and which building, together with its accessory buildings and uses, is maintained and controlled by a religious body organized to sustain public worship.
    \item[CLEAR-CUTTING] The removal of an entire stand of vegetation.
    \item[CLUSTERING/CLUSTER HOUSING] The development pattern and technique whereby structures are arranged in closely related groups to make the most efficient use of the natural amenities of the land.
    \item[COMMISSIONER] Commissioner of the Department of Natural Resources. \emph{(Ord. 537, effective 7-1-83)}
    \item[COMPREHENSIVE PLAN] A compilation of goals, policy statements, standards, programs and maps for guiding the physical, social and economic development, both public and private, of the city and its environs, as defined in the Municipal Planning Act, and includes any unit or part of the plan separately adopted and any amendment to the plan or parts thereof.
    \item[CONDITIONAL USE] A specific type of structure or land use that may be allowed but only after an in-depth review procedure and with appropriate conditions or restrictions as provided in the official zoning controls or building codes and upon a finding that:
        \begin{enumerate}[{\indent\indent}1)]
            \item Certain conditions as detailed in the zoning regulations exist; and 
            \item The structure and/or land use conform to the comprehensive land use plan (if one exists) and are compatible with the existing neighborhood.
        \end{enumerate}
        \emph{(Ord. 86, 2nd Series, effective 12-23-93)}
    \item[CONDOMINIUM] A form of individual ownership of a multi-family building with joint responsibility for maintenance and repairs of the common property.  In a condominium, each apartment or townhouse unit is owned outright by its occupant and each occupant also owns a share of the land and other common property of the building.
    \item[COOPERATIVE] A multi-unit development operated for and owned by its occupants.  Individual occupants do not own their specific housing unit outright as in a condominium, but they own shares in the total enterprise.
    \item[CURB LEVEL] The grade elevation established by the Council of the curb in front of the center of the building.  Where no curb level has been established, the engineering staff shall determine a curb level or its equivalent for the purpose of this chapter.
    \item[DRIVE-IN] Any use where products and/or services are provided to the customer under conditions where the customer does not have to leave the car or where service to the automobile occupants is offered regardless of whether service is also provided within a building.
    \item[DWELLING UNIT] A residential building or portion thereof intended for occupancy by a single family but not including hotels, motels, boarding or rooming houses or tourist homes.  There are three principal types:
        \begin{enumerate}[{\indent\indent}1)]
            \item \textbf{MULTIPLE-FAMILY} A residence designed for or occupied by three or more families, either wholly (attached) or partially a part of a larger structure (detached), with separate housekeeping and cooking facilities for each.
            \item \textbf{SINGLE-FAMILY} A free-standing (detached) residence structure designed for or occupied by one family only.
            \item \textbf{TWO-FAMILY} A residence designed for or occupied by two families only, with separate housekeeping and cooking facilities for each.
        \end{enumerate}
    \item[DWELLING, ATTACHED] One which is joined to another dwelling or building at one or more sides by a party wall or walls.
    \item[DWELLING, DETACHED] One which is entirely surrounded by open space on the same lot.
    \item[EARTH SHELTERED BERM] An earth covering on the above grade portions of building walls.
    \item[EARTH SHELTERED BUILDING] A building constructed so that 50\% or more of the completed structure is covered with earth.  Earth covering is measured from the lowest level of livable space in residential units and of usable space in non-residential buildings.  An earth sheltered building is a complete structure that does not serve just as a foundation or substructure for aboveground construction.  A partially completed building shall not be considered earth sheltered.
    \item[EASEMENT] A grant by a property owner for the use of a strip of land for the purpose of constructing and maintaining walkways; roadways; utilities, including but not limited to sanitary sewers, water mains, electric lines, telephone lines, storm sewer or storm drainageways and gas lines.
    \item[EFFICIENCY UNIT] A dwelling unit with one primary room which doubles as a living room, kitchen and bedroom. \emph{(Ord. 537, effective 7-1-83)}
    \item[EQUAL DEGREE OF ENCROACHMENT] A method of determining the location of floodway boundaries so that flood plain lands on both sides of a stream are capable of conveying a proportionate share of flood flows. \emph{(Ord. 86, 2nd Series, effective 12-23-93)}
    \item[ESSENTIAL SERVICES] Overhead or underground electrical, gas, steam or water transmission or distribution systems and structures or collection, communication, supply or disposal systems and structures used by public utilities or governmental departments or commissions or as are required for the protection of the public health, safety or general welfare, including towers, poles, wires, mains, drains, sewers, pipes, conduits, cables, fire alarm boxes, police call boxes and accessories in connection herewith but not including buildings.  For the purpose of this chapter, the word “buildings” does not include “structures” for essential services.
    \item[EXTERIOR STORAGE (INCLUDES OPEN STORAGE)] The storage of goods, materials, equipment, manufactured products and similar items not fully enclosed by a building.
    \item[EXTRACTION AREA] Any non-agricultural artificial excavation of earth exceeding 50 square feet of surface area of two feet in depth, other than activity involved in preparing land for earth sheltered or conventional construction of residential, commercial, and industrial buildings, excavated or made by the removal from the natural surface of the earth, of sod, soil, sand, gravel, stone or other natural matter, or made by turning, or breaking or undermining the surface of the earth, except that public improvement projects shall not be considered \emph{EXTRACTION AREAS}.
    \item[FAMILY] An individual, or two or more persons related by blood, marriage or adoption, living together as a single housekeeping unit in a dwelling unit.
    \item[FARM] A tract of land which is principally used for agricultural activities such as the production of cash crops, livestock or poultry farming.  The farms may include agricultural dwelling and accessory buildings and structures necessary to the operation of the farm.
    \item[FEEDLOTS, LIVESTOCK] The place of confined feeding of livestock, poultry or other animals for food, fur, pleasure or resale purposes in yards, lots, pens, buildings, or other areas not normally used for pasture or crops and in which substantial amounts of manure or related other wastes may originate by reason of the feeding of animals.
    \item[FENCE] Any partition, structure, wall or gate erected as a divider marker, barrier or enclosure and located along the boundary, or within the required yard. \emph{(Ord. 537, effective 7-1-83)}
    \item[FLOOD] A temporary increase in the flow or stage of a stream or in the stage of a wetland or lake that results in inundation of normally dry areas.
    \item[FLOOD FREQUENCY] The frequency for which it is expected that a specific flood stage or discharge may be equaled or exceeded.
    \item[FLOOD FRINGE] That portion of the flood plain outside of the floodway.  Flood fringe is synonymous with the term “floodway fringe” used in the Flood Insurance Study for the City of Crookston.
    \item[FLOOD PLAIN] The beds proper and the areas adjoining a wetland, lake or watercourse which have been or hereafter may be covered by the regional flood. \emph{(Ord. 86, 2nd Series, effective 12-23-93)}
    \item[FLOOD PROOFING] A combination of structural provisions, changes or adjustments to properties and structures subject to flooding, primarily for the reduction or elimination of flood damages. \emph{(Ord. 537, effective 7-1-83)}
    \item[FLOODWAY] The bed of a wetland or lake and the channel of a watercourse and those portions of the adjoining flood plain which are reasonably required to carry or store the regional flood discharge. \emph{(Ord. 86, 2nd Series, effective 12-23-93)}
    \item[FLOOR AREA] The gross area of the main floor of a residential building measured in square feet and not an attached garage, breezeway or similar attachment.
    \item[FLOOR AREA, GROSS]The sum of the gross area of the various floors of a building measured in square feet. The basement floor area shall not be included unless the area constitutes a story.
    \item[FLOOR AREA RATIO] The numerical value obtained through dividing the gross floor area of a building or buildings by the net area of the lot or parcel of land on which the building or buildings are located.
    \item[FORESTRY] The use and management including logging of a forest, woodland or plantation and related research and educational activities, including the construction, alteration or maintenance of woodroads, skidways, landings, and fences.
    \item[FRONTAGE] That boundary of a lot which abuts an existing or dedicated public street.
    \item[GARAGE, PRIVATE] An accessory building or accessory portion of the principal building which is intended for and used to store the private passenger vehicles of the family or families resident upon the premises.
    \item[GRADE] The average of the finished level at the center of the exterior walls of the building.  For an earth sheltered building GRADE means the average of the finished level at the center of the lot.  For a building with earth berms but less than 50\% earth covering grade means the average of the finished level at the center of the building at the beginning of the earth berm.
    \item[GROUP FACILITY OR HOME] A residence utilized by unrelated people for the purpose of rehabilitation. \emph{(Ord. 537, effective 7-1-83)}
    \item[HISTORIC DISTRICT] That portion of the Central Business District (C-1) which was placed on the National Register of Historic Places on November 23, 1984, a detailed description of which is on file and available for public inspection in the office of the Clerk-Treasurer. \emph{(Ord. 85, 2nd Series, effective 7-17-93)}
    \item[HOME OCCUPATION] Any gainful occupation or profession engaged in by the occupant of a dwelling at or from the dwelling when carried on within a dwelling unit or accessory structure.  The uses include professional offices, minor repair services, photo or art studios, dressmaking, barber shops, beauty shops, tourist homes, or similar uses.
    \item[HORTICULTURE] Horticulture uses and structures designed for the storage of products and machinery pertaining and necessary thereto.
    \item[HOTEL] A building which provides a common entrance, lobby, halls and stairway and in which 20 or more people can be, for compensation, lodged with or without meals.
    \item[JUNK YARD] An open area where waste, used, or second-hand materials are bought, sold, exchanged, stored, baled, packed, disassembled or handled, including but not limited to, scrap iron and other metals, paper, rags, rubber, tires, and bottles.  A junk yard includes an auto wrecking yard but does not include uses established entirely within enclosed buildings.  This definition does not include sanitary landfills.
    \item[KENNEL] Any structure or premises on which four or more dogs over four months of age are kept for sale, breeding, profit, and the like.
    \item[LANDSCAPING] Plantings, including trees, grass, ground cover, and shrubs.
    \item[LODGING ROOM] A room rented as sleeping and living quarters, but without cooking facilities. In a suite of rooms without cooking facilities each room which provides sleeping accommodations shall be counted as one lodging room.
    \item[LOT] A parcel or portion of land in a subdivision or plat of land, separated from other parcels or portions by description as on a subdivision or record of survey map, for the purpose of sale or lease or separate use thereof.
    \item[LOT AREA] The area of a lot in a horizontal plane bounded by the lot lines.
    \item[LOT, CORNER] A lot situated at the junction of, and abutting on two or more intersecting streets, or a lot at the point of deflection in alignment of a continuous street, the interior angle of which does not exceed 135 degrees.
    \item[LOT COVERAGE]The area of the zoning lot occupied by the principal buildings and accessory buildings. Earth berms are not to be included in calculating lot coverage.  Only the above grade portions of an earth sheltered building should be included in lot coverage calculations.
    \item[LOT DEPTH] The mean horizontal distance between the front lot line and the rear lot lime of a lot.
    \item[LOT LINE] The property line bounding a lot except that where any portion of a lot extends into the public right-of-way shall be the lot line for applying this chapter.
    \item[LOT LINE, FRONT] That boundary of a lot which abuts an existing or dedicated public street, and in the case of a corner lot it shall be the shortest dimension on a public street. If the dimensions of a corner lot are equal, the front line shall be designated by the owner and filed with the County Recorder.
    \item[LOT LINE, REAR] That boundary of a lot which is opposite the front lot line.  If the rear line is less than ten feet in length, or if the lot forms a point at the rear, the rear lot line shall be a line ten feet in length within the lot, parallel to, and at the maximum distance from the front lot line.
    \item[LOT LINE, SIDE] Any boundary of a lot which is not a front lot line or a rear lot line.
    \item[LOT OF RECORD] Any lot which is one unit of a plat heretofore duly approved and filed, or one unit of an Auditor’s Subdivision or a Registered Land Survey that has been recorded in the office of the County Recorder for Polk County, Minnesota, prior to the effective date of this chapter.
    \item[LOT, SUBSTANDARD] A lot or parcel of land for which a deed has been recorded in the office of the County Recorder upon or prior to the effective date of this chapter which does not meet the minimum lot area, structure setbacks or other dimensional standards of this chapter.
    \item[LOT, THROUGH] A lot which has a pair of opposite lot lines abutting two substantially parallel streets, and which is not a corner lot. On a through lot, both street lines shall be front lines for applying this chapter.
    \item[LOT WIDTH] The maximum horizontal distance between the side lot lines of a lot measured within the first 30 feet of the lot depth.
    \item[MANUFACTURED HOME] A structure, transportable in one or more sections, which in the traveling mode, is eight feet or more in width or 40 body feet or more in length, or, when erected on site, is 320 or more square feet, and which is built on a permanent chassis and designed to be used as a dwelling with or without a permanent foundation when connected to the required utilities, and includes the plumbing, heating, air conditioning, and electrical systems contained therein; except that the term includes any structure which meets all the requirements and with respect to which the manufacturer voluntarily files a certification required by the secretary and complies with the standards established under this chapter.
    \item[METES AND BOUNDS] A method of property description by means of their direction and distance from an easily identifiable point.
    \item[MINING] The extraction of sand, gravel, rock, soil or other material from the land in the amount of 1,000 cubic yards or more and the removing thereof from the site.  The only exclusion from this definition shall be removal of materials associated with construction of a building, provided the removal is an approved item in the building permit.
    \item[MOBILE HOME] A manufactured relocatable residential unit providing complete, independent living facilities for one family including permanent provisions for living, sleeping, eating, cooking and sanitation.
    \item[MOBILE HOME LOT] A parcel of land for the placement of a mobile home and the exclusive use of its occupants.
    \item[MOBILE HOME PARK] A contiguous parcel of land which has been developed for the placement of mobile homes and is owned by an individual, a firm, trust, partnership, public or private association or corporation.
    \item[MODULAR HOME] A non-mobile housing unit which is basically fabricated at a central factory and transported to a building site where final installations are made, permanently affixing the module to the site.
    \item[MOTEL (TOURIST COURT)] A building or group of detached, semi-detached, or attached buildings containing guest rooms or dwellings, with garage or parking space conveniently located to each unit, and which is designed, used or intended to be used primarily for the accommodation of automobile transients.
    \item[MULTIPLE RESIDENT (APARTMENT BUILDING)] Three or more dwelling units in one structure.
    \item[NURSERY LANDSCAPE] A business growing and selling trees, flowering and decorative plants and shrubs and which may be conducted within a building or without, for the purpose of landscape construction.
    \item[NURSING HOME] A building with facilities for the care of children, the aged, infirm, or place of rest for those suffering bodily disorder. Said nursing home shall be licensed by the State Board of Health as provided for in M.S. SEC. 144.50, as it may be amended from time to time. \emph{(Ord. 537, effective 7-1-83)}
    \item[OBSTRUCTION] Any dam, wall, wharf, embankment, levee, dike, pile, abutment, projection, excavation, channel modification, culvert, building, wire, fence, stockpile, refuse, fill, structure or matter in, along, across, or projecting into any channel, watercourse, or regulatory flood plain which may impede, retard or change the direction of the flow of water, either in itself or by catching or collecting debris carried by the water. \emph{(Ord. 86, 2nd Series, effective 12-23-93)}
    \item[OFF-STREET LOADING SPACE] A space accessible from a street, alley or driveway for the use of commercial trucks or other vehicles while loading or unloading merchandise or materials.
    \item[OPEN SALES LOT (EXTERIOR STORAGE)] Any land used or occupied for the purpose of buying and selling any goods, materials, or merchandise and for the storing of same under the open sky prior to sale.
    \item[ORDINARY HIGHWATER MARK] A mark delineating the highest water level which has been maintained for a sufficient period of time to leave evidence upon the landscape. The ordinary highwater mark is commonly that point where the natural vegetation changes from predominantly aquatic to predominantly terrestrial. In areas where the ordinary highwater mark is not evident, setbacks shall be measured from the stream bank of the following water bodies that have permanent flow or open water: the main channel, adjoining side channels, backwaters and sloughs.
    \item[PARKING SPACE] A suitably surfaced and permanently maintained area on privately owned property either within or outside of a building of sufficient size to store one standard automobile.
    \item[PEDESTRIAN WAY] A public or private right-of-way across or within a block, to be used by pedestrians.
    \item[PLANNED UNIT DEVELOPMENT] A residential development whereby buildings are grouped or clustered in and around common open space areas in accordance with a prearranged site plan and where the common open space is owned by the homeowners and usually maintained by a homeowners association.
    \item[PLANNING COMMISSION] The Planning Commission of Crookston except when otherwise designated.
    \item[PREFABRICATED HOME] Non-mobile housing unit, the walls, floors and ceilings of which are constructed at a central factory and transported to a building site where final construction is completed, permanently affixing the unit to the site. \emph{(Ord. 537, effective 7-1-83)}
    \item[PRINCIPAL USE OR STRUCTURE] All uses or structures that are not accessory uses or structures. \emph{(Ord. 86, 2nd Series, effective 12-23-93)}
    \item[PROPERTY LINE] The legal boundaries of a parcel of property which may also coincide with a right-of-way line of a road, cartway, and the like.
    \item[PROPERTY OWNER] Any person, association or corporation having a freehold estate interest, leasehold interest extending for a term or having renewal options for a term in excess of one year, a dominant easement interest, or an option to purchase any of same, but not including owners or interests held for security purposes only.
    \item[PROTECTIVE COVENANTS] A contract entered into between private parties which constitutes a restriction of the use of a particular parcel of property.
    \item[PUBLIC LAND] Land owned or operated by city, school district, county, state or other governmental units.
    \item[PUBLIC WATER] A body of water capable of substantial beneficial public use. This shall be construed to mean any body of water which has the potential to support any type of recreational pursuit or water supply purpose. However, no lake, pond or flowage of less than 25 acres in size and no river or stream having a total drainage area less than two square miles need be regulated by the city for the purpose of these regulations. A body of water created by a private user where there was no previous shoreland, as defined herein, for a designated private use authorized by the Commissioner shall be exempt from the provisions of the statewide standards and criteria. \emph{(Ord. 537, effective 7-1-83)}
    \item[REACH] A hydraulic engineering term to describe a longitudinal segment of a stream or river influenced by a natural or man-made obstruction. In an urban area, the segment of a stream or river between two consecutive bridge crossings would most typically constitute a reach. \emph{(Ord. 86, 2nd Series, effective 12-23-93)}
    \item[RECLAIMED LAND] The improvement of land by deposition of material to elevate the grade.  Any parcel upon which 400 cubic yards or more of fill are deposited shall be considered as reclaimed land.
    \item[RECREATION, COMMERCIAL] Includes all uses such as bowling alleys, roller and skating rinks, driving ranges, and movie theaters that are privately owned and operated with the intention of earning a profit by providing entertainment for the public.
    \item[RECREATION EQUIPMENT] Play apparatus such as swing sets and slides, sandboxes, poles for nets, unoccupied boats and trailers not exceeding 20 feet in length, picnic tables, lawn chairs, barbecue stands, and similar equipment or structures but not including tree houses, swimming pools, play houses exceeding 25 square feet of floor area, or sheds utilized for storage of equipment.
    \item[RECREATION, PUBLIC] Includes all uses such as tennis courts, ball fields, picnic areas, and the like that are commonly provided for the purpose at parks, playgrounds, community centers, and other sites owned and operated by a unit of government for the purpose of providing recreation.
    \item[REGIONAL FLOOD] A flood which is representative of large floods known to have occurred generally in Minnesota and reasonably characteristic of what can be expected to occur on an average frequency in the magnitude of the 100-year recurrence interval. REGIONAL FLOOD is synonymous with the term “base flood” used in the Flood Insurance Study.
    \item[REGISTERED LAND SURVEY] A survey map of registered land designed to simplify a complicated metes and bounds description, designating the same into a tract or tracts of a Registered Land Survey Number (see M.S. \textsection 508.47, as it may be amended from time to time). \emph{(Ord. 537, effective 7-1-83)}
    \item[REGULATORY FLOOD PROTECTION ELEVATION] The elevation no lower than one foot above the elevation of the regional flood plus any increases in flood elevation caused by encroachments on the flood plain that result from designation of a floodway. \emph{(Ord. 86, 2nd Series, effective 12-23-93)}
    \item[ROAD] A public right-of-way affording primary access by pedestrians and vehicles to abutting properties, whether designated as a street, highway, thoroughfare, parkway, throughway, road, avenue, boulevard, land, place or however otherwise designated. Ingress and egress easements shall not be considered roads.
    \item[ROOMING HOUSE] A building designed for or used as a single-family or two-family dwelling, all or a portion of which contains rooming units which accommodate three or more persons who are not members of the keeper’s family.  Rooms, or meals, or both, are provided for compensation on a periodic payment basis. \emph{(Ord. 537, effective 7-1-83)}
    \item[SATELLITE DISH ANTENNA] An outside parabolic antenna used or useful for the reception of communication signals transmitted by satellite. \emph{(Ord. 27, 2nd Series, effective 3-20-86)}
    \item[SELECTIVE CUTTING] The removal of single scattered trees. \emph{(Ord. 537, effective 7-1-83)}
    \item[SIGN] Any words, lettering, parts of letters, figures, numerals, phrases, sentences, emblems, devices, designs, trade names, or trademarks by which anything is made known which are visible from any public highway, street, or public property and used to attract attention.
    \item[SIGN, ADVERTISING] Any sign making anything known but excluding real estate signs, identification signs, home occupation signs, memorial signs, and public signs.
    \item[SIGN AWNING] Any sign affixed to or contained as a part of a hood or canopy constructed of flexible, translucent or fabric type material extending over all or a part of the area immediately adjacent to a face of a building and supported from the building. \emph{(Ord. 85, 2nd Series, effective 7-17-93)}
    \item[SIGN, BILLBOARD or POSTER PANEL] Any sign or advertising used on an outdoor display by painting, posting, or affixing on any surface of a picture, emblem, words, figures, numbers, or lettering for the purpose of making anything known, the sign or its structure being remote from the origin or point of sale of the matter advertised.
    \item[SIGN, COMBINATION ROOF AND PROJECTING] A sign or combination of signs of projecting and roof sign definition, a portion of the sign or signs being anchored to the roof and/or parapet wall of the building, the sign or combination of signs conveying the total message.  If a combination of signs, each sign must convey an integral part of one message.
    \item[SIGN, FLASHING] Any illuminated sign on which the artificial light is not maintained constant in intensity and color at all times when the sign is in use. \emph{(Ord. 537, effective 7-1-83)}
    \item[SIGN, GROUND] A sign affixed to or erected directly upon the ground or upon a base or foundation on or in the ground.
    \item[SIGN, HOME OCCUPATION] Any sign announcing the name of and/or logo associated with a home occupation.
    \item[SIGN, IDENTIFICATION] Any sign announcing the name of and/or logo associated with an organization, business (except home occupation) or place. \emph{(Ord. 85, 2nd Series, effective 7-17-93)}
    \item[SIGN, ILLUMINATED] Any sign upon which artificial light is directed or which has an interior light source.
    \item[SIGN, MARQUEE] A sign affixed to or contained as a part of any hood or canopy over the sidewalk and entrance to stores, buildings and places of public assembly extending wholly or in part across the sidewalk and supported from the building. \emph{(Ord. 537, effective 7-1-83)}
    \item[SIGN, MEMORIAL] Any sign announcing the name of a building and date of erection or similar information when cut into any masonry surface or when constructed of bronze or other non-combustible material and attached to the building.  Letters shall not exceed six inches in height when mounted or cut at an elevation less than eight feet above the sidewalk grade immediately below.  Temporary signs denoting the architect, engineer, public utility, or contractor are memorial signs when placed upon the work during construction or maintenance. \emph{(Ord. 85, 2nd Series, effective 7-17-93)}
    \item[SIGN, OFF-SITE DIRECTIONAL] A sign erected remote from the premises to which it has reference usually containing an arrow or other directional information to assist the public in locating the function to which it has reference.
    \item[SIGN, POLE] A sign constructed of metal, plastic or other approved material affixed to or erected upon a single or double metal pole, or pole of other generally acceptable materials which conform to requirements of the sign code. \emph{(Ord. 537, effective 7-1-83)}
    \item[SIGN, PORTABLE] A sign, flashing, illuminated, or non-illuminated, so designed as to be movable from one location to another and which is not permanently attached to ground, sales device, or structure. \emph{(Ord. 537, effective 7-1-83)}
    \item[SIGN, PROJECTING] A sign or poster that may be affixed to the front, rear, or side wall of any building and extending over the sidewalk, the message conveyed by the sign being perpendicular with or at an angle or angles to the wall to which it is affixed.  Projecting illuminated or non-illuminated signs shall, for the purpose of this chapter, be divided into four classifications:
        \begin{enumerate}[{\indent\indent}1)]
            \item Projecting signs which are affixed directly to the building wall with no visible sign structure other than the sign itself;
            \item Projecting signs which are supported from above by a visible projecting rod or sign structure affixed to the building but which are free-swinging at the bottom;
            \item Projecting signs supported by visible additional sign structure at two or more elevations with no part of the sign proper contacting the building to which it is affixed;
            \item Projecting signs on corner buildings which signs are anchored at approximately 45 degrees from the building corner at street intersections, designed to be viewed equally from four directions.
        \end{enumerate}
    \item[SIGN, PUBLIC] Any sign placed by governmental subdivisions or their agents announcing pedestrian and vehicular traffic directions, traffic controls, non-commercial activities, legal notices, railroad crossings, and temporary dangers or emergencies. \emph{(Ord. 85, 2nd Series, effective 7-17-93)}
    \item[SIGN, REAL ESTATE] Any sign with an area of no more than six square feet announcing the sale or lease of the real property upon which it is located while the property is actually for sale or lease. \emph{(Ord. 85, 2nd Series, effective 7-17-93)}
    \item[SIGN, ROOF] A sign erected, constructed, or maintained on the roof of any building, not to be so construed as to include an upward extension of a wall or projecting sign.
    \item[SIGN, STRUCTURE] The foundation, supports, uprights, bracing and framework for a sign, including the sign surface itself.  In the case of a sign painted on the wall of a building, the sign surface is the entire sign structure.
    \item[SIGN, TEMPORARY] A sign placed in a manner as not to be solidly affixed to any building, structure or land and used for advertising or leasing property or denoting architecture, engineer, public utility, or contractor.
    \item[SIGN, WALL] A sign or poster that may be affixed to or painted on the front, rear, or side wall of any business building to which it has reference, the message conveyed by the sign being parallel with the wall to which it is affixed.
    \item[SOLAR ACCESS SPACE] That airspace above all lots within the district necessary to prevent any improvement or tree located on said lots from casting a shadow upon any solar device located within said zone greater than the shadow cast by a hypothetical vertical wall ten feet high located along the property lines of said lots between the hours of 9:30 a.m. and 2:30 p.m., Central Standard Time on December 21; provided, however, this chapter shall not apply to any improvement or tree which casts a shadow upon a solar device at the time of the installation of said device, or to vegetation existing at the time of installation of the solar device.
    \item[SOLAR COLLECTOR] A device, or combination of devices, structure, or part of a device or structure that transforms direct solar energy into thermal, chemical or electrical energy and that contributes significantly to a structure’s energy supply.
    \item[SOLAR ENERGY] Radiant energy (direct, diffuse, and reflected) received from the sun.
    \item[SOLAR ENERGY SYSTEM] A complete design or assembly consisting of a solar energy collector, an energy storage facility (where used), and components to the distribution of transformed energy (to the extent they cannot be used jointly with a conventional energy system). To qualify as a solar energy system, the system must be permanently located for not less than 90 days in any calendar year beginning with the first calendar year after completion of construction. Passive solar energy systems are included in this definition but not to the extent that they fulfill other functions such as structural and recreational.
    \item[SOLAR SKYSPACE] The space between a solar energy collector and the sun which must be free of obstructions that shade the collector to an extent which precludes its cost effective operation.
    \item[SOLAR SKYSPACE EASEMENT] A right, expressed as an easement, covenant, condition, or other property interest in any deed or other instrument executed by or on behalf of any landowner, which protects the solar skyspace of an actual, proposed, or designated solar energy collector at a described location by forbidding or limiting activities or land uses that interfere with access to solar energy. The solar skyspace must be described as the three-dimensional space in which obstruction is prohibited or limited, or as the times of day during which direct sunlight to the solar collector may not be obstructed, or as a combination of the two methods.
    \item[SOLAR STRUCTURE] A structure designed to utilize solar energy as an alternate for, or supplement to, a conventional energy system.
    \item[STREET] A public right-of-way which affords primary means of access to abutting property, and shall also include avenue, highway, road, or way.
    \item[STREET, COLLECTOR] A street which serves or is designed to serve as a trafficway for a neighborhood or as a feeder to a major street.
    \item[STREET, LOCAL] A street intended to serve primarily as an access to abutting properties.
    \item[STREET, MAJOR OR THOROUGHFARE] A street which serves or is designed to serve, heavy flows of traffic and which is used primarily as a route for traffic between communities and/or other heavy traffic generating areas.
    \item[STREET PAVEMENT]The wearing or exposed surface of the roadway used by vehicular traffic.
    \item[STREET WIDTH] The width of the right-of-way, measured at right angles to the centerline of the street.
    \item[STORY] That portion of a building included between the surface of any floor and the surface of the floor next above, including below ground portions of earth sheltered buildings. \emph{(Ord. 537, effective 7-1-83)}
    \item[STRUCTURAL ALTERATION] Any change, other than incidental repairs, which would prolong or modify the life of the supporting members of a building, such as bearing walls, columns, beams, girders or foundations.
    \item[STRUCTURE] Anything constructed or erected on the ground or attached to the ground or on-site utilities, including, but not limited to, buildings, factories, sheds, detached garages, cabins, manufactured homes, travel trailers/vehicles not meeting the exemption criteria specified in SEC. 152.098(C)(1), and other similar items. \emph{(Ord. 86, 2nd Series, effective 12-23-93)}
    \item[SUBDIVISION] The division or redivision of a lot, tract, or parcel of land into two or more lots either by plat or by metes and bounds description.
    \item[TOWNHOUSE] A single-family building attached by party walls with other single family buildings, and oriented so that all exits open to the outside.
    \item[TOXIC AND HAZARDOUS WASTES] Waste materials including, but not limited to, poisons, pesticides, herbicides, acids, caustics, pathological wastes, radioactive materials, flammable or explosive materials and similar harmful chemicals and wastes which require special handling and must be disposed of in a manner which conserves the environment and protects the public health and safety.
    \item[USE] The purpose or activity for which the land or building thereon is designated, arranged or intended, or for which it is occupied, utilized or maintained.
    \item[USE, ACCESSORY] A use subordinate to and serving the principal use or structure on the same lot and customarily incidental thereto. \emph{(Ord. 537, effective 7-1-83)}
    \item[USE, CONDITIONAL] See \emph{CONDITIONAL USE}.
    \item[USE, INTEGRATED] A use which does not comply with all the regulations of this chapter or any amendments hereto governing the zoning district in which the use is located, and is incidental to the principal use. \emph{(Ord. 60, 2nd Series, effective 3-22-90)}
    \item[USE, NONCONFORMING] Use of land, buildings or structures legally existing at the time of adoption of this chapter which does not comply with all the regulations of this chapter or any amendment hereto governing the zoning district in which the use is located.
    \item[USE, PERMITTED] A public or private use which itself conforms with the purposes, objectives, requirements, regulations and performance standards of a particular district.
    \item[USE, PRINCIPAL] The main use of land or buildings as distinguished from subordinate or accessory uses.  A PRINCIPAL USE may be either permitted or conditional.
    \item[VARIANCE] A modification of a specific permitted development standard required in an official control including this chapter to allow an alternative development standard not stated as acceptable in the official control, but only as applied to a particular property for the purpose of alleviating a hardship, practical difficulty or unique circumstance as defined and elaborated upon in the city’s planning and zoning regulations. \emph{(Ord. 86, 2nd Series, effective 12-23-93)}
    \item[WETLAND] Land which is annually subject to periodic or continual inundation by water and commonly referred to as a bog, swamp, or marsh.
    \item[YARD] A required open space on a lot which is unoccupied and unobstructed by a structure from its lowest level to the sky except as permitted in this chapter. The yard extends along the lot line at right angles to the lot line to a depth or width specified in the setback regulations for the zoning district in which the lot is located.
    \item[YARD, FRONT] A yard extending along the full width of the front lot line between side lot lines and extending from the abutting street right-of-way line to depth required in the setback regulations for the zoning district in which the lot is located.
    \item[YARD, REAR] The portion of the yard on the same lot with the principal building located between the rear line of the building and the rear lot line and extending for the full width of the lot.
    \item[YARD, SIDE] The yard extending along the side lot line between the front yard and rear yards to a depth or width required by setback regulations for the zoning district in which the lot is located.
    \item[ZONING ADMINISTRATOR] The duly appointed person charged with enforcement of this chapter.
    \item[ZONING AMENDMENT] A change authorized by the city either in the allowed use within a district or in the boundaries of a district.
    \item[ZONING DISTRICT] An area or areas within the limits of the city for which the regulations and requirements governing use are uniform as defined by this chapter. \emph{(Ord. 537, effective 7-1-83)}
\end{description}
\emph{(‘83 Code, SEC. 11.03)}
\section{Application and Interpretation}
\subsection{}
In their interpretation and application, the provisions of this chapter shall be held to be the minimum requirements for the promotion of the public health, safety, and welfare.
\subsection{}
Where the conditions imposed by any provision of this chapter are either more restrictive or less restrictive than comparable conditions imposed by any other law, city code provision, statute, resolution, or regulation of any kind, the regulations which are more restrictive or which impose higher standards or requirements shall prevail.
\subsection{}
Except as in this chapter specifically provided, no structure shall be erected, converted, enlarged, reconstructed, or altered; and no structure or land shall be used for any purpose nor in any manner which is not in conformity with this chapter.\\
\emph{(‘83 Code, SEC. 11.10, Subd. 1)}
\section{Existing Lots}
A lot or parcel of land in a residential district which was of record as a separate lot or parcel in the office of the Polk County Recorder or Registrar of Titles, on or before June 9, 1964 may be used for single-family detached dwelling purposes provided it can be demonstrated that safe and adequate sewage treatment systems can be installed to serve the permanent dwelling.\\
\emph{(‘83 Code, SEC. 11.10, Subd. 3)}
\section{Separability}
It is hereby declared to be the intention that the several provisions of this chapter are separable in accordance with the following:
\begin{enumerate}[{\indent}A)]
    \item If any court of competent jurisdiction shall judge any provisions of this chapter to be invalid, the judgment shall not affect any other provision of this chapter not specifically included in the judgment.
    \item If any court of competent jurisdiction shall judge invalid the application of any provision of this chapter to a particular property, building, or structure, the judgment shall not affect other property, buildings, or structures.
\end{enumerate}
\emph{(‘83 Code, SEC. 11.10, Subd. 2)}
\section{Zoning Coordination}
Any zoning district change on land adjacent to or across a public right-of-way from an adjoining community shall be referred to the Planning Commission and the adjacent community for review and comment prior to action by the Council granting or denying the zoning district classification change.  A period of at least ten days shall be provided for receipt of comments; the comments shall be considered as advisory only.\\
\emph{(‘83 Code, SEC. 11.10, Subd. 6)  (Ord. 537, effective 7-1-83)}
\section{Foundations}
Any structure designed to be used as a dwelling shall be placed on a foundation constructed of masonry, concrete or treated wood. Constructed and installed as required by the Minnesota State Building Code.\\
\emph{(‘83 Code, SEC. 11.10, Subd. 7)  (Ord. 27, 2nd Series, effective 3-20-86)}\\

\begin{center}
    \emph{\textbf{\LARGE{ZONING DISTRICTS}}}
\end{center}

\setcounter{section}{19}
\section{Division of City into Districts}
The zoning districts are so designed as to assist in carrying out the intents and purposes of the Comprehensive Plan and are based upon the Comprehensive Plan which has the purpose of protecting the public health, safety, convenience and general welfare. For the purposes of this chapter, the city is hereby divided into the following zoning districts.
\begin{center}
    \begin{tabular}{|c|p{5cm}|}
    \hline
    \textbf{Symbol} & \textbf{Name}\\
    \hline
    FR & Farm Residence\\
    \hline
   R-1 & Single Family Residential\\
    \hline
   R-2 & One and Two Family Residential\\
    \hline
   R-3 & Multi-Family Residential\\
    \hline
   C-1 & Central Business District\\
    \hline
   C-2 & Highway Commercial\\
    \hline
   C-3 & Shopping Center\\
    \hline
   I-1 & Heavy Industrial\\
    \hline
   I-2 & Light Industrial\\
    \hline
    IN & Institutional\\
    \hline
    FP & Floodplain\\
    \hline
\end{tabular}
\end{center}
\emph{(‘83 Code, SEC. 11.20, Subd. 1)}
\section{Zoning Map}
\subsection{}
The location and boundaries of the districts established by this chapter are set forth on the Official Zoning Map which is hereby incorporated as part of this chapter and which is on file with the City Administrator’s Office.
\subsection{}
District boundary lines recorded on the zoning map are intended to follow lot lines, the centerlines of streets or alleys, the centerlines of streets or alleys projected, railroad rights-of-way lines, the center of watercourses or the corporate limit lines as they exist at the time of the enactment of this chapter.  The Floodplain District is an exception and includes the actual area subject to inundation.
\subsection{}
Whenever any street, alley or other public way is vacated, the zoning district adjoining that of the vacated street, alley or public way shall be automatically extended to the center of the vacated area and all area included therein shall be then and henceforth subject to all regulations of the extended district.
\subsection{}
No annexation petition shall be considered unless and until a hearing has also been petitioned for placing the annexed territory in a zoning district or districts.  No building permits shall be issued in annexed territory until the hearing has been held and assigned a zoning district.
\subsection{}
It shall be the responsibility of the City Engineer to maintain and amend the zoning map.  The Zoning Administrator shall make or cause to have made any corrections or amendments to said map after all of the procedures outlined in this chapter for the making of the revisions or amendments shall have been followed by the Planning Commission and the Council.
\subsection{}
Amendments to this zoning map shall be recorded on the map within 15 days after adoption by the Council.  The copy of the official zoning map shall be kept on file in the office of the City Engineer and shall be open to public inspection at all times during which the office of the Zoning Administrator is customarily open.\\
\emph{(‘83 Code, SEC. 11.20, Subd. 2)  (Ord. 537, effective 7-1-83)}\\

\begin{center}
    \emph{\textbf{\LARGE{RESIDENTIAL DISTRICTS}}}
\end{center}

\setcounter{section}{34}
\section{Farm Residence (FR)}
\subsection{Purpose}
The major purpose of this district is to allow existing agricultural and conservancy areas in the outlying parts of the city that does not have central sewer services. Limited residential development will be allowed in this district and clustering of housing units will be encouraged.
\subsection{Permitted Uses}
\begin{enumerate}[{\indent}1)]
    \item Commercial agriculture and horticulture.
    \item Farm buildings and structures.
    \item Single-family residential structures.
    \item Farm drainage and irrigation systems.
    \item Roadside stands for the sale of agricultural products.
    \item Historic sites.
    \item Public recreation.
    \item Essential services - telephone, telegraph, power lines and necessary appurtenant equipment and structures.
    \item Signs subject to the standards in SEC. 152.177.
    \item Churches, schools.
    \item City buildings, including police and fire stations.
    \item Solar structures. \emph{(Ord. 537, effective 7-1-83)}
    \item Bed and breakfast inns. \emph{(Ord. 37, 2nd Series, effective 9-16-86)}
\end{enumerate}
\subsection{Accessory Uses}
\subsubsection{}
Any incidental machinery, structure or buildings necessary to the conduct of agricultural, single-family residential, and other permitted uses subject to standards set forth in SEC. 152.167(A)(1).
\subsubsection{}
Private garages, carports, screen houses, swimming pools and storage buildings for use of occupants of the principal structures subject to standards set forth in SEC. 152.167(A)(1).\\
\emph{(Ord. 42, 2nd Series, effective 4-23-87)}
\subsection{Conditional Uses}
\begin{enumerate}[{\indent}1)]
    \item Multi-family residential.
    \item Cemeteries.
    \item Home occupations.
    \item Agricultural products and livestock processing plants.
    \item Hobby farms and stables.
    \item Kennels.
    \item Resorts.
    \item Nursery and garden supplies.
    \item Mining, sand and gravel operations.
    \item Wind energy conversion systems.
\end{enumerate}
\subsection{Performance Standards}
\subsubsection{Height Regulations}
\begin{enumerate}[{\indent}a)]
    \item The maximum height of all buildings shall not exceed two and one-half stories or 35 feet.
    \item This height limitation shall not apply to grain elevators, silos, windmills, elevator lags, cooling towers, water towers, chimneys, and smokestacks, church spires, or wind energy conversion systems.
\end{enumerate}
\subsubsection{Front Yard Regulations}
\begin{enumerate}[{\indent}a)]
    \item Required setback distances from right-of-way.
        \begin{center}
        \begin{tabular}{|c|c|}
            \hline
            \textbf{Road Right-of-Way} & \textbf{Road Classification}\\
            \hline
            70 ft. & State Highway\\
            \hline
            50 ft. & County Road\\
            \hline
            25 ft. & City Street\\
            \hline
        \end{tabular}
        \end{center}
    \item Where a lot is located at the intersection of two or more roads or highways, there shall be a front yard setback on each road or highway side of each corner lot.
\end{enumerate}
\subsubsection{Side and Rear Yard Regulations}
There shall be a side yard width of not less than ten feet on each side of the building and a rear yard of not less than 50 feet.
\subsubsection{Lot Width and Depth Regulations}
\begin{enumerate}[{\indent}a)]
    \item For farm dwellings - none.
    \item For non-farm single-family residences - minimum width of 200 feet and depth of 200 feet.
\end{enumerate}
\subsubsection{Lot Area Regulations}
\begin{enumerate}[{\indent}a)]
    \item For farm residences - none.
    \item For non-farm single-family residences - one acre.
\end{enumerate}
\subsubsection{Location of Structures}
Structures shall be so located on each lot as to permit resubdivision if and when central sewer and water systems become available.
\subsubsection{General Requirements}
Additional requirements for parking, signs, sewage systems and other regulations are set forth in SEC. 152.155 through SEC. 152.181.\\
\emph{(‘83 Code, SEC. 11.25)}
\section{Single-Family Residential (R-1)}
\subsection{Purpose}
The major purpose of this district is to allow low density single-family dwelling units in the developing portions of the city where central sewer and water is available.
\subsection{Permitted Uses}
\begin{enumerate}[{\indent}1)]
    \item Single-family residential structures.
    \item Public recreation including parks and playgrounds.
    \item Historic sites.
    \item Churches, chapels, temples and synagogues including parish houses.
    \item Elementary schools.
    \item City buildings including police and fire stations.
    \item Signs subject to standards in SEC. 152.177.
    \item Essential services - telephone, telegraph, and power lines and necessary appurtenant equipment and structures.
    \item Solar structures.
    \item Home occupations.
\end{enumerate}
\emph{(Ord. 537, effective 7-1-83)}
\subsection{Accessory Uses}
\begin{enumerate}[{\indent}1)]
    \item Any incidental structure or building necessary to the conduct of a permitted use subject to standards set forth in SEC. 152.167(A)(1).
    \item Private garages, carports, screen houses, swimming pools and storage buildings for use of occupants of the principal structures subject to standards set forth in SEC. 152.167(A)(1).
\end{enumerate}
\emph{(Ord. 42, 2nd Series, effective  4-23-87)}
\subsection{Conditional Uses}
\begin{enumerate}[{\indent}1)]
    \item Junior and senior high schools.
    \item Lodging and rooming houses.
    \item Cemeteries.  
    \item Local neighborhood commercial.  
    \item Wind energy conversion systems.
\end{enumerate}
\subsection{Performance Standards}
\subsubsection{Height Regulations}
The maximum height of all buildings shall not exceed two and one-half stories or 35 feet.
\subsubsection{Front Yard Regulations}
\begin{enumerate}[{\indent}a)]
    \item Required setback distances.
        \begin{center}
        \begin{tabular}{|c|c|}
            \hline
            \textbf{Road Right-of-Way} & \textbf{Road Classification}\\
            \hline
            70 ft. & State Highway\\
            \hline
            50 ft. & County Road\\
            \hline
            25 ft. & City Street\\
            \hline
        \end{tabular}
        \end{center}
        \emph{(Ord. 537, effective 7-1-83)}
    \item Where a lot is located at the intersection of two or more roads or highways, there shall be a front yard setback on each road or highway side of the corner lot. Upon proper application therefor, the Zoning Administrator may authorize, in writing, a yard of not less than 15 feet on not more than one yard of a corner lot if a reduced yard is appropriate under the circumstances. In determining whether an application for a reduced corner lot yard is appropriate, the Zoning Administrator shall consider all relevant information available, including, but not limited to, the extent to which maintenance of maximum setback is desirable given the present and anticipated use of the public ways adjoining the lot in question and the distance between the proposed structure on the corner lot and the principal structure on the adjacent lot (a distance less than that required for a rear yard shall not be allowed).\\
        \emph{(Ord. 45, 2nd Series, effective 7-18-87)}
\end{enumerate}
\subsubsection{Side and Rear Yard Regulations}
\begin{enumerate}[{\indent}a)]
    \item Side yard - 5 feet.  
    \item Rear yard - 18 feet.
\end{enumerate}
\subsubsection{Lot Area}
The minimum lot size shall be 7,500 square feet.
\subsubsection{Lot Width and Depth Regulations}
\begin{enumerate}[{\indent}a)]
    \item Lot width - 70 feet.  
    \item Lot depth - 100 feet.
\end{enumerate}
\subsubsection{General Regulations}
Additional regulations for parking, signs, sewage systems and other regulations are set forth in SEC. 152.155 through SEC. 152.181.
\subsubsection{Dwelling Structures}
Dwelling structures shall meet the following minimum standards:
\begin{enumerate}[{\indent}a)]
    \item Exceed 24 feet in width.
    \item Have a minimum floor area of 800 square feet.
    \item Placed on a permanent foundation.
    \item Meet all other requirements of law and city code provisions.
\end{enumerate}
\emph{(‘83 Code, SEC. 11.26)}
\section{One- and Two-Family Residential (R-2)}
\subsection{Purpose}
The major purpose of this district is to allow one- or two-family residential dwelling units at medium density in or near major activity centers or highways.
\subsection{Permitted Uses}
\begin{enumerate}[{\indent}1)]
    \item Any use permitted in the R-1 District.  
    \item Two-family dwelling units. \emph{(Ord. 537, effective 7-1-83)}
    \item Bed and breakfast inns. \emph{(Ord. 37, 2nd Series, effective 9-16-86)}
\end{enumerate}
\subsection{Accessory Uses}
Any accessory use permitted in the R-1 District.
\subsection{Conditional Uses}
\begin{enumerate}[{\indent}1)]
    \item Any conditional uses permitted in the R-1 District.
    \item Four-family dwellings.
    \item Townhouses and condominiums.  
    \item Mobile home parks.
\end{enumerate}
\subsection{Performance Standards}
\subsubsection{Height Regulations}
The maximum height of all buildings shall not exceed two and one-half stories or 35 feet.
\subsubsection{Front Yard Regulations}
\begin{enumerate}[{\indent}a)]
    \item Required setback distances.
        \begin{center}
        \begin{tabular}{|c|c|}
            \hline
            \textbf{Road Right-of-Way} & \textbf{Road Classification}\\
            \hline
            70 ft. & State Highway\\
            \hline
            50 ft. & County Road\\
            \hline
            25 ft. & City Street\\
            \hline
        \end{tabular}
        \end{center}
        \emph{(Ord. 537, effective 7-1-83)}
    \item Where a lot is located at the intersection of two or more roads or highways, there shall be a front yard setback on each road or highway side of the corner lot. Upon proper application therefor, the Zoning Administrator may authorize, in writing, a yard of not less than 15 feet on not more than one yard of a corner lot if a reduced yard is appropriate under the circumstances. In determining whether an application for a reduced corner lot yard is appropriate, the Zoning Administrator shall consider all relevant information available, including, but not limited to, the extent to which maintenance of maximum setback is desirable given the present and anticipated use of the public ways adjoining the lot in question and the distance between the proposed structure on the corner lot and the principal structure on the adjacent lot (a distance less than that required for a rear yard shall not be allowed).\\
        \emph{(Ord. 45, 2nd Series, effective 7-18-87)}
\end{enumerate}
\subsubsection{Side and Rear Yard Regulations}
\begin{enumerate}[{\indent}a)]
    \item Side yard - 4 feet.
    \item Rear yard - 18 feet.
\end{enumerate}
\subsubsection{Lot Area}
\begin{enumerate}[{\indent}a)]
    \item Single-family dwelling unit - 6,000 square feet.
    \item Two-family dwelling unit - 6,000 square feet per unit.
\end{enumerate}
\subsubsection{Lot Width and Depth}
\begin{enumerate}[{\indent}a)]
    \item Lot width - 50 feet.
    \item Lot depth - 100 feet.
\end{enumerate}
\subsubsection{General Regulations}
Additional requirements for parking, signs, sewage systems and other items are set forth in SEC. 152.155 through SEC. 152.181.
\subsubsection{Dwelling Structure Standards}
Dwelling structures shall meet the following minimum standards:
\begin{enumerate}[{\indent}a)]
    \item Exceed 24 feet in width.
    \item Have a minimum floor area of 800 square feet.
    \item Placed on a permanent foundation.
    \item Meet all other requirements of law and city code provisions.
\end{enumerate}
\emph{(‘83 Code, SEC. 11.27)}
\section{Multi-Family Residential (R-3)}
\subsection{Purpose}
The major purpose of this district is to allow multi-family dwelling units including apartments and townhouses at or adjacent to major commercial concentrations or highways.
\subsection{Permitted Uses}
\begin{enumerate}[{\indent}1)]
    \item Any use permitted in the R-2 District.
    \item Townhouses.
    \item Apartment buildings.
\end{enumerate}
\subsection{Accessory Uses}
\begin{enumerate}[{\indent}1)]
    \item Any accessory use permitted in the R-2 District.
    \item Putting greens, shuffle board courts, picnic areas, swimming pools, community buildings and similar recreational or service areas for the use of the residents of the buildings.
\end{enumerate}
\subsection{Conditional Uses}
Any use permitted in the R-2 District.
\subsection{Performance Standards}
\subsubsection{Height Regulations}
The maximum height of all buildings shall not exceed three stories or 40 feet.
\subsubsection{Front Yard Regulations}
\begin{enumerate}[{\indent}a)]
    \item Required setback distances.
        \begin{center}
        \begin{tabular}{|c|c|}
            \hline
            \textbf{Road Right-of-Way} & \textbf{Road Classification}\\
            \hline
            70 ft. & State Highway\\
            \hline
            50 ft. & County Road\\
            \hline
            25 ft. & City Street\\
            \hline
        \end{tabular}
        \end{center}
        \emph{(Ord. 537, effective 7-1-83)}
    \item Where a lot is located at the intersection of two or more roads or highways, there shall be a front yard setback on each road or highway side of the corner lot.  Upon proper application therefor, the Zoning Administrator may authorize, in writing, a yard of not less than 15 feet on not more than one yard of a corner lot if a reduced yard is appropriate under the circumstances. In determining whether an application for a reduced corner lot yard is appropriate, the Zoning Administrator shall consider all relevant information available, including, but not limited to, the extent to which maintenance of maximum setback is desirable given the present and anticipated use of the public ways adjoining the lot in question and the distance between the proposed structure on the corner lot and the principal structure on the adjacent lot (a distance less than that required for a rear yard shall not be allowed).\\
        \emph{(Ord. 45, 2nd Series, effective 7-18-87)}
\end{enumerate}
\subsubsection{Side and Rear Yard Regulations}
\begin{enumerate}[{\indent}a)]
    \item Side yard - 15 feet.
    \item Rear yard - 35 feet.
\end{enumerate}
\subsubsection{Lot Area}
\begin{enumerate}[{\indent}a)]
    \item The following minimum lot areas shall be required for each multi-family unit:
        \begin{center}
        \begin{tabular}{|l|c|}
            \hline
            One-bedroom unit & 2,000 square feet\\
            \hline
            Two-bedroom unit & 2,600 square feet\\
            \hline
            Three-bedroom unit & 2,700 square feet\\
            \hline
            Four or more bedrooms & 3,000 square feet\\
            \hline
        \end{tabular}
        \end{center}
    \item No multi-family building shall be erected that provides less than 7,500 square feet of lot area.
\end{enumerate}
\subsubsection{Lot Coverage}
The maximum lot coverage of multi-family units including necessary buildings shall not exceed 35\%.
\subsubsection{Parking Sign Requirements}
Other general requirements for parking, signs, and the like, are set forth in SEC. 152.155 through SEC. 152.181.
\subsubsection{Dwelling Structure Standards}
Dwelling structures shall meet the following minimum standards:
\begin{enumerate}[{\indent}a)]
    \item Exceed 24 feet in width.
    \item Have a minimum floor area of 800 square feet.
    \item Placed on a permanent foundation.
    \item Meet all other requirements of law and city code provisions.
\end{enumerate}
\emph{(‘83 Code, SEC. 11.28)  (Ord. 537, effective 7-1-83)}

\begin{center}
    \emph{\textbf{\LARGE{BUSINESS DISTRICTS}}}
\end{center}

\setcounter{section}{49}
\section{Central Business District (C-1)}
\subsection{Purpose}
The purpose of this district is to encourage the continuation of a viable downtown area by allowing retail, service, office and entertainment facilities as well as public and semi-public uses.  In addition, residential uses will be allowed to locate above the commercial establishments.
\subsection{Permitted Uses}
Commercial establishments offering merchandise or services to the general public in return for compensation. The establishment to include, but not be limited to, the following:
\begin{enumerate}[{\indent}1)]
    \item Retail establishments such as groceries, bakery, department stores, hardware, drug, clothing and furniture stores.
    \item Personal services such as laundry, barber, shoe repair shop and photography studios.
    \item Existing drinking establishments, including restaurants, cafes and supper clubs.
    \item Professional services such as medical and dental clinics, architects and attorneys fees.
    \item Repair services such as jewelry and radio and television repair shops.
    \item Banks, finance, insurance and real estate services.
    \item Entertainment and amusement services such as motion picture theaters, bowling alleys, art galleries.
    \item Lodging services such as hotel and motel.
    \item Public and semi-public buildings such as post office, city hall, fire and police stations.
    \item Private clubs.
    \item Hospitals and medical centers.
    \item Automobile parking lots, parking garages, bus stations.
    \item Solar structures. \emph{(Ord. 537, effective 7-1-83)}
    \item Churches, chapels, temples, and synagogues, including parish houses. \emph{(Ord. 57, 2nd Series, effective 7-22-89)}
\end{enumerate}
\subsection{Conditional Uses}
\begin{enumerate}[{\indent}1)]
    \item Apartments, provided they are located above the first floor level.
    \item Auto body shops.
    \item On and off-sale liquor establishments.
    \item Light industry such as printing shops that require direct contact with the public
    \item Wholesaling.
    \item Integrated uses.
    \item Extension or relocation of non-conforming uses.
    \item Other uses which in the opinion of the Planning Commission and the Council are of the same general character as the permitted uses and which will not be detrimental to the Central Business District or any other district.
\end{enumerate}
\emph{(Ord. 60, 2nd Series, effective 3-22-90)}
\subsection{Accessory Uses}
Uses incidental to the principal uses such as off-street parking and loading and unloading areas, storage of merchandise.
\subsection{Performance Standards}
\subsubsection{Height Regulations}
The maximum height of any building shall be four stories or 45 feet.
\subsubsection{Front Yard Regulations}
\begin{enumerate}[{\indent}a)]
    \item There shall be a front yard setback having a depth of not less than ten feet except in a block where two or more structures have been built facing the same street, the setback for the remaining lots in that block fronting on the same street shall be determined by the average setback of existing buildings.
    \item Where a lot is located at the intersection of two or more roads or highways, there shall be a front yard setback on each road or highway side of each corner lot.  No accessory buildings shall project beyond the front yard of either road.
\end{enumerate}
\subsubsection{Side and Rear Yard Regulations}
\paragraph{Side Yard}
\begin{enumerate}[{\indent}1)]
    \item No side yard is required on commercial lots where no openings are provided in the walls of commercial buildings adjacent to the interior lot lines.
    \item There shall be a side yard on the street side of all corner lots which shall have a width of not less than 50\% of the front yard depth required for the adjacent lot to the rear of the corner lot, when the adjacent lot fronts on the side of the corner lot.  In no case should the side yard be less than 15 feet.
\end{enumerate}
\paragraph{Rear Yard}
The minimum rear yard shall be 15 feet.
\subsubsection{Lot Area and Coverage Standards}
\paragraph{Lot Area}
None.
\paragraph{Lot Coverage}
No restrictions except that space shall be reserved either inside or outside the building for the loading or unloading of goods, materials and merchandise on every commercial lot. The space shall not be less than 15 feet in width for every 50 feet of building width or fraction thereof, nor less than 30 feet in length, nor less than 15 feet in height, and shall be provided with access to a street unless provided otherwise by the means as customer or employee parking space on the same premises. The location, dimensions, and means of ingress and egress of the loading and unloading space shall be designated upon the plans and specifications submitted for a building permit to the Planning Commission to determine therefrom whether or not this provision is being complied with.
\subsubsection{Screening and Fencing}
The city may require the screening or fencing of commercial uses on side and rear yards which face residential districts.
\subsubsection{General Regulations}
Requirements for signs, parking, shopping centers, and other regulations are set forth in SEC. 152.155 through SEC. 152.181.\\
\emph{(‘83 Code, SEC. 11.35)}
\section{Highway Business District (C-2)}
\subsection{Purpose}
This district is established to accommodate the type of businesses that are oriented to the traveling public and require highway access. To minimize unmanageable strip development, these districts should only allow the type of businesses that absolutely require highway access and exposure.
\subsection{Permitted Uses}
\begin{enumerate}[{\indent}1)]
    \item Farm implement dealers.
    \item Drive-in restaurant.
    \item Recreation equipment sales.
    \item Motels and hotels.
    \item Auto service station.
    \item Seasonal produce stand.
    \item Auto sales lot.
    \item Cafes and restaurants.
    \item Solar structures. \emph{(Ord. 537, effective 7-1-83)}
    \item Churches, chapels, temples, and synagogues, including parish houses. \emph{(Ord. 57, 2nd Series, effective 7-22-89)}
\end{enumerate}
\subsection{Accessory Uses}
The same accessory uses as permitted in the C-1 District.\\
\emph{(Ord. 537, effective 7-1-83)}
\subsection{Conditional Uses}
\begin{enumerate}[{\indent}1)]
    \item Drive-in movie theater.  
    \item Campgrounds.  
    \item Wind energy conversion systems.
    \item Integrated uses.
    \item Extension or relocation of nonconforming uses.
    \item Other uses which in the opinion of the Planning Commission and the Council are of the same general character as the permitted uses and which will not be detrimental to the Highway Business District or any other district.
\end{enumerate}
\emph{(Ord. 60, 2nd Series, effective 3-22-90)}
\subsection{Performance Standards}
\subsubsection{Height Regulations}
The maximum height of all buildings shall not exceed two and one-half stories or 35 feet.
\subsubsection{Service Roads (C-2)}
\paragraph{}
C-2 Districts shall be located only adjacent to existing and proposed major highways. Each C-2 District shall be provided with a service road between the highway and the business establishment. To the extent possible, service roads shall have access only to the major highways and highway-business oriented traffic shall be discouraged from local residential streets.
\paragraph{Service Road Standards}
\begin{enumerate}[{\indent}1)]
    \item Each service road shall have a minimum of 30 feet of right-of-way exclusive of adjoining thoroughfare right-of-way.
    \item Each service road shall be at least 24 feet wide and must be surfaced and have curbs.
    \item Two-way traffic shall be allowed on service roads.
    \item No parking shall be allowed on service roads.
    \item Access from service roads to thoroughfares shall be no more frequent than one access for each 500 feet of thoroughfare frontage.
    \item Maintenance of service roads shall be the obligation of either the city or the serviced businesses.
\end{enumerate}
\subsubsection{Setback Regulations from Road Right-of-Way}
\begin{enumerate}[{\indent}a)]
    \item State highway - 130 feet.
    \item County road - 110 feet.
    \item City street - 90 feet.
    \item Side lot - 20 feet.
    \item Rear lot - 35 feet.
    \item Lot line along Residential District - 75 feet.
    \item The Council may also require screening and fencing along the lot lines adjacent to residential districts. \emph{(Ord. 537, effective 7-1-83)}
    \item Lots platted prior to June 9, 1964 may use lot size and setback regulations in effect prior to the effective date of this chapter. \emph{(Ord. 12, 2nd Series, effective 5-15-84)}
\end{enumerate}
\subsubsection{General Standards}
Other standards and regulations related to parking, signs, and the like, are set forth in SEC. 152.155 through SEC. 152.181.\\
\emph{(‘83 Code, SEC. 11.36)}
\section{Shopping Center District (C-3)}
\subsection{Purpose}
The purpose of this district is to allow those types of businesses found in shopping centers and which serve several residential neighborhoods. These areas should have good accessibility and to the extent possible should be integrated with the public facilities as community buildings, libraries, health centers, and day care centers.\\
\emph{(Ord. 537, effective 7-1-83)}
\subsection{Permitted Uses}
Retail shopping center developed under the following conditions: an overall concept plan shall be submitted and approved by the city including architectural style of all structures, parking, driveways, landscaping and screening and adequate spaces for future community facilities when said facilities are to be part of the center; and, initial construction in new shopping centers shall include a minimum of 20,000 square feet of floor area. Individual stores, shops and businesses consistent with the following uses are allowed and are not subject to the concept plan approval and minimum square feet conditions required of retail shopping centers in this section:
\begin{enumerate}[{\indent}1)]
    \item Stores and shops selling the personal service or goods over a counter.  These include antiques, art and school supplies, bakeries, barber shop, beauty parlor, bicycles, books and stationery, candy, cameras and photographical supplies, carpets and rugs, catering establishments, china and glassware, Christmas tree sales, clothes pressing, clothing and costume rental, custom dressmaking, department stores and junior department stores, drugs, dry goods, electrical and household appliances, florists, food, furniture, furrier shops, garden supplies, gifts, hardware, hats, hobby shops, interior decorating, jewelry, watch repair, laundry and dry cleaning pick up, laundromats, leather goods and luggage, locksmith shops, musical instruments, office supply, paint and wallpaper, phonograph records, photography studios, restaurants, shoes, sporting goods, tailoring, theater, except open air drive-in, tobacco, toys, variety stores, wearing apparel, grocery store and off-sale liquor store.
    \item Offices for doctors, dentists, lawyers, real estate and similar uses to serve the adjoining residential area.
    \item Automobile service stations.
    \item Community facilities such as library, swimming pool, health center, day care center, religious facilities or community center.
    \item On-sale wine and/or 3.2 beer in conjunction with a restaurant facility.
    \item Health clubs or athletic clubs and facilities.
    \item Animal hospital or clinic when contained within a building.
    \item Small engine and appliance repair conducted entirely within a building and accessory to a principal use.
    \item Commercial recreation including theater, athletic club, billiard room, and similar facilities when contained within a building.
    \item Retail sales of auto accessories except that of installation facilities.
    \item Hotels and motels.  
    \item Solar energy systems.
\end{enumerate}
\emph{(Ord. 97, 2nd Series, effective 4-20-95)}
\subsection{Accessory Uses}
Same accessory uses as in C-1 District.
\subsection{Conditional Uses}
\begin{enumerate}[{\indent}1)]
    \item Outdoor display or sales conducted by an occupant of the shopping center.
    \item On-sale liquor.  
    \item Multiple dwellings.
    \item Wind energy conversion systems.
    \item Service bays as accessory uses for the installation of auto accessories in conjunction with an appliance store or auto accessory store provided there are no more than two bays, shall be screened and oriented as required by the Council.
    \item Other uses, which in the opinion of the Planning Commission and Council, are of the same general character as the permitted uses, and will not have an adverse effect on the Central Business District.
\end{enumerate}
\subsection{Performance Standards}
\subsubsection{Height Regulations}
The maximum height of all buildings shall not exceed two and one-half stories or 35 feet.
\subsubsection{Setback Requirements}
\begin{enumerate}[{\indent}a)]
    \item State highway - 70 feet.  
    \item County road - 50 feet. \emph{(Ord. 537, effective 7-1-83)}
    \item City street - 50 feet. \emph{(Ord. 97, 2nd Series, effective 4-20-95)}
    \item Side lot - 20 feet.  
    \item Rear lot - 35 feet.
    \item Lot line along residential district - 50 feet.
    \item The Council may also require screening and fencing along the lot lines adjacent to residential districts.
\end{enumerate}
\subsubsection{Frontage Roads}
The Council may require frontage roads, if appropriate.  If frontage roads are required, the same setbacks and frontage road standards as listed in SEC. 152.051(E) shall be applicable.
\subsubsection{General Requirements}
Other standards and regulations related to parking, signs, and the like, are set forth in SEC.SEC. 152.155 through 152.181.\\
\emph{(Ord. 537, effective 7-1-83)}\\
\emph{(‘83 Code, SEC. 11.37)}\\

\begin{center}
    \emph{\textbf{\LARGE{INDUSTRIAL DISTRICTS}}}
\end{center}

\setcounter{section}{64}
\section{Heavy Industry (I-1)}
\section{Light Industry}

\begin{center}
    \emph{\textbf{\LARGE{SPECIAL USE DISTRICTS}}}
\end{center}

\setcounter{section}{79}
\section{Institutional (IN)}

\begin{center}
    \emph{\textbf{\LARGE{FLOODPLAIN DISTRICT (FP)}}}
\end{center}

\setcounter{section}{89}
\section{Statutory Authorization, Findings of Fact and Purpose}
\section{General Provisions}
\section{Establishment of Zoning Districts}
\section{Floodway District (FW)}
\section{Flood Fringe District (FF)}
\section{AO Zone}
\section{Subdivisions}
\section{Public Utilities, Railroads, Roads and Bridges}
\section{Manufactured Homes, Manufactured Home Parks, Placement of Travel Trailers and Travel Vehicles}
\section{Administration}
\section{Nonconforming Uses}
\section{Violations}
\section{Amendments}

\begin{center}
    \emph{\textbf{\LARGE{NONCONFORMING USES\\[.1cm] AND STRUCTURES}}}
\end{center}

\setcounter{section}{114}
\section{Continuation of Existing Use or Building}
\section{Nonconforming Buildings}
\section{Nonconforming Use of Building or Land}

\begin{center}
    \emph{\textbf{\LARGE{PLANNED UNIT DEVELOPMENT\\[.1cm] RESIDENTIAL ONLY}}}
\end{center}

\setcounter{section}{129}
\section{Purpose}
\section{Permitted Uses}
\section{General Requirements}
\section{Density and Density Transfer}
\section{Coordination with Subdivision Regulations}
\section{Pre-Application Meeting}
\section{Preliminary Development Plan}
\section{Final Development Plan}
\section{Enforcing Development Schedule}
\section{Conveyance and Maintenance of Common Open Space}
\section{Standards for Common or Public Open Space}
\section{PUD Review and Amendments}

\begin{center}
    \emph{\textbf{\LARGE{PERFORMANCE STANDARDS}}}
\end{center}

\setcounter{section}{154}
\section{Purpose}
\section{Solar and Earth Sheltered Structures}
\section{Wind Energy Conversion Systems (WECS)}
\section{Exterior Storage}
\section{Refuse}
\section{Blight or Blighting Factors}
\section{Glare}
\section{[Reserved]}
\section{Nuisances}
\section{Screening}
\section{Landscaping}
\section{Permitted Encroachments}
\section{Accessory Buildings and Structures; Prohibited Dwelling Units}
\section{Tree and Woodland Preservation}
\section{Wetland Preservation}
\section{Traffic Control; Vacated Streets}
\section{Access Drives and Access}
\section{Private Sewer Systems}
\section{Mobile Home Parks}
\section{Off-Street Parking}
\section{Auto Service Station Standards}
\section{Drive-In Business Standards}
\section{Signs}
\section{Home Occupations}
\section{Manufactured Homes}
\section{Satellite Dish Antennas}
\section{Bed and Breakfast Inns}

\begin{center}
    \emph{\textbf{\LARGE{ADMINISTRATION AND ENFORCEMENT}}}
\end{center}

\setcounter{section}{194}
\section{Enforcing Officer}
\section{Appeals and Board of Zoning Appeals (See also SEC. 32.130 and SEC. 32.131)}
\section{Powers and Duties of Planning Commission (See also SEC. 32.120 and SEC. 32.121)}
\section{Zoning Amendments}
\section{Conditional Use Permits}
\section{Variances}
\section{Enforcement Provisions and Procedures}

\setcounter{section}{998}
\section{Penalty}

\section*{Appendix: Table of Lot Area, Width and Setbacks for Land Use Districts}
\addcontentsline{toc}{section}{Appendix: Table of Lot Area, Width and Setbacks for Land Use Districts}
