\chapter*{Chapter 151: \\
	Subdivision Regulations}
    \addstarredchapter{Chapter 151: Subdivision Regulations}
    \minitoc
    \pagebreak

\begin{center}
    \emph{\textbf{\LARGE{GENERAL PROVISIONS}}}
\end{center}

\section{Purpose}
The process of dividing raw land into separate parcels for other uses including residential, industrial and commercial sites, is one of the most important factors in the growth of any community. Once the land has been subdivided and the streets, homes and other structures have been constructed, the basic character of this permanent addition to the community has become firmly established. It is, therefore, to the interest of the general public, the developer, and the future land owners that subdivisions be conceived, designed, and developed in accordance with the highest possible standards of excellence. All subdivisions of land hereafter submitted for approval shall fully comply, in all respects, with the regulations set forth herein. It is the purpose of these regulations to:
\begin{enumerate}[{\indent}A)]
    \item Encourage well planned, efficient, and attractive subdivisions by establishing adequate standards for design and construction; 
    \item Provide for the health and safety of residents by requiring the necessary services such as properly designed streets and adequate sewage and water service; 
    \item Place the cost of improvements against those benefiting from their construction; 
    \item Secure the rights of the public with respect to public lands and waters; 
    \item Improve land records by establishing standards for surveys and plats; 
    \item Protect the environmentally sensitive areas in the city; and 
    \item Preserve energy through the encouragement of solar and earth-sheltered structures.
\end{enumerate}
\emph{(‘83 Code, SEC. 12.01)}
\section{Scope and Legal Authority}
\subsection{Scope}
The rules and regulations governing plats and subdivision of land contained herein shall apply within the city and other land as permitted by Minnesota State Statutes and as approved by the city. Except in the case of resubdivision, this chapter shall not apply to any lot or lots forming a part of a subdivision recorded in the office of the Polk County Recorder prior to the effective date of this chapter, nor is it intended by this chapter to repeal, annul or in any way impair or interfere with existing provisions of other laws or city code provisions except those specifically repealed by, or in conflict with this chapter, or with restrictive covenants running with the land. Where this chapter imposes a greater restriction upon the land than is imposed or required by the existing provisions of law, city code provisions, contract or deed, the provisions of this chapter shall control.
\subsection{Amendments}
The provisions of this chapter may be amended by the Council.
\subsection{Restrictions on Filing and Recording Conveyances}
\subsubsection{}
No conveyance of land in which the land conveyed is described by metes and bounds or by reference to an unapproved registered land survey made after April 21, 1969 or to an unapproved plat made after these regulations become effective, shall be made or recorded unless the parcel described in the conveyance:
\begin{enumerate}[{\indent}a)]
    \item Was a separate parcel of record April 1, 1945 or the date of adoption of subdivision regulations under Laws 1945, Chapter 287, whichever is the later, or of the adoption of subdivision regulations pursuant to a home rule charter;
    \item Was the subject of a written agreement to convey entered into prior to such time;
    \item Was a separate parcel of not less than two and one-half acres in area and 150 feet in width on January 1, 1966;
    \item Was a separate parcel of not less than five acres in area and 300 feet in width on July 1, 1980;
    \item Is a single parcel of commercial or industrial land of not less than five acres and having a width of not less than 300 feet and its conveyance does not result in the division of the parcel into two or more lots or parcels, any one of which is less than five acres in area or 300 feet in width; or
    \item Is a single parcel of residential or agricultural land of not less than 20 acres and having a width of not less than 500 feet and its conveyance does not result in the division of the parcel into two or more lots or parcels any one of which is less than 20 acres in area or 500 feet in width.
\end{enumerate}
\subsubsection{}
In any case in which compliance with the foregoing restrictions will create an unnecessary hardship and failure to comply does not interfere with the purpose of the subdivision regulations, the platting authority may waive the compliance by adoption of a resolution to that effect and the conveyance may then be filed or recorded. Any owner or agent of the owner of land who conveys a lot or parcel in violation of the provisions of this subdivision shall forfeit and pay to the city a penalty of not less than \$100 for each lot or parcel so conveyed. The city may enjoin the conveyance or may recover the penalty by a civil action in any court of competent jurisdiction.
\subsubsection{}
These subdivision regulations shall be applicable to any parcels which are taken from existing parcels of record by metes and bounds description and the city may deny the issuance of building permits to any parcels so divided, pending compliance with the subdivision regulations.
\subsection{Platting}
Any subdivision creating parcels, tracts or lots after the adopting of these regulations shall be platted.\\
\emph{(‘83 Code, SEC. 12.02)  Penalty, see SEC. 151.99}
\section{Definitions}
For the purpose of this chapter the following definitions shall apply, unless the context clearly indicates or requires a different meaning.
\begin{description}
    \item[ALLEY] A public right-of-way usually 20 feet or less in width which normally affords a secondary means of vehicular access to abutting property.
    \item[ATTORNEY] The attorney employed by the city unless otherwise stated.
    \item[BLOCK] The enclosed area within the perimeter of roads, property lines or boundaries of the subdivision.
    \item[BOULEVARD] The portion of the street right-of-way between the curb line and the property line.
    \item[BUTT LOT] A lot at the end of a block and located between two corner lots.
    \item[CLUSTER DEVELOPMENT] A subdivision development planned and constructed so as to group housing units into patterns while providing a unified network of open space and wooded areas, and meeting the overall density regulations of this chapter and the zoning chapter.
    \item[COMPREHENSIVE PLAN] A plan prepared by the city including a compilation of policy statements, goals, standards and maps indicating the general locations recommended for the various functional classes of land use and for the general physical development of the city and includes any plan or parts thereof.
    \item[CONTOUR MAP] A map on which irregularities of land surface are shown by lines connecting points of equal elevations. Contour interval is the vertical height between contour lines.
    \item[COPY] A print or reproduction made from a tracing.
    \item[CORNER LOT] A lot bordered on at least two sides by streets.
    \item[CONCEPT PLAN or SKETCH PLAN] A generalized plan of a proposed subdivision indicating lot layouts, streets, park areas, and water and sewer systems presented to the city officials at the pre-application meeting.
    \item[COUNTY BOARD] Polk County Board of Commissioners.
    \item[DEVELOPMENT] The act of building structures and installing site improvements.
    \item[DOUBLE FRONTAGE LOTS] Lots which have a front line abutting on one street and a back or rear line abutting on another street.
    \item[DRAINAGE COURSE] A water course or indenture for the drainage of surface water.
    \item[EASEMENT] A grant by an owner of land for a specific use by persons other than the owner.
    \item[ENGINEER] The registered engineer employed by the city unless otherwise stated.
    \item[FINAL PLAT] The final map, drawing or chart on which the subdivider’s plan of subdivision is presented to the Council for approval and which, if approved, will be submitted to the County Recorder.
    \item[KEY MAP] A map drawn to comparatively small scale which definitely shows the area proposed to be platted and the areas surrounding it to a given distance.
    \item[LOT] A parcel or portion of land in a subdivision or plat of land, separated from other parcels or portions by description as on a subdivision or record of survey map, for the purpose of sale or lease or separate use thereof.
    \item[METES AND BOUNDS DESCRIPTION] A description of real property which is not described by reference to a lot or block shown on a map, but is described by starting at a known point and describing the bearing and distances of the lines forming the boundaries of the property or delineates a fractional portion of a section, lot or area by describing lines or portions thereof.
    \item[MINIMUM SUBDIVISION DESIGN STANDARDS] The guides, principles and specifications for the preparation of subdivision plats indicating, among other things, the minimum and maximum dimensions of the various elements set forth in the plan.
    \item[OWNER] An individual, firm, association, syndicate, co-partnership, corporation, trust, or any other legal entity having sufficient proprietary interest in the land sought to be subdivided to commence and maintain proceedings to subdivide the same under these regulations.
    \item[PEDESTRIAN WAY] A public right-of-way across or within a block intended to be used by pedestrians.
    \item[PLAT] The drawing or map of a subdivision prepared for filing of record pursuant to M.S. Chapter 505, as it may be amended from time to time, and containing all elements and requirements set forth in applicable local regulations adopted pursuant to M.S. SEC. 462.358 and Chapter 505, as may be amended from time to time.
    \item[PRELIMINARY APPROVAL] Official action taken by the city on an application to create a subdivision which establishes the rights and obligations set forth in M.S. SEC. 462.358, as it may be amended from time to time, and the applicable subdivision regulation.  In accordance with M.S. SEC. 462.358, as it may be amended from time to time, and unless otherwise specified in the applicable subdivision regulation, preliminary approval may be granted only following the review and approval of a preliminary plat or other map or drawing establishing without limitation the number, layout, and location of lots, tracts, blocks and parcels to be created, location of streets, roads, utilities and facilities, park and drainage facilities, and lands to be dedicated for public use.
    \item[PRELIMINARY PLAT] The preliminary map, drawing or chart indicating the proposed layout of the subdivision to be submitted to the Planning Commission and Council for their consideration.
    \item[PRIVATE STREET] A street serving as vehicular access to one or more parcels of land which is not dedicated to the public but is owned by one or more private parties.
    \item[PROTECTIVE COVENANTS] Contracts entered into between private parties and constituting a restriction on the use of all private property within a subdivision for the benefit of the property owners, and to provide mutual protection against undesirable aspects of development which would tend to impair stability of values.
    \item[RIGHT-OF-WAY] The publicly owned land along a street or highway corridor a portion of which is covered by the street or highway pavement.
    \item[STREET] A public way for vehicular traffic, whether designed as a street, highway, thoroughfare, arterial parkway, throughway road, avenue, lane, place or however otherwise designated.
        \begin{enumerate}[{\indent\indent}1)]
            \item \emph{ARTERIAL STREET} A street or highway with access restrictions designed to carry large volumes of traffic between various sections of the city and beyond.
            \item \emph{COLLECTOR STREET} A street which carries traffic from local streets to arterials.
            \item \emph{CUL-DE-SAC} A street turn-around with only one outlet.
            \item \emph{LOCAL STREET} A street of limited continuity used primarily for access to the abutting properties and the local need of a neighborhood.
            \item \emph{SERVICE STREET} Marginal access street, or otherwise designated, is a minor street, which is parallel and adjacent to a thoroughfare and which provides access to abutting properties and protection from through traffic.
        \end{enumerate}
    \item[STREET WIDTH] For the purpose of this chapter, the shortest distance between the lines delineating the right-of-way.
    \item[SUBDIVIDER] Any person commencing proceedings under this chapter to effect a subdivision of land hereunder for himself or for another.
    \item[SUBDIVISION] The separation of an area, parcel, or tract of land under single ownership into two or more parcels, tracts, lots, or long-term leasehold interests where the creation of the leasehold interest necessitates the creation of streets, roads, or alleys, for residential, commercial, industrial, or other use or any combination thereof, except those separations:
        \begin{enumerate}[{\indent\indent}1)]
            \item Where all the resulting parcels, tracts, lots, or interests will be 20 acres or larger in size and 500 feet in width for residential uses and five acres or larger in size for commercial and industrial uses;
            \item Creating cemetery lots; and
            \item Resulting from court orders, or the adjustment of a lot line by the relocation of a common boundary.
        \end{enumerate}
    \item[TRACING] A plat or map drawn on transparent paper or cloth which can be reproduced by using regular reproduction procedure.
\end{description}
\emph{(Ord. 536, effective 7-1-83)}\\
\emph{(‘83 Code, SEC. 12.03)}

\begin{center}
    \emph{\textbf{\LARGE{PRELIMINARY AND FINAL PLATS}}}
\end{center}

\setcounter{section}{14}
\section{Identification and Description of Preliminary Plat}
The following data is required for a preliminary plat:
\begin{enumerate}[{\indent}A)]
    \item Proposed name of subdivision and street names, which shall not duplicate or be similar in pronunciation or spelling to the name of any plat heretofore recorded in Polk County.
    \item Location by section, township, range, and by legal description.
    \item Name of city.
    \item Names and addresses of the record owner and any agent having control of the land, subdivider, land surveyor, engineer, and designer of the plan.
    \item Graphic scale not less than one inch to 100 feet.
    \item North point.
    \item Key map including area within one mile radius of plat.
    \item Date of preparation.
    \item A current abstract of title or a registered property certificate along with any unrecorded documents and an opinion of title by the subdivider’s attorney.
\end{enumerate}
\emph{(‘83 Code, SEC. 12.20, Subd. 1.A.)}
\section{Existing Conditions in Proposed Tract}
\subsection{}
Boundary line of proposed subdivision, clearly indicated and to a close degree of accuracy.
\subsection{}
Existing zoning classifications for land within and abutting the subdivision including floodplain, and shoreland districts, if applicable.
\subsection{}
A general statement of the approximate acreage and dimensions of the lots.
\subsection{}
Location, right-of-way width, and names of existing or platted streets, or other public ways, parks, and other public lands, permanent buildings and structures, easements and section and corporate lines within the plan.
\subsection{}
Boundary lines of adjoining unsubdivided or subdivided land, identified by name and ownership, including all contiguous land owned or controlled by the subdivider.
\subsection{}
Topographic data, including contours at vertical intervals of two feet, watercourses, marshes, rock outcrops, power transmission poles and lines, and other significant features shall also be shown, including without limitation, shading all areas containing a grade greater than 7\%.
\subsection{}
An analysis of the soils by representatives of the Polk County Soil and Water Conservation District and soil borings may be required, if deemed necessary by the Planning Commission or Council.
\subsection{}
If applicable, limits of the floodplain, floodway and flood fringe areas.
\subsection{}
Existing zoning and land use in the area within 50 feet of the boundaries of the tract.
\subsection{}
Plans for water supply, sewage disposal, drainage and flood control.  Location and size of existing sewers, water mains, culverts or other underground facilities within the preliminary plan area.  The data as existing grades, invert elevations, and location of catch basins, manholes, hydrants and street pavement width and type, shall also be shown.\\
\emph{(‘83 Code, SEC. 12.20, Subd. 1.B.)}
\section{Subdivision Design Features}
\subsection{}
Layout and width of proposed streets and utility easements, pedestrian ways showing street names, lot dimensions, parks and other public areas. The street layout shall include all contiguous land owned or controlled by the subdivider.
\subsection{}
Proposed use of all parcels, and if zoning change is contemplated, proposed zoning amendment.
\subsection{}
Preliminary street grades and drainage plan.
\subsection{}
Layout, numbers and preliminary dimensions of lots and blocks.
\subsection{}
When lots are located on a curve, the width of the lot at the building setback line.\\
\emph{(‘83 Code, SEC. 12.20, Subd. 1.C.)}
\section{Other Information}
\subsection{}
Where a subdivider owns property adjacent to that which is being proposed for the subdivision, the Planning Commission may require that the subdivider submit a sketch plan of the remainder of the property so as to show the possible relationships between the proposed subdivision and the future subdivision.
\subsection{}
Potential resubdivision and use of excessively deep or wide (over 200 feet) lots shall be indicated in a satisfactory manner.
\subsection{}
A plan for soil erosion and sediment control both during construction and after development has been completed.
\subsection{}
The other information as may be requested by the city staff, Planning Commission, or Council.\\
\emph{(‘83 Code, SEC. 12.20, Subd. 1.D.)}
\section{Data and Requirements for Final Plats}
\subsection{}
The plat shall be prepared by a land surveyor who is registered in the State of Minnesota and shall comply with the appropriate provisions of Minnesota Statutes and of these regulations.
\subsection{}
Data as required by the City Engineer, for example, accurate angular and linear dimensions for all lines, angles and curvatures used to describe boundaries, streets, easements, and other important features.
\subsection{}
Identification and description data as required for the preliminary plat.
\subsection{}
Boundaries of the property lines of all proposed streets and alleys, with their width, and any other areas intended for public use.
\subsection{}
Lines of adjoining streets and alleys, with their widths and names.
\subsection{}
All lot lines and easements, with figures showing their dimensions.
\subsection{}
An identification system for all lots and blocks.
\subsection{}
Certification by a registered land surveyor to the effect that the plat represents a survey made by him and that monuments and markers thereon exist as located and that all dimensional and geodetic details are correct.
\subsection{}
Notarized certification by owner, and by any mortgage holder of record, of adoption of the plat and the dedication of streets and other public areas.
\subsection{}
Certification showing that all taxes currently due have been paid and that all special assessments have been paid in full.
\subsection{}
Title opinion by a practicing attorney-at-law based upon an examination of an abstract of the records of the Polk County Recorder for the lands included within the plat and showing the title to be in the name of the owner or subdivider. The date of continuation of the abstract examined or the date of the examination of the records shall be within 30 days prior to the date the final plat is filed with the County Auditor. The owner or subdivider shown in the title opinion shall be the owner of record of the platted lands on the date of recording of the plat with the Polk County Recorder.
\subsection{}
Execution by all owners of any interest in the land and any holders of a mortgage therein of the certificate required by Minnesota Statutes and which certificate shall include an accurate legal description of any area to be dedicated for public use and shall include a dedication to the city of sufficient easements to accommodate utility services in the form as shall be approved by the City Attorney.\\
\emph{(‘83 Code, SEC. 12.20, Subd. 2)}
\section{Certifications}
The final plat shall include the required certification by the city and county officials. This shall include a signature by the Chairman of the Planning Commission indicating that the plat has been reviewed by the Planning Commission.
\subsection{}
Form for approval by signature of county officials concerned with the recording of the plat.
\subsubsection{}
Checked and approved as to compliance with M.S. Chapter 505, as it may be amended from time to time.\\
Dated this _____ day of __________, A.D.,  20_____.\\
____________________\\
(Name) Polk County Engineer
\subsubsection{}
No delinquent taxes and transfer entered this _____ day of __________ 20_____.\\
____________________\\
(Name) Polk County Auditor
\subsubsection{}
Document Number __________.\\
I hereby certify this instrument was filed\\
in the office of the County Recorder for\\
record on this _____ day of __________,  20_____,\\
at _____ o’clock _____.m.,\\
and was duly recorded in Book _____ of _____, on Page _____.\\
____________________\\
(Name) County Recorder, Polk County
\subsubsection{}
If property being platted is in the Torrens System, use the following:\\
Document Number _____.\\
I hereby certify this instrument was filed\\
in the office of the Registrar of Titles for\\
record on this _____ day of __________, 20_____,\\
at _____ o’clock _____.m.,\\
and was duly recorded in Book _____ of _____, on Page _____.\\
_____________________\\
(Name) Registrar of Titles, Polk County
\subsubsection{}
Checked and approved as in compliance with the zoning ordinance and the subdivision regulations chapter.\\
____________________\\
Chairman, Crookston Planning Commission
\subsection{Form for Approval by the City Attorney}
I hereby certify that proper evidence of title has been presented to and examined by me, and I hereby approve this plat as to form and execution.\\
Dated this _____ day of __________, A.D., 20_____.\\
____________________\\
(Name) Crookston Attorney
\subsection{}
Approved by Crookston City Council on this _____ day of __________, A.D., 20_____.\\
____________________\\
Mayor, Crookston\\
Attest:\\
____________________\\
Clerk-Treasurer\\
\emph{(‘83 Code, SEC. 12.20, Subd. 3)  (Ord. 536, effective 7-1-83)}

\begin{center}
    \emph{\textbf{\LARGE{SUBDIVISION DESIGN STANDARDS}}}
\end{center}

\setcounter{section}{29}
\section{Conformity with Comprehensive Plan}
\section{Streets and Thoroughfares}
\section{Blocks and Lots}
\section{Easements}
\section{Tree Removal, Conservation, Soil Erosion, Density Credit}
\section{Parks, Open Space and Public Use}

\begin{center}
    \emph{\textbf{\LARGE{REQUIRED IMPROVEMENTS}}}
\end{center}

\setcounter{section}{49}
\section{Improvements Required for All Subdivisions}
\section{Payment for Installation of Improvements}

\begin{center}
    \emph{\textbf{\LARGE{ADMINISTRATION AND ENFORCEMENT}}}
\end{center}

\setcounter{section}{64}
\section{Plat Presentation Procedures}
\section{Pre-Application Meeting}
\section{Preliminary Plat}
\section{Final Plat}
\section{Effect of Subdivision Approval}
\section{Disclosure by Seller; Purchaser’s Action for Damages}
\section{Modifications, Exceptions and Variances}
\section{Planned Unit Developments}
\section{Minor Subdivisions}

\setcounter{section}{98}
\section{Penalty}
