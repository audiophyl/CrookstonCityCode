\chapter*{Chapter 151: \\
	Subdivision Regulations}
    \addstarredchapter{Chapter 151: Subdivision Regulations}
    \minitoc
    \pagebreak

\begin{center}
    \emph{\textbf{\LARGE{GENERAL PROVISIONS}}}
\end{center}

\section{Purpose}
The process of dividing raw land into separate parcels for other uses including residential, industrial and commercial sites, is one of the most important factors in the growth of any community. Once the land has been subdivided and the streets, homes and other structures have been constructed, the basic character of this permanent addition to the community has become firmly established. It is, therefore, to the interest of the general public, the developer, and the future land owners that subdivisions be conceived, designed, and developed in accordance with the highest possible standards of excellence. All subdivisions of land hereafter submitted for approval shall fully comply, in all respects, with the regulations set forth herein. It is the purpose of these regulations to:
\begin{enumerate}[{\indent}A)]
    \item Encourage well planned, efficient, and attractive subdivisions by establishing adequate standards for design and construction; 
    \item Provide for the health and safety of residents by requiring the necessary services such as properly designed streets and adequate sewage and water service; 
    \item Place the cost of improvements against those benefiting from their construction; 
    \item Secure the rights of the public with respect to public lands and waters; 
    \item Improve land records by establishing standards for surveys and plats; 
    \item Protect the environmentally sensitive areas in the city; and 
    \item Preserve energy through the encouragement of solar and earth-sheltered structures.
\end{enumerate}
\emph{(‘83 Code, SEC. 12.01)}
\section{Scope and Legal Authority}
\subsection{Scope}
The rules and regulations governing plats and subdivision of land contained herein shall apply within the city and other land as permitted by Minnesota State Statutes and as approved by the city. Except in the case of resubdivision, this chapter shall not apply to any lot or lots forming a part of a subdivision recorded in the office of the Polk County Recorder prior to the effective date of this chapter, nor is it intended by this chapter to repeal, annul or in any way impair or interfere with existing provisions of other laws or city code provisions except those specifically repealed by, or in conflict with this chapter, or with restrictive covenants running with the land. Where this chapter imposes a greater restriction upon the land than is imposed or required by the existing provisions of law, city code provisions, contract or deed, the provisions of this chapter shall control.
\subsection{Amendments}
The provisions of this chapter may be amended by the Council.
\subsection{Restrictions on Filing and Recording Conveyances}
\subsubsection{}
No conveyance of land in which the land conveyed is described by metes and bounds or by reference to an unapproved registered land survey made after April 21, 1969 or to an unapproved plat made after these regulations become effective, shall be made or recorded unless the parcel described in the conveyance:
\begin{enumerate}[{\indent}a)]
    \item Was a separate parcel of record April 1, 1945 or the date of adoption of subdivision regulations under Laws 1945, Chapter 287, whichever is the later, or of the adoption of subdivision regulations pursuant to a home rule charter;
    \item Was the subject of a written agreement to convey entered into prior to such time;
    \item Was a separate parcel of not less than two and one-half acres in area and 150 feet in width on January 1, 1966;
    \item Was a separate parcel of not less than five acres in area and 300 feet in width on July 1, 1980;
    \item Is a single parcel of commercial or industrial land of not less than five acres and having a width of not less than 300 feet and its conveyance does not result in the division of the parcel into two or more lots or parcels, any one of which is less than five acres in area or 300 feet in width; or
    \item Is a single parcel of residential or agricultural land of not less than 20 acres and having a width of not less than 500 feet and its conveyance does not result in the division of the parcel into two or more lots or parcels any one of which is less than 20 acres in area or 500 feet in width.
\end{enumerate}
\subsubsection{}
In any case in which compliance with the foregoing restrictions will create an unnecessary hardship and failure to comply does not interfere with the purpose of the subdivision regulations, the platting authority may waive the compliance by adoption of a resolution to that effect and the conveyance may then be filed or recorded. Any owner or agent of the owner of land who conveys a lot or parcel in violation of the provisions of this subdivision shall forfeit and pay to the city a penalty of not less than \$100 for each lot or parcel so conveyed. The city may enjoin the conveyance or may recover the penalty by a civil action in any court of competent jurisdiction.
\subsubsection{}
These subdivision regulations shall be applicable to any parcels which are taken from existing parcels of record by metes and bounds description and the city may deny the issuance of building permits to any parcels so divided, pending compliance with the subdivision regulations.
\subsection{Platting}
Any subdivision creating parcels, tracts or lots after the adopting of these regulations shall be platted.\\
\emph{(‘83 Code, SEC. 12.02)  Penalty, see SEC. 151.99}
\section{Definitions}
For the purpose of this chapter the following definitions shall apply, unless the context clearly indicates or requires a different meaning.
\begin{description}
    \item[ALLEY] A public right-of-way usually 20 feet or less in width which normally affords a secondary means of vehicular access to abutting property.
    \item[ATTORNEY] The attorney employed by the city unless otherwise stated.
    \item[BLOCK] The enclosed area within the perimeter of roads, property lines or boundaries of the subdivision.
    \item[BOULEVARD] The portion of the street right-of-way between the curb line and the property line.
    \item[BUTT LOT] A lot at the end of a block and located between two corner lots.
    \item[CLUSTER DEVELOPMENT] A subdivision development planned and constructed so as to group housing units into patterns while providing a unified network of open space and wooded areas, and meeting the overall density regulations of this chapter and the zoning chapter.
    \item[COMPREHENSIVE PLAN] A plan prepared by the city including a compilation of policy statements, goals, standards and maps indicating the general locations recommended for the various functional classes of land use and for the general physical development of the city and includes any plan or parts thereof.
    \item[CONTOUR MAP] A map on which irregularities of land surface are shown by lines connecting points of equal elevations. Contour interval is the vertical height between contour lines.
    \item[COPY] A print or reproduction made from a tracing.
    \item[CORNER LOT] A lot bordered on at least two sides by streets.
    \item[CONCEPT PLAN or SKETCH PLAN] A generalized plan of a proposed subdivision indicating lot layouts, streets, park areas, and water and sewer systems presented to the city officials at the pre-application meeting.
    \item[COUNTY BOARD] Polk County Board of Commissioners.
    \item[DEVELOPMENT] The act of building structures and installing site improvements.
    \item[DOUBLE FRONTAGE LOTS] Lots which have a front line abutting on one street and a back or rear line abutting on another street.
    \item[DRAINAGE COURSE] A water course or indenture for the drainage of surface water.
    \item[EASEMENT] A grant by an owner of land for a specific use by persons other than the owner.
    \item[ENGINEER] The registered engineer employed by the city unless otherwise stated.
    \item[FINAL PLAT] The final map, drawing or chart on which the subdivider’s plan of subdivision is presented to the Council for approval and which, if approved, will be submitted to the County Recorder.
    \item[KEY MAP] A map drawn to comparatively small scale which definitely shows the area proposed to be platted and the areas surrounding it to a given distance.
    \item[LOT] A parcel or portion of land in a subdivision or plat of land, separated from other parcels or portions by description as on a subdivision or record of survey map, for the purpose of sale or lease or separate use thereof.
    \item[METES AND BOUNDS DESCRIPTION] A description of real property which is not described by reference to a lot or block shown on a map, but is described by starting at a known point and describing the bearing and distances of the lines forming the boundaries of the property or delineates a fractional portion of a section, lot or area by describing lines or portions thereof.
    \item[MINIMUM SUBDIVISION DESIGN STANDARDS] The guides, principles and specifications for the preparation of subdivision plats indicating, among other things, the minimum and maximum dimensions of the various elements set forth in the plan.
    \item[OWNER] An individual, firm, association, syndicate, co-partnership, corporation, trust, or any other legal entity having sufficient proprietary interest in the land sought to be subdivided to commence and maintain proceedings to subdivide the same under these regulations.
    \item[PEDESTRIAN WAY] A public right-of-way across or within a block intended to be used by pedestrians.
    \item[PLAT] The drawing or map of a subdivision prepared for filing of record pursuant to M.S. Chapter 505, as it may be amended from time to time, and containing all elements and requirements set forth in applicable local regulations adopted pursuant to M.S. SEC. 462.358 and Chapter 505, as may be amended from time to time.
    \item[PRELIMINARY APPROVAL] Official action taken by the city on an application to create a subdivision which establishes the rights and obligations set forth in M.S. SEC. 462.358, as it may be amended from time to time, and the applicable subdivision regulation.  In accordance with M.S. SEC. 462.358, as it may be amended from time to time, and unless otherwise specified in the applicable subdivision regulation, preliminary approval may be granted only following the review and approval of a preliminary plat or other map or drawing establishing without limitation the number, layout, and location of lots, tracts, blocks and parcels to be created, location of streets, roads, utilities and facilities, park and drainage facilities, and lands to be dedicated for public use.
    \item[PRELIMINARY PLAT] The preliminary map, drawing or chart indicating the proposed layout of the subdivision to be submitted to the Planning Commission and Council for their consideration.
    \item[PRIVATE STREET] A street serving as vehicular access to one or more parcels of land which is not dedicated to the public but is owned by one or more private parties.
    \item[PROTECTIVE COVENANTS] Contracts entered into between private parties and constituting a restriction on the use of all private property within a subdivision for the benefit of the property owners, and to provide mutual protection against undesirable aspects of development which would tend to impair stability of values.
    \item[RIGHT-OF-WAY] The publicly owned land along a street or highway corridor a portion of which is covered by the street or highway pavement.
    \item[STREET] A public way for vehicular traffic, whether designed as a street, highway, thoroughfare, arterial parkway, throughway road, avenue, lane, place or however otherwise designated.
        \begin{enumerate}[{\indent\indent}1)]
            \item \emph{ARTERIAL STREET} A street or highway with access restrictions designed to carry large volumes of traffic between various sections of the city and beyond.
            \item \emph{COLLECTOR STREET} A street which carries traffic from local streets to arterials.
            \item \emph{CUL-DE-SAC} A street turn-around with only one outlet.
            \item \emph{LOCAL STREET} A street of limited continuity used primarily for access to the abutting properties and the local need of a neighborhood.
            \item \emph{SERVICE STREET} Marginal access street, or otherwise designated, is a minor street, which is parallel and adjacent to a thoroughfare and which provides access to abutting properties and protection from through traffic.
        \end{enumerate}
    \item[STREET WIDTH] For the purpose of this chapter, the shortest distance between the lines delineating the right-of-way.
    \item[SUBDIVIDER] Any person commencing proceedings under this chapter to effect a subdivision of land hereunder for himself or for another.
    \item[SUBDIVISION] The separation of an area, parcel, or tract of land under single ownership into two or more parcels, tracts, lots, or long-term leasehold interests where the creation of the leasehold interest necessitates the creation of streets, roads, or alleys, for residential, commercial, industrial, or other use or any combination thereof, except those separations:
        \begin{enumerate}[{\indent\indent}1)]
            \item Where all the resulting parcels, tracts, lots, or interests will be 20 acres or larger in size and 500 feet in width for residential uses and five acres or larger in size for commercial and industrial uses;
            \item Creating cemetery lots; and
            \item Resulting from court orders, or the adjustment of a lot line by the relocation of a common boundary.
        \end{enumerate}
    \item[TRACING] A plat or map drawn on transparent paper or cloth which can be reproduced by using regular reproduction procedure.
\end{description}
\emph{(Ord. 536, effective 7-1-83)}\\
\emph{(‘83 Code, SEC. 12.03)}

\begin{center}
    \emph{\textbf{\LARGE{PRELIMINARY AND FINAL PLATS}}}
\end{center}

\setcounter{section}{14}
\section{Identification and Description of Preliminary Plat}
The following data is required for a preliminary plat:
\begin{enumerate}[{\indent}A)]
    \item Proposed name of subdivision and street names, which shall not duplicate or be similar in pronunciation or spelling to the name of any plat heretofore recorded in Polk County.
    \item Location by section, township, range, and by legal description.
    \item Name of city.
    \item Names and addresses of the record owner and any agent having control of the land, subdivider, land surveyor, engineer, and designer of the plan.
    \item Graphic scale not less than one inch to 100 feet.
    \item North point.
    \item Key map including area within one mile radius of plat.
    \item Date of preparation.
    \item A current abstract of title or a registered property certificate along with any unrecorded documents and an opinion of title by the subdivider’s attorney.
\end{enumerate}
\emph{(‘83 Code, SEC. 12.20, Subd. 1.A.)}
\section{Existing Conditions in Proposed Tract}
\subsection{}
Boundary line of proposed subdivision, clearly indicated and to a close degree of accuracy.
\subsection{}
Existing zoning classifications for land within and abutting the subdivision including floodplain, and shoreland districts, if applicable.
\subsection{}
A general statement of the approximate acreage and dimensions of the lots.
\subsection{}
Location, right-of-way width, and names of existing or platted streets, or other public ways, parks, and other public lands, permanent buildings and structures, easements and section and corporate lines within the plan.
\subsection{}
Boundary lines of adjoining unsubdivided or subdivided land, identified by name and ownership, including all contiguous land owned or controlled by the subdivider.
\subsection{}
Topographic data, including contours at vertical intervals of two feet, watercourses, marshes, rock outcrops, power transmission poles and lines, and other significant features shall also be shown, including without limitation, shading all areas containing a grade greater than 7\%.
\subsection{}
An analysis of the soils by representatives of the Polk County Soil and Water Conservation District and soil borings may be required, if deemed necessary by the Planning Commission or Council.
\subsection{}
If applicable, limits of the floodplain, floodway and flood fringe areas.
\subsection{}
Existing zoning and land use in the area within 50 feet of the boundaries of the tract.
\subsection{}
Plans for water supply, sewage disposal, drainage and flood control.  Location and size of existing sewers, water mains, culverts or other underground facilities within the preliminary plan area.  The data as existing grades, invert elevations, and location of catch basins, manholes, hydrants and street pavement width and type, shall also be shown.\\
\emph{(‘83 Code, SEC. 12.20, Subd. 1.B.)}
\section{Subdivision Design Features}
\subsection{}
Layout and width of proposed streets and utility easements, pedestrian ways showing street names, lot dimensions, parks and other public areas. The street layout shall include all contiguous land owned or controlled by the subdivider.
\subsection{}
Proposed use of all parcels, and if zoning change is contemplated, proposed zoning amendment.
\subsection{}
Preliminary street grades and drainage plan.
\subsection{}
Layout, numbers and preliminary dimensions of lots and blocks.
\subsection{}
When lots are located on a curve, the width of the lot at the building setback line.\\
\emph{(‘83 Code, SEC. 12.20, Subd. 1.C.)}
\section{Other Information}
\subsection{}
Where a subdivider owns property adjacent to that which is being proposed for the subdivision, the Planning Commission may require that the subdivider submit a sketch plan of the remainder of the property so as to show the possible relationships between the proposed subdivision and the future subdivision.
\subsection{}
Potential resubdivision and use of excessively deep or wide (over 200 feet) lots shall be indicated in a satisfactory manner.
\subsection{}
A plan for soil erosion and sediment control both during construction and after development has been completed.
\subsection{}
The other information as may be requested by the city staff, Planning Commission, or Council.\\
\emph{(‘83 Code, SEC. 12.20, Subd. 1.D.)}
\section{Data and Requirements for Final Plats}
\subsection{}
The plat shall be prepared by a land surveyor who is registered in the State of Minnesota and shall comply with the appropriate provisions of Minnesota Statutes and of these regulations.
\subsection{}
Data as required by the City Engineer, for example, accurate angular and linear dimensions for all lines, angles and curvatures used to describe boundaries, streets, easements, and other important features.
\subsection{}
Identification and description data as required for the preliminary plat.
\subsection{}
Boundaries of the property lines of all proposed streets and alleys, with their width, and any other areas intended for public use.
\subsection{}
Lines of adjoining streets and alleys, with their widths and names.
\subsection{}
All lot lines and easements, with figures showing their dimensions.
\subsection{}
An identification system for all lots and blocks.
\subsection{}
Certification by a registered land surveyor to the effect that the plat represents a survey made by him and that monuments and markers thereon exist as located and that all dimensional and geodetic details are correct.
\subsection{}
Notarized certification by owner, and by any mortgage holder of record, of adoption of the plat and the dedication of streets and other public areas.
\subsection{}
Certification showing that all taxes currently due have been paid and that all special assessments have been paid in full.
\subsection{}
Title opinion by a practicing attorney-at-law based upon an examination of an abstract of the records of the Polk County Recorder for the lands included within the plat and showing the title to be in the name of the owner or subdivider. The date of continuation of the abstract examined or the date of the examination of the records shall be within 30 days prior to the date the final plat is filed with the County Auditor. The owner or subdivider shown in the title opinion shall be the owner of record of the platted lands on the date of recording of the plat with the Polk County Recorder.
\subsection{}
Execution by all owners of any interest in the land and any holders of a mortgage therein of the certificate required by Minnesota Statutes and which certificate shall include an accurate legal description of any area to be dedicated for public use and shall include a dedication to the city of sufficient easements to accommodate utility services in the form as shall be approved by the City Attorney.\\
\emph{(‘83 Code, SEC. 12.20, Subd. 2)}
\section{Certifications}
The final plat shall include the required certification by the city and county officials. This shall include a signature by the Chairman of the Planning Commission indicating that the plat has been reviewed by the Planning Commission.
\subsection{}
Form for approval by signature of county officials concerned with the recording of the plat.
\subsubsection{}
Checked and approved as to compliance with M.S. Chapter 505, as it may be amended from time to time.\\*
Dated this \fillable{1cm} day of \fillable{2cm}, A.D.,  20\fillable{1cm}.\\*[1cm]
\fillable{5cm}\\*
$^{(Name) Polk County Engineer}$
\subsubsection{}
No delinquent taxes and transfer entered this \fillable{1cm} day of \fillable{2cm} 20\fillable{1cm}.\\*[1cm]
\fillable{5cm}\\*
$^{(Name) Polk County Auditor}$
\subsubsection{}
Document Number \fillable{2cm}.\\*
I hereby certify this instrument was filed in the office of the County Recorder for record on this \mbox{\fillable{1cm} day of \fillable{2cm},  20\fillable{1cm},} at \mbox{\fillable{1cm} o’clock \fillable{1cm}.m.,} and was duly recorded in Book \fillable{2cm} of \fillable{2cm}, on Page \fillable{1cm}.\\*[1cm]
\fillable{5cm}\\*
$^{(Name) County Recorder, Polk County}$
\subsubsection{}
If property being platted is in the Torrens System, use the following:\\*
Document Number \fillable{2cm}.\\*
I hereby certify this instrument was filed in the office of the Registrar of Titles for record on this \fillable{1cm} day of \fillable{2cm}, 20\fillable{1cm}, at \fillable{1cm} o’clock \fillable{1cm}.m., and was duly recorded in Book \fillable{2cm} of \fillable{2cm}, on Page \fillable{1cm}.\\*[1cm]
\fillable{5cm}\\*
$^{(Name) Registrar of Titles, Polk County}$
\subsubsection{}
Checked and approved as in compliance with the zoning ordinance and the subdivision regulations chapter.\\*[1cm]
\fillable{5cm}\\*
$^{Chairman, Crookston Planning Commission}$
\subsection{Form for Approval by the City Attorney}
I hereby certify that proper evidence of title has been presented to and examined by me, and I hereby approve this plat as to form and execution.\\*
Dated this \fillable{1cm} day of \fillable{2cm}, A.D., 20\fillable{1cm}.\\*[1cm]
\fillable{5cm}\\*
$^{(Name) Crookston Attorney}$
\subsection{}
Approved by Crookston City Council on this \fillable{1cm} day of \fillable{2cm}, A.D., 20\fillable{1cm}.\\*[1cm]
\fillable{5cm}\\*
$^{Mayor, Crookston}$\\*
Attest:\\*[1cm]
\fillable{5cm}\\*
$^{Clerk-Treasurer}$\\*
\emph{(‘83 Code, SEC. 12.20, Subd. 3)  (Ord. 536, effective 7-1-83)}

\begin{center}
    \emph{\textbf{\LARGE{SUBDIVISION DESIGN STANDARDS}}}
\end{center}

\setcounter{section}{29}
\section{Conformity with Comprehensive Plan}
The proposed subdivision shall conform to the comprehensive plan adopted by the city.\\*
\emph{(‘83 Code, SEC. 12.30, Subd. 1)}

\section{Streets and Thoroughfares}
\subsection{General Street Design}
\subsubsection{}
The design of all streets shall be considered in their relation to existing and planned streets, to reasonable circulation of traffic, topographic conditions, to runoff of storm water and to the proposed uses of the area to be served.
\subsubsection{}
Where new streets extend existing adjoining streets their projection shall be no less than the minimum required width, and when new streets extend existing adjoining streets, their centerlines shall be continuous.
\subsubsection{}
Where adjoining areas are not subdivided, the arrangement of streets in new subdivisions shall make provision for the proper projection of streets so that parcels will not be land-locked.  When a new subdivision adjoins developable land, then the new streets shall be carried to the boundaries of the unsubdivided land.
\subsection{Street Names}
Street names shall not duplicate the names of other streets.
\subsection{Street Width and Grade}
Street right-of-way widths shall be as determined in the policies plan and official map and, where applicable, shall conform to county and state standards for trunk highways.  If there is no plan or standard, right-of-way widths shall conform to the following minimum dimensions:\\
\begin{center}
\begin{tabular}{|p{2.5cm}|p{2.5cm}|p{2.5cm}|p{2.5cm}|p{2.5cm}|}
    \hline
    \textbf{Street Category} & \textbf{Minimum Width R.O.W} & \textbf{Minimum Width Pavement} & \textbf{Maximum Grade} & \textbf{Minimum Drainage Grade}\\
    \hline
    Collector & 80 ft. & 38 ft. & 5\% & .5\%\\
    \hline
    Local & 60 ft. & 26 ft. & 7\% & .5\%\\
    \hline
    Frontage roads & 60 ft. & 24 ft. & 7\% & .5\%\\
    \hline
    Cul-de-sac street & 60 ft. & 32 ft. & 7\% & .5\%\\
    \hline
    Turn-around radius of cul-de-sac & 55 ft. & & & .5\%\\
    \hline
\end{tabular}
\end{center}
\subsection{Street Intersections}
Insofar as practical, streets shall intersect at right angles.  In no case shall the angle formed by the intersection of two streets be less than 60 degrees. Intersections having more than four corners shall be prohibited. Adequate land for future intersection and interchange construction needs shall be dedicated to the city.
\subsection{Tangents}
A tangent of at least 100 feet shall be introduced between reverse curves on streets.
\subsection{Deflections}
When connecting street lines deflect from each other at one point by more than ten degrees they shall be connected by a curve with a radius adequate to ensure a sight distance of no less than 100 feet for all streets.
\subsection{Street Jogs}
Street jogs with centerline offsets of less than 150 feet shall be avoided.
\subsection{Local Streets}
Local streets shall be laid out so as not to encourage through traffic.
\subsection{Cul-de-sac}
The maximum length of a street terminating in a cul-de-sac shall be 500 feet, measured from the centerline of the street of origin to the end of the right-of-way.
\subsection{Access to Arterial Streets}
In the case where a proposed plat is adjacent to a limited access highway (arterial), there shall be no direct vehicular or pedestrian access from individual lots to the highways. As a general requirement, access arterials shall be at intervals of not less than one-fourth mile and through existing and established cross roads where possible. The Council may require the developer to provide local service drives along the right-of-way of the facilities, or it may require that lots should back on the arterial, in which case, vehicular and pedestrian access between the lots and arterial shall be prohibited.
\subsection{Half Streets}
Half streets shall be prohibited except where it will be practical to require the dedication of the other half when the adjoining property is subdivided, in which case the dedication of a half street may be permitted.
\subsection{Corners}
Curb lines at street intersections shall be rounded at a radius of not less than 15 feet.
\subsection{Hardship to Owners of Adjoining Property}
The street arrangements shall not be such as to cause hardship to owners of adjoining property in platting their own land and providing convenient access to it.\\
\emph{(‘83 Code, SEC. 12.30, Subd. 2)}

\section{Blocks and Lots}
\subsection{Blocks}
The length, width and acreage of blocks shall be sufficient to provide for convenient access, circulation, control and safety of street design.\\*
\emph{(‘83 Code, SEC. 12.30, Subd. 3)}
\subsection{Lots}
\subsubsection{Size}
The lot dimensions shall be such as to comply with the minimum lot areas specified in the zoning chapter.
\subsubsection{Side Lot Lines}
Side lines of lots shall be substantially at right angles to straight street lines or radial to curved street lines.
\subsubsection{Drainage}
Lots shall be graded so as to provide drainage away from building locations.
\subsubsection{Natural Features}
In the subdividing of any land, due regard shall be shown for all natural features, such as tree growth, wetlands, steep slopes, water courses, or similar conditions, and plans adjusted to preserve those which will add attractiveness, safety and stability to the proposed development.
\subsubsection{Lot Remnants}
All remnants of lots below minimum size left over after subdividing of a larger tract must be added to adjacent lots rather than allowed to remain as unusable parcels unless the owner can show plans for future use of the remnant.
\subsubsection{Double Frontage Lots}
Double frontage (lots with frontage on two parallel streets) or reverse frontage shall be discouraged except where lots back on an arterial or collector street.  The lots shall have an additional depth of at least ten feet in order to allow for screen planting along the back lot line.\\
\emph{(‘83 Code, SEC. 12.30, Subd. 4)}

\section{Easements}
\subsection{Utilities}
Easements of at least ten feet wide centered on rear lot lines shall be provided for utilities where necessary. Easements for storm or sanitary sewer shall be at least 20 feet wide. They shall have continuity of alignment from block to block. Temporary construction easements may be required where installation depths are greater than ten feet. Utility easements shall be kept free of any vegetation or structures which would interfere with the free movement of utility service vehicles.
\subsection{Water Courses}
When a subdivision is traversed by a water course, drainage way, channel or stream, there shall be provided a storm water easement or drainage right-of-way conforming substantially with the lines of the water courses, and with the further width or construction as may be determined to be necessary by the City Engineer.\\*
\emph{(‘83 Code, SEC. 12.30, Subd. 5)}
\section{Tree Removal, Conservation, Soil Erosion, Density Credit}
\subsection{Tree Removal and Conservation of Vegetation}
The standards related to tree removal contained in the performance standards of the zoning chapter shall be applicable to all proposed subdivisions.\\*
\emph{(‘83 Code, SEC. 12.30, Subd. 6)}
\subsection{Soil Erosion and Sediment Control}
The standards related to soil erosion and sediment control contained in the performance standards of the zoning chapter shall be applicable to all proposed subdivisions.\\*
\emph{(‘83 Code, SEC. 12.30, Subd. 7)}
\subsection{Density Credit}
In order to protect environmentally sensitive areas such as wetlands, marshes, steep slopes, woodlands, a density credit system shall be allowed for residential developments.  The overall density of the zoning districts shall not be exceeded and the standards set forth in the planned unit development provisions of the zoning chapter shall be applicable.\\*
\emph{(‘83 Code, SEC. 12.30, Subd. 8)}

\section{Parks, Open Space and Public Use}
\subsection{}
Where a proposed park, playground, school site, or other public site shown on an adopted comprehensive plan or official map is embraced in part or in whole by a boundary of a proposed subdivision, and the public ground shall be shown as reserved land on the preliminary plat to allow the Council, Board of Education or county and state agency the opportunity to consider and take action toward acquisition of the public ground or park or school site by purchase or other means prior to approval of the final plat.
\subsection{}
It is declared general policy that in all new subdivisions, percentage of the gross area of all property subdivided shall be dedicated for parks, playgrounds, or other public use. The percentage shall be in addition to the property dedicated for streets, alleys, waterways, pedestrian ways or other public ways. The following schedule shall be applicable to all subdivisions. This schedule is based upon the density of the development allowed in each district and is intended to equalize the amount and value of land dedicated for parks per dwelling unit in the various districts.
\begin{center}
    \begin{tabular}{|c|c|}
        \hline
        \textbf{In areas zoned:} & \\
        \hline
        R-1 & 5\% of the total land area\\
        \hline
        R-2 & 6\% of the total land area\\
        \hline
        R-3 & 8\% of the total land area\\
        \hline
    \end{tabular}
\end{center}
\subsubsection{}
No areas may be dedicated as parks, playgrounds, or public lands until the areas have been approved for the purpose to which they are to be dedicated. The park land shall be graded to the contours set forth in the preliminary plat.
\subsubsection{}
The developer shall provide a minimum of three inches of black dirt over the entire park area and the area shall be seeded with a type of seed approved by the city. The financial guarantees by the developer to the city shall be in effect at least until the time that the park land is graded and seeded.
\subsection{}
At least 50\% of the gross area dedicated for parks, open space or public use shall be suitable for active recreation use. Active recreation meaning organized playground activities such as softball, football, and the like. These areas to be used for organized playground activities shall have a slope of less than 2\% grade and be largely clear of forest vegetation. Other areas to be dedicated may be forested and may have steeper slopes.
\subsection{}
When the subdivision is small or does not include a park or public area shown on the comprehensive plan, or if in the judgment of the Council the area proposed to be dedicated is not suitable or desirable for park/playground purposes because of location, size or other reason, the Council may require, in lieu of land dedication, a payment to the city of a sum equal to the percentage listed above of the market value of the land to be subdivided as determined by the City Assessor. The undeveloped land value shall be the value of the land when ready to be platted but not including utility costs. The Council and/or its agents shall have the authority to make the final determination of the value of the land for purposes of park dedication. If requested, the Council shall provide the developer or landowner with the methodology used to calculate the value of the land.
\subsection{}
The dedication of land for public use shall be without restrictions or reservations and shall be transferred to the city by deed or by plat. Money given to the city in lieu of land shall be used by the city only for acquiring or developing public park land.\\*
\emph{(‘83 Code, SEC. 12.30, Subd. 9)  (Ord. 536, effective 7-1-83)}\\


\begin{center}
    \emph{\textbf{\LARGE{REQUIRED IMPROVEMENTS}}}
\end{center}

\setcounter{section}{49}
\section{Improvements Required for All Subdivisions}
The subdivider and/or developer shall be required to provide the following improvements for all subdivisions unless the Council elects to do so under a subdivision agreement and financial guarantees as provided for herein or by assessing benefited property for all or a portion of the cost of required improvements under the development contract and financial guarantees from the subdivider-developer as the Council may require.
\subsection{Monuments}
Steel monuments shall be placed at lot corners, block corners, angle points, points of curves and at intermediate points as shown on the final plat. The installation shall be the subdivider’s expense and responsibility. All U.S., state, county or other official benchmarks, monuments, or triangulation stations in or adjacent to property shall be preserved in precise position.
\subsection{Streets}
\subsubsection{Grading}
Streets shall be graded to the full width of the right-of-way in accordance with street grades submitted to and approved by the Engineer or as approved by him. All street grading and gravel base construction shall be in accordance with specifications on file in the Engineer’s office. Grading shall be complete prior to installation of applicable underground utilities, either private or public in nature. Gravel base construction shall be undertaken after completion of the installation of underground utilities.
\subsubsection{Surfacing}
Following the City Engineer approval of street grading and after utility installation, streets shall be surfaced and provided with concrete curbs and gutters in accordance with the latest recommended plans and specifications prepared by the Engineer, approved by the Council, and on file in the Recorder’s office.\\*
\emph{(Ord. 536, effective 7-1-83)}
\subsubsection{Sidewalks and Driveways}
In cases where driveways are constructed after curbing and sidewalk are in place, the sidewalk shall be reconstructed in accordance with driveway specifications to the width of the driveway.\\*
\emph{(Ord. 13, 2nd Series, effective 5-15-84)}
\subsection{Utilities}
\subsubsection{Installation}
All utilities, whether private or public, shall be installed underground so as to enhance the visual appearance of the area, unless special permission is granted by the Council for other installations. Where utilities are to be installed in street or alley rights-of-way, the installations shall take place prior to street surfacing. Water and sewer services shall be laid to the property line.
\subsubsection{Sanitary Sewer}
Sanitary sewer facilities adequate to serve the subdivision shall be installed in accordance with the latest plans and specifications of the Engineer and shall meet the requirements of the master plan for sanitary sewer extensions of the city. All new construction shall be connected to the sanitary sewer system.
\subsubsection{Water Supply}
Water distribution facilities adequate to serve the subdivision shall be installed in accordance with the latest plans and specifications of the Engineer and shall meet the requirements of the master plan for watermain extensions of the city. All new construction shall be connected to the water system.
\subsection{Drainage Facilities}
Storm sewer and/or other surface drainage facilities shall be installed as determined to be necessary by the Engineer for the proper drainage of surface waters.\\*
\emph{(Ord. 536, effective 7-1-83)}
\subsection{Tree Planting or Street Trees}
In areas lacking trees, the type, size and location shall be as determined by the city.\\*
\emph{(Ord. 13, 2nd Series, effective 5-15-84)}
\subsection{Specifications and Inspections}
Unless otherwise stated, all of the required improvements shall conform to engineering standards and specifications as required by the Council. The improvements shall be subject to inspection and approval by, and shall be made in sequence as determined by the City Engineer.\\*
\emph{(‘83 Code, SEC. 12.40)}
\section{Payment for Installation of Improvements}
\subsection{General}
The required improvements as listed in this chapter are to be furnished and installed at the sole expense of the subdivider. However, if the cost of an improvement would by general policy be assessed only in part to the improved property and the remaining cost paid out of general tax levy, provision may be made for the payment of a portion of the cost by the city. Further, if any improvement installed within the subdivision will be of substantial benefit to lands beyond the boundaries of the subdivision, provision may be made for causing a portion of the cost of the improvement, representing the benefit to the lands, to be assessed against the same. In such a situation, the subdivider will be required only to pay for the portion of the whole cost of said improvement as will represent the benefit to the property within the subdivision.
\subsection{Agreement Providing for the Installation of Improvements}
\subsubsection{}
Prior to the installation of any required improvements and prior to approval of the plat, the subdivider shall enter into a contract in writing with the city requiring the subdivider or the city to furnish and construct said improvements at the subdivider’s sole cost and in accordance with plans and specifications and usual contract conditions. This shall include provisions for supervision for details of construction by the Engineer and shall grant to the Engineer authority to correlate the work to be done under said contract by any subcontractor authorized to proceed thereunder and with any other work being done or contracted by the city in the vicinity. The agreement shall require the subdivider to make an escrow deposit or, in lieu thereof, to furnish a performance bond, the amount of the deposit or penal amount of the bond to be equal to 125\% of the Engineer’s estimate of the total cost of the improvements to be furnished under the contract, or the lesser amount as the Council has authorized, including the cost of inspection. On request of the subdivider, the contract may provide for completion of part or all of the improvements covered thereby prior to acceptance of the plat. In the event the amount of the deposit or bond may be reduced in a sum equal to the estimated cost of the improvements so completed prior to the acceptance of the plat. The time for completion of the work and the several parts thereof shall be determined by the city upon recommendation of the Engineer after consultation with the subdivider. It shall be reasonable with relation to the work to be done, the seasons of the year, and proper correlation with construction activities in the plat and subdivision.
\subsubsection{}
No subdivider shall be permitted to start work on any other subdivision without special approval of the Council if he or she has previously defaulted on work or commitments.
\subsection{Financial Guarantee}
\subsubsection{}
The contract provided for in division (B) of this section shall require the subdivider to make an escrow deposit or, in lieu thereof, furnish a performance bond. The escrow deposit or performance bond shall conform to the requirements of this section.
\subsubsection{}
An escrow deposit shall be made with the city in a sum equal to 125\% of the total cost as estimated by the Engineer of all the improvements to be furnished and installed by the subdivider pursuant to the contract, which have not been completed prior to approval of the plat. The total costs shall include costs of inspection by the city. The city shall be entitled to reimburse itself out of said deposit for any cost and expense incurred by the city for completion of the work in case of default of the subdivider under said contract, and for any damages sustained on account of any breach thereof.  Upon completion of the work and termination of any liability, the balance remaining in said deposit shall be refunded to the subdivider.
\subsubsection{}
In lieu of making the escrow deposit, the subdivider may furnish a bank letter or credit or performance bond with corporate surety in a penal sum equal to 125\% of the total cost as estimated by the Engineer of all the improvements to be furnished and installed by the subdivider pursuant to the contract, which have not been completed prior to the approval of the plat. The bond shall be approved as to form by the attorney and filed with the Clerk-Treasurer.
\subsubsection{}
In the event the subdivider defaults in the terms in addition to all other rights and remedies authorized by this chapter and as otherwise provided by law, the city may complete the project referred to in the contract and assess all costs of the completion incurred by the city against the real property being subdivided as a special assessment and collect it the same as if it were any other special assessment levied by the city against real property.
\subsection{Construction Plans and Inspection}
\subsubsection{}
Construction plans for the required improvements conforming in all respects with the standards and city code provisions shall be prepared at the subdivider’s expense by a professional engineer who is registered in the State of Minnesota, and the plans shall contain his or her certificate. The plans together with the quantities of construction items shall be submitted to the Engineer for his or her approval and for his or her estimate of the total costs of the required improvement. Upon approval, the plans shall become a part of the required contract. The tracings of the plans approved by the Engineer plus two prints shall be furnished to the city to be filed as a public record.
\subsubsection{}
All required improvements on the site that are to be installed under the provisions of this regulation shall be inspected during the course of construction by the City Engineer at the subdivider’s expense, and acceptance by the city shall be subject to the Engineer’s certificate of compliance with the contract.
\subsection{Improvements Completed Prior to Approval of the Plat}
Improvements within a subdivision which have been completed prior to application for approval of the plat or execution of the contract for installation of the required improvements shall be accepted as equivalent improvements in compliance with the requirements only if the Engineer shall certify that he or she is satisfied that the existing improvements conform to applicable standards.\\*
\emph{(Ord. 536, effective 7-1-83)}\\*
\emph{(‘83 Code, SEC. 12.41)}

\begin{center}
    \emph{\textbf{\LARGE{ADMINISTRATION AND ENFORCEMENT}}}
\end{center}

\setcounter{section}{64}
\section{Plat Presentation Procedures}
The following procedures shall be followed in the administration of this chapter and no real property within the jurisdiction of this chapter shall be subdivided and offered for sale or a plat recorded until a preliminary plat and a final plat of the proposed subdivision have been reviewed by the Planning Commission and the city staff, and until the final plat has been approved by the Council as set forth in the procedures provided herein. PUD’s shall be presented in the same manner as other plats for the review of the Planning Commission and the approval of the Council.\\*
\emph{(‘83 Code, SEC. 12.10, Subd. 1)}
\section{Pre-Application Meeting}
\subsection{}
Prior to the preparation of a preliminary plat, the subdividers or owners shall meet with the Zoning Administrator, Engineer, and other appropriate officials in order to be made fully aware of all applicable city code provisions, regulations, and plans in the area to be subdivided. At this time or at subsequent informal meetings, the subdivider may submit a general sketch plan of the proposed subdivision and preliminary proposals for the provision of water supply and waste disposal. The sketch plan can be presented in simple form but should show that consideration has been given to the relationship of the proposed subdivision to existing city facilities that would serve it, to neighboring subdivisions and developments, and to the natural resources and topography of the site.
\subsection{}
 The subdivider is urged to avail himself of the advice and assistance of the local planning staff at this point in order to save time and effort, and to facilitate the approval of the preliminary plat.\\*
 \emph{(‘83 Code, SEC. 12.10, Subd. 2)}
\section{Preliminary Plat}
\subsection{}
After the pre-application meeting, the subdivider shall submit six or more copies of the preliminary plat to the Zoning Administrator at least seven days prior to the Planning Commission meeting at which the plat is to be considered. The subdivider shall include a written statement along with the preliminary plat describing the proposed subdivision. The written statement shall include the anticipated development of existing natural features and vegetation, and any other information required by the subdivision regulations.
\subsection{}
The Zoning Administrator shall submit one copy of the preliminary plat to the Planning Commission, the City Engineer, and any other appropriate city officials. One copy shall also be submitted to the County Engineer if the plat abuts a county road and one copy to the State Department of Transportation if the plat abuts a state highway for review and comment.
\subsection{}
The City Engineer and Zoning Administrator and other appropriate city officials shall review the preliminary plat and shall transmit a report of their findings and recommendations together with any supporting material to the Planning Commission prior to the meeting at which the plat is to be considered. The subdivider shall be required to pay the cost of the services and the Council shall establish a fee from time to time to cover the costs. The City Engineer or Zoning Administrator may require additional information which shall be provided at the subdivider’s cost.
\subsection{}
The Planning Commission may require qualified technical and staff services such as economic and legal to review the preliminary plat and advise on its suitability regarding general planning; conformity with plans of other private and public organizations and agencies; adequacy of proposed water supply, sewage disposal, drainage and flood control, special assessment procedures and other features. The subdivider shall also be required to pay the cost of the services.
\subsection{}
Within 30 days after the plat has been filed and after reports and certifications have been received as requested, the Planning Commission shall hold a public hearing on the preliminary plat after notice of the time and place thereof has been published once in the official newspaper at least ten days before the day of the hearing. This shall constitute the public hearing on the plat as required by state law. Within 30 days of the conclusion of the public hearing, the Planning Commission shall make its report to the Council.
\subsection{}
The Planning Commission may forward to the Council a favorable, conditional, or unfavorable report and the reports shall contain a statement of findings and recommendations.
\subsection{}
The Council shall act to approve or disapprove. The preliminary application must be approved or disapproved by the Council within 120 days following the delivery of an application completed in compliance with these regulations by the applicant, unless an extension of the review period has been agreed to by the applicant. If the city fails to preliminary approve or disapprove an application within the review period, the application shall be deemed approved, and upon demand the city shall execute a certificate to that effect. If the Council disapproves the preliminary plat, the grounds for any disapproval shall be set forth in the minutes of the Council meeting and reported to the owners or subdividers.
\subsection{}
The approval of a preliminary plat is an acceptance of the general layout as submitted, and indicates to the subdivider that he or she may proceed toward preparation of a final plat in accordance with the terms of approval and provisions of the subdivision regulations.
\subsection{}
During the intervening time between approval of the preliminary plat and the signing of the final plat, the subdivider must submit acceptable engineering plans for all required improvements to the City Engineer for review.
\subsection{}
In the case of all subdivisions, the Planning Commission shall recommend denial of, and the Council may deny, approval of a preliminary or final plat if it makes any of the following findings:
\begin{enumerate}[{\indent}1)]
    \item That the proposed subdivision, including the design, is in conflict with any adopted component of the comprehensive plan.
    \item That the physical characteristics of this site, including but not limited to topography, vegetation, susceptibility to erosion and siltation, susceptibility to flooding, water storage, drainage and retention, are such that the site is not suitable for the type of development or use contemplated;
    \item That the site is not physically suitable for the proposed density of development;
    \item That the design of the subdivision or the proposed improvements are likely to cause substantial environmental damage;
    \item That the design of the subdivision or the type of improvements is likely to cause serious public health problems; and
    \item That the design of the subdivision or the type of improvements will conflict with easements of record.
\end{enumerate}
\emph{(‘83 Code, SEC. 12.10, Subd. 3)}
\section{Final Plat}
\subsection{}
The subdivider shall engage a registered land surveyor to prepare a final plat which shall constitute that portion of the preliminary plat which the subdivider proposes to record and develop at the time.
\subsection{}
The subdivider shall submit six or more copies of the final plat to the Zoning Administrator at least seven days before the Planning Commission meeting at which the plat is to be considered.  The final plat shall be submitted within one year of preliminary plat approval; otherwise, the approval shall become null and void.  In the event the preliminary plat is not entirely platted in final form within five years of approval, the preliminary plat shall be considered null and void.
\subsection{}
The final plat shall have incorporated all changes required by the city, County Engineer regarding county roads, and State Department of Transportation regarding state highways, but in all other respects it shall conform to the preliminary plat as approved.
\subsection{}
The Zoning Administrator shall transmit one copy of the final plat to each member of the Planning Commission, City Engineer, City Attorney, and other appropriate city officials.
\subsection{}
The city staff shall review the final plat and shall transmit reports of their recommendations to all Planning Commission members prior to the meeting at which plat is to be considered.
\subsection{}
The Planning Commission shall study the final plat, considering the reports of the City Engineer, City Attorney, and other city departments and/or employees, and then shall transmit its recommendations to the Council within 30 days of submittal to the Zoning Administrator.
\subsection{}
The Council shall act upon the final plat within 30 days of receiving the recommendations of the Planning Commission, whereupon the Clerk-Treasurer shall notify the subdivider of the Council’s action. Upon request by the applicant for final approval by the city, the city shall certify final approval within 60 days if the applicant has complied with all conditions and requirements of the regulations and all conditions and requirements upon which the preliminary approval was conditioned, either through performance or agreements assuring performance. If the city fails to certify final approval within the time frame, and if the applicant has complied with all conditions or requirements, the final plat shall be deemed approved and upon demand, the city shall execute a certificate to that effect.
\subsection{}
Upon approval of the final plat by the Council, the subdivider shall record the final plat with the Polk County Recorder, as provided for by that office, within 60 days after approval.  Otherwise the approval of the final plat shall be considered void. The subdivider shall, within 30 days of recording, furnish the Clerk-Treasurer with a reproducible print of the final plat showing evidence of the recording.\\*
\emph{(‘83 Code, SEC. 12.10, Subd. 4)}
\section{Effect of Subdivision Approval}
For one year following preliminary approval and for two years following final approval, unless the subdivider and the city agree otherwise, no amendment to a comprehensive plan or official control shall apply to or affect the use, development, density, lot size, lot layout, or dedication or platting required or permitted by the approved application. Thereafter, pursuant to its regulations, the city may extend the period by agreement with the subdivider and subject to all applicable performance conditions and requirements, or it may require submission of a new application unless substantial physical activity and investment has occurred in reasonable reliance on the approved application and the subdivider will suffer substantial financial damage as a consequence of a requirement to submit a new application. In connection with a subdivision involving planned and staged development, the city may by resolution or agreement grant the rights referred to herein for the period of time longer than two years which it determines to be reasonable and appropriate.\\*
\emph{(‘83 Code, SEC. 12.10, Subd. 5)}
\section{Disclosure by Seller; Purchaser’s Action for Damages}
A person conveying a new parcel of land which, or the plat for which, has not previously been filed or recorded, and which is part of or would constitute a subdivision to which adopted subdivision regulations apply, shall attach to the instrument of conveyance either: recordable certification by the Clerk-Treasurer that the subdivision regulations do not apply, or that the subdivision has been approved by the Council or that the restrictions on the division of taxes and filing and recording have been waived by resolution of the Council in this case because compliance will create an unnecessary hardship and failure to comply will not interfere with the purpose of the regulations; or a statement which names and identifies the location of the appropriate city offices and advises the grantee that subdivision and zoning regulations may restrict the use or restrict or prohibit the development of the parcel, or construction on it, and that the division of taxes and the filing or recording of the conveyance may be prohibited without prior recordable certification of approval, nonapplicability, or waiver from the city. In any action commenced by a buyer of a parcel against the seller thereof, the misrepresentation of or the failure to disclose material facts in accordance with this subdivision shall be grounds for damages.  If the buyer establishes his or her right to damages, a district court hearing the matter may in its discretion also award to the buyer an amount sufficient to pay all or any part of the costs incurred in maintaining the action, including reasonable attorney fees, and an amount for punitive damages not exceeding 5\% of the purchase of the land.\\*
\emph{(‘83 Code, SEC. 12.10, Subd. 6)  (Ord. 536, effective 7-1-83)}
\section{Modifications, Exceptions and Variances}
The Council may grant a variance upon receiving a report from the Planning Commission in any particular case where the subdivider can show by reason of exceptional topography or any other physical conditions that strict compliance with these regulations would cause exceptional and undue hardship provided the relief may be granted without detriment to the public welfare and without impairing the intent and purpose of these regulations. The Planning Commission may recommend variations from the requirements of this chapter in specific which, in its opinion, do not affect the comprehensive plan or the intent of this chapter.  Any modifications thus recommended shall be entered in the minutes of the Planning Commission in setting forth the reasons which justify the modifications.  The Council may approve variances from these requirements in specific cases which in its opinion meets the above requirements and do not adversely affect the purposes of this chapter.\\*
\emph{(‘83 Code, SEC. 12.50, Subd. 1)}
\section{Planned Unit Developments}
Upon receiving a report from the Planning Commission, the Council may grant a variance from the provisions of these regulations in the case of a planned unit development, as defined in the zoning chapter, provided that the Council shall find that the proposed development is fully consistent with the purposes and intent of these regulations. This provision is intended to provide the necessary flexibility for new land planning and land development trends and techniques.\\*
\emph{(‘83 Code, SEC. 12.50, Subd. 2)}
\section{Minor Subdivisions}
\subsection{}
In the case of a subdivision resulting in five parcels or less situated in the city where conditions are well defined, the Council may exempt the subdivider from complying with some of the requirements of these regulations. In the case of a request to subdivide a lot which is a part of a recorded plat, or where the subdivision is to permit the adding of a parcel of land to an abutting lot or to create not more than three new lots, and the newly created property lines will not cause any resulting lot to be in violation of these regulations or the zoning chapter, the division may be approved by the Council, after submission of a survey by a registered land surveyor showing the original lot and the proposed subdivision.
\subsection{}
In the case of a request to divide a lot which is a part of a recorded plat where the division is to permit the adding of a parcel of land to an abutting lot or to create two lots and the newly created property line will not cause the other remaining portion of the lot to be in violation with this regulation or the zoning chapter, the division may be approved by the Council after submission of a survey by a registered land surveyor showing the original lot and the proposed subdivision.\\*
\emph{(‘83 Code, SEC. 12.50, Subd. 3)  (Ord. 536, effective 7-1-83)}

\setcounter{section}{98}
\section{Penalty}
Every person who violates a section, subdivision, paragraph or provision of this chapter when he or she performs an act thereby prohibited or declared unlawful, or fails to act when the failure is thereby prohibited or declared unlawful, and upon conviction thereof, shall be punished as for a misdemeanor except as otherwise stated in specific provisions hereof. The penalty which may be imposed for any crime which is a misdemeanor under this code, including Minnesota Statutes specifically adopted by reference, shall be a sentence of not more than 90 days or a fine of not more than \$1,000, or both. The costs of prosecution may be added. A separate offense shall be deemed committed upon each day during which a violation occurs or continues.\\*
\emph{(‘83 Code, SEC. 12.99)  (Ord. 536, effective 7-1-83)}
