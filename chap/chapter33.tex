\chapter*{Chapter 33: \\
	Special Service Districts}
    \addstarredchapter{Chapter 33: Special Service Districts}
    \vfill
    \minitoc
    \pagebreak
    
\subchapter{FLOOD CONTROL DISTRICT}

\section{Establishment}
\index{SPECIAL SERVICE DISTRICTS!FLOOD CONTROL DISTRICT!Establishment}
Pursuant to the authority granted by the Legislature in Laws of 1997, 2nd Special Session, Chapter 2, Section 29, and M.S. Chapter 428A, as it may be amended from time to time, a Flood Control District is established within the city.  The Flood Control District shall include all property located within the corporate limits of the city.  The Flood Control District shall be a special service district as described in M.S. Chapter 428A, as it may be amended from time to time, and, except as specifically provided in this section, shall be governed by, and implemented in accordance with, the provisions of M.S. Chapter 428A, as it may be amended from time to time, except that:
\subsection{}
The provisions of M.S. \textsection 428A.08, as it may be amended from time to time, shall not apply; and
\subsection{}
The special service charges shall be levied on all property within the Flood Control District, and not limited to commercial, industrial and public utility property.

\section{Flood Control Improvements}
\index{SPECIAL SERVICE DISTRICTS!FLOOD CONTROL DISTRICT!Flood Control Improvements}
\subsection{}
The city may undertake within the Flood Control District, from time to time, flood control improvements, including:
\begin{enumerate}
    \item The acquisition of properties within or adjacent to the flood plain;
    \item The demolition or removal of structures or improvements within or adjacent to the flood plain or where necessary to permit the construction or extension of flood control works; and
    \item The construction, reconstruction, extension or maintenance of levees, dikes and other flood control works.
\end{enumerate}
\subsection{}
The city shall, before undertaking any flood control improvements under this subchapter, submit to the Commissioner of Natural Resources a description of the proposed flood control improvements.  The city may proceed with the flood control improvements upon either  approval of the Commissioner of Natural Resources or failure of the Commissioner of Natural Resources to either approve or reject the improvements within 30 days of such submission.
\subsection{}
The costs of acquiring, constructing, reconstructing, extending or maintaining the flood control improvements may be paid for by the special service charges described in SEC. 33.03, from special assessments or improvement bonds issued under M.S. Chapter 429, as it may be amended from time to time, from federal or state grants, from money appropriated by the city from other sources, or from any combination of those sources.

\section{Special Service Charges}
\index{SPECIAL SERVICE DISTRICTS!FLOOD CONTROL DISTRICT!Special Service Charges}
The city shall, by resolution, establish special service charges which will be imposed annually on all owners of real property within the district. The amount of the special service charges to be imposed upon each parcel of real property shall be determined in accordance with the formula established by the governing body of the city, which formula may take into consideration such factors as the governing body shall consider relevant, including without limitation, the proximity of the property to the flood plain and the classification of the property under M.S. \textsection 273.13, as it may be amended from time to time. The special service charges will be imposed annually for the number of years determined by the governing body, provided that no special service charges shall be imposed more than 20 years from the date of adoption of this subchapter. The collections of the special service charges shall be deposited by the city into a flood control improvement fund, to be used to pay the costs of flood control improvements or to pay principal of and interest on bonds issued pursuant to M.S. Chapter 429, as it may be amended from time to time, to pay the costs of flood control improvements.



\subchapter{DOWNTOWN SPECIAL SERVICE DISTRICT}

\setcounter{section}{14}
\section{Establishment}
\index{SPECIAL SERVICE DISTRICTS!DOWNTOWN SPECIAL SERVICE DISTRICT!Establishment}
Pursuant to the authority granted by the Legislature in Laws of Minnesota, 1991, Chapter 291, Article 4, Section 24, and M.S. \textsection 428A.01 to \textsection 428A.10, as it may be amended from time to time, a special service district is established wherein the city may render or contract for public services to be rendered, of a kind or degree not ordinarily provided throughout the city from general fund revenues. The special service district consists of that area described as follows:\\
\\
Beginning at the intersection of the northerly side of Sixth Street and the easterly side of Burlington Northern Mainline rail bridge; then proceeding in a southerly direction along the easterly side of the main rail line to a point on the northerly bank of the Red Lake River; then proceeding in an easterly direction along the north bank of the Red Lake River to the point where it intersects with the westerly boundary of Broadway; then proceeding in an easterly direction parallel to the northerly line of Sixth Street to a point on the midline of the block bounded by Broadway on the west, Loring on the north and Ash on the east; then proceeding in a northerly direction along such midline extended to a point on the south boundary of Fletcher Street; then proceeding east along the south line of Fletcher Street to a point on the east line of Ash Street; then proceeding north along the east line of Ash Street to a point on the north line Third Street; thence proceeding west along the north line of Third Street to a point on the east line of Ash Street; thence proceeding north along the east line of Ash Street to a point on the south side of Fourth Street; thence proceeding east along the south side of Fourth Street to a point on the east side of Ash Street extended; then proceeding north along the east side of Ash Street as extended and the east side of Ash Street to a point on the north side of Fifth Street; then proceeding west along the north side of Fifth Street to a point on the east side of Broadway; then proceeding north along the east side of Broadway to a point on the north side of Sixth Street; then proceeding west along the north side of Sixth Street to the point of beginning.

\section{Services}
\index{SPECIAL SERVICE DISTRICTS!DOWNTOWN SPECIAL SERVICE DISTRICT!Services}
Within the special service district, the city may render or contract for any service or services to the extent that the service or services are of a kind or degree that is not ordinarily provided throughout the city from general revenues of the city, including, but not limited to, the repair, maintenance, operation, and construction of any improvement authorized by M.S. \textsection 429.021, as it may be amended from time to time, parking services rendered or contracted for by the city, and any other service or improvement provided by the city or the Crookston Development Authority that is authorized by law or charter.

\section{Service Charges}
\index{SPECIAL SERVICE DISTRICTS!DOWNTOWN SPECIAL SERVICE DISTRICT!Service Charges}
The city may impose service charges that are reasonably related to the special services provided.  In imposing the service charges, the city must follow the procedures specified in M.S. \textsection 428A.01 to \textsection 428A.10, as it may be amended from time to time, including, but not limited to, imposing the service charges only after the filing of the required petition and only upon property within the special service district having the specified classification and use.  Charges for service must be as nearly as possible proportionate to the cost of furnishing the service, and may be fixed on the basis of the service directly rendered, or by reference to a reasonable classification of the types of premises to which service is furnished, or on any other equitable basis.

\section{Governing Law}
\index{SPECIAL SERVICE DISTRICTS!DOWNTOWN SPECIAL SERVICE DISTRICT!Governing Law}
The provisions of Laws of Minnesota 1991, Chapter 291, Article 4, Section 24 and M.S. \textsection 428A.01 to \textsection 428A.10, as it may be amended from time to time, in all respects govern the creation, existence, and operation of the special service district and the manner of imposing service charges therein and this subchapter must be construed consistently with said statutes.


\subchapter{SIDEWALK IMPROVEMENT DISTRICT}

\setcounter{section}{19}
\section{Establishment}
\index{SPECIAL SERVICE DISTRICTS!SIDEWALK IMPROVEMENT DISTRICT!Establishment}
Pursuant to the authority granted by the Legislature in MS. \textsection 435.44, as it may be amended from time to time, a Sidewalk improvement District is established within the City. The Sidewalk Improvement District shall be all property located within the City.

\section{Costs}
\index{SPECIAL SERVICE DISTRICTS!SIDEWALK IMPROVEMENT DISTRICT!Costs}
The total costs of sidewalk district improvements may be, by resolution, apportioned and assessed to all parcels or tracts of land located in the district on a uniform basis as to each classification of real estate. Where sidewalk widths are wider than the standard width of the district, the additional costs may be assessed as a direct benefit to the abutting property. An indirect district benefit assessment may involve all parcels or tracts of land located in the district without regard to the location of sidewalks.

\section{Maximum Amortization of Costs}
\index{SPECIAL SERVICE DISTRICTS!SIDEWALK IMPROVEMENT DISTRICT!Maximum Amortization of Costs}
The Council may assess the costs on all district sidewalk improvements up to a maximum of five years on equal annual installments, plus interest on the unpaid balance.

\section{Notice and Hearing}
\index{SPECIAL SERVICE DISTRICTS!SIDEWALK IMPROVEMENT DISTRICT!Notice and Hearing}
Before imposition of the costs as authorized in this subchapter, the City shall provide for notice and hearing substantially conforming to the material provisions of MS. \textsection 428A.03(1), as it may be amended from time to time.

\section{Not Exclusive Finance Method}
\index{SPECIAL SERVICE DISTRICTS!SIDEWALK IMPROVEMENT DISTRICT!Not Exclusive Finance Method}
The costs of sidewalk district improvements may be paid for by apportionment and assessment under this subchapter, from special assessments or improvement bonds issued under MS. Chapter 429, as it may be amended from time to time, from state and federal grants, from money appropriated by the City from other sources, or from any combination of such sources.
