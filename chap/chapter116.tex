\chapter*{Chapter 116: \\
	Licensing of Certain Trades}
    \addstarredchapter{Chapter 116: Licensing of Certain Trades}
    \minitoc
    \pagebreak

\begin{center}
    \emph{\textbf{\LARGE{PLUMBERS}}}
\end{center}

\section{License Required; Restriction}
\subsection{}
It is unlawful for any person to engage in the work or business of plumbing or the installation of water or sewer pipes without a license therefor from the city.\\
\emph{(‘83 Code, SEC.6.52, Subd. 1)}
\subsection{}
No person shall be licensed under this subchapter unless he or she shall have a license from the Minnesota State Board of Health, and any revocation or suspension by the State Board of Health shall immediately revoke or suspend the license issued by the city.\\
\emph{(‘83 Code, SEC. 6.52, Subd. 2)}\\
\emph{(Ord. 35, effective 2-10-04; Ord. 233, effective 2-13-35; Ord. 319, effective 9-14-54)  Penalty, see SEC. 110.99}
\section{Bond Required}
Before a license shall be granted to any person as a plumber, he or she shall execute and deposit with the city a corporate surety bond in the penal sum of \$2,000 conditioned upon the faithful and lawful performance of all work entered upon by him or her. The bond shall be for the benefit of persons injured or suffering financial loss by reason of failure of performance. The term of the bond shall be concurrent with the term of the license. Provided, however, no bond shall be required of a person having a license from the Minnesota State Board of Health and a bond running to the State of Minnesota.\\
\emph{(‘83 Code, SEC. 6.52, Subd. 3)}
\section{Insurance Required}
Before a license shall be granted to any person as a plumber, he or she shall have filed with the Clerk-Treasurer a policy or certificate of public liability insurance, including products liability insurance, for coverage concurrent with the license term with limits of \$100,000 for injury to one person, \$300,000 for each occurrence, and \$25,000 for property damage.\\
\emph{(‘83 Code, SEC. 6.52, Subd. 4)  (Ord. 8, 2nd Series, effective 5-15-84)}

\begin{center}
    \emph{\textbf{\LARGE{GAS FITTERS}}}
\end{center}

\setcounter{section}{14}
\section{License Required}
It is unlawful for any person to install, alter, service or repair gas piping, appliances or appurtenances, without a license therefor from the city.\\
\emph{(‘83 Code, SEC. 6.53, Subd. 1) Penalty, see SEC.110.99}
\section{Insurance Required}
Before a license shall be granted to any person as a gas fitter, he or she shall have filed with the Clerk-Treasurer a policy or certificate of public liability insurance, including products liability insurance, for coverage concurrent with the license term with limits of \$100,000 for injury to one person, \$300,000 for each occurrence, and \$25,000 for property damage.\\
\emph{(‘83 Code, SEC.6.53, Subd 2)}

\begin{center}
    \emph{\textbf{\LARGE{CEMENT CONTRACTORS}}}
\end{center}

\setcounter{section}{29}
\section{Definition}
For the purpose of this subchapter the following definition shall apply, unless the context clearly indicates or requires a different meaning.
\begin{description}
    \item[CEMENT CONTRACTOR] Any person who constructs, reconstructs or repairs concrete sidewalks, curbs or gutters upon the public streets of the city.
\end{description}
\emph{(‘83 Code, SEC. 6.58, Subd. 1)}
\section{License Required}
It is unlawful for any cement contractor to engage in the business without a license therefor from the city.\\
\emph{(‘83 Code, SEC. 6.58, Subd. 2)  Penalty, see SEC. 110.99}
\section{Bond and Insurance}
\subsection{}
No license shall be issued until the applicant has filed with the city a bond in the penal sum of \$1,000 on which the city is obligee, conditioned on the use of quality materials for all work, and performance of the work in a good and workmanlike manner.\\
\emph{(Ord. 8, 2nd Series, effective 5-15-84)}
\subsection{}
No license shall issue until the applicant has filed with the city a policy or certificate of public liability insurance for coverage concurrent with the licensed term and with limits of at least \$100,000 for injury to one person, \$300,000 for each occurrence, and \$25,000 property damage.\\
\emph{(‘83 Code, SEC. 6.58, Subd. 3)}
