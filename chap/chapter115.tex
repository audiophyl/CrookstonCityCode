\chapter*{Chapter 115: \\
	Garbage and Refuse Haulers}
    \addstarredchapter{Chapter 115: Garbage and Refuse Haulers}
    \vfill
    \minitoc
    \pagebreak

\section{Definitions\footnote{(‘83 Code, SEC. 6.51, Subd. 1)}}
\index{GARBAGE AND REFUSE HAULERS!Definitions}
For the purpose of this chapter the following definitions shall apply, unless the context clearly indicates or requires a different meaning.
\begin{description}
    \item[GARBAGE] All putrescible wastes, including animal offal and carcasses of dead animals but excluding human excreta, sewage and other water-carried wastes.
    \item[OTHER REFUSE] Ashes, glass, crockery, cans, paper, boxes, rags and similar non-putrescible wastes but excluding sand, earth, brick, stone, concrete, trees, tree branches and wood.
\end{description}

\section{License Required}
\index{GARBAGE AND REFUSE HAULERS!License Required}
It is unlawful for any person to haul garbage or other refuse for hire without a license therefor from the city, or to haul garbage or other refuse from his or her own residence or business property other than as herein excepted.\footnote{(‘83 Code, SEC. 6.51, Subd. 2)  Penalty, see SEC. 110.99}

\section{Exception}
\index{GARBAGE AND REFUSE HAULERS!Exceptions}
\subsection{}
Nothing in this chapter shall prevent persons from hauling garbage or other refuse from their own residences or business properties provided the following rules are observed:
\begin{enumerate}[{\indent}1)]
    \item That all garbage is hauled in containers that are water-tight on all sides and the bottom and with tight-fitting covers on top;
    \item That all other refuse is hauled in vehicles with leak-proof bodies and completely covered or enclosed by canvas or other means or material so as to completely eliminate the possibility of loss of cargo; and
    \item That all garbage and other refuse shall be dumped or unloaded only at the designated sanitary landfill.
\end{enumerate}
\subsection{}
Haulers acting under contract with the city shall not be required to be licensed under this chapter.\footnote{(‘83 Code, SEC. 6.51, Subd. 3)  Penalty, see SEC. 110.99}

\section{Hauler License Requirements}
\index{GARBAGE AND REFUSE HAULERS!Hauler License Requirements}
\subsection{}
Hauler licenses shall be granted only upon the condition that the licensee have water-tight, packer-type vehicles in good condition to prevent loss in transit of liquid or solid cargo, that the vehicle be kept clean and as free from offensive odors as possible and not allowed to stand in any street longer than reasonably necessary to collect garbage or refuse, and that the same be dumped or unloaded only at the designated sanitary land-fill, and strictly in accordance with regulations relating thereto.
\subsection{}
Before a garbage and refuse hauler’s license shall be issued, the applicant shall file with the Clerk-Treasurer evidence that he or she has provided public liability insurance on all vehicles in at least the sum of \$100,000 for the injury of one person, \$300,000 for the injury of two or more persons in the same accident, and \$50,000 for property damages.
\subsection{}
Licensees shall deliver all refuse to the sanitary land-fill and shall be required to pay non-resident rates for any refuse collected outside the city.  Collection outside the city and failure to pay non-resident rates therefor shall be grounds for revocation of the license.
\subsection{}
The Council, in the interest of maintaining healthful and sanitary conditions in the city, hereby reserves the right to specify and assign certain areas to all licensees, and to limit the number of licenses issued.
\subsection{}
Each applicant shall file with the Clerk-Treasurer, before a garbage and refuse hauler’s license is issued or renewed, a schedule of proposed rates to be charged by him or her during the licensed period for which the application is made.  The schedule of proposed rates, or a compromise schedule thereof, shall be approved by the Council before granting the license.  Nothing herein shall prevent a licensee from petitioning the Council for review of the rates during the licensed period, and the Council may likewise consider the petition and make new rates effective at any time.  No licensee shall charge rates in excess of the rates approved by the Council.
\subsection{}
No license shall be issued under this chapter for hauling refuse from residential property.\footnote{(‘83 Code, SEC. 6.51, Subd. 4)}
