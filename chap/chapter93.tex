\chapter*{Chapter 93: \\
	Fire Prevention}
    \addstarredchapter{Chapter 93: Fire Prevention}
    \vfill
    \minitoc
    \emph{\textbf{Cross-reference:}\\
        {\indent}Fire Department established, see SEC. 32.030\\
        {\indent}Fireworks not permitted in city parks, see SEC. 95.27\\
        {\indent}Display, Sale Storage, Possession and Use of Fireworks, see Chapter 120}
    \pagebreak

\subchapter{INTERNATIONAL FIRE CODE}

\section{Adoption of Minnesota State Fire Code}
\index{FIRE PREVENTION!INTERNATIONAL FIRE CODE!Adoption of Minnesota State Fire Code}
The Minnesota State Fire Code, one copy of which has been marked as the official copy and which is on file in the office of the Fire Chief, is hereby adopted as the fire code for the city for the purpose of prescribing regulations governing conditions hazardous to life and property from fire or explosion.  Every provision contained in the code, except as modified or amended by this section, is hereby adopted and made a part of this section as if fully set forth herein.

\section{Enforcement}
\index{FIRE PREVENTION!INTERNATIONAL FIRE CODE!Enforcement}
\subsection{}
The Chief of the Fire Department, or representative authorized by him or her, is authorized to enforce the provisions of this chapter.
\subsection The Chief of the Fire Department may detail the members of the Fire Department as inspectors as shall from time to time be necessary.  The Chief of the Fire Department may recommend the employment of technical inspectors.

\section{Definitions}
\index{FIRE PREVENTION!INTERNATIONAL FIRE CODE!Definitions}
\subsection{}
Wherever the term CORPORATION COUNSEL is used in the Minnesota State Fire Code, it shall be held to mean the attorney for the city.
\subsection Wherever the word JURISDICTION is used in the Minnesota State Fire Code, it shall be held to mean the city.

\section{Districts in which Certain Substances Prohibited}
\index{FIRE PREVENTION!INTERNATIONAL FIRE CODE!Districts in which Certain Substances Prohibited}
\subsection{Establishment of Limits of Districts in which Storage of Flammable or Combustible Liquids in Outside Tanks is to be Prohibited}
\subsubsection{}
The limits referred to in the Minnesota State Fire Code in which storage of flammable or combustible liquids in outside tanks is prohibited, are hereby established as follows: Except for outside aboveground storage of Class 1B liquids in the I-1 or I-2 Zoning Districts pursuant to conditional use permit, outside aboveground storage of Class 1 liquids is prohibited within the city limits. Outside underground storage of Class 1 liquids is prohibited in the R-1, R-2, and R-3 Zoning Districts.
\subsubsection The limits referred to in the Minnesota State Fire Code in which new bulk plants for flammable or combustible liquids are prohibited, are hereby established as follows: The limits shall be within the city limits.
\subsection{Establishment of Limits in which Bulk Storage of Liquified Petroleum Gases is to be Restricted}
The limits referred to in the Minnesota State Fire Code in which bulk storage of liquified petroleum gas is restricted, are hereby established as follows: The limits shall be within the city limits.
\subsection{Establishment of Limits of Districts in which Storage of Explosives and Blasting Agents is to be Prohibited}
The limits referred to in the Minnesota State Fire Code in which storage of explosives and blasting agents is prohibited, are hereby established as follows: The limits shall be within the city limits.
\subsection{New Materials, Processes or Occupancies which may Require Permits}
The City Administrator, the Building Inspector/Official and the Fire Chief shall act as a committee to determine and specify, after giving affected persons an opportunity to be heard, any new materials, processes or occupancies which shall require permits, in addition to those now enumerated in said Code. The Fire Chief shall post the list in a conspicuous place in his or her office, and distribute copies thereof to interested persons.

\subchapter{FIRE PREVENTION REGULATION}

\setcounter{section}{14}
\section{Permit Required for Installation of Tank}
\index{FIRE PREVENTION!FIRE PREVENTION REGULATION!Permit Required for Installation of Tank}
\subsection{Definition}
As used in this section, the term \textbf{TANKS} means any above-ground or underground container used, or to be used, for the storage of flammable or non-flammable matter, except containers regulated by an agency of this state and for which a permit has been issued by the agency.
\subsection{Permit Required}
It is unlawful for the owner of any premises to install, or permit the installation, thereon without a permit therefor from the city.\footnote{Penalty, see SEC. 10.99}

\section{Storage of Wood}
\index{FIRE PREVENTION!FIRE PREVENTION REGULATION!Storage of Wood}
\subsection{Definition}
\textbf{WOOD} shall include, but not be limited to, firewood and lumber, whether rough, pre-cut, construction grade or finished, which is stored or kept on property in the city.
\subsection{Persons Exempt}
This section shall not apply to:
\begin{enumerate}[{\indent}1)]
    \item Persons having property upon which new construction is taking place and the wood on the property is being used for said construction, unless the wood has remained on the property for more than six months and is not a permanent part of the new construction at the end of that time;
    \item Persons storing or keeping wood on property, when the wood is stored or kept in neat and secured stacks in a covered structure impervious to the elements; and
    \item Commercial construction businesses, including, but not necessarily limited to, lumberyards, when operating on property other than residential property.
\end{enumerate}
\subsection{Conditions of Outside Storage}
Wood stored or kept in the city which is not contained within a covered enclosure impervious to the elements shall be stored or kept in neat and secure stacks in accordance with this section.  Stacks shall not exceed four feet in height.  Stacks shall be no less than five feet from any side property line on corner lots, 30 feet from any front property line or any front of a house, whichever is closer, and two feet from any rear property line or any side property line of interior lots.  Grass height around all wood stacks shall be maintained at a maximum height of six inches.
\subsection{Maximum Amount}
No more than four cords of wood shall be stored on any residential property.
\subsection{Existing Wood Piles}
Any woodpile in existence on the effective date of this section which does not comply with the provisions of this section shall be removed or placed in compliance.\footnote{Penalty, see SEC. 10.99}
