\chapter*{Chapter 72: \\
	Parking Regulations}
    \addstarredchapter{Chapter 72: Parking Regulations}
    \vfill
    \minitoc
    \pagebreak


\subchapter{GENERAL PROVISIONS}

\section{Presumption as to Owner and Driver}
\index{PARKING REGULATIONS!GENERAL PROVISIONS!Presumption as to Owner and Driver}
As to any vehicle parking in violation of this title or Chapter 96, when the driver thereof is not present, it shall be presumed that the owner parked the same, or that the driver was acting as the agent of the owner.\footnote{(‘83 Code, SEC. 9.01)}

\section{Impounding and Removing Vehicles}
\index{PARKING REGULATIONS!GENERAL PROVISIONS!Impounding and Removing Vehicles}
When any police officer finds a vehicle standing upon a street or city-owned parking lot in violation of any parking regulation, the officer is hereby authorized to require the driver or other person in charge of the vehicle to remove the same to a position in compliance with this chapter or Chapter 96. When any police officer finds a vehicle unattended upon any street or city-owned parking lot in violation of any parking regulation, the officer is hereby authorized to impound the unlawfully parked vehicle and to provide for the removal thereof and to remove the same to a convenient garage or other facility or place of safety; provided, that if any charge shall be placed against the vehicle for cost of removal or storage, or both, by anyone called upon to assist therewith the same shall be paid prior to removal from the place of storage or safekeeping.\footnote{(‘83 Code, SEC. 9.12)}\\


\subchapter{PARKING PROHIBITIONS AND RESTRICTIONS}

\setcounter{section}{14}
\section{Parking Contrary to Posted Restrictions or Prohibitions}
\index{PARKING REGULATIONS!PARKING PROHIBITIONS AND RESTRICTIONS!Parking Contrary to Posted Restrictions or Prohibitions}
It is unlawful for any person to park a vehicle, except an emergency vehicle, contrary to lane restrictions or prohibitions painted on any curb, or contrary to sign-posted, fenced, or barricaded restrictions or prohibitions.\footnote{(‘83 Code, SEC. 7.04, Subd. 4)  Penalty, see SEC. 70.99}

\section{General Parking Prohibitions\footnote{(‘83 Code, SEC. 9.02)  (Ord. 14, 2nd Series, effective 5-18-85)  Penalty, see SEC. 70.99}}
\index{PARKING REGULATIONS!PARKING PROHIBITIONS AND RESTRICTIONS!General Parking Prohibitions}
It is unlawful for any person to stop, stand or park a vehicle except when necessary to avoid conflict with other traffic or in compliance with the specific directions of a police officer or traffic-control device in any of the following places:
\begin{enumerate}[{\indent}A)]
\item On a sidewalk;
\item In front of a public or private driveway;
\item Within an intersection;
\item Within ten feet of a fire hydrant;
\item On a crosswalk;
\item Within 20 feet of a crosswalk at any intersection;
\item In a signposted fire lane located upon a street or alley or upon other public or private property;
\item Within 30 feet upon the approach to any flashing beacon, stop sign or traffic-control signal located at the side of a roadway;
\item Within 50 feet of the nearest rail of a railroad crossing;
\item Within 20 feet of the driveway entrance to any fire station and on the side of a street opposite the entrance to any fire station within 75 feet of the entrance when properly sign-posted;
\item Alongside or opposite any street excavation or obstruction when the stopping, standing or parking would obstruct traffic;
\item On the roadway side of any vehicle stopped or parked at the edge or curb of a street;
\item Upon any bridge or other elevated structure upon a street;
\item At any place where official signs prohibit or restrict stopping, parking or both;
\item In any alley, except for loading or unloading and then only so long as reasonably necessary for the loading and unloading to or from adjacent premises; or
\item On any boulevard which has been curbed.
\end{enumerate}

\section{Unauthorized Removal}
\index{PARKING REGULATIONS!PARKING PROHIBITIONS AND RESTRICTIONS!Unauthorized Removal}
It is unlawful for any person to move a vehicle not owned by the person into any prohibited area or away from a curb a distance as is unlawful.\footnote{(‘83 Code, SEC. 9.03)  Penalty, see SEC. 70.99}

\section{Stopping or Parking in Violation of Direction of Officer}
\index{PARKING REGULATIONS!PARKING PROHIBITIONS AND RESTRICTIONS!Stopping or Parking in Violation of Direction of Officer}
It is unlawful for any person to stop or park a vehicle on a street when directed or ordered to proceed by any police officer invested by law with authority to direct, control or regulate traffic.\footnote{(‘83 Code, SEC. 9.04)  Penalty, see SEC. 70.99}

\section{Parallel Parking}
\index{PARKING REGULATIONS!PARKING PROHIBITIONS AND RESTRICTIONS!Parallel Parking}
Except where angle parking is specifically allowed and indicated by curb marking or sign-posting, or both, each vehicle stopped or parked upon a two-way road where there is an adjacent curb shall be stopped or parked with the right-hand wheels of the vehicle parallel with, and within 12 inches of, the right-hand curb, and, where painted markings appear on the curb or the street, the vehicle shall be within the markings, front and rear; provided that upon a one-way roadway all vehicles shall be so parked, except that the left-hand wheels of the vehicle may be parallel with and within 12 inches from the left-hand curb, but the front of the vehicle in any event and with respect to the remainder of the vehicle, shall be in the direction of the flow of traffic upon the one-way street; and it is unlawful to park in violation of this section.\footnote{(‘83 Code, SEC. 9.05)  Penalty, see SEC. 70.99}

\section{Angle Parking}
\index{PARKING REGULATIONS!PARKING PROHIBITIONS AND RESTRICTIONS!Angle Parking}
Where angle parking has been established by Council resolution, and is allowed, as shown by curb marking or sign-posting, or both, each vehicle stopped or parked shall be at an angle of approximately 45 to 60 degrees with the front wheel touching the curb and within any parking lines painted on the curb or street, provided that the front wheel not touching the curb shall be the portion of the vehicle furthest in the direction of one-way traffic.  It is unlawful to park in violation of this section.\footnote{(‘83 Code, SEC. 9.06)  Penalty, see SEC. 70.99}

\section{Manner of Parking on Streets Without Curbs}
\index{PARKING REGULATIONS!PARKING PROHIBITIONS AND RESTRICTIONS!Manner of Parking on Streets Without Curbs}
Upon streets not having a curb each vehicle shall be stopped or parked parallel and to the right of the paving, improved or main traveled part of the street.  It is unlawful to park in violation of this section.\footnote{(‘83 Code, SEC. 9.07)  Penalty, see SEC. 70.99}

\section{Parking Time Limits\footnote{(‘83 Code, SEC. 9.08)  Penalty, see SEC. 70.99}}
\index{PARKING REGULATIONS!PARKING PROHIBITIONS AND RESTRICTIONS!Parking Time Limits}
Parking on streets shall be limited as follows:
\subsection{}
It is unlawful for any person to stop, park or leave standing any vehicle upon any street for a continuous period in excess of 24 hours.
\subsection{}
The Chief of Police may, when authorized by resolution of the Council, designate certain streets, blocks or portions of streets or blocks as prohibited parking zones, or five-minute, ten-minute, 15-minute, 30-minute, one-hour, two-hour, four-hour, six-hour, eight-hour, morning or afternoon rush hour limited parking zones and shall mark by appropriate signs any zones so established.  The zones shall be established whenever necessary for the convenience of the public or to minimize traffic hazards and preserve a free flow of traffic.  It is unlawful for any person to stop, park or leave standing any vehicle in a prohibited parking zone, for a period of time in excess of the sign-posted limitation, or during sign-posted hours of prohibited parking.
\subsection{}
It is unlawful for any person to remove, erase or otherwise obliterate any mark or sign placed upon a tire or other part of a vehicle by a police officer for the purpose of measuring the length of time the vehicle has been parked.
\subsection{}
For the purpose of enforcement of this section, any vehicle moved less than one block in a limited time parking zone shall be deemed to have remained stationary.
\subsection{}
It is unlawful to park or leave standing any vehicles, as follows:
\subsubsection{}
On the following downtown East-West streets from 1:00 a.m. to 7:00 a.m. on Tuesdays, Thursdays and Saturdays:
\begin{enumerate}[{\indent}a)]
\item Loring Street from Market Street to Ash Street;
\item Fletcher Street from Market Street to Ash Street;
\item Robert Street from Sampson’s Addition Bridge to the Robert Street Bridge;
\item Second Street from Market Street to Ash Street;
\item Third Street from the Burlington Northern tracks to Elm Street;
\item Fourth Street from the Burlington Northern tracks to Elm Street;
\item Fifth Street from the Burlington Northern tracks to North Ash Street; and
\item Sixth Street from Main Street to Broadway.
\item Elm Street from 3rd Street to 4th Street.
\end{enumerate}
\subsubsection{}
On the following downtown North-South streets from 1:00 a.m. to 7:00 a.m. on Mondays, Wednesdays and Fridays:
\begin{enumerate}[{\indent}a)]
\item Market Street from Fletcher Street to Second Street;
\item Main Street from the Red Lake River Bridge to Sixth Street;
\item Broadway from the Red Lake River Bridge to Sixth Street;
\item Ash Street from Loring Street to Third Street; and
\item Ash Street from Third Street to Fourth Street.
\end{enumerate}

\section{Snow Emergencies}
\index{PARKING REGULATIONS!PARKING PROHIBITIONS AND RESTRICTIONS!Snow Emergencies}
\subsection{Purpose}
The rapidly increasing use of the public streets in the City for the movement and parking of motor vehicles requires the adoption of special emergency snow and ice removal regulations and procedures in order that hazards and obstructions to the movement of fire, police, emergency and other vehicles will be minimized, the public safety, health and welfare protected, and public funds preserved.
\subsection{Definitions}
For the purpose of this Section, the following definitions shall apply unless the context clearly indicates or requires a different meaning:
\begin{description}
    \item[SNOW EMERGENCY] A condition of snowfall, snow accumulation or ice accumulation or anticipated snowfall, snow accumulation, or ice accumulation which creates or is likely to create hazardous street conditions endangering or impeding or likely to endanger or impede the movement of fire, police, emergency or other vehicular traffic, or otherwise endanger the safety, health or welfare of the public.
    \item[RESIDENTIAL STREETS] All City streets except those designated in City Code, Section 72.22 (E).
\end{description}
\subsection{Declaration of Emergency}
Whenever the Public Works Director, in the exercise of sound judgment and discretion, determines that a snow emergency exists in the City, the Public Works Director may declare a snow emergency by communicating such declaration to the local news media and through other available media/means.  Failure to effectively communicate shall not invalidate such declaration.
\subsection{Duration}
The snow emergency shall begin and end at times stated by the Public Works Director as part of the snow emergency declaration.  If at least two (2) hours before the expiration of the initial or any extension of a snow emergency, a declaration of extension is made by the Public Works Director by communicating the same in the manner(s) provided for the initial declaration, the snow emergency shall be extended for an additional period as stated in the declaration of extension.  Failure to effectively communicate shall not invalidate the extension.
\subsection{Parking Restrictions}
It is unlawful to park or leave standing any motor vehicle on the side of any residential street designated by the Public Works Director for prohibited parking as part of the snow emergency declaration or any extension(s) thereof.

\section{Truck Parking}
\index{PARKING REGULATIONS!PARKING PROHIBITIONS AND RESTRICTIONS!Truck Parking}
\subsection{}
It is unlawful to park a detached semi-trailer upon any street, city-owned parking lot, or other public property except streets as specifically designated by the Council by resolution and sign-posted.
\subsection{}
It is unlawful to park a truck (other than a truck of 12,000 pounds gross vehicle weight, or less), truck-trailer, tractor-trailer or truck-tractor within an area zoned as a residential district except for the purpose of loading or unloading the same.
\subsection{}
It is unlawful to park a commercial vehicle of more than 12,000 pounds gross vehicle weight upon any street in the business district except streets as specifically designated by the Council by resolution and sign-posted, but parking of the vehicle for a period of not more than 20 minutes shall be permitted in the space for the purpose of necessary access to abutting property while actively loading or unloading when the access cannot reasonably be secured from an alley or from an adjacent street where truck parking is not so restricted.
\subsection{}
It is unlawful to park a truck or other vehicle using or equipped with a trailer, or extended body or other extension or projection beyond the original length of the vehicle diagonally along any street except for a time sufficient to load or unload.
\subsection{}
Parking of commercial vehicles is permitted in duly designated and sign-posted loading zones, and in alleys, for a period of up to 20 minutes, provided that the alley parking does not prevent the flow of traffic therein, all of which shall be for the purpose of access to abutting or adjacent property while actively loading or unloading.\footnote{(‘83 Code, SEC. 9.10)  Penalty, see SEC. 70.99}

\section{Physically Handicapped Parking}
\index{PARKING REGULATIONS!PARKING PROHIBITIONS AND RESTRICTIONS!Physically Handicapped Parking}
\subsection{}
Statutory parking privileges for physically handicapped shall be strictly observed and enforced.  Police officers are authorized to tag vehicles on either private or public property in violation of the statutory privileges.
\subsection{}
It is unlawful for any person, whether or not physically handicapped, to stop, park, or leave standing, a motor vehicle in a sign-posted fire lane at any time, or in lanes where, and during the hours as, parking is prohibited to accommodate heavy traffic during morning and afternoon rush hours. \footnote{(Ord. 10, 2nd Series, effective 5-15-84)}
\subsection{}
Citizen volunteers may aid in the enforcement of statutory parking privileges for the physically handicapped by providing the Police Department with proof of the violations.  Upon receipt of the proof, the police may issue a notice of violation to the registered owner of the vehicle by mail. \footnote{(Ord. 104, 2nd Series, effective 5-13-95) (‘83 Code, SEC. 9.14) Penalty, see SEC. 70.99}

\section{Parking Rules in City Parking Lots}
\index{PARKING REGULATIONS!PARKING PROHIBITIONS AND RESTRICTIONS!Parking Rules in City Parking Lots}
In city-owned parking lots, the Council may limit the sizes and types of motor vehicles to be parked thereon, hours of parking, and prescribed method of parking, provided that the limitations and restrictions are marked or sign-posted thereon.  It is unlawful to park or leave standing any vehicle backed into a parking place, to drive in a direction opposite the flow of traffic marked by “one-way” signs or arrows, or to park any vehicle in any city-owned parking lot contrary to the restrictions or limitations marked or sign-posted therein.\footnote{(‘83 Code, SEC. 9.11)  Penalty, see SEC. 70.99}

\section{Unattended Vehicle}
\index{PARKING REGULATIONS!PARKING PROHIBITIONS AND RESTRICTIONS!Unattanded Vehicle}
\subsection{}
It is unlawful for any person to leave a motor vehicle unattended while the engine is running, unless all of the doors are locked.
\subsection{}
It is unlawful for any person to leave a motor vehicle unattended with the key in the ignition, unless all of the doors are locked.\footnote{(‘83 Code, SEC. 9.13)  Penalty, see SEC. 70.99}
