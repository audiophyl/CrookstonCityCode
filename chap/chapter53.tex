\chapter*{Chapter 53: \\
	Water Service}
    \addstarredchapter{Chapter 53: Water Service}
    \minitoc
    \textbf{\emph{{Cross-reference:}}}\\
    \emph{Penalty surcharge, see SEC. 50.98}\\
    \emph{Public Works Department established, see SEC. 32.080}
    \pagebreak

\section{Deficiency of Water and Shutting Off Water}
The city is not liable for any deficiency or failure in the supply of water to customers whether occasioned by shutting the water off for the purpose of making repairs or connections or by any other cause whatever.  In case of fire, or alarm of fire, water may be shut off to insure a supply for fire fighting.  In making repairs or construction of new works, water may be shut off at any time and kept off so long as may be necessary.\\
\emph{(‘83 Code, SEC. 3.30, Subd. 1)}
\section{Owner's Responsibility to Repair Leaks}
The City realizes that water service line repairs occurring from the curb stop to the water main pose a large hardship on property owners.  In order to relieve this burden the City Council has authorized a water service line insurance policy.  The policy components are as follows:  The City will contract to have water service line leaks repaired from the water main up to and including the curb stop or gate valve.  If there is a question as to which valve is applicable it will be determined by the Public Works Director. Service line repairs from beyond the curb stop or gate valve to the building will still be the property owner’s responsibility.  This policy only covers water leaks or inoperable curb stops.  It does not cover frozen service lines.  The service line repairs will be funded through an increase to the customer meter charge based on size of meter. The Public Works Director will review the funding annually to insure the program is properly funded.  This policy will cover all water accounts with the exception of mobile home parks.\\
\emph{(Res. 24504, 11/28/2006)}
\section{Abandoned Services}
All service installations connected to the water system that have been abandoned or, for any reason, have become useless for further service shall be disconnected at the main.  The owner of the premises, served by this service, shall pay the cost of the excavation.  The city shall perform the actual disconnection and all pipe and appurtenances removed from the street right-of-way shall become the property of the city.  When new buildings are erected on the site of old ones, and it is desired to increase the old water service, a new permit shall be taken out and the regular tapping charge and utility connection fee shall be made as if this were a new service.  It is unlawful for any person to cause or allow any service pipe to be hammered or squeezed together at the ends to stop the flow of water, or to save expense in improperly removing the pipe from the main.  Also, the improper disposition thereof shall be corrected by the city and the cost incurred shall be borne by the person causing or allowing the work to be performed.\\
\emph{(‘83 Code, SEC. 3.30, Subd. 3)  Penalty, see SEC. 50.99}
\section{Service Pipes}
Every service pipe must be laid in a manner as to prevent rupture by settlement.  The service pipe shall be placed not less than seven feet below the surface in all cases so arranged as to prevent rupture and stoppage by freezing.  Frozen service pipes between the main and the building shall be the responsibility of the owner.  Service pipes must extend from the corporate valve to the inside of the building; or if not taken into a building then to the hydrant or other fixtures which they are intended to supply.  A valve, the same size as the service pipe, shall be placed close to the inside wall of the building, ahead of the meter and well protected from freezing.  Service installations shall comply with requirements of the Minnesota State Plumbing Code.  All joints shall be left uncovered until inspected.  Minimum size connection with the water mains for all new permits shall be one inch in diameter.\\
\emph{(‘83 Code, SEC. 3.30, Subd. 4) (Ord. 101, 2nd Series, effective 5-13-95)}
\section{Private Water Supplies}
No water pipe of the city water system shall be connected with any pump, well, pipe, tank or any device that is connected with any other source of water supply and when such are found, the city shall notify the owner or occupant to disconnect the same and, if not immediately done, the city water shall be turned off before any new connections to the city system are permitted, the city shall ascertain that no cross-connections will exist when the new connection is made. When a building is connected to “city water” the private water supply may be used only for the purposes as the city may allow.\\
\emph{(‘83 Code, SEC. 3.30, Subd. 5)}
\section{Prohibited Uses or Restricted Hours}
Whenever the city shall determine that a shortage of water threatens the city, it may entirely prohibit water use or limit the times and hours during which water may be used from the city water system for lawn and garden sprinkling, irrigation, car washing, air conditioning, and other uses, or either or any of them.  It is unlawful for any water consumer to cause or permit water to be used in violation of the determination after public announcement thereof has been made through the news media specifically indicating the restrictions thereof.\\
\emph{(‘83 Code, SEC. 3.30, Subd. 6)}
\section{Private Fire Hose Connections}
Owners of structures with self-contained fire protection systems may apply for and obtain permission to connect the street mains with hydrants, large pipes, and hose couplings, for use in case of fire only, at their own installation expense and at the rates as the Council may adopt by resolution as herein provided.\\
\emph{(‘83 Code, SEC. 3.30, Subd. 7)}
\section{Opening Hydrants}
It is unlawful for any person, other than members of the Fire Department or other person duly authorized by the city, in pursuance of lawful purpose, to open any fire hydrant or attempt to draw water from the same or in any manner interfere therewith.  It is also unlawful for any person so authorized to deliver or suffer to be delivered to any other person any hydrant key or wrench, except for the purposes strictly pertaining to their lawful use.\\
\emph{(‘83 Code, SEC. 3.30, Subd. 8)  Penalty, see SEC. 50.99}
\section{Unmetered Service}
Unmetered service may be provided for construction, flooding skating rinks, and any other purpose.  The service shall be at a duly adopted rate.  Where it is difficult or impossible to accurately measure the amount of water taken, unmetered service may be provided and the unmetered rate applied; provided, however, that by acceptance thereof the consumer agrees to have the city estimate the water used.  In so estimating, the city shall consider the use to which the water is put and the length of time of unmetered service.\\
\emph{(‘83 Code, SEC. 3.30, Subd. 9) Penalty, see SEC. 50.99}
\section{Water Meters}
All water meters shall be purchased and maintained by the city.  All repairs of water meters not resulting from normal usage shall be the responsibility of the property owner, as shall any maintenance and repair of meters which are not of the remote reading type.  All water meters shall be installed and controlled by the city and the cost of installation shall be the responsibility of the property owner.  Any remote type meter in need of replacement by reason of normal usage shall be furnished to the owner, and installed at the expense of the city.  All meters shall be sealed and the seal broken only by a city employee acting within the course of his or her employment.\\
\emph{(‘83 Code, SEC. 3.30, Subd. 10)}
\section{Work to be in Accordance with Minnesota Plumbing Code}
All piping, connections and appurtenances shall be installed and performed strictly in accordance with the Minnesota Plumbing Code.  Failure to install or maintain the same in accordance therewith, or failure to have or permit required inspections shall, upon discovery by the city, be an additional ground for termination of water service to any consumer.\\
\emph{(‘83 Code, SEC. 3.30, Subd. 11)}
