\documentclass[code.tex]{subfiles}
\begin{document}
\chapter*{Chapter 70: \newline
	General Provisions}
\addcontentsline{toc}{chapter}{Chapter 70: General Provisions}

MiniTOC here
\pagebreak

\section{Definitions}
Except as otherwise defined in the city code, or where the context clearly indicates a contrary intent, the words and terms defined in M.S. Chapter 169, as it may be amended from time to time, shall be applicable to this title.\newline
\emph{(‘83 Code, SEC. 7.01)}
\section{Application}
Except as otherwise provided in this code, the provisions of this title are applicable to the drivers of all vehicles and animals upon streets, including, but not limited to, those owned or operated by the United States, the State of Minnesota, or any county, town, city, district, or other political subdivision.\newline
\emph{(‘83 Code, SEC. 7.02)}
\section{Minnesota Statutes Adopted by Reference}
Except as otherwise provided in this title, the regulatory and procedural provisions of M.S. Chapter 168, Chapter 169 (commonly referred to as the Highway Traffic Regulation Act) and Chapter 171, as may be amended from time to time, are hereby incorporated herein and adopted by reference, including the penalty provisions thereof.\newline
\emph{(‘83 Code, SEC. 8.01) (Ord.129, 2nd Series, effective 5-16-98)}
\section{Scope}
Except as otherwise provided in this code, the provisions of this title relate exclusively to the streets and alleys in the city, and the operation and parking of vehicles refer exclusively to the operation and parking of vehicles upon the streets and alleys.\newline
\emph{(‘83 Code, SEC. 7.03, Subd. 1) (Ord. 15, 2nd Series, effective 5-18-85)}
\section{Failing to Comply with Orders of Officer}
It is a misdemeanor for any person to willfully fail or refuse to comply with any lawful order or direction of any police or peace officer invested by law with authority to direct, control or regulate traffic.\newline
\emph{(‘83 Code, SEC. 7.03, Subd. 2)  Penalty, see SEC. 70.99}
\section{Placing Unauthorized Traffic-Control Signals}
No device, sign or signal shall be erected or maintained for traffic or parking control unless the Council shall first have approved and directed the same, except as otherwise provided in this section; provided, that when traffic and parking control is marked or sign-posted, the marking or sign-posting shall attest to Council action thereon.\newline
\emph{(‘83 Code, SEC. 7.04, Subd. 1)  Penalty, see SEC. 70.99}
\section{Tampering with Traffic-Control Devices}
It is a misdemeanor for any person to deface, mar, damage, move, remove, or in any way tamper with any structure, work, material, equipment, tools, sign, signal, barricade, fence, painting or appurtenance in any street unless the person has written permission from the city or is an agent, employee or contractor for the city, or other authority having jurisdiction over a particular street, and acting within the authority or scope of a contract with the city or the other authority.\newline
\emph{(‘83 Code, SEC. 7.04, Subd. 5)  Penalty, see SEC. 70.99}
\section{City May Authorize Temporary Restrictions}
The city, acting through the Chief of Police, may temporarily restrict traffic or parking for any private, public or experimental purpose.  It is the duty of the Chief of Police to so restrict traffic or parking when a hazardous condition arises or is observed.\newline
\emph{(‘83 Code, SEC. 7.04, Subd. 2)  Penalty, see SEC. 70.99}

\setcounter{section}{98}
\section{Penalty}
Every person violates a section, division, or provision of this title when he or she performs an act thereby prohibited or declared unlawful, or fails to act when the failure is thereby prohibited or declared unlawful, and upon conviction thereof, shall be punished as follows:
\subsection{}
Where the specific section, division, or provision specifically makes violation a misdemeanor, the person shall be punished as for a misdemeanor; where a violation is committed in a manner or under circumstances so as to endanger or be likely to endanger any person or property, the person shall be punished as for a misdemeanor; where the person stands convicted of violation of any provision of this title, exclusive of violations relating to the standing or parking of an unattended vehicle, within the immediate preceding 12-month period for the third or subsequent time, he or she shall be punished as for a misdemeanor.  The penalty which may be imposed for any crime which is a misdemeanor under this code, including Minnesota Statutes specifically adopted by reference, shall be a sentence of not more than 90 days or a fine of not more than \$1,000, or both.
\subsection{}
As to any violation not constituting a misdemeanor under the provisions of division (A) above,  the person shall be punished as for a petty misdemeanor.  The penalty which may be imposed for any petty offense which is a petty misdemeanor shall be a sentence of a fine of not more than \$300.
\subsection{}
As to any violation of a provision adopted by reference, he or she shall be punished as specified in the provision, so adopted.
\subsection{}
In the case of a misdemeanor or a petty misdemeanor, the costs of prosecution may be added.  A separate offense shall be deemed committed upon each day during which a violation occurs or continues.\newline
\emph{(‘83 Code, SEC. 7.99, SEC. 8.99, and SEC. 9.99)}

\end{document}
