\chapter*{Chapter 112: \\
	Amusements}
    \addstarredchapter{Chapter 112: Amusements}
    \vfill
    \minitoc
    \pagebreak

\subchapter{AMUSEMENT DEVICES}

\section{Definitions\footnote{(‘83 Code, SEC. 6.30, Subd. 1)}}
\index{AMUSEMENTS!AMUSEMENT DEVICES!Definitions}
For the purpose of this subchapter the following definitions shall apply, unless the context clearly indicates or requires a different meaning.
\begin{description}
    \item[AMUSEMENT DEVICE] Includes game of skill, coin amusement, and video game as herein defined.
    \item[ARCADE] A contiguous area in which more than six amusement devices are kept for use by the public generally.
    \item[COIN AMUSEMENT] Any machine which upon the insertion of a coin, token or slug, operates or may be operated and is available to the public generally for entertainment or amusement, which machine emits music, noise or displays motion pictures.
    \item[DISTRIBUTOR] The person who places amusement devices on premises not owned by him or under his or her control, which devices may be played by the public generally for a price paid either directly or indirectly.
    \item[GAME OF SKILL] Any device excepting pool and billiard tables, bowling alleys and shooting lanes, but including miniatures thereof, played by manipulating special equipment and propelling balls or other projectiles across a board or field into respective positions whereby a score is established, which is available to be played by the public generally at a price paid either directly or indirectly for the privilege.
    \item[VIDEO GAME] Any electrical device which displays objects on a screen and upon insertion of a coin, token or slug may be played by the public generally for entertainment or amusement.
\end{description}

\section{License Required}
\index{AMUSEMENTS!AMUSEMENT DEVICES!License Required}
It is unlawful for any person to have upon premises owned or controlled by him any amusement device, or operate an arcade, without a license therefor from the city. It is unlawful for any person to be a distributor without a license therefor from the city.\footnote{(‘83 Code,SEC.6.30,Subd.2) Penalty, see SEC.110.99}

\section{Unlawful Use and Devices\footnote{(‘83 Code, SEC. 6.30, Subd. 3)  Penalty, see SEC. 110.99}}
\index{AMUSEMENTS!AMUSEMENT DEVICES!Unlawful Use and Devices}
It is unlawful for any person to:
\begin{enumerate}[{\indent}A)]
    \item Sell or maintain a machine or device which is for gambling or contains an automatic pay-off device; 
    \item Give any prize, award, merchandise, gift, or thing of value to any person on account of operation of the device; 
    \item Sell or maintain, or permit to be operated in his or her place of business, any amusement device equipped with an automatic pay-off device; 
    \item Equip any amusement device with an automatic pay-off device; 
    \item Permit the playing of coin amusement machines between the hours of 1:00 a.m. and 6:00 a.m. of any day; or 
    \item If he or she is a licensee, agent or employee of a licensee, knowingly permit a minor to be present on the premises in violation of the curfew laws or permit a person under the age of 17 to be present on the premises when school is in session unless on a valid excused absence. 
\end{enumerate}

\subchapter{DANCES}

\setcounter{section}{14}
\section{Definitions\footnote{(‘83 Code, SEC. 6.31, Subd. 1)}}
\index{AMUSEMENTS!DANCES!Definitions}
For the purpose of this subchapter the following definitions shall apply, unless the context clearly indicates or requires a different meaning.
\begin{description}
    \item[PUBLIC DANCE] Any dance wherein the public may participate by payment, directly or indirectly, of an admission fee or price for dancing, which fee may be in the form of a club membership, or payment of money, directly or indirectly.
    \item[PUBLIC DANCING PLACE] Any room, place, or space open to public patronage in which dancing, wherein the public may participate, is carried on and to which admission may be had by the public by payment, directly or indirectly, of an admission fee or price for dancing.
\end{description}

\section{License Required}
\index{AMUSEMENTS!DANCES!License Required}
It is unlawful for any person to operate a public dancing place, or hold a public dance, without a license therefor from the city.\footnote{(‘83 Code, SEC. 6.31, Subd. 2)  Penalty, see SEC. 110.99}

\section{Application; Conditions of License}
\index{AMUSEMENTS!DANCES!Application; Conditions of License}
\subsection{}
Notwithstanding any provisions to the contrary contained in this Code, the Clerk shall act upon all applications for dance licenses when no more than 400 people are expected to attend the dance and when alcohol is not expected to be served.
\subsection{}
A verified application for a dance license shall be filed with the Clerk on a form approved by the Council and shall specify, at a minimum, the names and addresses of the person, persons, committee or organization that is to hold the dance, time and place thereof, the area of the dance floor, the number of persons expected to attend, if alcohol will be served and whether the applicant has been convicted of a felony, gross misdemeanor, or violation of any public dance laws within the past five years.  No license shall be issued to any person who has been so convicted.
\subsection{}
All applications shall be referred to the Chief of Police to verify that the applicant has not been convicted of a felony, gross misdemeanor, or violation of any public dance laws within the past five years.  No license shall be issued to any person who has been so convicted.
\subsection{}
Applications that are not acted upon by the Clerk shall be referred by the Council to the Chief of Police for investigation and report prior to being acted upon by the Council.
\subsection{}
The Council shall act upon all dance license applications not acted upon by the Clerk at a regular or special meeting thereof, whether or not it is included in the call or agenda of the meeting.
\subsection{}
At least three officers of the law shall be designated by the Chief of Police and employed by the city to be present at every public dance, the license for which is acted upon by the Clerk, during the entire time the dance is being held.  For purposes of this division (F), the term OFFICER OF THE LAW means any person who is a full-time peace officer, part-time peace officer, reserve officer, or person deputized by the Chief of Police.  In the discretion of the Council, more than three officers of the law may be required.
\subsection{}
The dance license shall be posted in the public dancing place and shall state the name of the licensee, the amount paid therefor, and the time and place licensed.  The license shall also state that the licensee is responsible for the manner of conducting the dance.
\subsection{}
No license shall be issued to any applicant under the age of 18 years.
\subsection{}
Before a license is issued under this section to more than one individual or to a corporation, partnership or association, the applicant or applicants shall appoint in writing a natural person as its agent.  The agent, by the terms of the agent’s written consent, shall take full responsibility for the conduct of the public dancing place.  The agent must be a person who, by reason of age, character, reputation and other attributes, could qualify individually as a licensee.\footnote{(Ord. 67, 2nd Series, effective 7-21-90).}
\subsection{}
Notwithstanding any provisions of law to the contrary, the Council may, pursuant to general Council policy established by resolution or upon a special finding of the necessity therefor, place the regulations, conditions and restrictions, in addition to those stated in this chapter, upon any license as it, in its discretion, may deem reasonable and justified.  All the regulations, conditions and restrictions shall be stated on the license either verbatim or by reference to the Council resolution.\footnote{(Ord. 78, 2nd Series, effective 5-27-92) (’83 Code, SEC. 6.31, Subd. 3)  Penalty, see SEC. 110.99}\\

\subchapter{SHOWS}

\setcounter{section}{29}
\section{License Required}
\index{AMUSEMENTS!SHOWS!License Required}
It is unlawful for any person to present any public show, movie, caravan, circus, carnival, theatrical or other performance or exhibition without first having obtained a license therefor from the city.\footnote{(‘83 Code, SEC. 6.32, Subd. 1)  Penalty, see SEC. 110.99}

\section{Exceptions\footnote{(‘83 Code, SEC. 6.32, Subd. 2)}}
\index{AMUSEMENTS!SHOWS!Exceptions}
No license shall be required in the following instances:
\begin{enumerate}[{\indent}A)]
    \item Performances presented in the local schools and colleges, under the sponsorship of the schools and colleges, and primarily for the students thereof only.
    \item Performances of athletic, musical or theatrical events sponsored by local schools or colleges using student talent only.
    \item Any performance or event in, or sponsored by, bona fide local church and non-profit organizations, provided that the organization shall be incorporated.
\end{enumerate}

\subchapter{BILLIARDS, POOL, BOWLING, AND THE LIKE}

\setcounter{section}{44}
\section{License Required}
\index{AMUSEMENTS!BILLIARDS, POOL, BOWLING, AND THE LIKE!License Required}
It is unlawful for any person to keep or maintain any pool, foosball, billiard, snooker or other game table, or any bowling alley (bowling lane) available for public use without first having obtained a license from the city.\footnote{(‘83 Code, SEC. 6.33, Subd. 1)  Penalty, see SEC. 110.99}

\section{Practices Prohibited\footnote{(‘83 Code, SEC. 6.33, Subd. 2)  Penalty, see SEC. 110.99}}
\index{AMUSEMENTS!BILLIARDS, POOL, BOWLING, AND THE LIKE!Practices Prohibited}
It is unlawful for any:
\begin{enumerate}[{\indent}A)]
    \item Pool, foosball, billiard, snooker or other game table licensee to be open between 1:00 a.m. and 8:00 a.m. of any weekday, or between 1:00 a.m. and 12:00 noon on any Sunday, and permit use of the licensed facilities.
    \item Person under the age of 19 years to play pool, foosball, billiards, snooker or other game table where beer or liquor is sold or consumed, unless accompanied by his or her parent or guardian.
    \item Licensee to cause or permit any person under the age of 19 years to play pool, foosball, billiards, snooker or other similar game table where beer or liquor is sold or consumed unless the minor is accompanied by his or her parent or guardian.
    \item Licensee to permit any form of gambling thereon.
    \item Licensee to permit any person to become disorderly or to use profane, obscene or indecent language.
    \item Licensee, not having an on-sale liquor license, to sell or possess, or knowingly allow any person on the licensed premises to sell or possess, intoxicating liquor.
\end{enumerate}

\subchapter{ROLLER SKATING RINKS}

\setcounter{section}{59}
\section{License Required}
\index{AMUSEMENTS!ROLLER SKATING RINKS!License Required}
It is unlawful for any person to operate a roller skating rink without a license therefor from the city.\footnote{(‘83 Code, SEC. 6.71, Subd. 1)  Penalty, see SEC. 110.99}

\section{Regulations}
\index{AMUSEMENTS!ROLLER SKATING RINKS!Regulations}
\subsection{}
It is the responsibility of every licensee to maintain good order on licensed premises.
\subsection{}
No license shall issue until the applicant has filed with the Clerk-Treasurer a policy or certificate of public liability insurance coverage concurrent with the license term with limits of at least \$10,000 for injury to one person, and \$20,000 for each occurrence.\footnote{(‘83 Code, SEC. 6.71, Subd. 2)  Penalty, see SEC. 110.99}\\

\subchapter{MINIATURE GOLF COURSES}

\setcounter{section}{74}
\section{License Required}
\index{AMUSEMENTS!MINIATURE GOLF COURSES!License Required}
It is unlawful for any person to operate a miniature golf course without a license therefor from the city.\footnote{(‘83 Code, SEC. 6.72, Subd. 1)  Penalty, see SEC. 110.99}

\section{Insurance}
\index{AMUSEMENTS!MINIATURE GOLF COURSES!Insurance}
No license shall issue until the applicant has filed with the Clerk-Treasurer a policy or certificate of public liability insurance for coverage concurrent with the license term with limits of at least \$10,000 for injury to one person, and \$20,000 for each occurrence.\footnote{(‘83 Code, SEC. 6.72, Subd. 2)}
