\chapter*{Chapter 91: \\
	Animals}
    \addstarredchapter{Chapter 91: Animals}
    \vfill
    \minitoc
    \emph{\textbf{Cross-reference} Animals in parks, see SEC. 95.29}
    \pagebreak

\subchapter{ANIMALS AND FOWL}

\section{Definitions\footnote{(M.S. Chapter 343.20, Subd. 7)  (‘83 Code, SEC. 10.21, Subd. 1)}}
\index{ANIMALS!ANIMALS AND FOWL!Definitions}
For the purpose of this subchapter the following definitions shall apply, unless the context clearly indicates or requires a different meaning.
\begin{description}
    \item[ANIMALS] Any mammal, reptile, amphibian, fish, bird (including all fowl and poultry) or other member commonly accepted as a part of the animal kingdom.  \textbf{ANIMALS} shall be classified as follows:
    \begin{enumerate}[{\indent}1)]
        \item \textbf{DOMESTIC ANIMALS} shall mean those animals commonly accepted as domesticated household pets.  Unless otherwise defined, the animals shall include dogs, cats, caged birds, gerbils, hamsters, guinea pigs domesticated rabbits, fish, non-poisonous, non-venomous and non-constricting reptiles or amphibians, and other similar animals.
        \item \textbf{FARM ANIMALS} shall mean those animals commonly associated with a farm or performing work in an agricultural setting.  Unless otherwise defined, the animals shall include members of the equestrian family (horses and mules), bovine family (cows, bulls and oxen), sheep, poultry (chickens and turkeys), fowl (ducks and geese), swine, goats, bees, and other animals associated with a farm, ranch, or stable.
        \item \textbf{NON-DOMESTIC ANIMALS} shall mean those animals commonly considered to be naturally wild and not naturally trained or domesticated, or which are commonly considered to inherently dangerous to the health, safety and welfare of people.  Unless otherwise defined, the animals shall include:
        \begin{enumerate}[{\indent\indent}a)]
            \item Any member of the large cat family (family felidae) including lions, tigers, cougars, bobcats, leopards and jaguars, but excluding commonly accepted and domesticated house cats.
            \item Any naturally wild member of the canine family (family canidae) including wolves, foxes, coyotes, dingoes, and jackals, but excluding commonly accepted domesticated dogs.
            \item Any crossbreeds such as the crossbreed between a wolf and a dog, unless the crossbreed is commonly accepted as a domesticated house pet.
            \item Any relative of the rodent family, including any skunk (whether or not descended), raccoon, or squirrel, but excluding those members otherwise defined or commonly accepted as domesticated pets.
            \item Any poisonous, venomous, constricting, or inherently dangerous member of the reptile or amphibian families including rattlesnakes, boa constrictors, pit vipers, crocodiles and alligators.
            \item Any other animal which is not explicitly listed above but which can be reasonably defined by the terms of this subpart, including but not limited to bears and deer.
        \end{enumerate}
        \item \textbf{SERVICE ANIMAL} shall mean any animal trained to assist a person with a disability.
    \end{enumerate}
\end{description}

\section{Keeping or Harboring Animals\footnote{(‘83 Code, SEC. 10.21, Subd. 2)  Penalty, see SEC. 10.99}}
\index{ANIMALS!ANIMALS AND FOWL!Keeping or Harboring Animals}
It is unlawful for any person to keep or harbor any animal, other than a domestic animal or a service animal as defined in this chapter, not in transit, except:
\begin{enumerate}[{\indent}A)]
    \item Farm animals kept in that portion of the city zoned for agricultural purposes or animals that have been kept continuously by right of prior land use;
    \item Animals kept as part of a show licensed under the city code;
    \item Animals used in a parade for which a permit has been issued;
    \item Animals kept in a laboratory for scientific or experimental purposes; or
    \item Animals kept in an animal hospital or clinic for treatment by a licensed veterinarian.
\end{enumerate}

\section{Transporting Animals\footnote{(‘83 Code, SEC. 10.21, Subd. 3)  Penalty, see SEC. 10.99}}
\index{ANIMALS!ANIMALS AND FOWL!Transporting Animals}
It is unlawful for any person to transport animals unless they are:
\begin{enumerate}[{\indent}A)]
    \item Confined within a vehicle, cage or other means of conveyance;
    \item Farm animals being transported in a portion of the city zoned for agricultural purposes; or
    \item Restrained by means of bridles, halters, ropes or other means of individual restraint, or in the case of companion animals or service animals under the immediate control of the animal owner or designee.
\end{enumerate}

\section{Cruel Treatment; Inadequate Housing; Doghouses}
\index{ANIMALS!ANIMALS AND FOWL!Cruel Treatment; Inadequate Housing; Doghouses}
\subsection{}
It is unlawful for any person to treat any animal as herein defined, or any other animal, in a cruel or inhumane manner.\footnote{(‘83 Code, SEC. 10.21, Subd. 4)}
\subsection{}
It is unlawful for any person to keep any animal as herein defined, or any other animal, in any structure infested by rodents, vermin, flies or insects, or inadequate for protection against the elements.
\subsection{}
The provisions of M.S. Chapter 343, Cruelty to Animals, as it may be amended from time to time, are adopted relating to doghouses and incorporated herein.\footnote{(‘83 Code, SEC. 10.21, Subd. 5)  Penalty, see SEC. 10.99}

\section{Trespass}
\index{ANIMALS!ANIMALS AND FOWL!Trespass}
It is unlawful for any person to herd, drive or ride any animal over and upon any grass, turf, boulevard, city park, cemetery, garden or lot without specific permission therefor from the owner.\footnote{(‘83 Code, SEC. 10.21, Subd. 6)  Penalty, see SEC. 10.99}

\section{Dangerous Animals}
\index{ANIMALS!ANIMALS AND FOWL!Dangerous Animals}
\subsection{Attack by an Animal}
It shall be unlawful for any person’s animal to inflict or attempt to inflict bodily injury on any person or other animal. The provisions of SEC. 91.06 shall not apply to a dog under the control of an on-duty law enforcement officer or to an attack upon an uninvited intruder.
\subsection{Destruction of Dangerous Animal}
The Chief of Police or his designee shall have the authority to order the destruction of a dangerous animal in accordance with the terms established by SEC. 91.06.
\subsection{Definitions}
For the purpose of SEC. 91.06 (dangerous animals) through 91.16 (regulation of licensing of dogs and cats), the following definitions shall apply.
\begin{description}
    \item[DANGEROUS ANIMAL] An animal which has:
    \begin{enumerate}[{\indent\indent}a)]
        \item Without provocation, inflicted substantial bodily harm on a human being on public or private property;
        \item Killed a domestic animal without provocation while off the owner’s property;
        \item Bitten one or more persons or animals on two or more occasions; or
        \item Been found to be potentially dangerous and/or the owner has notice of the same, the animal aggressively bites, attacks, or endangers the safety of humans or animals.
    \end{enumerate}
    \item[POTENTIALLY DANGEROUS ANIMAL]  An animal which has:
    \begin{enumerate}[{\indent\indent}a)]
        \item When unprovoked, inflicts bites on a human or domestic animal on public or private property;
        \item When unprovoked, chases or approaches a person upon public or private property; or has engaged in unprovoked conduct threatening the safety of humans or domestic animals.
    \end{enumerate}
    \item[PROPER ENCLOSURE]  Securely confined indoors or in a securely locked pen or suitable structure to prevent the animal from escaping and to provide protection for the animal from the elements.  A proper enclosure does not include a porch, patio, or any part of a house, garage, or other structure that would allow the animal to exit of its own volition, or any house or structure in which windows are open or in which doors or window screens are the only barriers which prevent the animal from exiting.  The enclosure shall not allow the egress of the animal in any manner without human assistance.  A pen or suitable structure shall meet the following minimum specifications:
    \begin{enumerate}[{\indent\indent}a)]
        \item Have a minimum overall floor size of 32 square feet.
        \item Sidewalls shall have a minimum height of five feet and be constructed of 11-gauge or heavier wire.  Openings in the wire shall not exceed two inches, support posts shall be 1 ¼ inch or larger steel pipe buried in the ground 18 inches or more.  When a concrete floor is not provided, the sidewalls shall be buried a minimum of 18 inches in the ground.
        \item A cover over the entire pen or suitable structure shall be provided.  The cover shall be constructed of the same gauge wire or heavier as the sidewalls and shall also have no openings in the wire greater than two inches.
        \item An entrance/exit gate shall be provided and be constructed of the same material as the sidewalls and shall also have no openings in the wire greater than two inches.  The gate shall be equipped with a device capable of being locked and shall be locked at all times when the animal is in the pen or suitable structure.
    \end{enumerate}
    \item[UNPROVOKED] The condition in which the animal is not purposely excited, stimulated, agitated or disturbed.
    \item[OWNER] “Owner” means any person, firm, corporation, organization, or department possessing, harboring, keeping, feeding, boarding, having an interest in, or having care, custody, regulation or control of an animal.
    \item[ANIMAL POUND] The location designated by the city administrative authority for the purpose of impounding and caring for animals held under the authority of this section.\footnote{(Ord. 41, 2nd Series, effective 4-23-87).}
    \item[MUZZLE or MUZZLED] The muzzle must be made in a manner which will not cause injury to the animal or interfere with its vision or respiration, but must prevent it from biting any person or animal.
\end{description}
\subsection{Designation as Potentially Dangerous Animal}
The Chief of Police or designee shall designate any animal as a potentially dangerous animal upon receiving evidence that the potentially dangerous animal has, engaged in conduct set forth in SEC. 91.06 (C)(2).  When an animal is declared potentially dangerous, the Chief of Police or designee shall cause the owner of the potentially dangerous animal to be notified in writing that the animal is potentially dangerous.
\subsection{Evidence Justifying Designation}
The Chief of Police or designee shall have the authority to designate any animal as a dangerous animal upon receiving evidence of the following:
\begin{enumerate}[{\indent}1)]
    \item When the animal has exhibited any of the behavior outlined in SEC. 91.06 (C)(1).
\end{enumerate}
\subsection{Authority to Order Destruction}
The Chief of Police or designee, upon finding that an animal is dangerous, is authorized to order, as part of disposition, that the animal be destroyed based upon a finding that an animal is dangerous as defined by SEC. 91.06 (C)(1).
\subsection{Procedure}
The Chief of Police or designee, after having determined that an animal is dangerous, may proceed in the following manner: The Chief of Police or designee shall cause one owner of the animal to be notified in writing or in person that the animal is dangerous and may order the animal seized or make orders as deemed proper. This owner shall be notified as to the basis for the designation of the animal as dangerous in accordance with SEC. 91.06 (C)(1), and shall be given ten (10) days to appeal this order by requesting a hearing before a hearing officer.
\subsubsection{}
If an owner requests in writing, directed to the Chief of Police or designee, and posts a bond of \$200.00, a hearing for determination as to the dangerous nature of the animal shall be conducted. The hearing shall be held before a hearing officer who shall set a date for a hearing not more than ten (10) days after demand for the hearing. The records of the Police Department shall be admissible for consideration by the hearing officer without further foundation. After considering all evidence pertaining to the temperament of the animal, the hearing officer shall make an order as he or she deems proper. The hearing officer may order that the Chief of Police or designee take the animal into custody for destruction if the animal is not currently in custody. If the animal is ordered into custody for destruction, the owner shall immediately make the animal available to the Chief of Police or designee.
\subsubsection{}
No person shall harbor an animal after it has been found by a hearing officer to be dangerous.
\subsection{Stopping an Attack}
If any police officer or Animal Control Officer is witness to an attack by an animal upon a person or another animal, the officer may take whatever means the officer deems appropriate to bring the attack to an end and prevent further injury to the victim.
\subsection{Notification of New Address}
The owner of an animal which has been identified as dangerous or potentially dangerous shall notify the Chief of Police or designee in writing if the animal is to be relocated from its current address or given, transferred, or sold to another person.  The notification shall be given in writing at least fourteen (14) days prior to the relocation or transfer of ownership.  The notification shall include the current owner’s name and address, the relocation address, the transfer address and the name of the new owner or custodian, if any.

\section{Dangerous Animal Requirements}
\index{ANIMALS!ANIMALS AND FOWL!Dangerous Animal Requirements}
\subsection{Requirements}
If the Chief of Police or designee does not order the destruction of an animal that has been declared dangerous, the owner shall, within fourteen (14) days after mailing of the notice that the animal has been declared dangerous, complete all of the following:
\begin{enumerate}[{\indent}1)]
    \item The owner shall provide and maintain a proper enclosure for the dangerous animal as specified in SEC. 91.06 (C)(3);
    \item Post the front and the rear of the premises with clearly visible warning signs, including a warning symbol to inform children, that there is a dangerous animal on the property, as specified in M.S. \textsection 347.51, as may be amended from time to time;
    \item Provide as your proof annually of public liability insurance in the minimum amount of \$100,000.00;
    \item If the animal is outside the proper enclosure, the animal must be muzzled and restrained by a substantial chain or leash (not to exceed six feet in length) and under the physical restraint of a person 16 years of age or older.  The muzzle must be made in a manner as defined by this section.
    \item If the animal is a dog, it must have an easily identifiable, standardized tag identifying the dog as dangerous affixed to its collar at all times as specified in M.S. \textsection 347.51, as it may be amended from time to time, and shall have a microchip implanted as provided by M.S. \textsection 347.151, as it may be amended from time to time.
    \item All animals deemed dangerous by the Chief of Police or designee or hearing officer shall be registered with the city within fourteen (14) days after the date the animal was so deemed and provide satisfactory proof thereof to the Chief of Police or designee.
    \item If the animal is a dog, the dog must be licensed and current on rabies vaccination.  If the animal is a cat or ferret, it must be current on rabies vaccination.
\end{enumerate}

\section{Nuisances}
\index{ANIMALS!ANIMALS AND FOWL!Nuisances}
\subsection{Definitions}
\subsubsection{}
For purposes of this Section, an animal shall be declared a public nuisance when the animal is the subject of three or more, in any combination, of the following ordinances of the Crookston City Code:
\begin{enumerate}[{\indent}a)]
    \item A violation of Section 91.18.
    \item A violation of Section 91.19 (A).
    \item A violation of Section 91.19 (D)(6).
    \item A designation of the animal as a potentially dangerous animal or dangerous animal as defined by Section 91.06, Subd. (C)(1) or (C)(2).
\end{enumerate}
\subsubsection{}
For purposes of Section 91.08 (A), a violation is defined as:
\begin{enumerate}[{\indent}a)]
    \item A plea of, or adjudication of guilt to a citation; or,
    \item The payment of a City Administrative fine.
\end{enumerate}
\subsection{Procedure}
\subsubsection{}
The Chief of Police or designee, after having determined that an animal is a nuisance as defined by this Section, may proceed in the following manner: the Chief of Police or designee shall serve upon the owner or keeper of the animal notice of the determination by registered mail or by personal service. The Chief of Police or designee may further order the animal seized or make orders as deemed proper. This owner shall be notified as to the basis for the designation of the animal as a nuisance in accordance with Section 91.08 and shall be given ten (10) days to appeal this order by requesting a hearing before a hearing officer.
\subsubsection{}
An owner or keeper of an animal determined to be a nuisance can challenge the determination by requesting a hearing before a hearing officer. A hearing must be requested by submitting a written demand for hearing to the Chief of Police or designee and cash and/or a bond in the amount of \$200.00. The \$200.00 cash or bond will be held by the Chief of Police or designee to secure payment of the costs which the owner or keeper of the animal is responsible for, including the costs of impoundment and destruction of the animal. The hearing shall be held on a date determined by the hearing officer but in no event shall the hearing be held more than ten (10) days after receipt of the written demand for hearing and bond is received by the Chief of Police or designee. The record of the Police Department shall be admissible for consideration by the hearing officer without further foundation. After considering all evidence, the hearing officer shall make an order as he or she deems proper. The hearing officer may order that the Chief of Police or designee take the animal into custody if the animal is not currently in custody. if the animal is ordered to be disposed of as defined by this Section, and the owner chooses destruction, the owner shall immediately make the animal available to the Chief of Police or designee. if the animal is ordered to be disposed of and the owner chooses to have the animal permanently removed from the City, the owner shall do so within five (5) days of the date of the hearing officer's order, and the owner shall provide to the Chief or Police or designee a written and notarized statement regarding steps taken to permanently remove the animal from the City.
\subsection{Abatement}
An animal found to be a nuisance under this Section shall be abated by the owner or keeper of such animal by the disposition of the animal within ten (10) days after receipt of notice to the owner or keeper thereof. "Disposition” shall mean the destruction of the animal or its permanent removal from the City. Once an owner or keeper has received notice that the animal has been found to be a nuisance, if the owner or keeper of the animal fails to dispose of the animal as provided above, the City’s designated animal control authority is authorized and directed to capture and immediately dispose of the animal. The owner or keeper of the animal shall immediately make the animal available to the City’s designated animal control authority.
\subsection{Keeping or Harboring Animals}
It shall be unlawful for any person to harbor an animal within the City after it has been determined to be a nuisance and the applicable time periods for appeal and disposition have expired.\\

\subchapter{\mbox{REGULATION AND LICENSING} \mbox{OF DOGS AND CATS}}

\setcounter{section}{14}
\section{Definitions\footnote{(‘83 Code, SEC. 10.20, Subd. 1)}}
\index{ANIMALS!REGULATION AND LICENSING OF DOGS AND CATS!Definitions}
For the purpose of this subchapter the following definitions shall apply, unless the context clearly indicates or requires a different meaning.
\begin{description}
    \item[ANIMAL POUND] The location designated by the city administrative authority for the purpose of impounding and caring for animals held under the authority of this section.\footnote{(Ord. 41, 2nd Series, effective 4-23-87)}
    \item[MUZZLE or MUZZLED] A device or the securing of a dog’s jaws by a device constructed of strong, soft material or a metal muzzle such as that used commercially with greyhounds; the muzzle must be made in a manner which will not cause injury to the dog or interfere with its vision or respiration, but must prevent it from biting any person or animal.\footnote{(Ord. 43, 2nd Series, effective 6-11-87)}
    \item[OWNER] A person who owns, harbors, feeds, boards or keeps an animal hereby regulated.
    \item[UNCONFINED] Not securely confined indoors or confined in a securely enclosed and locked pen or structure upon the premises of the owner; the pen or structure must have secure sides and a secure top and if the pen or structure has no bottom secured to the sides, the sides must be embedded into the ground no less than one foot.
\end{description}

\section{License Required; Application, Fee and Tag}
\index{ANIMALS!REGULATION AND LICENSING OF DOGS AND CATS!License Required; Application, Fee and Tag}
\subsection{License Required}
It is unlawful for the owner of any dog or cat, six months of age or more, to fail to obtain a license therefor from the city.
\subsection{Application}
Application for a dog or cat license shall be upon a form supplied by the city and accompanied by a certificate of a veterinarian, duly licensed to practice veterinary medicine within the State of Minnesota, which certificate shall state that the dog or cat for which application for a license is made, has been inoculated against rabies for at least the period for which the license is applied.
\subsection{Period and Fee}
All dog and cat licenses shall expire on December 31 of each year.  License fees shall be established by the Council by resolution, which may in the same manner be amended from time to time, and uniformly enforced.
\subsection{Tag Required}
All licensed dogs and cats shall wear a collar and have a tag firmly affixed thereto evidencing the license for the current year.  A duplicate for a lost tag may be issued by the city upon presentation of the receipt showing the payment of the license fee for the current year.  Tags shall not be transferable, and no refund shall be made on any license fee because of leaving the city or death of the dog or cat before the expiration of the license.\footnote{(‘83 Code, SEC. 10.20, Subds. 3-6)  Penalty, see SEC. 10.99}

\section{Maximum Number of Animals Permitted}
\index{ANIMALS!REGULATION AND LICENSING OF DOGS AND CATS!Maximum Number of Animals Permitted}
The number of licensed dogs and cats permitted per dwelling unit shall not exceed three animals exceeding three months of age, which number shall consist of no more than two dogs and one cat or two cats and one dog. Any existing dwelling unit which becomes nonconforming on the effective date of this subchapter shall not have the number of permitted dogs and cats enlarged, but may continue with the existing dogs and cats until the expiration of the lives of those excess dogs and cats.\footnote{(‘83 Code, SEC. 10.20, Subd. 18)  Penalty, see SEC. 10.99}

\section{Running at Large Prohibited}
\index{ANIMALS!REGULATION AND LICENSING OF DOGS AND CATS!Running at Large Prohibited}
It shall be unlawful for the dog or cat of any person who owns, harbors, or keeps a dog or cat, to run at large.  A person, who owns, harbors, or keeps a dog or cat which runs at large shall be guilty of a misdemeanor.  Dogs or cats on a leash and accompanied by a responsible person or accompanied by and under the control and direction of a responsible person, so as to be effectively restrained by command as by leash, shall be permitted in streets or on public land unless the city has posted an area with signs reading “Dogs or Cats Prohibited.”\footnote{Penalty, see SEC. 10.99}

\section{Unlawful Acts}
\index{ANIMALS!REGULATION AND LICENSING OF DOGS AND CATS!Unlawful Acts}
\subsection{}
It is unlawful for any owner to suffer or permit any dog or cat to defecate upon public property, or the private property of another, without immediately removing the excrement and disposing of it in a sanitary manner.
\subsection{}
It is unlawful for any owner to suffer or permit any dog or cat to be upon public property or the private property of another, unless the dog or cat is in the custody of a person of suitable age and discretion having in his or her possession equipment and supplies for excrement removal.
\subsection{}
The provisions of this section do not apply to a guide dog accompanying a blind person, a service dog accompanying a disabled person or a dog while engaged in police or rescue activity.\footnote{(‘83 Code, SEC. 10.20, Subd. 17)}
\subsection{}
It is unlawful for the owner of any dog or cat to\footnote{(‘83 Code, SEC. 10.20, Subd. 19)  (Ord. 115, 2nd Series, effective 1-18-97)  Penalty, see SEC. 10.99}:
\begin{enumerate}[{\indent}1)]
    \item Fail to have the animal currently immunized for rabies;
    \item Fail to have the license tag issued by the city firmly attached to a collar worn at all times by the licensed animal;
    \item Interfere with any police officer, or other city employee, in the performance of his or her duty to enforce this section;
    \item Abandon any animal in the city;
    \item Fail to provide the animal with sufficient good and wholesome food and water, proper shelter and protection from the elements, veterinary care when needed to prevent suffering, and humane care and treatment; or 
    \item Fail to keep his or her dog from barking, howling or whining.
\end{enumerate}

\section{Animal Bites; Quarantine}
\index{ANIMALS!REGULATION AND LICENSING OF DOGS AND CATS!Animal Bites; Quarantine}
\subsection{Reporting}
\subsubsection{Animal Owner}
When any person owning a dog or cat has been notified by any person injured or by someone in his or her behalf, or by someone with knowledge of the injury, that the person has been bitten or attacked by the animal, or when any person owning a dog or cat has been notified by any person that the animal has been bitten by a rabid animal, the owner shall immediately report his or her name and address, the nature of any attack, so far as is known, and the circumstances of the animal to the Police Department.
\subsubsection{All Persons}
Anyone having knowledge or reason to believe that any dog or cat in the city has bitten a person shall immediately report, so far as is known, the name and address of the owner and circumstances of the animal.
\subsection{Control/Quarantine}
\subsubsection{}
Whenever any dog or cat has bitten a person, it shall be confined in the animal pound for a period of ten days under the care and observation of a licensed veterinarian with the expense thereof borne by the owner of the dog or cat.  This confinement may be waived by the Police Department under the following conditions:
\begin{enumerate}[{\indent}a)]
    \item The attack was not excessively violent in nature and that the injuries inflicted do not meet the definition of substantial bodily harm as set out in M.S. Chapter 609.02, Subd. 7a, as it may be amended from time to time;
    \item The animal owner can provide proof that the animal is current on all applicable immunizations;
    \item The Police Department can reasonably determine that the animal will be confined by the owner for the required 10-day period, and that the animal will be accessible to the Police Department during that period;
    \item The animal owner agrees to have the animal examined, at the owner’s expense, by a licensed veterinarian at the conclusion of the ten-day confinement; and
    \item The victim does not request impoundment.
\end{enumerate}
\subsubsection
It shall be lawful for the Chief of Police, or an agent of the Chief of Police, to destroy in a humane manner any dog or cat that has been determined to have rabies.  If at the end of the ten-day period, a licensed veterinarian is convinced that the dog or cat is then free from rabies, the dog or cat may be released from the pound, provided the provisions of SEC. 91.21 are met.\footnote{(‘83 Code, SEC. 10.20, Subd. 7)  (Ord. 41, 2nd Series, effective 4-23-87)}

\section{Impoundment; Notice and Release Procedures}
\index{ANIMALS!REGULATION AND LICENSING OF DOGS AND CATS!Impoundment; Notice and Release Procedures}
\subsection{Impoundment}
Any dangerous dog, as defined in SEC. 91.06, any dog or cat found in the city without a tag, any dog or cat running at large, or any dog or cat found to be or to be kept in violation of this chapter, may be captured and impounded.\footnote{(‘83 Code, SEC. 10.20, Subd. 8)  (Ord. 115, 2nd Series, effective 1-18-97)}
\subsection{Notice of Impounding}
Upon the impounding of any dog or cat, except under M.S. \textsection 343.22 or \textsection 343.29 (Investigation of Cruelty Complaints) or M.S. \textsection 343.29 (Exposure of Animals), as these sections may be amended from time to time, reasonable effort will be made to identify and notify the owner of the animal as to the time and place of the taking and the reason for impoundment.\footnote{(‘83 Code, SEC. 10.20, Subd. 9)}
\subsection{Animal Pound Records}
An accurate record of the time of the placement of the dog or cat in the animal pound shall be kept on each animal. Every dog or cat so placed in the animal pound shall be held for redemption by the owner for a period of not less than five regular business days. In the case of a dog or cat quarantined because it has bitten a person, the five-day period shall begin after the termination of the quarantine. A \textbf{REGULAR BUSINESS DAY} is one during which the pound is open for business to the public for at least four hours between 8:00 a.m. and 7:00 p.m. Impoundment records shall be preserved for a minimum of six months and shall show the description of the animal by specie, breed, sex, approximate age, and other distinguishing traits; the location at which the animal was seized; the date of seizure; the name and address of the person from whom any animal three months of age or over was received; and, the name and address of the person to whom any animal three months of age or over was transferred.
\subsection{Release from Animal Pound\footnote{(‘83 Code, SEC. 10.20, Subd. 11)}}
Dogs and cats shall be released to their owners, as follows:
\begin{enumerate}[{\indent}1)]
    \item If the dog or cat is owned by a resident of the city, after purchase of a license as aforesaid, and payment of the impounding fee for the period for which the dog or cat was impounded and all expenses of observation or treatment.
    \item If the dog or cat is owned by a person not a resident of the city, after immunization of any the dog or cat for rabies, and payment of the impounding fee for the period for which the dog or cat was impounded and all expenses of observation or treatment.
    \item All impounding fees shall be established, and may be amended, by resolution of the Council.\footnote{(Ord. 41, 2nd Series, effective 4-23-87)}
    \item A dangerous dog shall not be released to its owner but shall be destroyed as provided in this section.\footnote{(Ord. 115, 2nd Series, effective 1-18-97)}
\end{enumerate}
\subsection{Exceptions}
\subsubsection{}
Any dog or other animal seized under M.S. \textsection 343.22 or \textsection 343.29, as it may be amended from time to time, shall be held for ten regular business days.  For the purposes of this section, the term \textbf{REGULAR BUSINESS DAY} shall have the meaning as stated in division (C) of this section.  A person claiming an interest in an animal in custody under this section may prevent disposition of the animal by posting security in an amount sufficient to provide for the animal’s actual costs of care and keeping.  The security must be posted within ten days of the seizure inclusive of the date of the seizure.\footnote{(‘83 Code, SEC. 10.20, Subd. 12)}
\subsubsection{}
Any dog or other animal seized as dangerous as defined in this chapter shall be held for ten regular business days as set out in division (C) of this section. In the case of a dog or other animal quarantined because it has bitten a person, the ten-day period shall be concurrent with the period of the quarantine. A person claiming an interest in an animal in custody under this section, or a person contesting the determination of the animal as dangerous, may prevent disposition of the animal by posting security in an amount sufficient to provide for the animal’s actual costs of care and keeping, and costs associated with the destruction of the animal should the determination that the animal be destroyed be upheld. This security must be posted within ten days of the seizure inclusive of the date of the seizure.
\subsection{Notice of Impounding under State Law\footnote{(‘83 Code, SEC. 10.20, Subd. 13)}}
Upon impounding an animal under division (E) above, notice shall be given the owner or person claiming interest in the animal by delivering or mailing it to a person claiming an interest in the animal or by posting a copy of it at the place where the animal is taken into custody or by delivering it to a person residing on the property, and telephoning, if possible. The notice shall include:
\begin{enumerate}[{\indent}1)]
    \item A description of the animal seized, the authority and purpose for the seizure, the time, place, and circumstances under which the animal was seized, and the location, address, telephone number, and contact person where the animal is kept;
    \item A statement that a person claiming an interest in the animal may post security to prevent disposition of the animal and may request a hearing concerning the seizure or impoundment and that failure to do so within ten days of the date of the notice will result in disposition of the animal.
    \item A statement that all actual costs of the care, keeping, and disposal of the animal are the responsibility of the person claiming an interest in the animal, except to the extent that a court or hearing officer finds that the seizure or impoundment was not substantially justified by law; and
    \item A form or specific instructions that can be used by a person claiming an interest in the animal for requesting a hearing under this section.
\end{enumerate}
\subsection{Right to Hearing and Release from Animal Pound}
Upon request of a person claiming interest in the animal, which request must be made within ten days of the date of seizure, a hearing shall be held within five business days of the request to determine the validity of the seizure and impoundment.  If the seizure was done pursuant to a warrant under M.S. \textsection 343.22, as it may be amended from time to time, the hearing must be conducted by the judge who issued the warrant.  If the seizure was done under M.S. \textsection 343.29, as it may be amended from time to time, the city may either authorize a licensed veterinarian with no financial interest in the matter or professional association with either party or use the services of a hearing officer to conduct the hearing.  If the seizure was done under the dangerous animal provisions of this chapter, the city shall use the services of a hearing officer to conduct the hearing.  A person claiming interest in the animal who is aggrieved by a decision of a hearing officer under this section may seek a court order governing the seizure or impoundment within five days of the notice of the order.
\subsubsection{}
The judge or hearing officer may authorize the return of the animal, if the judge or hearing officer finds that:
\begin{enumerate}[{\indent}a)]
    \item The animal is physically fit;
    \item The person claiming an interest in the animal can and will provide the care required by law for the animal; and
    \item If the animal is determined dangerous and the person claiming interest in the animal consents to abide by the provisions of M.S. \textsection 347.50 through \textsection 347.56, as amended from time to time, regarding the registration and keeping of dangerous dogs.
\end{enumerate}
\subsubsection{}
The person claiming an interest in the animal is liable for all actual costs of the care, keeping, and disposal of the animal, except if a court or hearing officer finds that the seizure or impoundment was not substantially justified by law.  The costs shall be paid in full or a mutually satisfactory arrangement for payment must be made between the city and the person claiming an interest in the animal before return of the animal to the person.\footnote{(‘83 Code, SEC. 10.20, Subd. 14)  (Ord. 130, 2nd Series, effective 5-16-98)}
\subsection{Personal Liability}
The owner of a dog or cat which is impounded under this section is personally liable for the impounding fees and expenses of observation, disposal or treatment associated with the animal’s impoundment.  If the owner fails to properly claim the dog or cat or the animal, if found to have rabies or to be a dangerous dog representing a continuing threat of severe injury to human beings, is destroyed, the city may prepare a bill for the fees and expenses and mail it to the owner and the amount thereof shall then be due and payable to the city.\footnote{(‘83 Code, SEC. 10.20, Subd. 15)  (Ord. 43, 2nd Series, effective 6-11-87)}
\subsection{Unclaimed Dogs and Cats}
If an impounded dog or cat is unclaimed, the animal shall be humanely destroyed and the carcass disposed of, unless it is requested by a licensed educational or scientific institution under authority of M.S. \textsection 35.71, as it may be amended from time to time.  Provided, however, that if a tag affixed to the animal, or a statement by the animal’s owner after seizure specifies that the animal should not be used for research, the animal shall not be made available to any institution but be destroyed after the expiration of the five-day period.\footnote{(‘83 Code, SEC. 10.20, Subd. 16)  (Ord. 41, 2nd Series, effective 4-23-87)}
