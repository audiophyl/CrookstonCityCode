\chapter*{Chapter 150: \\
	Building Regulations}
    \addstarredchapter{Chapter 150: Building Regulations}
    \vfill
    \minitoc
    \pagebreak

\section{Building Code Adopted}
\index{BUILDING REGULATIONS!Building Code Adopted}
\subsection{}
The following are hereby adopted by reference as though set forth verbatim herein:
\subsubsection{}
The Minnesota State Building Code (880), which includes the following chapters of Minnesota Rules:
\begin{enumerate}[{\indent\indent}a)]
    \item 1300, Administration of Minnesota State Building Code;
    \item 1301, Building Official Certification;
    \item 1302, State Building Code Construction Approvals;
    \item 1303, Minnesota Provisions of the State Building Code;
    \item 1305, Adoption of the 2006 international Building Code;
    \item 1307, Elevators and Related Devices;
    \item 1309, Adoption of the 2006 International Residential Code;
    \item 1311, Adoption of the Guidelines for the Rehabilitation of Existing Buildings;
    \item 1315, Adoption of the 2005 National Electrical Code;
    \item 1325, Solar Energy Systems;
    \item 1335, Floodproofing Regulations;
    \item 1341, Minnesota Accessibility Code; (m)1346, Adoption of the 2000 Minnesota State Mechanical Code;
    \item 1360, Prefabricated Structures;
    \item 1361, Industrialized/Modular Buildings;
    \item 1370, Storm Shelters (Manufactured Home Parks);
    \item Minnesota Energy Code — consists of Minnesota Statutes 168.617 (7670) and Minnesota Rules Chapters, 7672, 7674, 7676 and 7678;
\end{enumerate}
\subsection{}
One copy of each of the codes shall be marked \textbf{CITY OF CROOKSTON - OFFICIAL COPY} and kept on file in the office of the Clerk-Treasurer and open to inspection and use by the public.\footnote{(‘83 Code, SEC. 4.01)  (Ord. 126, 2nd Series, effective 5-16-98; Am. Ord. 144, 2nd Series, passed 9-11-01)}

\section{Housing Code Adopted}
\index{BUILDING REGULATIONS!Housing Code Adopted}
The Uniform Housing Code, 1988 Edition, published by the International Conference of Building Officials, is hereby adopted by reference as though set forth verbatim herein.  One copy of the code shall be marked \textbf{CITY OF CROOKSTON - OFFICIAL COPY} and kept on file in the office of the Building Official and open to inspection and use by the public.\footnote{(‘83 Code, SEC. 4.05)  (Ord. 34, 2nd Series, effective 4-16-86)}

\section{Permit Fees}
\index{BUILDING REGULATIONS!Permit Fees}
Fees for permits under this chapter, which may include a surcharge, shall be determined by the Council and fixed by its resolution, a copy of which shall be in the office of the Building Official and uniformly enforced.\footnote{(‘83 Code, SEC. 4.02)}

\section{Building Permits Required; Exterior Work Time Limits}
\index{BUILDING REGULATIONS!Building Permits Required; Exterior Work Time Limits}
It is unlawful for any person to erect, construct, enlarge, alter, repair, move, improve, remove, convert, or demolish any building or structure, or any part or portion thereof, including, but not limited to, the plumbingr electrical, ventilating, heating or air conditioning systems therein, or cause the same to be done, without first obtaining a separate building or mechanical permit for each building, structure or mechanical components from the city, except for exemptions listed in the Building Code. No permit shall be required for minor or insignificant work for which no permit fee is charged. It is unlawful for any person to fail to complete exterior work authorized by a building permit issued in accordance with the state building code within 365 days following issuance of the building permit when the project involves a one to four family residence or within 730 days following issuance of the building permit when the project does not involve a one to four family residence.

\section{Permits and Special Requirements for Moving Buildings}
\index{BUILDING REGULATIONS!Permits and Special Requirements for Moving Buildings}
\subsection{Definitions}
For the purpose of this section, the following definitions shall apply, unless the context clearly indicates or requires a different meaning.
\begin{description}
    \item[COMBINED MOVING PERMIT] A permit to move a building on both a street and a highway.
    \item[HIGHWAY] A public thoroughfare for vehicular traffic which is a state trunk highway, county state-aid highway, or county road.
    \item[HIGHWAY MOVING PERMIT] A permit to move a building on a highway for which a fee is charged which does not include route approval, but does include regulation of activities which do not involve the use of the highway; which activities include, but are not limited to, repairs or alterations to a municipal utility required by reason of the movement.
    \item[MOVING PERMIT] A document allowing the use of a street or highway for the purpose of moving a building.
    \item[STREET] A public thoroughfare for vehicular traffic which is not a state trunk highway, county state-aid highway or county road.
    \item[STREET MOVING PERMIT] A permit to move a building on a street for which a fee is charged which does include route approval, together with use of the street and activities including, but not limited to, repairs or alterations to a municipal utility required by reason of the movement.
\end{description}
\subsection{Application}
The application for a moving permit shall state the dimensions, weight, and approximate loaded height of the structure or building proposed to be moved, the places from which and to which it is to be moved, the route to be followed, the dates and times of moving and parking, the name and address of the mover, and the municipal utility and public property repairs or alterations that will be required by reason of the movement. In the case of a street moving permit or combined moving permit the application shall also state the size and weight of the structure or building proposed to be moved and the street alterations or repairs that will be required by reason of the movement. All applications shall be referred to the Public Works Department. All applications for street and combined moving permits shall also be referred to the Police Department and no permits shall be issued until route approval has been obtained from the Departments.
\subsection{Permit and Fee}
The moving permit shall state date or dates of moving, hours, routing, movement and parking. Permits shall be issued only for moving buildings by building movers licensed by the State of Minnesota. Fees to be charged shall be separate for each of the following: a moving permit fee to cover use of streets and route approval, and a fee equal to the anticipated amount required to compensate the city for any municipal utility and public property (other than streets) repairs or alterations occasioned by the movement. All permit fees shall be paid in advance of issuance.
\subsection{Building Permit and Code Compliance}
Before any building is moved from one location to another within the city, or from a point of origin without the city to a destination within the city, regardless of the route of movement, it shall be inspected and a building permit shall have been issued for at least the work necessary to bring it into full compliance with the State Building Code.
\subsection{Unlawful Acts}
\subsubsection{}
It is unlawful for any person to move a building on any street without a street moving permit from the city.
\subsubsection{}
It is unlawful for any person to move a building on any highway without a highway moving permit from the city.
\subsubsection{}
It is unlawful to move any building (including a manufactured home) if the point of origin or destination (or both) is within the city, and regardless of the route of movement, without having paid in full all real and personal property taxes, special assessments and municipal utility charges due on the premises of origin and filing written proof of the payment with the city.\footnote{(‘83 Code, SEC. 4.04)  Penalty, see SEC. 150.99}

\section{Sign Permits}
\index{BUILDING REGULATIONS!Sign Permits}
\subsection{Permit Required}
Except for exemptions listed in SEC. 152.177, it is unlawful for any person to erect, construct, enlarge, alter, move, remove, repair, maintain, or convert any sign without first obtaining a separate permit therefor for each sign.\footnote{(Ord. 84, 2nd Series, effective 7-17-93)}
\subsection{Requirements and Regulations}
\subsubsection{}
A separate permit application shall be required for each activity sign requiring a permit.\footnote{(Ord. 18, 2nd Series, effective 5-18-85)}
\subsubsection{}
No permit application shall be accepted unless it is accompanied by specific plans and all necessary information upon which to base a decision as to whether or not it complies fully with the city code.
\subsubsection{}
Sign permit fees shall be established by resolution of the Council.
\subsubsection{}
No permit shall issue until the applicant has filed with the Clerk-Treasurer a policy or certificate of public liability insurance for coverage of the sign-related activity concurrent with the permit term with limits of at least \$100,000 for injury to one person, \$300,000 for each occurrence, and \$50,000 property damage.
\subsubsection{}
It is the primary responsibility of all owners and occupants of private property to see that all signs located thereon are erected, constructed, enlarged, altered, moved, removed, repaired, maintained and converted in compliance with the city code.  It is unlawful for any person to allow any sign located on real property owned or occupied by him or her to be erected, constructed, enlarged, altered, moved, removed, repaired, maintained, or converted without a required permit.
\subsubsection{}
In the event any work or activity relating to a sign is performed without the securing of the required permit prior to the work or activity, in addition to all other rights and remedies authorized by the city code and as otherwise provided by law, the city may cause the sign to be inspected for compliance with the requirements of the city code.  The city shall charge an inspection fee for any inspection which may include a penalty for failure to secure the required permit.
\subsubsection{}
The inspection fee under this section (including any penalty for failure to secure the required permit) shall be fixed and determined by the Council, adopted by resolution, and uniformly enforced.  The inspection fee may, from time to time, be amended by the Council by resolution.  A copy of the resolution setting forth the currently effective inspection fee shall be kept on file in the office of the Clerk-Treasurer, and open to inspection during regular business hours.
\subsubsection{}
If upon the inspection the sign is found to comply with city code requirements, a permit therefor may be issued upon payment of the inspection fee and the regular permit fee.
\subsubsection{}
If any sign is found to have been erected, constructed, enlarged, altered, moved, removed, repaired, maintained, or converted in violation of any of the provisions of the city code, the city may give the owner and occupant of the property upon which the sign is located written notice of the violation. Unless the sign is owned by the owner or occupant of the property, a copy of the notice shall also be given to the owner of the sign, if known, or, if unknown, affixed to the sign, sign structure, or building. If the violation is not remedied within 30 days after the notice, or within three days after any notice relating to a portable sign, the city may remove, or cause to be removed, the sign at the expense of the owner and occupant of the property.
\subsubsection{}
Any rights or remedies conferred by this section shall not preclude other civil or criminal action by the city under this section, the city code or other applicable law.
\subsubsection{}
The owner and occupant of the property on or upon which the sign is located which is inspected under division (B)(7) of this section or otherwise found to be in violation of the requirements of the city code shall be jointly and severally liable for the payment of the inspection fee and the cost of removal of the sign by the city. The city may prepare a bill and mail it to the owner and occupant and the amount shall then be due and payable. The city may collect the same in a civil action.\footnote{(Ord. 18, 2nd Series, effective 5-18-85)}\footnote{(‘83 Code, SEC. 4.06)  Penalty, see SEC. 150.99}

\section{Satellite Dish Antenna Permits}
\index{BUILDING REGULATIONS!Satellite Dish Antenna Permits}
\subsection{Definition}
The term \textbf{SATELLITE DISH ANTENNA} has the meaning set forth in city code SEC. 152.003.\footnote{(Ord. 35, 2nd Series, effective 5-17-86)}
\subsection{Unlawful Activity}
It is unlawful for any person to erect, construct, move, or maintain, or cause to be erected, constructed, moved, or maintained within the city any satellite dish antenna greater than 30” inches in diameter without first obtaining a separate permit for each antenna.\footnote{(Ord. 117, 2nd Series, effective 5-17-97)}
\subsection{Requirements and Regulations}
\subsubsection{}
A separate permit application shall be required for each activity and satellite dish antenna requiring a permit.
\subsubsection{}
No permit application shall be accepted unless it is accompanied by specific plans and all necessary information upon which to base a decision as to whether or not it complies fully with the city code.
\subsubsection{}
In the event that the Building Official, by reason of conditions imposed herein, or omitted therefrom, is unable to grant a requested permit, the application therefor shall be presented to the Planning Commission at the next regular meeting for consideration, interpretation, and recommendation to the Council.  In the event that the decision of the Council is favorable, the permit shall be granted under considered special conditions.
\subsubsection{}
Satellite dish antenna permit fees shall be established by resolution of the Council.\footnote{(Ord. 28, 2nd Series, effective 3-20-86)}\footnote{(‘83 Code, SEC. 4.07)  Penalty, see SEC. 150.99}

\section{Fences}
\index{BUILDING REGULATIONS!Fences}
\subsection{Definition}
For purposes of this section, the term \textbf{FENCE} means any partition, structure or gate erected as a dividing marker, barrier or enclosure of a property.
\subsection{Application}
Notwithstanding any provisions to the contrary contained in this Code, it is unlawful for any person to construct or erect any fence without first obtaining a permit from the city.
\subsection{Application for Permit}
Every application for a fence permit shall be submitted to the Building Official.  The application shall set forth the type of fence, the materials to be used in the construction thereof, its height, and its location, particularly as to its proximity to the lot lines of the applicant.  The applicant is responsible to properly locate the property lines.  A survey may be needed if the applicant is unable to locate the property markers.  The fee shall be a fixed fee as required in \textsection 150.03.
\subsection{Prohibited}
Notwithstanding any provisions to the contrary contained in this Code, it is unlawful for any person to construct and maintain or allow to be constructed or maintained upon any property a barbed wire fence or any fence charged or connected with any electrical current in such a manner as to transmit the current to persons, animals or things which intentionally or unintentionally come in contact with the same.
\subsection{Restricted}
Barbed wire may be used to top security fencing in an industrial district but shall not be closer to the ground than a height of six and one-half feet.  Fences charged or connected with an electrical current may be allowed in the Farm Residence district for agricultural use only.
\subsection{Temporary Fencing}
The following fencing materials may be installed without a permit if they meet the listed restrictions:
\begin{enumerate}[{\indent}1)]
    \item Snow fences:  Plastic or wood fencing used for the limited purpose of reducing the drifting of snow shall be allowed from October 1st through May 1st.
    \item Construction-related fencing:  Silt fences or fencing for the purpose of security at a construction jobsite, protection of excavation or protection of plants, shall be allowed during the duration of the project, while the building permit is valid.
\end{enumerate}
\subsection{Common Fences}
Fences may be placed on an adjoining property line, provided the property owners agree in writing and the writing is recorded in the office of the County Recorder on each of the subject properties.  A copy of this recorded agreement must be provided to the Building Official prior to issuance of a permit.
\subsection{Standards}
Notwithstanding any provisions to the contrary contained in this Code, all fences erected or maintained in the city shall meet and comply with the following requirements:
\begin{enumerate}[{\indent}1)]
    \item All fences shall not exceed six feet in height in a residential district.  
    \item All fences more than a height of six feet in a non-residential district are permitted to the extent they comply with the Minnesota State Building Code.  
    \item All fences on the street side of the principle structure shall not exceed four feet in height if erected closer to the street line than the average set back in the area.  
    \item All fences shall comply with the traffic control restrictions in \textsection 152.170.
    \item No fence shall be allowed to be erected closer than two feet from the property line or within a utility or drainage easement.
    \item No fence shall be allowed to be erected parallel to an alleyway closer than five feet from the property line.
    \item No fence shall be constructed of solid sheathing.
    \item The Council shall have the power upon application, for cause shown, to waive the strict application of the preceding requirements and compliance therewith.
\end{enumerate}
\subsection{Construction}
Every fence shall be constructed in a substantial manner and of a substantial material reasonably suitable for the purpose for which the fence is proposed to be used.  Supports and structural members of any fence shall be placed in the interior side of the fence so that the finished side is exposed.  
\subsection{Maintenance}
Every fence shall be structurally sound and maintained in a condition of reasonable repair and shall not by reason of age, decay, accident or other cause be allowed to become and remain in a state of disrepair so as to be a nuisance of the public or any abutting property owner.  Furthermore, all fences shall be constructed and maintained so that the exposed, outer-face is smooth, in neat condition and appearance at all times, and free of rust, rotting, chipped paint, loose/broken fencing panels, or exposed structural members.  The maintenance requirements apply to every fence, regardless of whether it was constructed before or after the adoption of this provision.  If the Building Official determines that more than 25\% of the fence is in disrepair, the Building Official shall notify the owner of the property on which the fence is located, in writing, of the existence of said disrepair, and a permit shall be secured by the owner within 14 days after receiving the notice from the Building Official.  Failure to secure the permit within 14 days after receiving the notice, or repair the fence in a reasonable time, may result in the city ordering the fence be repaired, painted or removed.    

\setcounter{section}{98}
\section{Penalty}
\index{BUILDING REGULATIONS!Penalty}
Every person who violates a section, division, or provision of this chapter when he or she performs an act thereby prohibited or declared unlawful, or fails to act when the failure is thereby prohibited or declared unlawful, or performs an act prohibited or declared unlawful or fails to act when the failure is prohibited or declared unlawful by a code adopted by reference by this chapter, and upon conviction thereof, shall be punished as for a misdemeanor except as otherwise stated in specific provisions hereof. The penalty which may be imposed for any crime which is a misdemeanor under this code, including Minnesota Statutes specifically adopted by reference, shall be a sentence of not more than 90 days or a fine of not more than \$1,000, or both. The costs of prosecution may be added. A separate offense shall be deemed committed upon each day during which a violation occurs or continues.\footnote{(‘83 Code, SEC. 4.99)}
