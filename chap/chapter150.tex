\chapter*{Chapter 150: \\
	Building Regulations}
    \addstarredchapter{Chapter 150: Building Regulations}
    \vfill
    \minitoc
    \pagebreak

\section{Building Code Adopted By Reference}
\index{BUILDING REGULATIONS!Building Code Adopted By Reference}
The Minnesota State Building Code, as adopted by the Commissioner of Labor and Industry pursuant to Minnesota Statutes, Chapter 326B, including all of the amendments, rules and regulations established, adopted and published from time to time by the Minnesota Commissioner of Labor and Industry, through the Building Codes and Standards Unit, is hereby adopted by reference with the exception of any optional chapters, amendments, rules or regulations, unless specifically adopted herein. 

\section{Property Maintenance Code}
\index{BUILDING REGULATIONS!Property Maintenance Code}
\subsection{International Property Maintenance Code Adopted}
The International Property Maintenance Code, 2012 Edition, (“IPMC”) published by the International Code Council, Inc., is hereby adopted by reference as though set forth verbatim herein, as modified and amended by \textsection 150.02(B) and shall be referred to as the Property Maintenance Code. One copy of the Property Maintenance Code shall be marked CITY OF CROOKSTON - OFFICIAL COPY and kept on file in the office of the Building Official and open to inspection and use by the public.  
\subsection{Amendments to the International Property Maintenance Code}
The IPMC adopted by reference in \textsection 150.02(A) is hereby modified and amended as follows:
\begin{enumerate}[{\indent}1)]
    \item \textbf{Section 101.1 Title.} This Section is amended in its entirety to read as follows: “These regulations shall be known as the Property Maintenance Code, hereinafter referred to as “this code.””
    \item \textbf{Section 101.3 Intent.} This Section is amended in its entirety to read as follows: “This code shall be construed to secure its expressed intent, which is to promote the public health, safety, general welfare, and zoning and land use objectives insofar as they are affected by the continued occupancy and maintenance of structures and premises.”
    \item \textbf{Section 102.1 General.} This Section is amended by adding the following sentence at the end of the paragraph: “Except with respect to the Minnesota State Fire Code, when there is a conflict between this code and any other provision of the City Code or other law adopted by reference in the City Code, the more restrictive shall govern.”
    \item \textbf{Section 102.3 Application of other codes.} This Section is amended in its entirety to read as follows: “Repairs, additions or alterations to a structure, or changes of \textit{occupancy}, shall be done in accordance with the procedures and provisions of the Minnesota State Building Code, the Minnesota State Fire Code, the Minnesota Plumbing code, the Minnesota Residential Code, the Minnesota Fuel Gas Code, the Minnesota Mechanical Code, the Minnesota Electrical Code, the Minnesota Residential Energy Code and the Minnesota Commercial Energy Code.  Nothing in this code shall be construed to cancel, modify or set aside any provision of Chapter 152 of the City Code.”
    \item \textbf{Section 102.7 Referenced codes and standards.} This Section is amended in its entirety to read as follows: “The codes and standards referenced in this code shall be those that are listed in the Minnesota State Building Code, the Minnesota State Fire Code, the Minnesota Plumbing Code, the Minnesota Residential Code, the Minnesota Fuel Gas Code, the Minnesota Mechanical Code, the Minnesota Electrical Code, the Minnesota Residential Energy Code and the Minnesota Commercial Energy Code and considered part of the requirements of this code to the prescribed extent of each such reference and as further regulated in Sections 102.7.1 and 102.7.2.  \textbf{Exception:} Where enforcement of a code provision would violate the conditions of the listing of the equipment or appliance, the conditions of the listing shall apply.”
    \item \textbf{SECTION 103 DEPARTMENT OF PROPERTY MAINTENANCE INSPECTION.} The title of this Section is amended in its entirety to read as follows: “\textbf{ADMINISTRATION OF THE PROPERTY MAINTENANCE CODE.}”
    \item \textbf{Section 103.1. General.} This Section is amended in its entirety to read as follows: “The City Administrator shall have the authority to appoint an official to administer and enforce this code.”
    \item \textbf{Section 103.2. Appointment.} This Section is amended in its entirety to read as follows: “The City Administrator shall appoint an official responsible for the enforcement of this code.  This official shall be known as the \textit{code official}.
    \item \textbf{Section 103.3. Deputies.} This Section is amended in its entirety to read as follows: “The City Administrator shall have the authority to appoint deputy(s) to assist the \textit{code official} in administering and enforcing this code.  Such deputy(s) shall have the powers delegated by the \textit{code official}.”
    \item \textbf{Section 103.5. Fees.} This Section is amended in its entirety to read as follows: “The fees for activities, services, administration, and enforcement of this code are set forth in the City’s fee schedule as adopted by the Council from time to time.”
    \item \textbf{Section 106.3. Prosecution of violation.} The title of this Section is amended in its entirety to read: “\textbf{Administrative fine and prosecution of violation.}”  This Section is amended in its entirety to read as follows: “Any person failing to comply with a notice of violation or order served in accordance with Section 107 shall be charged an administrative fine by the \textit{code official} through the issuance and service of an administrative fine.  For purposes of issuing an administrative fine, any violation shall be deemed a \textit{strict liability offense}.  The administrative citation shall include a statement that the administrative fine may be appealed in accordance with Section 111.1. It is the intent of the City to impose an administrative fine to defray costs associated with inspection and compliance services.  The administrative fine must reflect the costs associated with inspection, notice and order, posting, enforcement, administration, and/or abatement of unlawful conditions set forth in this code and shall be set and may, from time-to-time, be amended, by the City Council by resolution.  Failure by the owner or responsible party to pay the administrative fine shall be cause for the amount of the fine to be levied against the subject property as a special assessment and collected as in the case of other special assessments.  Any person failing to comply with three or more notices of violations or orders within a twelve-month period shall be guilty of a misdemeanor punishable by up to ninety days in jail, a \$1,000.00 fine, or both.  Action under this section does not preclude any other civil enforcement procedure.”
    \item \textbf{Section 107.5. Penalties.} This Section is amended in its entirety to read as follows: “Penalties for noncompliance with orders and notices shall be as set forth in Sections 106.3 and 106.4.”
    \item \textbf{Section 111.1. Application for appeal.} This Section is amended in its entirety to read as follows: “Any person directly affected by a decision of the \textit{code official} or a notice or order issued under this code shall have the right to appeal to the Board of Property Maintenance Appeals, provided that a written application for appeal is filed within twenty (20) days after the day the decision, notice or order was served. In the case of an appeal from a notice issued to vacate pending elimination of imminent dangers, the appeal shall be heard as soon as possible after the time of filing. An application for appeal shall be based on a claim that the true intent of this code or the rules legally adopted thereunder have been incorrectly interpreted, the provisions of this code do not fully apply, or the requirements of this code are adequately satisfied by other means.”
    \item \textbf{Sections 111.2, 111.2.1, 111.2.2, 111.2.3, 111.2.4, 111.2.5, 111.4.1, 111.5, 111.6, 111.6.1, 111.6.2, 111.7} are deleted and removed in their entirety.
    \item \textbf{Section 112.4. Failure to comply.} This Section is amended in its entirety to read as follows: “Any person who shall continue any work after having been served with a stop work order, except such work as that person is directed to perform to remove a violation or unsafe condition, shall be in violation of this code.”
    \item \textbf{Section 201.3. Terms defined in other codes.} This Section is amended in its entirety to read as follows: “Where terms are not defined in this code and are defined in the Minnesota State Building Code, the Minnesota State Fire Code, the Minnesota Plumbing Code, the Minnesota Residential Code, the Minnesota Fuel Gas Code, the Minnesota Mechanical Code, the Minnesota Electrical Code, the Minnesota Residential Energy Code, the Minnesota Commercial Energy Code and the City Code, such terms shall have the meanings ascribed to them as stated in those codes.”
    \item \textbf{Section 301.3. Vacant structures and land.} This Section is amended in its entirety to read as follows: “All vacant \textit{premises} shall be maintained in a clean, safe, secure and sanitary condition as provided herein so as not to adversely affect the public health or safety and so that they promote the general welfare and zoning and land use objectives.  A \textit{premises} is vacant if no person or persons actually and currently conducts a lawful business on the \textit{premises} or lawfully resides or lives in any part of a \textit{structure} on a permanent, non-transient basis in accordance with the City Code.  An owner shall not permit or maintain exterior storage on a vacant \textit{premises}. Exterior storage includes, but is not limited to, storage of motor vehicles, trailers, recreational vehicles, construction materials, firewood, refuse, garbage, salvage materials, or other junk or debris.  \textbf{Exceptions:} Those persons who leave their residential structures on a temporary basis for vacation purposes or reside elsewhere on  a seasonal basis and have the intent to return are exempt from this Section. Those persons who have received approval from the \textit{code official} to store construction materials and equipment and have an active building permit for the \textit{premises} are exempt from this Section.
    \item \textbf{Section 302.4. Weeds.} This Section is deleted and removed in its entirety.
    \item \textbf{Section 302.8. Motor vehicles.} This Section is deleted and removed in its entirety.
    \item \textbf{Section 303.2. Enclosures.} This Section is amended in its entirety to read as follows: “Private swimming pools, hot tubs and spas, containing water more than 24 inches (610 mm) in depth shall be completely surrounded by a fence or barrier at least 48 inches (1219 mm) in height above the finished ground level measured on the side of the barrier away from the pool.  Gates and doors in such barriers shall be self-closing and self-latching.  Where the self-latching device is a minimum of 54 inches (1372 mm) above the bottom of the gate, the release mechanism shall be located on the pool side of the gate. Self-closing and self-latching gates shall be maintained such that the gate will positively close and latch when released form an open position of 6 inches (152 mm) from the gatepost.  No existing pool enclosure shall be removed, replaced or changed in a manner that reduces its effectiveness as a safety barrier.  \textbf{Exception:} Pools that are temporarily installed above ground and disassembled and removed at the end of the summer season shall be exempt from the provisions of this section.  However, during nonuse, such pools shall be covered and access thereto shall be removed.”
    \item \textbf{Section 304.1. General.} This Section is amended in its entirety to read as follows: “The exterior of a structure shall be maintained in good repair, structurally sound and sanitary so that it is not detrimental to the public health or safety and so that it promotes the general welfare and zoning and land use objectives.”
    \item \textbf{Section 304.1.1. Unsafe conditions.} The first sentence in this Section is amended by deleting and removing the phrase “\textit{International Building Code} or the \textit{International Existing Building Code}” and by adding in its place the term “\textit{Minnesota State Building Code}.”
    \item \textbf{Section 304.14. Insect screens.} This Section is amended in its entirety to read as follows: “During the period of May 1\textsuperscript{st} to October 1\textsuperscript{st}, every door, window and other outside opening required for \textit{ventilation} of habitable rooms, food preparation areas, food service areas or any areas where products to be included or utilized in food for human consumption are processed, manufactured, packaged or stored shall be supplied with \textit{approved} tightly fitting screens of minimum 16 mesh per inch (16 mesh per 25 mm), and every screen door used for insect control shall have a self-closing device in good working condition.
    \item \textbf{Section 305.1.1. Unsafe conditions.} The first sentence in this Section is amended by deleting and removing the phrase “\textit{International Building Code} or the \textit{International Existing Building Code}” and by adding in its place the term “\textit{Minnesota State Building Code}.”
    \item \textbf{Section 306.1.1. Unsafe conditions.} The first sentence in this Section is amended by deleting and removing the phrase “\textit{International Building Code} or the \textit{International Existing Building Code}” and by adding in its place the term “\textit{Minnesota State Building Code}.”
    \item \textbf{Section 308.2.1. Rubbish storage facilities.} This Section is amended in its entirety to read as follows: “The occupant of every premises shall supply approved covered containers for rubbish and shall be responsible for the removal of rubbish from the premises.  Notwithstanding the foregoing requirement, the owner of a premises that contains more than four dwelling units shall provide a covered dumpster for the occupants of such premises sufficient to collect the rubbish produced from all units, and such owner shall empty the covered dumpster on a weekly periodic basis, unless it is filled earlier, in which event the owner shall immediately empty the dumpster.”
    \item \textbf{Section 401.3. Alternative devices.} The first sentence in this Section is amended by deleting and removing the term “\textit{International Building Code}” and by adding in its place the term “\textit{Minnesota State Building Code}.”
    \item \textbf{Section 502.5. Public toilet facilities.} The first sentence in this Section is amended by deleting and removing the term “\textit{International Building Code}” and by adding in its place the term “\textit{Minnesota State Building Code}.”
    \item \textbf{Section 505.1. General.} The first sentence in this Section is amended by deleting and removing the term “\textit{International Building Code}” and by adding in its place the term “\textit{Minnesota State Building Code}.”
    \item \textbf{Section 602.3. Heat supply.} The first sentence in this Section is amended in its entirety to read as follows: “Every \textit{owner} and \textit{operator} of any building who rents, leases or lets one or more \textit{dwelling units} or \textit{sleeping units} on terms, either expressed or implied, to furnish heat to the \textit{occupants} thereof shall provide and maintain a heating system during the period from October 1\textsuperscript{st} to May 1\textsuperscript{st} that allows the \textit{occupants} to maintain a minimum temperature of 68°F (20°C) in all habitable rooms, \textit{bathrooms} and \textit{toilet rooms}.”
    \item \textbf{Section 602.4. Occupiable work spaces.} The first sentence in this Section is amended in its entirety to read as follows: “Indoor occupiable work spaces shall be supplied with heat during the period from October 1\textsuperscript{st} to May 1\textsuperscript{st} to maintain a minimum temperature of 65°F (18°C) during the period the spaces are occupied.”
    \item \textbf{Section 604.3.1.1. Electrical equipment.} The first sentence in this Section is amended by deleting and removing the term “\textit{International Building Code}” and by adding in its place the term “\textit{Minnesota State Building Code}.”
    \item \textbf{Section 604.3.2.1. Electrical equipment.} The first sentence in this Section is amended by deleting and removing the term “\textit{International Building Code}” and by adding in its place the term “\textit{Minnesota State Building Code}.”
    \item \textbf{Section 606.1. General.} The first sentence in this Section is amended in its entirety to read as follows: “Elevators, dumbwaiters and escalators shall be maintained in compliance with ASME A17 and Chapter 1307 of the Minnesota State Building Code.”
    \item \textbf{Section 702.1. General.} The second sentence in this Section is amended by deleting and removing the term “\textit{International Fire Code}” and by adding in its place the term “\textit{Minnesota State Fire Code}.”
    \item \textbf{Section 702.2. Aisles.} The sentence in this Section is amended by deleting and removing the term “\textit{International Fire Code}” and by adding in its place the term “\textit{Minnesota State Fire Code}.”
    \item \textbf{Section 702.3. Locked doors.} The sentence in this Section is amended by deleting and removing the term “\textit{International Fire Code}” and by adding in its place the term “\textit{Minnesota State Fire Code}.”
    \item \textbf{Section 704.1. General.} The sentence in this Section is amended by deleting and removing the term “\textit{International Fire Code}” and by adding in its place the term “\textit{Minnesota State Fire Code}.”
    \item \textbf{SECTION 705.} A new Section is added entitled “\textbf{SECTION 705 CARBON MONOXIDE ALARMS.}”
    \item \textbf{Section 705.1.} A new Section is added entitled “\textbf{Section 705.1 General.}” This Section shall read as follows: “Carbon monoxide alarms shall be installed in accordance with Minnesota Statutes, Sections 299F.50-51, as amended from time to time.”
    \item \textbf{CHAPTER 8 REFERENCED STANDARDS.} The following sentence in this Chapter is removed and deleted in its entirety: “The application of the referenced standards shall be as specified in Section 102.7.”
    \item \textbf{CHAPTER 8 REFERENCED STANDARDS.} The standard reference number “A17.1/CSA B44- 2007" cited in the American Society of Mechanical Engineers table in this Chapter is replaced with “A17.1/CSA B44-2010.”
    \item \textbf{CHAPTER 8 REFERENCED STANDARDS.} The table regarding the International Code Council in this Chapter is deleted in its entirety.
\end{enumerate}

\section{Permit Fees}
\index{BUILDING REGULATIONS!Permit Fees}
Fees for permits under this chapter, which may include a surcharge, shall be determined by the Council and fixed by its resolution, a copy of which shall be in the office of the Building Official and uniformly enforced.

\section{Building Permits Required; Exterior Work Time Limits}
\index{BUILDING REGULATIONS!Building Permits Required; Exterior Work Time Limits}
It is unlawful for any person to erect, construct, enlarge, alter, repair, move, improve, remove, convert, or demolish any building or structure, or any part or portion thereof, including, but not limited to, the plumbingr electrical, ventilating, heating or air conditioning systems therein, or cause the same to be done, without first obtaining a separate building or mechanical permit for each building, structure or mechanical components from the city, except for exemptions listed in the Building Code. No permit shall be required for minor or insignificant work for which no permit fee is charged. It is unlawful for any person to fail to complete exterior work authorized by a building permit issued in accordance with the state building code within 365 days following issuance of the building permit when the project involves a one to four family residence or within 730 days following issuance of the building permit when the project does not involve a one to four family residence.

\section{Permits and Special Requirements for Moving Buildings}
\index{BUILDING REGULATIONS!Permits and Special Requirements for Moving Buildings}
\subsection{Definitions}
For the purpose of this section, the following definitions shall apply, unless the context clearly indicates or requires a different meaning.
\begin{description}
    \item[COMBINED MOVING PERMIT] A permit to move a building on both a street and a highway.
    \item[HIGHWAY] A public thoroughfare for vehicular traffic which is a state trunk highway, county state-aid highway, or county road.
    \item[HIGHWAY MOVING PERMIT] A permit to move a building on a highway for which a fee is charged which does not include route approval, but does include regulation of activities which do not involve the use of the highway; which activities include, but are not limited to, repairs or alterations to a municipal utility required by reason of the movement.
    \item[MOVING PERMIT] A document allowing the use of a street or highway for the purpose of moving a building.
    \item[STREET] A public thoroughfare for vehicular traffic which is not a state trunk highway, county state-aid highway or county road.
    \item[STREET MOVING PERMIT] A permit to move a building on a street for which a fee is charged which does include route approval, together with use of the street and activities including, but not limited to, repairs or alterations to a municipal utility required by reason of the movement.
\end{description}
\subsection{Application}
The application for a moving permit shall state the dimensions, weight, and approximate loaded height of the structure or building proposed to be moved, the places from which and to which it is to be moved, the route to be followed, the dates and times of moving and parking, the name and address of the mover, and the municipal utility and public property repairs or alterations that will be required by reason of the movement. In the case of a street moving permit or combined moving permit the application shall also state the size and weight of the structure or building proposed to be moved and the street alterations or repairs that will be required by reason of the movement. All applications shall be referred to the Public Works Department. All applications for street and combined moving permits shall also be referred to the Police Department and no permits shall be issued until route approval has been obtained from the Departments.
\subsection{Permit and Fee}
The moving permit shall state date or dates of moving, hours, routing, movement and parking. Permits shall be issued only for moving buildings by building movers licensed by the State of Minnesota. Fees to be charged shall be separate for each of the following: a moving permit fee to cover use of streets and route approval, and a fee equal to the anticipated amount required to compensate the city for any municipal utility and public property (other than streets) repairs or alterations occasioned by the movement. All permit fees shall be paid in advance of issuance.
\subsection{Building Permit and Code Compliance}
Before any building is moved from one location to another within the city, or from a point of origin without the city to a destination within the city, regardless of the route of movement, it shall be inspected and a building permit shall have been issued for at least the work necessary to bring it into full compliance with the State Building Code.
\subsection{Unlawful Acts}
\subsubsection{}
It is unlawful for any person to move a building on any street without a street moving permit from the city.
\subsubsection{}
It is unlawful for any person to move a building on any highway without a highway moving permit from the city.
\subsubsection{}
It is unlawful to move any building (including a manufactured home) if the point of origin or destination (or both) is within the city, and regardless of the route of movement, without having paid in full all real and personal property taxes, special assessments and municipal utility charges due on the premises of origin and filing written proof of the payment with the city.\footnote{Penalty, see SEC. 150.99}

\section{Sign Permits}
\index{BUILDING REGULATIONS!Sign Permits}
\subsection{Permit Required}
Except for exemptions listed in SEC. 152.177, it is unlawful for any person to erect, construct, enlarge, alter, move, remove, repair, maintain, or convert any sign without first obtaining a separate permit therefor for each sign.
\subsection{Requirements and Regulations}
\subsubsection{}
A separate permit application shall be required for each activity sign requiring a permit.
\subsubsection{}
No permit application shall be accepted unless it is accompanied by specific plans and all necessary information upon which to base a decision as to whether or not it complies fully with the city code.
\subsubsection{}
Sign permit fees shall be established by resolution of the Council.
\subsubsection{}
No permit shall issue until the applicant has filed with the Clerk-Treasurer a policy or certificate of public liability insurance for coverage of the sign-related activity concurrent with the permit term with limits of at least \$100,000 for injury to one person, \$300,000 for each occurrence, and \$50,000 property damage.
\subsubsection{}
It is the primary responsibility of all owners and occupants of private property to see that all signs located thereon are erected, constructed, enlarged, altered, moved, removed, repaired, maintained and converted in compliance with the city code.  It is unlawful for any person to allow any sign located on real property owned or occupied by him or her to be erected, constructed, enlarged, altered, moved, removed, repaired, maintained, or converted without a required permit.
\subsubsection{}
In the event any work or activity relating to a sign is performed without the securing of the required permit prior to the work or activity, in addition to all other rights and remedies authorized by the city code and as otherwise provided by law, the city may cause the sign to be inspected for compliance with the requirements of the city code.  The city shall charge an inspection fee for any inspection which may include a penalty for failure to secure the required permit.
\subsubsection{}
The inspection fee under this section (including any penalty for failure to secure the required permit) shall be fixed and determined by the Council, adopted by resolution, and uniformly enforced.  The inspection fee may, from time to time, be amended by the Council by resolution.  A copy of the resolution setting forth the currently effective inspection fee shall be kept on file in the office of the Clerk-Treasurer, and open to inspection during regular business hours.
\subsubsection{}
If upon the inspection the sign is found to comply with city code requirements, a permit therefor may be issued upon payment of the inspection fee and the regular permit fee.
\subsubsection{}
If any sign is found to have been erected, constructed, enlarged, altered, moved, removed, repaired, maintained, or converted in violation of any of the provisions of the city code, the city may give the owner and occupant of the property upon which the sign is located written notice of the violation. Unless the sign is owned by the owner or occupant of the property, a copy of the notice shall also be given to the owner of the sign, if known, or, if unknown, affixed to the sign, sign structure, or building. If the violation is not remedied within 30 days after the notice, or within three days after any notice relating to a portable sign, the city may remove, or cause to be removed, the sign at the expense of the owner and occupant of the property.
\subsubsection{}
Any rights or remedies conferred by this section shall not preclude other civil or criminal action by the city under this section, the city code or other applicable law.
\subsubsection{}
The owner and occupant of the property on or upon which the sign is located which is inspected under division (B)(7) of this section or otherwise found to be in violation of the requirements of the city code shall be jointly and severally liable for the payment of the inspection fee and the cost of removal of the sign by the city. The city may prepare a bill and mail it to the owner and occupant and the amount shall then be due and payable. The city may collect the same in a civil action.\footnote{(Ord. 18, 2nd Series, effective 5-18-85)}\footnote{Penalty, see SEC. 150.99}

\section{Satellite Dish Antenna Permits}
\index{BUILDING REGULATIONS!Satellite Dish Antenna Permits}
\subsection{Definition}
The term \textbf{SATELLITE DISH ANTENNA} has the meaning set forth in city code SEC. 152.003.
\subsection{Unlawful Activity}
It is unlawful for any person to erect, construct, move, or maintain, or cause to be erected, constructed, moved, or maintained within the city any satellite dish antenna greater than 30” inches in diameter without first obtaining a separate permit for each antenna.
\subsection{Requirements and Regulations}
\subsubsection{}
A separate permit application shall be required for each activity and satellite dish antenna requiring a permit.
\subsubsection{}
No permit application shall be accepted unless it is accompanied by specific plans and all necessary information upon which to base a decision as to whether or not it complies fully with the city code.
\subsubsection{}
In the event that the Building Official, by reason of conditions imposed herein, or omitted therefrom, is unable to grant a requested permit, the application therefor shall be presented to the Planning Commission at the next regular meeting for consideration, interpretation, and recommendation to the Council.  In the event that the decision of the Council is favorable, the permit shall be granted under considered special conditions.
\subsubsection{}
Satellite dish antenna permit fees shall be established by resolution of the Council.\footnote{Penalty, see SEC. 150.99}

\section{Fences}
\index{BUILDING REGULATIONS!Fences}
\subsection{Definition}
For purposes of this section, the term \textbf{FENCE} means any partition, structure or gate erected as a dividing marker, barrier or enclosure of a property.
\subsection{Application}
Notwithstanding any provisions to the contrary contained in this Code, it is unlawful for any person to construct or erect any fence without first obtaining a permit from the city.
\subsection{Application for Permit}
Every application for a fence permit shall be submitted to the Building Official.  The application shall set forth the type of fence, the materials to be used in the construction thereof, its height, and its location, particularly as to its proximity to the lot lines of the applicant.  The applicant is responsible to properly locate the property lines.  A survey may be needed if the applicant is unable to locate the property markers.  The fee shall be a fixed fee as required in \textsection 150.03.
\subsection{Prohibited}
Notwithstanding any provisions to the contrary contained in this Code, it is unlawful for any person to construct and maintain or allow to be constructed or maintained upon any property a barbed wire fence or any fence charged or connected with any electrical current in such a manner as to transmit the current to persons, animals or things which intentionally or unintentionally come in contact with the same.
\subsection{Restricted}
Barbed wire may be used to top security fencing in an industrial district but shall not be closer to the ground than a height of six and one-half feet.  Fences charged or connected with an electrical current may be allowed in the Farm Residence district for agricultural use only.
\subsection{Temporary Fencing}
The following fencing materials may be installed without a permit if they meet the listed restrictions:
\begin{enumerate}[{\indent}1)]
    \item Snow fences:  Plastic or wood fencing used for the limited purpose of reducing the drifting of snow shall be allowed from October 1st through May 1st.
    \item Construction-related fencing:  Silt fences or fencing for the purpose of security at a construction jobsite, protection of excavation or protection of plants, shall be allowed during the duration of the project, while the building permit is valid.
\end{enumerate}
\subsection{Common Fences}
Fences may be placed on an adjoining property line, provided the property owners agree in writing and the writing is recorded in the office of the County Recorder on each of the subject properties.  A copy of this recorded agreement must be provided to the Building Official prior to issuance of a permit.
\subsection{Standards}
Notwithstanding any provisions to the contrary contained in this Code, all fences erected or maintained in the city shall meet and comply with the following requirements:
\begin{enumerate}[{\indent}1)]
    \item All fences shall not exceed six feet in height in a residential district.  
    \item All fences more than a height of six feet in a non-residential district are permitted to the extent they comply with the Minnesota State Building Code.  
    \item All fences on the street side of the principle structure shall not exceed four feet in height if erected closer to the street line than the average set back in the area.  
    \item All fences shall comply with the traffic control restrictions in \textsection 152.170.
    \item No fence shall be allowed to be erected closer than two feet from the property line or within a utility or drainage easement.
    \item No fence shall be allowed to be erected parallel to an alleyway closer than five feet from the property line.
    \item No fence shall be constructed of solid sheathing.
    \item The Council shall have the power upon application, for cause shown, to waive the strict application of the preceding requirements and compliance therewith.
\end{enumerate}
\subsection{Construction}
Every fence shall be constructed in a substantial manner and of a substantial material reasonably suitable for the purpose for which the fence is proposed to be used.  Supports and structural members of any fence shall be placed in the interior side of the fence so that the finished side is exposed.  
\subsection{Maintenance}
Every fence shall be structurally sound and maintained in a condition of reasonable repair and shall not by reason of age, decay, accident or other cause be allowed to become and remain in a state of disrepair so as to be a nuisance of the public or any abutting property owner.  Furthermore, all fences shall be constructed and maintained so that the exposed, outer-face is smooth, in neat condition and appearance at all times, and free of rust, rotting, chipped paint, loose/broken fencing panels, or exposed structural members.  The maintenance requirements apply to every fence, regardless of whether it was constructed before or after the adoption of this provision.  If the Building Official determines that more than 25\% of the fence is in disrepair, the Building Official shall notify the owner of the property on which the fence is located, in writing, of the existence of said disrepair, and a permit shall be secured by the owner within 14 days after receiving the notice from the Building Official.  Failure to secure the permit within 14 days after receiving the notice, or repair the fence in a reasonable time, may result in the city ordering the fence be repaired, painted or removed.    

\section{Temporary Family Health Care Dwellings}
\index{BUILDING REGULATIONS!Temporary Family Health Care Dwellings}
Pursuant to the authority granted by Minnesota Statutes, \textsection 462.3593, Subd. 9, the City of Crookston optsout of the requirements of Minnesota Statutes, \textsection 462.3593, which defines and regulates Temporary Family Health Care Dwellings.

\setcounter{section}{98}
\section{Penalty}
\index{BUILDING REGULATIONS!Penalty}
Every person who violates a section, division, or provision of this chapter when he or she performs an act thereby prohibited or declared unlawful, or fails to act when the failure is thereby prohibited or declared unlawful, or performs an act prohibited or declared unlawful or fails to act when the failure is prohibited or declared unlawful by a code adopted by reference by this chapter, and upon conviction thereof, shall be punished as for a misdemeanor except as otherwise stated in specific provisions hereof. The penalty which may be imposed for any crime which is a misdemeanor under this code, including Minnesota Statutes specifically adopted by reference, shall be a sentence of not more than 90 days or a fine of not more than \$1,000, or both. The costs of prosecution may be added. A separate offense shall be deemed committed upon each day during which a violation occurs or continues.
