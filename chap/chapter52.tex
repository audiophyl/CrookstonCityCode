\chapter*{Chapter 52: \\
	Sewer Service}
    \addstarredchapter{Chapter 52: Sewer Service}
    \vfill
    \minitoc
    \textbf{\emph{{Cross-reference:}}}\\
    \indent\emph{Penalty surcharge, see SEC. 50.98}
    \pagebreak


\subchapter{GENERAL PROVISIONS}

\section{Definitions\footnote{(‘83 Code, SEC. 3.40, Subd. 1)}}
\index{SEWER SERVICE!GENERAL PROVISIONS!Definitions}
\begin{description}
\item[ACT] The Federal Water Pollution Control Act also referred to as the Clean Water Act, as amended, 33 USC 1251, \emph{et seq.}
\item[ASTM] American Society for Testing Materials.
\item[APPROVING AUTHORITY] Concurrent approval by the Council, the City Engineer, the City Health Officer, and the Public Works Director.
\item[AUTHORITY] The City of Crookston, Minnesota, or its representative thereof.
\item[BOD$_{5}$ or BIOCHEMICAL OXYGEN DEMAND] The quantity of oxygen utilized in the biochemical oxidation of organic matter under standard laboratory procedure in five days at 20 degrees Centigrade in terms of milligrams per liter (mg/l).
\item[BUILDING DRAIN] That part of the lowest horizontal piping of a drainage system which receives the discharge from soil, waste, and other drainage pipes inside the walls of the building and conveys it to the building sewer, beginning five feet (1.5 meters) outside the building wall.
\item[BUILDING SEWER] The extension from the building drain to the public sewer or other place of disposal.
\item[CHEMICAL OXYGEN DEMAND (COD)] The quantity of oxygen utilized in the chemical oxidation of organic matter as determined by standard laboratory procedures, and as expressed in terms of milligrams per liter (mg/l).
\item[COMPATIBLE POLLUTANT] Biochemical oxygen demand, suspended solids, pH, and fecal coliform bacteria, plus additional pollutants identified in the NPDES/SDS permit if the treatment facilities are designed to treat the pollutants to a degree which complies with effluent concentration limits imposed by the permit.
\item[CONTROL MANHOLE] A structure specially constructed for the purpose of measuring flow and sampling of wastes.
\item[DIRECTOR] The Public Works Director or his or her authorized deputy, agent or representative.
\item[EASEMENT] An acquired legal right for the specific use of land owned by others.
\item[FECAL COLIFORM] Any number of organisms common to the intestinal tract of man and animals whose presence in sanitary sewage is an indicator of pollution.
\item[FLOATABLE OIL] Oil, fat, or grease in a physical state, such that it will separate by gravity from wastewater.
\item[GARBAGE] Animal and vegetable waste resulting from the handling, preparation, cooking, and serving of food.
\item[INCOMPATIBLE POLLUTANT] Any pollutant that is not defined as a compatible pollutant, including non-biodegradable dissolved solids.
\item[INDUSTRIAL WASTE] Gaseous, liquid, and solid wastes resulting from industrial or manufacturing processes, trade or business, or from the development, recovery, and processing of natural resources, as distinct from residential or domestic strength wastes.
\item[INDUSTRY] Any non-governmental or nonresidential user of a publicly owned treatment works which is identified in the Standard Industrial Classification Manual, latest edition, which is categorized in Divisions A, D, D, E, and I.
\item[INFILTRATION] Water entering the sewage system (including building drains and pipes) from the ground through such means as defective pipes, pipe joints, connections and manhole walls.
\item[INFILTRATION/INFLOW (I/I)] The total quantity of water from both infiltration and inflow.
\item[INFLOW] Water other than wastewater that enters a sewer system (including building drains) from sources such as, but not limited to, roof leaders, cellar drains, yard and area drains, foundation drains, drains from springs and swampy areas, manhole covers, cross-connections from storm sewers, catch basins, surface runoff, street wash waters or drainage.
\item[INTERFERENCE] The inhibition or disruption of the wastewater disposal system processes or operations which causes or significantly contributes to a violation of any requirement of the city’s NPDES and/or SDS Permit. The term includes a sewage sludge use or disposal by the city in accordance with published regulations providing guidelines under Section 405 of the Act or any regulations developed pursuant to the Solid Waste Disposal Act, the Clean Air Act, the Toxic Substance Control Act, or more stringent state criteria applicable to the method of disposal or use employed by the city.
\item[MPCA] Minnesota Pollution Control Agency.
\item[NATIONAL CATEGORICAL PRETREATMENT STANDARDS] Federal regulations establishing pretreatment standards for introduction of pollutants in publicly-owned wastewater treatment facilities which are determined to be not susceptible to treatment by the treatment facilities or would interfere with the operation of the treatment facilities, pursuant to Section 307(b) of the Act.
\item[NATIONAL POLLUTANT DISCHARGE ELIMINATION SYSTEM (NPDES) PERMIT] A permit issued by the MPCA, setting limits on pollutants that a permittee may legally discharge into navigable waters of the United States pursuant to Sections 402 and 405 of the Act.
\item[NATURAL OUTLET] Any outlet, including storm sewers and combined sewers, which overflows into a watercourse, pond, ditch, lake or other body of surface water or ground water.
\item[NON-CONTACT COOLING WATER] The water discharged from any use such as air conditioning, cooling or refrigeration, or during which the only pollutant added is heat.
\item[NORMAL DOMESTIC STRENGTH WASTE] Wastewater that is primarily introduced by residential users with a BOD$_{5}$ concentration not greater than 280 mg/l and a suspended solid (TSS) concentration not greater than 280 mg/l.
\item[OTHER WASTE] Municipal refuse, decayed wood, sawdust, shavings, bark, lime, sand, ashes, offal, oil tar, chemicals, and all other substances not sewage or industrial wastes.
\item[pH] The logarithm of the reciprocal of the concentration of hydrogen ions expressed in grams per liter of solution.
\item[PRETREATMENT] The treatment of wastewater from industrial sources prior to the introduction of the waste effluent into a publicly-owned treatment works.
\item[PROPERLY SHREDDED GARBAGE] Garbage that has been shredded to such a degree that all particles will be carried freely under the flow conditions normally prevailing in public sewers with no particle greater than ${^1/_2}$ inch (1.27 cm) in any dimension.
\item[SEWAGE] The spent water of a community.  The preferred term is wastewater.
\item[SEWER] A pipe or conduit that carries wastewater or drainage water.
\begin{enumerate}
\item \textbf{COLLECTION SEWER.} A sewer which has as its primary purpose the collection of wastewaters from individual point source discharges and connections.
\item \textbf{COMBINED SEWER.} A sewer intended to serve as a sanitary sewer and a storm sewer.
\item \textbf{FORCE MAIN.} A pipe in which wastewater is carried under pressure.
\item \textbf{INTERCEPTOR SEWER.} A sewer whose primary purpose is to transport wastewater from collection sewers to a treatment facility.
\item \textbf{PRIVATE SEWER.} A sewer which is not owned and maintained by a public authority.
\item \textbf{SANITARY SEWER.} A sewer intended to carry only liquid and water-carried wastes from residences, commercial buildings, industrial plants, and institutions together with minor quantities of ground, storm and surface waters which are not admitted intentionally.
\item \textbf{STORM SEWER or STORM DRAIN.} A drain or sewer intended to carry storm waters, surface runoff, ground water, sub-surface water, street wash water, drainage, and unpolluted water from any source.
\end{enumerate}
\item[SIGNIFICANT INDUSTRIAL USER] Any industrial user of the wastewater treatment facility which has a discharge flow in excess of 25,000 gallons per average work day, or has exceeded 5\% of the total flow received at the treatment facility, or whose waste contains a toxic pollutant in toxic amounts pursuant to Section 307(a) of the Act, or whose discharge has a significant effect, either singly or in combination with other contributing industries, on the wastewater disposal system, the quality of sludge, the system’s effluent quality, or emissions generated by the treatment system.
\item[SLUG] Any discharge of water or wastewater which in concentration of any given constituent, or in quantity of flow, exceeds for any period of duration longer than 15 minutes, more than five times the average 24-hour concentration of flows during normal operation, and which adversely affects the collection and/or performance of the wastewater treatment works.
\item[STATE DISPOSAL SYSTEM (SDS) PERMIT] Any permit (including any terms, conditions and requirements thereof) issued by the MPCA pursuant to M.S. \textsection 115.07, as it may be amended from time to time, for a disposal system as defined by M.S. \textsection 115.01, Subd. 8, as it may be amended from time to time.
\item[SUSPENDED SOLIDS (SS) or TOTAL SUSPENDED SOLIDS (TSS)] The total suspended matter that either floats on the surface of, or is in suspension in water, wastewater or other liquids, and is removable by laboratory filtering as prescribed in “Standard Methods for the Examination of Water and Wastewater”, latest edition, and referred to as non-filterable residue.
\item[TOXIC POLLUTANT] The concentration of any pollutant or combination of pollutants which upon exposure to or assimilation into any organism will cause adverse effects as defined in standards issued pursuant to Section 307 (a) of the Act.
\item[UNPOLLUTED WATER] Water of quality equal to or better than the effluent criteria in effect, or water that would not cause violation of receiving water quality standards, and would not be benefitted by discharge to the sanitary sewers and wastewater treatment facilities.
\item[USER] Any person who discharges or causes or permits the discharge of wastewater into the wastewater disposal system.
\item[WASTEWATER] The spent water of a community and referred to as sewage.  From the standpoint of source, it may be a combination of the liquid and water-carried wastes from residences, commercial buildings, industrial plants, and institutions together with any ground water, surface water and storm water that may be present.
\item[WASTEWATER TREATMENT WORKS or TREATMENT WORKS] An arrangement of any devices, facilities, structures, equipment, or processes owned or used by the city for the purpose of the transmission, storage, treatment, recycling and reclamation of municipal sewage, domestic sewage, or industrial wastewater, or structures necessary to recycle or reuse water including interceptor sewers, outfall sewers, collection sewers, pumping, power, and other equipment and their appurtenances; extensions, improvements, remodeling, additions, and alterations thereof, elements essential to provide a reliable recycled water supply such as standby treatment units and clear well facilities, and any works including land which is an integral part of the treatment process or is used for ultimate disposal of residues resulting from the treatment.
\item[WATERCOURSE] A natural or artificial channel for the passage of water, either continuously or intermittently.
\item[WPCF] The Water Pollution Control Federation.
\end{description}

\section{Use of Public Sewers Required}
\index{SEWER SERVICE!GENERAL PROVISIONS!Use of Public Sewers Required}
\subsection{}
It is unlawful for any person to place, deposit, or permit to be deposited in any unsanitary manner on public or private property within the city, or in any area under the jurisdiction of the city, any human or animal excrement, garbage, or other objectionable waste.
\subsection{}
It is unlawful to discharge to any natural outlet within the city, or in any area under the jurisdiction of the city, any sewage or other polluted waters, except where suitable treatment has been provided in accordance with subsequent provisions of this section and the city’s NPDES/SDS permit.
\subsection{}
Except as otherwise provided in this chapter, it is unlawful to construct, or maintain any privy, privy vault, septic tank, cesspool, or other facility intended or used for the disposal of sewage.
\subsection{}
The owner(s) of all houses, buildings, or properties used for human occupancy, employment, recreation or other purposes from which wastewater is discharged, and which are situated within the city and adjacent to any street, alley, or right-of-way, in which there is now located, or may in the future be located, a public sanitary sewer of the city, is required at the owner(s) expense to install a suitable service connection to the public sewer in accordance with provisions of this section, within 180 days of the date the public sewer is operational, provided the public sewer is within 100 feet of the structure generating the wastewater.  All future buildings constructed on property adjacent to the public sewer shall be required to immediately connect to the public sewer.  If sewer connections are not made pursuant to this section, an official 30-day notice shall be served instructing the affected property owner to make the connection.
\subsection{}
If the owner(s) fails to connect to a public sewer in compliance with a notice given under this chapter, the city may undertake to have the connection made and to collect the cost thereof under SEC. 50.21.\footnote{(‘83 Code, SEC. 3.40, Subd. 3) (Ord. 48, 2nd Series, effective 9-27-88) Penalty, see SEC. 50.99}

\section{Private Wastewater Disposal}
\index{SEWER SERVICE!GENERAL PROVISIONS!Private Wastewater Disposal}
\subsection{}
Where a public sewer is not available under the provisions of SEC. 52.02 of this chapter, the building sewer shall be connected to a private wastewater disposal system complying with the provisions of this section.
\subsection{}
Prior to commencement of construction of a private wastewater disposal system, the owner(s) shall first obtain a written permit signed by the city.  The application for the permit shall be made on a form furnished by the city, which the applicant shall supplement by any plans, specifications, and other information deemed necessary by the city.
\subsection{}
A permit for a private wastewater disposal system shall not become effective until the installation is completed to the satisfaction of the city or its authorized representative.  The city or its authorized representatives shall be allowed to inspect the work at any stage of construction, and, in any event, the applicant for the permit shall notify the city when work is ready for final inspection, and before any underground portions are covered.  The inspection shall be made within 24 hours of the receipt of notice.
\subsection{}
The type, capacities, location, and layout of a private wastewater disposal system shall comply with all requirements of Minn. Rules Chapter 7080, as it may be amended from time to time, entitled, “Individual Sewage Treatment System Standards.” No septic tank or cesspool shall be permitted to discharge to any natural outlet.
\subsection{}
At the time as a public sewer becomes available to a property serviced by a private wastewater disposal system, a direct connection shall be made to the public sewer within 180 days in compliance with this section, and within 30 days after the connection is made any septic tanks, cesspools, and similar private wastewater disposal system shall be cleaned of sludge, the bottom broken to permit drainage, and the tank or pit filled with suitable material.
\subsection{}
The owner(s) shall operate and maintain the private wastewater disposal facilities in a sanitary manner at all times at no expense to the city.
\subsection{}
No statement contained in this section shall be construed to interfere with any additional requirements that may be imposed by the MPCA or the Department of Health of the State of Minnesota.\footnote{(‘83 Code, SEC. 3.40, Subd. 4) (Ord. 48, 2nd Series, effective 9-27-88)}

\section{Tampering With, Damaging System Facilities}
\index{SEWER SERVICE!GENERAL PROVISIONS!Tampering With, Damaging System Facilities}
It is unlawful for any person to maliciously or willfully break, damage, destroy, uncover, deface or tamper with any structure, appurtenance, or equipment which is part of the wastewater facilities.\footnote{(‘83 Code, SEC. 3.40, Subd. 8) (Ord. 48, 2nd Series, effective 9-27-88) Penalty, see SEC. 50.99}

\section{User Rate Schedule and Charges}
\index{SEWER SERVICE!GENERAL PROVISIONS!User Rate Schedule and Charges}
Each user of sewer service shall pay the charge(s) applicable to the type of service, and in accordance with the provisions set forth in SEC. 52.55 through SEC. 52.60.\footnote{(‘83 Code, SEC. 3.40, Subd. 10) (Ord. 48, 2nd Series, effective 9-27-88)}\\


\subchapter{SEWER USE REGULATIONS}

\setcounter{section}{14}
\section{Building Sewers and Connections}
\index{SEWER SERVICE!SEWER USE REGULATIONS!Building Sewers and Connections}
\subsection{}
Any new connection(s) to the sanitary sewer system is prohibited unless sufficient capacity is available in all downstream facilities including, but not limited to, capacity for flow, BOD$_{5}$, and suspended solids, as determined by the Public Works Director.
\subsection{}
It is unlawful for any unauthorized person to uncover, make any connections with or opening into, use, alter, or disturb any public sewer or appurtenance thereof without first obtaining a written permit from the city.
\subsection{}
\subsubsection{}
There shall be two classes of building sewer permits:
\begin{enumerate}[{\indent}a)]
    \item For residential and commercial service; and
    \item For service to establishments producing industrial wastes.
\end{enumerate}
\subsubsection{}
In either case, the owner or his or her agent shall make application on a special form furnished by the city.  The permit application shall be supplemented by any plan, specifications, or other information considered pertinent in the judgment of the Public Works Director.  A permit and inspection fee shall be paid to the city at the time application is filed.
\subsection{}
All costs and expenses incidental to the installation and connection of the building sewer shall be borne by the owner.  The owner shall indemnify the city from any loss or damage that may directly or indirectly be occasioned by the installation of the building sewer.
\subsection{}
A separate and independent building sewer shall be provided for every building.
\subsection{}
Old building sewers may be used in connection with new buildings only when they are found, on examination and test by the Public Works Director, to meet all requirements of this section.
\subsection{}
The size, slope, alignment, materials of construction of a building sewer, and the methods to be used in excavating, placing of the pipe, jointing, testing, and backfilling the trench, shall all conform to the requirements of the building and plumbing codes or other applicable rules and regulations of the city and the current regulations of the Minnesota Pollution Control Agency.  In the absence of code provisions or in amplification thereof, the materials and procedures set forth in appropriate specifications of the ASTM and WPCF Manual of Practice No. 9 shall apply.
\subsection{}
Whenever possible, the building sewer shall be brought to the building at an elevation below the basement floor.  In all buildings in which any building drain is too low to permit gravity flow to the public sewer, sanitary sewage carried by the building drain shall be lifted in an approved means and discharged to the building sewer.
\subsection{}
It is unlawful for any person to make connection of roof downspouts, exterior foundation drains, areaway drains, or other sources of surface runoff or ground water to a building sewer or building drain which in turn is connected directly or indirectly to a public sanitary sewer.
\subsection{}
The connection of the building sewer into the public sewer shall conform to the requirements of the building and plumbing codes or other applicable rules and regulations of the city and State of Minnesota, to the procedures set forth in appropriate specifications of the ASTM and WPCF Manual of Practice No. 9.  All connections shall be made gastight and watertight, and verified by proper testing to prevent the inclusion of infiltration/inflow.  Any deviation from the prescribed procedures and materials must be approved by the Public Works Director before installation.
\subsection{}
The applicant for the building sewer permit shall notify the Public Works Director when the building sewer is ready for inspection and connection to the public sewer.  The connection shall be made under the supervision of the Public Works Director or his or her representative.
\subsection{}
All excavations for building sewer installation shall be adequately guarded with barricades and lights so as to protect the public from hazard.  Streets, sidewalks, parkways, and other public property disturbed in the course of the work shall be restored to their original condition.\footnote{(‘83 Code, SEC. 3.40, Subd. 5) (Ord. 48, 2nd Series, effective 9-27-88) Penalty, see SEC. 50.99}

\section{Discharges of Unpolluted Drainage}
\index{SEWER SERVICE!SEWER USE REGULATIONS!Discharges of Unpolluted Drainage}
\subsection{}
It is unlawful for any person to discharge or cause to be discharged any storm water, surface water, groundwater, roof runoff, subsurface drainage, non-contact cooling water, or unpolluted industrial process waters to any sanitary sewer.
\subsection{}
Storm water and all other unpolluted drainage shall be discharged to the sewers as are specifically designated as storm sewers, or to a natural outlet approved by the Public Works Director.  Industrial cooling water or unpolluted process waters may be discharged, on approval of the Public Works Director, to a storm sewer or natural outlet.\footnote{(‘83 Code, SEC. 3.40, Subd. 6 A., B.) (Ord. 48, 2nd Series, effective 9-27-88) Penalty, see SEC. 50.99}

\section{Prohibited Discharges}
\index{SEWER SERVICE!SEWER USE REGULATIONS!Prohibited Discharges}
It is unlawful for any person to discharge or cause to be discharged any of the following described waters or wastes to any public sewers:
\subsection{}
Any liquids, solids, or gases which by reason of their nature or quantity are, or may be, sufficient either alone or by interaction with other substances to cause fire or explosion or be injurious in any other way to the wastewater disposal system or to the operation of the system.  Prohibited materials include, but are not limited to, gasoline, kerosene, naphtha, benzene, toluene, xylene, ethers, alcohols, ketones, aldehydes, peroxides, chlorates, perchlorates, bromates, carbides, hydridges, and sulfides.
\subsection{}
Solid or viscous substances which will cause obstruction to the flow in a sewer or other interference with the operation of the wastewater treatment facilities such as, but not limited to, grease, garbage with particles greater than ${^1/_2}$ inch in any dimension, animal guts or tissues, paunch manure, bones, hair, hides or fleshings, entrails, whole blood, feathers, ashes, cinders, sand, spent lime, stone or marble dust, metal, glass, straw, shavings, grass clippings, rags, spent grains, spent hops, waste paper, wood, plastic, asphalt residues, residues from refining or processing of fuel, or lubricating oil, mud or glass grinding or polishing wastes.
\subsection{}
Any wastewater having a pH of less than 5.0 or greater than 9.5 or having any other corrosive property capable of causing damage or hazard to structures, equipment, or personnel of the wastewater disposal system.
\subsection{}
Any wastewater containing toxic pollutants in sufficient quantity, either singly or by interaction with other pollutants, to inhibit or disrupt any wastewater treatment process, constitute a hazard to humans or animals, or create a toxic effect in the receiving waters of the wastewater disposal system.  A toxic pollutant shall include, but not be limited to, any pollutant identified pursuant to Section 307(a) of the Clean Water Act.\footnote{(‘83 Code, SEC. 3.40, Subd. 6 C.) (Ord. 48, 2nd Series, effective 9-27-88) Penalty, see SEC. 50.99}

\section{Public Works Director May Limit Certain Discharges}
\index{SEWER SERVICE!SEWER USE REGULATIONS!Public Works Director May Limit Certain Discharges}
It is unlawful for any person to discharge or cause to be discharged the following described substances, materials, waters, or wastes if it appears likely in the opinion of the Public Works Director that the wastes can harm either the sewers, the sewage treatment process, or equipment, have an adverse effect on the receiving stream, or can otherwise endanger life, limb, public property, or constitute a nuisance. In forming his or her opinion as to the acceptability of these wastes, the Public Works Director will give consideration to the factors as the quantities of subject wastes in relation to flows and velocities in the sewers, materials of construction of the sewers, nature of the sewage treatment process, capacity of the sewage treatment plant, degree of treatability of wastes in the sewage treatment plant, and other pertinent factors. The Public Works Director may set limitations lower than limitations established in the regulations below if, in his or her opinion, more severe limitations are necessary to meet the above objectives. The substances prohibited are:
\subsection{}
Any wastewater having a temperature greater than 150 degrees F. (65.6 degrees C.), or causing, individually or in combination with other wastewater, the influent at the wastewater treatment plant to have a temperature exceeding 104 degrees F. (40 degrees C.), or having heat in amounts which inhibit biological activity in the wastewater treatment works resulting in interference therein.
\subsection{}
Any wastewater containing fats, wax, grease, or oils, whether emulsified or not, in excess of 100 mg/1 or containing substances which may solidify or become viscous at temperatures between 32 degrees F. and 150 degrees F. (0 degrees C. and 65.6 degrees C.); and any wastewater containing oil and grease concentrations of mineral origin of greater than 100 mg/1, whether emulsified or not.
\subsection{}
Any quantities of flow, concentrations, or both which constitute a “slug” as defined in SEC. 52.01.
\subsection{}
Any garbage not properly shredded.  Garbage grinders may be connected to sanitary sewers from homes, hotels, institutions, restaurants, hospitals, catering establishments, or similar places where garbage originates from the preparation of food on the premises or when served by caterers.
\subsection{}
Any noxious or malodorous liquids, gases, or solids which either singly or by interaction with other wastes are capable of creating a public nuisance or hazard to life, or are sufficient to prevent entry into the sewers for their maintenance and repair.
\subsection{}
Any wastewater with objectionable color not removed in the treatment process, such as, but not limited to, dye wastes and vegetable tanning solutions.
\subsection{}
Non-contact cooling water or unpolluted storm, drainage, or ground water.
\subsection{}
Wastewater containing inert suspended solids (such as, but not limited to, Fullers earth, lime slurries, and lime residues) or of dissolved solids (such as, but not limited to, sodium chloride and sodium sulfate) in the quantities that would cause disruption with the wastewater disposal system.
\subsection{}
Any radioactive wastes or isotopes of the half-life or concentration as may exceed limits established by the Public Works Director in compliance with applicable state or federal regulations.
\subsection{}
Any waters or wastes containing arsenic, cadmium, copper, cyanide, lead, mercury, nickel, silver, total chromium, zinc, phenolic compounds which cannot be removed by the wastewater treatment system, and similar objectionable or toxic substances; or wastes exerting an excessive chlorine requirement, to the degree that any material received in the composite sewage at the sewage treatment works exceeds the limits established by the Public Works Director for the materials.
\subsection{}
Any wastewater which creates conditions at or near the wastewater disposal system which violates any statute, rule, regulation or ordinance of any regulatory agency, or state or federal regulatory body.
\subsection{}
Any waters or wastes containing BOD$_{5}$ or suspended solids of the character and quantity that unusual attention or expense is required to handle the materials at the wastewater treatment works, except as may be permitted by specific written agreement subject to the provisions of this chapter.\footnote{(‘83 Code, SEC. 3.40, Subd. 6 D.) (Ord. 48, 2nd Series, effective 9-27-88) Penalty, see SEC. 50.99}

\section{Actions by Public Works Director}
\index{SEWER SERVICE!SEWER USE REGULATIONS!Actions by Public Works Director}
If any waters or wastes are discharged, or are proposed to be discharged to the public sewers, which waters contain the substances or possess the characteristics enumerated in SEC. 52.18, and which in the judgment of the Public Works Director, may have a deleterious effect upon the sewage works, processes, equipment, receiving waters, or which otherwise create a hazard to life or constitute a public nuisance, the Public Works Director may:
\subsection{}
Reject the wastes;
\subsection{}
Require pretreatment to an acceptable condition for discharge to the public sewers, pursuant to Section 307(b) of the Act and all addendums thereof;
\subsection{}
Require control over the quantities and rates of discharge; and/or
\subsection{}
Require payment to cover the added costs of handling and treatment of the wastes not covered by existing taxes or sewer charges under the provisions of SEC. 52.05 of this chapter.  If the Public Works Director permits the pretreatment or equalization of waste flows, the design and installation of the plants and equipment shall be subject to the review and approval of the Public Works Director pursuant to the requirements of the MPCA, and shall be located as to be readily and easily accessible for cleaning and inspection.\footnote{(‘83 Code, SEC. 3.40, Subd. 6 E.) (Ord. 48, 2nd Series, effective 9-27-88)}

\section{Dilution of Discharges Prohibited}
\index{SEWER SERVICE!SEWER USE REGULATIONS!Dilution of Discharges Prohibited}
It is unlawful for any person to increase the use of process water or, in any manner, attempt to dilute a discharge as a partial or complete substitute for adequate treatment to achieve compliance with the limitations contained in this section, or contained in the National Categorical Pretreatment Standards or any state requirements.\footnote{(‘83 Code, SEC. 3.40, Subd. 6 F.) (Ord. 48, 2nd Series, effective 9-27-88) Penalty, see SEC. 50.99}

\section{Maintenance of Pretreatment Facilities by Owner}
\index{SEWER SERVICE!SEWER USE REGULATIONS!Maintenance of Pretreatment Facilities by Owner}
Where pretreatment or flow-equalizing facilities are provided or required for any waters or wastes, they shall be maintained continuously in satisfactory and effective operation at the expense of the owner(s).\footnote{(‘83 Code, SEC. 3.40, Subd. 6 G.) (Ord. 48, 2nd Series, effective 9-27-88)}

\section{Grease, Oil, and Sand Interceptors}
\index{SEWER SERVICE!SEWER USE REGULATIONS!Grease, Oil, and Sand Interceptors}
The Public Works Director may require grease, oil, and sand interceptors when, in his or her opinion, they are necessary for the proper handling of liquid wastes containing floatable grease in excessive amounts, as specified in SEC. 52.18(D) of this subchapter, any flammable wastes as specified in SEC. 52.17(A) of this subchapter, sand or other harmful ingredients; except that the interceptors shall not be required for private living quarters or dwelling units.  All interceptors shall be of the type to be readily and easily accessible for cleaning and inspection.  In the maintaining of these interceptors, the owner(s) shall be responsible for the proper removal and disposal of the captured materials by appropriate means, and shall maintain a record of dates and means of disposal which are subject to review by the Public Works Director.  Any removal and hauling of the collecting materials not performed by the owner’s personnel, must be performed by a currently licensed waste disposal firm.\footnote{(‘83 Code, SEC. 3.40, Subd. 6 H.) (Ord. 48, 2nd Series, effective 9-27-88)}\\


\subchapter{INDUSTRIAL WASTE DISPOSAL}

\setcounter{section}{34}
\section{Required Measurement and Observation Facilities}
\index{SEWER SERVICE!INDUSTRIAL WASTE DISPOSAL!Required Measurement and Observation Facilities}
Any person discharging industrial waste into the city system or any connected disposal system, shall provide and maintain a suitable structure, or control manhole, with the necessary meters and other appurtenances in the building sewer to facilitate observation, sampling, and measurement of wastes by the Public Works Director. The structure shall be accessible and safely located, and shall be constructed in accordance with plans approved by the city. The structure shall be installed by the owner at his or her expense and shall be maintained by the owner to be safe and accessible at all times. The metered water supply to a source of industrial waste volume, where it can be established that the metered water supply and waste quantities are approximately the same, or where a measurable adjustment to the metered supply can be made to determine the waste volume, will be acceptable if approved by the Public Works Director.\footnote{(‘83 Code, SEC. 3.40, Subd. 7 A.) (Ord. 48, 2nd Series, effective 9-27-88)}

\section{Measurements, Tests and Analyses}
\index{SEWER SERVICE!INDUSTRIAL WASTE DISPOSAL!Measurements, Tests, and Analyses}
\subsection{Industrial Waste Analysis}
The owner of any property serviced by a building sewer carrying industrial wastes may, at the discretion of the Public Works Director, be required to provide laboratory measurements, tests, or analyses of waters or wastes to illustrate compliance with this chapter and any special condition for discharge established by the city or regulatory agencies having jurisdiction over the discharge.  The number, type, and frequency of sampling and laboratory analyses to be performed by the owner shall be as stipulated by the Public Works Director.  The industry must supply a complete analysis of the constituents of the wastewater discharge to assure that compliance with federal, state, and local standards are being met.  The owner shall report the results of measurements and laboratory analyses to the Public Works Director at the times and in the manner as prescribed by the Public Works Director.  The owner shall bear the expense of all measurements, analyses, and reporting required by the city.  At the times as deemed necessary, the city reserves the right to take measurements and samples for analysis by an independent laboratory.\footnote{(‘83 Code, SEC. 3.40, Subd. 7 B.)}
\subsection{Measurement and Test Procedures}
All measurements, tests and analyses of the characteristics of waters and wastes to which reference is made in this section shall be determined in accordance with the latest edition of “Standard Methods for the Examination of Water and Wastewater,” published by the American Public Health Association.  Sampling methods, location, times, duration and frequencies are to be determined on an individual basis subject to approval by the Public Works Director.\footnote{(‘83 Code, SEC. 3.40, Subd. 7 C.) (Ord. 48, 2nd Series, effective 9-27-88)}

\section{Accidental Discharges}
\index{SEWER SERVICE!INDUSTRIAL WASTE DISPOSAL!Accidental Discharges}
Where required by the city, the owner of any property serviced by a sanitary sewer shall provide protection from an accidental discharge of prohibited materials or other substances regulated by this chapter.  Where necessary, facilities to prevent accidental discharges of prohibited materials shall be provided and maintained at the owner’s expense.  Detailed plans showing facilities and operating procedures to provide this protection shall be submitted to the Public Works Director for review and approval prior to construction of the facility.  Review and approval of the plans and operating procedures shall not relieve any person from the responsibility to modify the person’s facility as necessary to meet the requirements of this chapter.  Accidental discharges of prohibited waste or slugs shall be reported to the Public Works Director by the person responsible for the discharge, or by the owner or occupant of the premises where the discharge or slug occurs, promptly upon obtaining knowledge of the fact of the discharge or slug.  The notification will not relieve any person of any liability for any expense, loss or damage to the wastewater treatment system or treatment process, or for any fines imposed on the city on account thereof under any state or federal law.  Employers shall insure that all employees who may cause or discover a discharge, are advised of the emergency notification procedure.\footnote{(‘83 Code, SEC. 3.40, Subd. 7 D.) (Ord. 48, 2nd Series, effective 9-27-88) Penalty, see SEC. 50.99}

\section{Special Agreements}
\index{SEWER SERVICE!INDUSTRIAL WASTE DISPOSAL!Special Agreements}
No statement contained in this chapter shall be construed as preventing any special agreement or arrangement between the city and any industrial discharger whereby an industrial waste of unusual strength or character may be accepted by the city for treatment, subject to payment thereof by the industrial discharger, provided that National Categorical Pretreatment Standards and the city’s NPDES and/or State Disposal System Permit limitations are not violated.\footnote{(‘83 Code, SEC. 3.40, Subd. 7 E.) (Ord. 48, 2nd Series, effective 9-27-88)}

\section{Obstructions; Catch Basin/Waste Trap}
\index{SEWER SERVICE!INDUSTRIAL WASTE DISPOSAL!Obstructions; Catch Basin/Waste Trap}
It is unlawful for any person having charge of any building or other premises which drains into the public sewer to permit any substance or matter which may form a deposit or obstruction to flow or pass into the public sewer.  Within 90 days after receipt of written notice from the city, the owner shall install a suitable and sufficient catch basin or waste trap, or if one already exists, shall clean out, repair or alter the same, and perform the other work as the Public Works Director may deem necessary.  Upon the owner’s refusal or neglect to install a catch basin or waste trap or to clean out, repair, or alter the same after the period of 90 days, the Public Works Director may cause the work to be completed at the expense of the owner or a representative thereof and the city may collect the cost thereof in the manner provided by SEC. 50.21 of this title.\footnote{(‘83 Code, SEC. 3.40, Subd. 7 F.) (Ord. 48, 2nd Series, effective 9-27-88) Penalty, see SEC. 50.99}

\section{Bad Service Connection}
\index{SEWER SERVICE!INDUSTRIAL WASTE DISPOSAL!Bad Service Connection}
Whenever any service connection becomes clogged, obstructed, broken or out-of-order, or detrimental to the use of the public sewer, or unfit for the purpose of drainage, the owner shall repair or cause the work to be done as the Public Works Director may direct.  Each day after 15 days that a person neglects or fails to so act shall constitute a separate violation of this section, and the Public Works Director may then cause the work to be done, and recover from the owner or agent the expense thereof in the manner provided by SEC. 50.21.\footnote{(‘83 Code, SEC. 3.40, Subd. 7 G.) (Ord. 48, 2nd Series, effective 9-27-88) Penalty, see SEC. 50.99}

\section{Car Washes}
\index{SEWER SERVICE!INDUSTRIAL WASTE DISPOSAL!Car Washes}
The owner or operator of any motor vehicle washing or servicing facility shall provide and maintain in serviceable condition at all times, a catch basin or waste trap in the building drain system to prevent grease, oil, dirt or any mineral deposit from entering the public sewer system.\footnote{(‘83 Code, SEC. 3.40, Subd. 7 H.) (Ord. 48, 2nd Series, effective 9-27-88) Penalty, see SEC. 50.99}


\subchapter{SEWER SERVICE CHARGE SYSTEM}

\setcounter{section}{54}
\section{Purpose}
\index{SEWER SERVICE!SEWER SERVICE CHARGE SYSTEM!Purpose}
The purpose of this subchapter is to provide for sewer service charges to recover costs associated with operation, maintenance and replacement to insure effective functioning of the wastewater treatment system and local capital costs incurred in the construction of the wastewater treatment system.\footnote{(‘83 Code, SEC. 3.41, Subd. 1) (Ord. 50, 2nd Series, effective 9-27-88)}

\section{Definitions}
\index{SEWER SERVICE!SEWER SERVICE CHARGE SYSTEM!Definitions}
For the purpose of this subchapter the following definitions shall apply, unless the context clearly indicates or requires a different meaning.
\begin{description}
\item[ADMINISTRATION] Those fixed costs attributable to administration of the wastewater treatment works (for example, billing and associated bookkeeping and accounting costs).
\item[BIOCHEMICAL OXYGEN DEMAND or BOD$_{5}$] The quantity of oxygen utilized in the biochemical oxidation of organic matter under standard laboratory procedure in five days at 20 degrees C., expressed in milligrams per liter.
\item[COMMERCIAL USER] Any place of business which discharges sanitary waste as distinct from industrial wastewater.
\item[COMMERCIAL WASTEWATERS] Domestic wastewater emanating from a place of business as distinct from industrial wastewater.
\item[DEBT SERVICE CHARGE] A charge levied on users of wastewater treatment facilities for the cost of repaying money bonded to construct the facilities.
\item[EXTRA STRENGTH WASTE] Wastewater having a BOD and/or TSS greater than Normal Domestic Strength Wastewater and not otherwise classified as an incompatible waste.
\item[GOVERNMENTAL USER] Users which are units, agencies or instrumentalities of federal, state or local government discharging Normal Domestic Strength Wastewater.
\item[INCOMPATIBLE WASTE] Waste that either singly or by interaction with other wastes interferes with any waste treatment process, constitutes a hazard to humans or animals, creates a public nuisance or creates any hazard in the receiving waters of the wastewater treatment works.
\item[INDUSTRIAL USERS or INDUSTRIES]
\begin{enumerate}
\item Entities that discharge into a publicly owned wastewater treatment works, liquid wastes resulting from the processes employed in industrial or manufacturing processes, or from the development of any natural resources.  These are identified in the Standard Industrial Classification Manual, 1972, Office of Management and Budget, as amended and supplemented under one of the following divisions:
    \begin{enumerate}[{\indent}a)]
        \item Division A:  Agriculture, Forestry and Fishing.
        \item Division B:  Mining.
        \item Division C:  Manufacturing.
        \item Division E:  Transportation, Communications, Gas, and Sanitary Sewers.
        \item Division I:  Services.
    \end{enumerate}
\item For the purpose of this definition, domestic waste shall be considered to have the following characteristics:
    \begin{enumerate}[{\indent}a)]
        \item BOD$_{5}$: Less than 280 mg/l.
        \item Suspended solids:  Less than 280 mg/l.
    \end{enumerate}
\item Any non-governmental user of a publicly owned treatment works which discharges wastewater to the treatment works which contains toxic pollutants or poisonous solids, liquids, or gases in sufficient quantity either singly or by interaction with other wastes, to contaminate the sludge of any municipal systems, or to injure or to interfere with any sewage treatment process, or which constitutes a hazard to humans or animals, creates a public nuisance, or creates any hazard in or has an adverse effect on the waters receiving any discharge from the treatment works.
\end{enumerate}
\item[INDUSTRIAL WASTEWATER] The liquid processing wastes from an industrial manufacturing process, trade, or business including, but not limited to, all Standard Industrial Classification Manual Divisions A, B, D, E and I manufacturers as distinct from domestic wastewater.
\item[INSTITUTIONAL USER] Users other than commercial, governmental, industrial or residential users, discharging primarily normal domestic strength wastewater (for example, non-profit organizations).
\item[NORMAL DOMESTIC STRENGTH WASTEWATER] Wastewater that is primarily produced by residential users, with BOD$_{5}$ concentrations not greater than 280 mg/l and suspended solids concentration not greater than 280 mg/l.
\item[OPERATION AND MAINTENANCE] Activities required to provide for the dependable and economical functioning of the treatment works, throughout the design or useful life, whichever is longer, of the treatment works, and at the level of performance for which the treatment works were constructed.  \textbf{OPERATION AND MAINTENANCE} includes replacement.
\item[OPERATION AND MAINTENANCE COSTS] Expenditures for operation and maintenance, including replacement.
\item[PUBLIC WASTEWATER COLLECTION SYSTEM] A system of sanitary sewers owned, maintained, operated and controlled by the city.
\item[REPLACEMENT] Obtaining and installing of equipment, accessories, or appurtenances which are necessary during the design life or useful life, whichever is longer, of the treatment works to maintain the capacity and performance for which the works were designed and constructed.
\item[REPLACEMENT COSTS] Expenditures for replacement.
\item[RESIDENTIAL USER] A user of the treatment facilities whose premises or building is used primarily as a residence for one or more persons, including dwelling units such as detached and semi-detached housing, apartments, and mobile homes; and which discharges primarily normal domestic strength sanitary wastes.
\item[SANITARY SEWER] A sewer intended to carry only liquid and water carried wastes from residences, commercial buildings, industrial plants, and institutions, together with minor quantities of ground, storm, and surface waters which are not admitted intentionally.
\item[SEWER SERVICE CHARGE] The aggregate of all charges, including charges for operation, maintenance, replacement, debt service, and other sewer related charges that are billed periodically to users of the wastewater treatment facilities.
\item[SEWER SERVICE FUND] A fund into which income from sewer service charges is deposited along with other income, including taxes intended to retire debt incurred through capital expenditure for wastewater treatment.  Expenditure of the sewer service fund will be for operation, maintenance, and replacement costs; and to retire debt incurred through capital expenditure for wastewater treatment.
\item[SLUG] Any discharge of water or wastewater which in concentration of any given constituent or in quantity of flow exceeds for any period of duration longer than 15 minutes more than five times the average 24 hour concentration of flows during normal operation and which adversely affects the collection system and/or performance of the wastewater treatment works.
\item[STANDARD INDUSTRIAL CLASSIFICATION MANUAL] Office of Management and Budget, 1972.
\item[SUSPENDED SOLIDS (SS) or TOTAL SUSPENDED SOLIDS] The total suspended matter that either floats on the surface or is in suspension in water, wastewater or other liquids, and is removable by laboratory filtering as prescribed in “Standard Methods for the Examination of Water and Wastewater,” latest edition, and referred to as non-filterable residue.
\item[TOXIC POLLUTANT] The concentration of any pollutant or combination of pollutants as defined in standards issued pursuant to Section 307(a) of the Act, which upon exposure to or assimilation into any organism will cause adverse effects.
\item[USER CHARGE] A charge levied on users of a treatment works for the user’s proportionate share of the cost of operation and maintenance, including replacement.
\item[USERS] Those residential, commercial, governmental, institutional and industrial establishments which are connected to the public sewer collection system.
\item[WASTEWATER] The spent water of a community, also referred to as sewage.  From the standpoint of source it may be a combination of the liquid and water-carried wastes from residences, commercial buildings, industrial plants, and institutions together with any ground water, surface water and storm water that may be present.
\item[WASTEWATER TREATMENT WORKS or TREATMENT WORKS] An arrangement of any devices, facilities, structures, equipment, or processes owned or used by the city for the purpose of the transmission, storage, treatment, recycling, and reclamation of municipal sewage, domestic sewage or industrial wastewater, or structures necessary to recycle or reuse water including interceptor sewers, outfall sewers, collection sewers, pumping, power, and other equipment and their appurtenances; extensions, improvements, remodeling, additions, and alterations thereof; elements essential to provide a reliable recycled water supply such as standby treatment units and clear well facilities; and any works including land which is an integral part of the treatment process or is used for ultimate disposal of residues resulting from the treatment.\footnote{(‘83 Code, SEC. 3.41, Subd. 2) (Ord. 50, 2nd Series, effective 9-27-88)}
\end{description}

\section{Establishment of Sewer Service Charge System}
\index{SEWER SERVICE!SEWER SERVICE CHARGE SYSTEM!Establishment of Sewer Service Charge System}
\subsection{}
A sewer service charge system is hereby established whereby all revenue collected from users of the wastewater treatment facilities will be used to affect all expenditures incurred for annual operation, maintenance, and replacement, and for debt service on capital expenditure incurred in constructing the wastewater treatment works.
\subsection{}
Each user shall pay its proportionate share of operation, maintenance and replacement costs of the treatment works, based on the user’s proportionate contribution to the total wastewater loading from all users.
\subsection{}
Each user shall pay debt service charges to retire local capital costs as determined by the Council.
\subsection{}
Sewer service rates and charges to users of the wastewater treatment facility shall be determined and fixed in a “sewer service charge system” developed according to the provisions of this subchapter. The sewer service charge system shall be adopted in the manner provided by SEC. 50.02.
\subsection{}
Revenues collected for sewer service shall be billed in a separate fund known as the “Sewer Service Fund.” Income from revenues collected will be expended to offset the cost of operation, maintenance and equipment replacement for the facility and to retire the debt for capital expenditure.
\subsection{}
Sewer service charges and the sewer service fund will be administered in accordance with the provisions of SEC. 52.60.\footnote{(‘83 Code, SEC. 3.41, Subd. 3) (Ord. 50, 2nd Series, effective 9-27-88)}

\section{Determination of Sewer Service Charges}
\index{SEWER SERVICE!SEWER SERVICE CHARGE SYSTEM!Determination of Sewer Service Charges}
\subsection{}
\begin{enumerate}
\item Users of the wastewater treatment works shall be identified as belonging to one of the following user classes:
    \begin{enumerate}[{\indent}a)]
        \item Residential.
        \item Commercial.
        \item Industrial.
        \item Institutional.
        \item Governmental.
    \end{enumerate}
\item The allocation of users to these categories for the purpose of assessing user charges and debt service charges shall be the responsibility of the Clerk-Treasurer.  Allocation of users to user classes shall be based on the substantive intent of the definitions of these classes contained herein.
\end{enumerate}
\subsection{}
Each user shall pay operation, maintenance, and replacement costs in proportion to the user’s proportionate contribution of wastewater flows and loadings to the treatment plant, with the minimum rate for loadings of BOD and TSS being the rate established for concentrations of 280 mg/l BOD and 280 mg/l TSS (for example, Normal Domestic Strength Wastewater).  Those industrial users discharging segregated normal domestic strength wastewater only, can be classified as commercial users for the purpose of rate determination.
\subsection{}
The charges assessed residential users and those users of other classes discharging normal domestic strength wastewater shall be established proportionately according to billable wastewater volume.  Billable wastewater volume shall be calculated as follows:
\subsubsection{Residential Users}
Billable wastewater volume for residential users shall be calculated on the basis of metered water usage.  The per month billable wastewater volume shall be equal to monthly metered water usage.  The city may require residential users to install water meters for the purpose of determining billable wastewater volume.
\subsubsection{Non-residential Users}
The billable wastewater volume of non-residential users may be determined in the same manner as for residential users, except that if the city determines that there are significant seasonal variations in the metered water usage of non-residential users resulting in a proportionate increase in wastewater volume; then billable wastewater volume shall be calculated on the basis of monthly metered usage as recorded throughout the year and calculated on the basis of wastewater flow meters.  The city may, at its discretion, require non-residential users to install the additional water meters or wastewater flow meters as may be necessary to determine billable wastewater volume.
\subsection{}
Determination of user charges.  User charges for normal domestic strength users shall be determined as follows:
\subsubsection{Calculation of Unit Cost for Treatment of Normal Domestic Strength Wastewater}
Where:
\begin{description}
\item[Uomr] = Unit Cost for Operation, Maintenance and Replacement in \$/Kgal
\item[Comr] = Total annual OM\&R costs plus nonpayment contingency.
\item[Tbwv] = Total annual billable wastewater in kgal.
\end{description}
\subsubsection{Calculation of User Charge}
Where:
\begin{description}
\item[Uc] = User charge.
\item[Uomr] = Unit cost for Operation, Maintenance and Replacement in \$/Kgal.
\item[bwv] = Billable wastewater volume of a particular user in kgal.
\end{description}
\subsection{}
Recovery of local construction costs.  Local construction costs of the wastewater treatment facilities will be recovered through a per-connection debt service charge determined as follows:
\begin{equation*}
Cc = Cds/Tc
\end{equation*}
Where:
\begin{description}
\item[Cc] = Debt service charge per connection.
\item[Cds] = Cost of annual debt service plus nonpayment contingency.
\item[Tc] = Total number of connections to the wastewater treatment facilities.
\end{description}
\subsection{}
Determination of sewer service charge relating to local construction costs.  The sewer service charge for local construction costs for a particular connection shall be determined as follows:
\begin{equation*}
SSC = UC + Cc
\end{equation*}
Where:
\begin{description}
\item[SSC] = Sewer Service Charge.
\item[Uc] = User Charge.
\item[Cc] = Connection Charge for Debt Service.
\end{description}
\subsection{}
The sewer service charges established in this section shall not prevent the assessment of additional charges to users who discharge wastes with concentrations greater than normal domestic strength or wastes of unusual character, or contractual agreements with the users, as long as the following conditions are met:
\subsubsection{}
The user pays operation, maintenance, and replacement costs in proportion to the user’s proportionate contribution of wastewater flows and loadings to the treatment plant, and no user is charged at a rate less than that of normal domestic strength wastewater.
\subsubsection{}
The measurements of the wastes are conducted according to the latest edition of “Standard Methods for the Examination of Water and Wastewater” in a manner acceptable to the city as provided for in this chapter.
\subsubsection{}
A study of unit costs of collection and treatment processes attributable to flow, BOD, TSS and other significant loadings shall be developed for determining the proportionate allocation of costs to flows and loadings for users discharging wastes of greater than normal domestic strength or wastes of unusual character.\footnote{(‘83 Code, SEC. 3.41, Subd. 4) (Ord. 50, 2nd Series, effective 9-27-88)}

\section{Sewer Service Fund}
\index{SEWER SERVICE!SEWER SERVICE CHARGE SYSTEM!Sewer Service Fund}
\subsection{}
A sewer service fund is hereby established as an income fund to receive all revenues generated by the sewer service charge system, and all other income dedicated to the operation, maintenance, replacement and construction of the wastewater treatment works, including taxes, special charges, fees, and assessments intended to retire construction debt.  The city also establishes the following accounts as income and expenditure accounts within the sewer service fund:
\begin{enumerate}
\item Operation and maintenance account;
\item Equipment replacement account; and
\item Debt retirement account.
\end{enumerate}
\subsection{}
All revenue generated by the sewer service charge system, and all other income pertinent to the treatment system, including taxes and special assessments dedicated to retire construction debt, shall be held by the Clerk-Treasurer separate and apart from all other funds of the city.  Funds received by the sewer service fund shall be transferred to the operation and maintenance account, the equipment replacement account, and the debt retirement account in accordance with state and federal regulations and the provisions of this section.
\subsection{}
Revenue generated by the sewer service charge system sufficient to insure adequate replacement throughout the design or useful life, whichever is longer, of the wastewater facility shall be held separate and apart in the equipment replacement account and dedicated to affecting replacement costs.  Interest income generated by the equipment replacement account shall remain in the equipment replacement account.
\subsection{}
Revenue generated by the sewer service charge system sufficient for operation and maintenance shall be held separate and apart in the operation and maintenance account.\footnote{(‘83 Code, SEC. 3.41, Subd. 5) (Ord. 50, 2nd Series, effective 9-27-88)}

\section{Administration}
\index{SEWER SERVICE!SEWER SERVICE CHARGE SYSTEM!Administration}
The sewer service charge system and sewer service fund shall be administered according to the following provisions.
\subsection{}
The Clerk-Treasurer shall maintain a proper system of accounts suitable for determining the operation and maintenance, equipment replacement and debt retirement costs of the treatment works, and shall furnish the Council with a report of the costs annually in May.  The Council shall annually determine whether or not sufficient revenue is being generated for the effective operation, maintenance, replacement, and management of the treatment works, and whether sufficient revenue is being generated for debt retirement. The Council will also determine whether the user charges are distributed proportionately to each user in accordance with division (B) of this section and Section 204(b)(2)(A) of the Federal Water Pollution Control Act, as amended. The city shall thereafter, but not later than the end of the year, reassess, and as necessary revise the sewer service charge system then in use to insure the proportionality of the user charges and to insure the sufficiency of funds to maintain the capacity and performance to which the facilities were constructed, and to retire the construction debt.
\subsection{}
In accordance with federal and state requirements each user will be notified annually in conjunction with a regular billing of that portion of the sewer service charge attributable to operation, maintenance and replacement.
\subsection{}
In accordance with federal and state requirements, the Clerk-Treasurer shall be responsible for maintaining all records necessary to document compliance with the sewer service charge system adopted.
\subsection{}
Bills for sewer service charges shall be rendered according to the provisions of SEC. 50.15 through SEC. 50.23.
\subsection{}
Any costs caused by discharges to the treatment works of toxics or other incompatible wastes, including the costs of restoring wastewater treatment services, clean up and restoration of the receiving water and environs, and sludge disposal, shall be borne by the discharger(s) of the wastes.  The city may collect any portion of the costs incurred by it in a civil action or, in the alternative and at the option of the city, as a special assessment against the premises causing the discharge and from which the discharge occurred. All costs are hereby made a lien upon the premises causing the discharge and upon which the discharge occurred. All charges which are on July 31 of each year more than 45 days past due, shall, after notice and hearing, be certified by the Clerk-Treasurer of the city to the County Auditor between the first and tenth day of October of each year, and the Clerk-Treasurer in so certifying the charges to the County Auditor shall specify the amount thereof, the description of the premises served and the name of the owner thereof. The amount so certified shall be extended by the Auditor on the tax rolls against the premises in the same manner as other taxes, and collected by the County Treasurer, and paid to the city along with other taxes.\footnote{(‘83 Code, SEC. 3.41, Subd. 6) (Ord. 50, 2nd Series, effective 9-27-88)}\\

\subchapter{ADMINISTRATION AND ENFORCEMENT}

\setcounter{section}{69}
\section{Control by Public Works Director}
\index{SEWER SERVICE!ADMINISTRATION AND ENFORCEMENT!Control by Public Works Director}
The Public Works Director has control of and general supervision over all public sewers and is responsible for administering the provisions of this chapter to the end that a proper and efficient public sewer is maintained.\footnote{(‘83 Code, SEC. 3.40, Subd. 2) (Ord. 48, 2nd Series, effective 9-27-88)}

\section{Powers and Authority of Inspectors}
\index{SEWER SERVICE!ADMINISTRATION AND ENFORCEMENT!Powers and Authority of Inspectors}
\subsection{}
The Public Works Director or other duly authorized employees of the city, bearing proper credentials and identification, shall be permitted to enter all properties for the purpose of inspection, observations, measurement, sampling, and testing pertinent to the discharges to the city’s sewer system in accordance with the provisions of this chapter.
\subsection{}
The Public Works Director or other duly authorized employees are authorized to obtain information concerning industrial processes which have a direct bearing on the type and source of discharge to the wastewater collection system.  Any industry may withhold information considered confidential; however, the industry must establish that the revelation to the public of the information in question, might result in an advantage to competitors.
\subsection{}
While performing necessary work on private properties, the Public Works Director or duly authorized employees of the city shall observe all safety rules applicable to the premises established by the company or owner, and the company or owner shall be held harmless for injury or death to the city employees and the city shall indemnify the company or owner against loss or damage to its or his or her property by city employees and against liability claims and demands for personal injury or property damage asserted against the company or owner and growing out of the gauging and sampling operation, except as may be caused by negligence or failure of the company or owner to maintain safe conditions as required in this chapter.
\subsection{}
The Public Works Director or other duly authorized employees of the city bearing proper credentials and identification shall be permitted to enter all private properties through which the city holds a duly negotiated easement for the purposes of, but not limited to, inspection, observation, measurement, sampling, repair, and maintenance of any portion of the wastewater facilities lying within the easement.  All entry and subsequent work, if any, on the easement, shall be done in full accordance with the terms of the duly negotiated easement pertaining to the private property involved.\footnote{(‘83 Code, SEC. 3.40, Subd. 9) (Ord. 48, 2nd Series, effective 9-27-88)}

\section{Violations}
\index{SEWER SERVICE!ADMINISTRATION AND ENFORCEMENT!Violations}
\subsection{}
Any person who violates any provision of this chapter, shall be served by the city with written notice stating the nature of the violation and providing a reasonable time limit for the satisfactory correction thereof.  The offender shall, within the period of time stated in the notice, permanently cease all violations.
\subsection{}
If a person continues any violation beyond the time limit provided for in division (A), each day in which any violation occurs shall be deemed as a separate offense.
\subsection{}
Any person violating a provision of this chapter is liable to the city for any expense, loss, or damage occasioned by the city by reason of the violation.\footnote{(‘83 Code, SEC. 3.40, Subd. 11) (Ord. 48, 2nd Series, effective 9-27-88) Penalty, see SEC. 50.99}
