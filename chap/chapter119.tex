\chapter*{Chapter 119: \\
	Tobacco Sales}
    \addstarredchapter{Chapter 119: Tobacco Sales}
    \minitoc
    \pagebreak

\section{Purpose}
Because the city recognizes that many persons under the age of 18 years purchase or otherwise obtain, possess and use tobacco and tobacco-related devices, and the sales, possession and use are violations of both state and federal laws; and because studies have shown that most smokers begin smoking before they have reached the age of 18 years and that those persons who reached the age of 18 years without having started smoking are significantly less likely to begin smoking; and because smoking has been shown to be the cause of several serious health problems which subsequently place a financial burden on all levels of government; this chapter is intended to regulate the sale, possession and use of tobacco and tobacco-related devices for the purpose of enforcing and furthering existing laws, to protect minors against the serious affects associated with the illegal use of tobacco and tobacco-related devices, and to further the official public policy of the State of Minnesota in regard to preventing young people from starting to smoke as stated in M.S. \textsection 144.391, as it may be amended from time to time.\\
\emph{(Ord. 131, 2nd Series, passed 6-9-98)}
\section{Definitions}
For the purpose of this chapter the following definitions shall apply, unless the context clearly indicates or requires a different meaning.
\begin{description}
    \item[COMPLIANCE CHECKS] The system the city uses under this chapter. COMPLIANCE CHECKS shall involve the use of minors as authorized by this chapter. COMPLIANCE CHECKS shall also mean the use of minors who attempt to purchase tobacco or tobacco-related devices for educational, research and training purposes, as authorized by state and federal laws. COMPLIANCE CHECKS may also be conducted by other units of government for the purpose of enforcing appropriate federal, state or local laws and regulations relating to tobacco and tobacco-related devices.
    \item[MINOR] Any natural person who has not yet reached the age of 18 years.
    \item[MOVABLE PLACE OF BUSINESS] Any form of business operated out of a truck, van, automobile or other type of vehicle or transportable shelter and not a fixed address store front or other permanent type of structure authorized for sales transactions.
    \item[SELF-SERVICE] The open display of tobacco or tobacco-related devices in any manner where any person has access to the tobacco or tobacco-related devices without the assistance or intervention of the licensee or the licensee’s employee.  The assistance or intervention entails the actual physical exchange of the tobacco or tobacco related device between the customer and the licensee or employee.  Self-service does not include vending machines.
    \item[TOBACCO] Any substance or item containing tobacco leaf, including, but not limited to, cigarettes; cigars; pipe tobacco; snuff; fine cut or other chewing tobaccos; cheroots; stogies; perique; granulated; plug cut; crimp cut, ready-rubbed; and other smoking tobacco; snuff flowers; cavendish; shorts; plug and twist tobaccos; dipping tobaccos; refuse scraps, clippings, cuttings and sweepings of tobacco; and other kinds and forms of tobacco leaf prepared in the manner as to be suitable for chewing, sniffing or smoking.
    \item[TOBACCO-RELATED DEVICES] Any tobacco as well as a pipe, rolling papers or other device intentionally designed or intended to be used in a manner which enables the chewing, sniffing or smoking of tobacco.
    \item[VENDING MACHINE] Any mechanical, electric or electronic, or other type of device which dispenses tobacco or tobacco-related devices upon the insertion of money, tokens, or other form of payment directly into the machine by the person seeking to purchase the tobacco or tobacco-related device.
\end{description}
\emph{(Ord. 131, 2nd Series, passed 6-9-98)}
\section{License Required}
It is unlawful for any person, directly or indirectly, to keep for retail sale, sell at retail or offer to sell at retail any tobacco or tobacco-related device unless a license to do so is first obtained from the city.  Separate licenses shall be issued for the sale of tobacco or tobacco-related devices at each fixed place of business and no license shall be issued for a movable place of business.\\
\emph{(Ord. 131, 2nd Series, passed 6-9-98)  Penalty, see SEC. 119.99}
\section{Prohibited Sales Methods}
It is unlawful for a licensee to sell or distribute tobacco or tobacco-related devices by vending machine unless minors are at all times prohibited from entering the licensed establishment.  It is unlawful for a licensee to sell or distribute tobacco or tobacco related devices by self-service unless the licensed establishment generates 90\% or more of its revenues from the sale of tobacco and minors are at all times prohibited from entering the licensed establishment.  It is unlawful for any person to permit or assist with the unlawful sale or distribution of tobacco or tobacco related devices by vending machine or self-service upon property the person owns or controls.\\
\emph{(Ord. 131, 2nd Series, passed 6-9-98)  Penalty, see SEC. 119.99}
\section{Compliance Checks and Inspections}
All licensed premises shall be open to inspection by local law enforcement or other authorized city officials during regular business hours. From time to time, but at least twice per year, the city shall conduct compliance checks by engaging, with written consent of their parents or guardians, minors over the age of 15 years but less than 18 years, to enter the licensed premises to attempt to purchase tobacco or tobacco-related devices. Minors used for the purpose of compliance checks shall be supervised by designated law enforcement officers or other designated city personnel.  Minors used for compliance checks shall not be guilty of the unlawful purchase or attempted purchase, nor the unlawful possession of tobacco or tobacco-related devices when the items are obtained or attempted to be obtained as a part of the compliance check. No minor used in compliance checks shall attempt to use a false identification misrepresenting the minor’s age and all minors lawfully engaged in a compliance check shall answer all questions about the minor’s age asked by the licensee or his or her employee and shall produce any identification, if any exists, for which he or she is asked.  Nothing in this chapter shall prohibit compliance checks authorized by state or federal laws for educational, research or training purposes, or required for the enforcement of a particular state or federal law.\\
\emph{(Ord. 131, 2nd Series, passed 6-9-98)}
\section{Illegal Acts}
Unless otherwise provided, the following acts are a violation of this chapter.
\subsection{Illegal Sales}
It is unlawful for any person to sell or otherwise provide any tobacco or tobacco-related device to a minor.
\subsection{Illegal Possession}
It is unlawful for any minor to have in his or her possession any tobacco or tobacco-related device. This division shall not apply to minors lawfully involved in a compliance check.
\subsection{Illegal Use}
It is unlawful for any minor to smoke, chew, sniff or otherwise use any tobacco or tobacco-related device.
\subsection{Illegal Procurement}
It is unlawful for any minor to purchase or attempt to purchase or otherwise obtain any tobacco or tobacco-related device and it is unlawful for any person to purchase or otherwise obtain the items on behalf of a minor.  It is further unlawful for any person to coerce or attempt to coerce a minor to illegally purchase or otherwise obtain or use any tobacco or tobacco-related device.  This division shall not apply to minors lawfully involved in a compliance check.
\subsection{Use of False Identification}
It is unlawful for any minor to attempt to disguise his or her true age by the use of a false form of identification, whether the identification is that of another person or one on which the age of the person has been modified or tampered with to represent an age older than the actual age of the person.\\
\emph{(Ord. 131, 2nd Series, passed 6-9-98)  Penalty, see SEC. 119.99}
\section{Exceptions and Defenses}
Nothing in this chapter shall prevent the providing of tobacco or tobacco-related devices to a minor as part of a lawfully recognized religious, spiritual, or cultural ceremony.  It shall be an affirmative defense to the violation of this chapter for a person to have reasonably relied on proof of age as described by state law.\\
\emph{(Ord. 131, 2nd Series, passed 6-9-98)}
\section{Violations}
\subsection{Notice}
Upon discovery of a suspected violation, the alleged violator shall be issued, either personally or by mail, a citation that sets forth the alleged violation and which shall inform the alleged violator of his or her right to be heard on the accusation.
\subsection{Hearings}
If a person accused of violating this chapter so requests, a hearing shall be scheduled, the time and place of which shall be published and provided to the accused violator.
\subsection{Hearing Officer}
The City Administrator shall appoint the hearing officer.
\subsection{Decision}
If the hearing officer determines that a violation of this chapter did occur, that decision, along with the hearing officer’s reasons for finding a violation and the penalty to be imposed, shall be recorded in writing, a copy of which shall be provided to the accused violator.  Likewise, if the hearing officer finds that no violation occurred or finds grounds for not imposing any penalty, the findings shall be recorded and a copy provided to the acquitted, accused violator.
\subsection{Appeals}
Appeals of any decision made by the hearing officer under this chapter shall be filed in the District Court for the County of Polk.
\subsection{Enforcement Alternatives}
Nothing in this chapter shall prohibit the city from seeking prosecution as a misdemeanor for any alleged violation of this section.  If the city elects to seek misdemeanor prosecution, no administrative penalties shall be imposed on the defendant in the prosecution.  Nothing in this chapter shall prohibit the city from seeking criminal prosecution or imposing administrative penalties on both the licensee’s employee and the licensee.
\subsection{Continued Violation}
Each violation, and every day in which a violation occurs or continues, shall constitute a separate offense.\\
\emph{(Ord. 131, 2nd Series, passed 6-9-98)}

\setcounter{section}{98}
\section{Penalty}
\subsection{Licensees}
Any licensee found to have violated this chapter, or whose employees shall have violated this chapter, shall be charged an administrative fine of at least \$75 for a first violation of this chapter; at least \$200 for a second offense at the same licensed premises within a 24-month period; and at least \$250 for a third or subsequent offense at the same location within a 24-month period.  In addition, after the third offense, the license shall be suspended for not less than seven days.
\subsection{Other Individuals}
Other individuals, other than minors regulated by division (C) of this section, found to be in violation of this chapter, shall be charged an administrative fee of at least \$50.
\subsection{Minors}
Minors found in unlawful possession of, or who unlawfully purchase or attempt to purchase tobacco or tobacco-related devices, shall be subject to a penalty or penalties as determined by the appropriate court having jurisdiction over the violations occurring within the city.  The city shall consult with interested educators, parents, children and representatives of the court system to develop alternate penalties for minors who purchase, possess and consume tobacco.  The city and the interested persons shall consider a variety of options, including, but not limited to, tobacco free education programs, notice to schools, parents, community service and other court diversion programs.\\
\emph{(Ord. 131, 2nd Series, passed 6-9-98)}
